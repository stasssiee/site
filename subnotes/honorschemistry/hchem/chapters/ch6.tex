\documentclass[../hchem.tex]{subfiles}
\graphicspath{{\subfix{../figures/}}}
\begin{document}
\chapter{Stoichiometry}
\section{Stoichiometry}
Stoichiometry is the measurement and calculation of the amounts of reactants and products in chemical reactions.

Balanced chemical equations represent the relationship between the number of moles of reactants and the number of moles of products.

Mole ratio is the conversion factor for any two reactants or products in a chemical reaction.

\textbf{Mole to Mole Problems}
\begin{enumerate}
    \item Write the balanced equation 
    \item Write the strategy (molar road map)
    \item Set up the correct calculation
\end{enumerate}

Stoichiometry Road Map - Grams of A $\leftrightarrow$ Moles of A $\leftrightarrow$ moles of B $\leftrightarrow$ grams of B 

Exercise - Iron reacts with carbon dioxide to form iron(III) oxide and carbon monoxide. How many moles of carbon dioxide are needed to produce 2.2 moles of iron(III) oxide? (6.6 mol CO$_2$)

Exercise - How many grams of magnesium chloride are produced when 0.500 moles of magnesium reacts with an excess of hydrochloric acid? (47.6 g MgCl$_2$)

Exercise - How many moles of zinc sulfate are produced when 4.55 g of zinc reacts with an excess of sulfuric acid? (0.0696 mol ZnSO$_4$)

Exercise - Calcium carbonate reacts with phosphoric acid to produce calcium phosphate, carbon dioxide, and water. 
Calculate the number of grams of CO$_2$ formed when 0.47 g of water is produced. (1.1 g CO$_2$)

\section{Percent Yield, Limiting Reactant, \& Gas and Solution Stoichiometry}
\begin{itemize}
    \item Stoichiometric calculations are based on ideal reactions.
    \item Many reactions do not go to completion and not as much product is produced as expected.
\end{itemize}

There are two different yields:

Theoretical Yield: what stoichiometry predicts.

Actual Yield: What is actually produced and measured in the lab.

Percent Yield:
\begin{center}
    \% yield = $\frac{\text{actual yield}}{\text{theoretical yield}}\times 100$
\end{center}

Exercise - What is the percent yield for the reaction 
\begin{center}
    Cl$_2$ + 2KBr $\rightarrow$ 2KCl + Br$_2$
\end{center}
in which 214 g of chlorine react with an excess of potassium bromide to produce 412 g of bromine? (85.4\%)

The limiting reactant is the chemical that is used up first in a chemical reaction. It limits the amount of product that can be made.

The other reactant(s) is/are called the excess reactant(s).

If you are given the amounts of both reactants and asked to predict the amounts of products, you must base your answer on the limiting reactant.

Exercise - The balanced equation for the reaction between 50.0 g of silicon dioxide and 50.0 g of carbon is 
\begin{center}
    SiO$_2$ + 3C $\rightarrow$ SiC + 2CO
\end{center}
Assuming the reaction is 100\% efficient, what is the excess reactant and how much in excess is it? (Carbon, 20.0 g)

We can also involve gases in stoichiometry:
\begin{itemize}
    \item Liters are used for a gas as STP 
    \item STP means ``standard temperature and pressure'', which we'll define as 0$\degree$C and 1 atm 
    \item Use 22.4 L/mol. This applies to ANY GAS AT STP.
\end{itemize}

Exercise - How many liters of 3.4 M copper(II) sulfate are needed to react fully with 2.00 grams of zinc? (0.0090 L CuSO$_4$)

Exercise - How many atoms of oxygen are there in a 3.0 mole sample of Mg(ClO$_3$)$_2$? ($1.1\times 10^{25}$ atoms)

\end{document}