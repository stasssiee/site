\documentclass[../hchem.tex]{subfiles}
\graphicspath{{\subfix{../figures/}}}
\begin{document}
\chapter{Organic Chemistry}
Organic chemistry: the chemistry of carbon compounds
\begin{itemize}
    \item Contain carbon 
    \item Have covalent bonds - low melting points, low boiling points, are soluble in nonpolar solvents, burn in air 
    \item Can form large molecules (polymers)
\end{itemize}

Hydrocarbons
\begin{itemize}
    \item Compounds that contain only carbon and hydrogen 
\end{itemize}

Complete structural formulas show all the atoms with the bonds between each of the atoms represented by lines.

Condensed structural formulas are a shorthand way and list all atoms in order and tells how they are bound together.

Molecular formula - atoms w/ subscripts making up molecule.

Skeleton structure/formula - leaves out H atoms, only C skeleton and connecting bonds shown and bonds are represented as lines

Bond-line Drawing - Only C-C bonds are shown. Each vertex represents a carbon atom. It is understood that hydrogen atoms are attached as needed to complete the bonding.

Alkanes
\begin{itemize}
    \item Alkanes are hydrocarbon chains that have only single bonds between the carbons
\end{itemize}

Branched chain alkanes 
\begin{itemize}
    \item Sometimes carbons are not bonded in a straight chain. Sometimes there are branches.
    \item The longest branch is called the parent chain. The side branches are called the substituent groups.
\end{itemize}

Name of Alkyl groups 
\begin{itemize}
    \item The group names have the same prefix as their corresponding parent chains, but the -ane suffix is replaced with -yl
\end{itemize}

Rules for Naming Organic Structures 
\begin{itemize}
    \item Count the carbons in the longest chain. This parent chain provides the base name of the structure.
    \item Number each carbon in the parent chain, starting with the end closest to a substituent group.
    \item Name each alykl group substituent before the name of the parent chain. Include the alykl name the number of the carbon it is attached to in the parent chain.
    \item If the alykl group occurs more than once, use a prefix before its name to indicate how many times it appears.
    \item Whenever different alykl groups are attached to the same parent chain, name them in alphabetical order 
    \item Write the entire name using hyphens to separate numbers from words and commas to separate numbers. No spaces.
\end{itemize}

Cyclic Alkanes 
\begin{itemize}
    \item Cyclic hydrocarbons are organic compounds that exist as carbon rings. 
    \item The prefix cyclo- is used 
    \item Alkyl groups added to front of name of ring 
\end{itemize}

Naming branched cycloalkanes
\begin{itemize}
    \item The ring is considered the parent chain 
    \item Number the carbons beginning with the one attached to a substituent group that gives the lowest possible sum of numbers in the name
\end{itemize}

Multiple Carbon-Carbon Bonds 
\begin{itemize}
    \item Alkanes all have single bonds between carbons. These are called saturated hydrocarbons.
    \item Some hydrocarbons contain double or triple bonds. These are referred to as unsaturated hydrocarbons.
\end{itemize}

Alkenes 
\begin{itemize}
    \item Alkenes contain at least one double bond between carbon atoms.
\end{itemize}

Naming alkenes 
\begin{itemize}
    \item Alkenes are named like alkanes, but their names have -ene at the end.
    \item The double bond will always be part of the parent chain.
    \item You also must specify the location of the double bond with a number. This is the number of the carbon atom where the double bond starts.
\end{itemize}

Naming cycloalkenes 
\begin{itemize}
    \item Named like cycloalkanes, but carbons \#1 \& 2 must be attached to the double bond.
\end{itemize}

Naming branched alkenes 
\begin{itemize}
    \item The longest carbon chain must contain the double bond. Always start numbering closest to the double bond.
    \item If there is more than one double bond, use a prefix before the -ene to indicate how many.
\end{itemize}

Alkene Geometric Isomers 
\begin{itemize}
    \item All atoms are bonded in the same order but are arranged differently in space 
    \item We'll discuss isomers further later 
    \item cis- the functional groups are on the same side of the molecule 
    \item trans- the functional groups are on opposite sides of the molecule 
\end{itemize}

Alkynes 
\begin{itemize}
    \item Alkynes are hydrocarbons that contain at least one triple bond. They are named in the same way as alkenes, but with suffix -yne.
    \item Can have \>1 triple bond 
    \item Parent c hain must contain both C atoms of triple bond 
\end{itemize}

Remember: each C forms a covalent bond with 4 other atoms.

Classes of hydrocarbons:
\begin{itemize}
    \item Aliphatic - does not contain benzene ring 
    \item Aromatic - contains benzene ring 
    \item Aliphatic -
    \begin{itemize}
        \item Alkanes - simplest class of organic compounds contain only C and H and have only single bonds 
        \item Alkenes have a C-C double bond 
        \item Alkynes have a C-C triple bond 
    \end{itemize}
\end{itemize}

Saturation:
\begin{itemize}
    \item Saturated: hydrocarbon has maximum \#H'2 
    \item Unsaturated: hydrocarbon has less than maximum \#H's (can be cyclic or have double or triple bonds)
\end{itemize}

Substituents: atoms that take the place of H 

Functional Groups 
\begin{itemize}
    \item Small structural units within molecules where most chemical reactions occur 
    \item R (radical) represents any hydrocarbon attached to functional groups 
    \item Since double \& triple bonds are chemically reactive, they are considered functional groups 
\end{itemize}

Benzene 
\begin{itemize}
    \item Compounds containing benzene rings are called aromatic compounds.
    \item There are not 3 single bonds and 3 double bonds in a benzene ring. Instead, the electron pairs are delocalized, which means they are shared among all 6 C's in the ring. 
    \item If there are other groups present on the benzene ring the compound is said to be a substituted benzene.
    \item When benzene is a substituent on a carbon chain, it is called phenyl.
\end{itemize}

Alcohols 
\begin{itemize}
    \item Carbon compounds containing a hydroxyl group, -OH 
    \item -ol added on to name of compound 
    \item Classified by how many substituents are attached to the C of -OH group 
\end{itemize}

Halocarbons 
\begin{itemize}
    \item Carbon compounds containing halogens (Cl, F, Br, I)
    \item Alkyl halide - halogen attached to carbon of aliphatic chain 
    \item Aryl halide - halogen attached to aromatic hydrocarbon 
    \item Occur by substitution reaction (hydrogen replaced by halogen)
\end{itemize}

Ethers
\begin{itemize}
    \item 2 C's single bonded to oxygen atom 
    \item General formula R$_1$-O-R$_2$ 
    \item Simple ethers can be named by namimg alykl groups alphabetically followed by word ``ether''. 
    \item Another way is name smaller hydrocarbon prefix, add -oxy, and join it to alkane name of the larger hydrocarbon group.
\end{itemize}

Ketones 
\begin{itemize}
    \item In middle of compound 
    \item Change final -e of alkane to -one.
    \item Indicate number before name to indicate position of ketone group 
    \item Less reactive than aldehydes, so popular as solvents 
\end{itemize}

Aldehydes 
\begin{itemize}
    \item at end of compound 
    \item Change final -e of alkane to -al 
    \item Many have characteristic odors/flavors 
\end{itemize}

Amines 
\begin{itemize}
    \item Contain N-C in aliphatic chains or aromatic rings 
    \item Primary amine general formula R-NH$_2$
    \item Called amino group (found in amino acids)
    \item Suffix -amine 
    \item Amines are stinky!
\end{itemize}

Amides 
\begin{itemize}
    \item -OH of carboxylic acid is replaced by N bonded to other atoms 
    \item Write name with same number of carbon atoms, replacing final -e with -amide 
\end{itemize}

Carboxylic Acids 
\begin{itemize}
    \item Organic acid with carboxyl group (-CO$_2$H or -COOH)
    \item Change -ane of alkane to -anoic acid 
    \item Weak acids (ionize slightly to give carboxylate ion and proton)
\end{itemize}

Esters 
\begin{itemize}
    \item Carboxyl group where H of hydroxyl group replaced by alkyl group (-OR)
    \item General formula R$_1$CO$_2$R$_2$
    \item Formed by reaction of an alcohol and a carboxylic acid 
    \item Substituent suffix -oate 
\end{itemize}

Aromatic Hydrocarbons 
\begin{itemize}
    \item Substituents attached to benzene ring are usually named as derivatives of benzene 
    \item Benzene is the parent compound if there is no continuous HC longer than 6 C's joined with it 
\end{itemize}

Disubstituted Aromatic Hydrocarbons
\begin{itemize}
    \item o-/m-/p- not used with 2 different substituents 
    \item Name substituents in alphabetical order 
\end{itemize}

Isomers 
\begin{itemize}
    \item Compounds with same molecular formula but different structures 
    \item Structural isomers - same molecular formula but different connections between atoms 
    \item Funtional isomerism - substances have same molecular formula but different functional groups 
    \item Positional isomerism - occurs when same functional groups are in different positions on same C chain 
    \item Stereoisomers - same molecular formula, same connections between atoms, but different arrangements of atoms in 3-D space 
    \item Geometric isomers - result from different arrangements of groups around double bond 
\end{itemize}

Properties and uses of Alkanes 
\begin{itemize}
    \item Nonpolar 
    \item Low boiling point/melting point 
    \item Immiscible in water but soluble in nonpolar solvents 
    \item Low reactivity due to nonpolarity 
    \item Relatively strong C-C and C-H bonds limit their use 
\end{itemize}

Properties and Uses of Alkenes 
\begin{itemize}
    \item Nonpolar/low solubility in water 
    \item Low melting/boiling points 
    \item More reactive than alkanes 
    \item Several occur naturally in living organisms 
    \item Ethene used in plastics 
    \item Scents of lemons, limes and pine trees 
\end{itemize}

Properties and Uses of Alkynes 
\begin{itemize}
    \item Useful starting material in many synthesis reactions due to reactivity
    \item Physical and chemical properties similar to alkenes.
\end{itemize}



\end{document}