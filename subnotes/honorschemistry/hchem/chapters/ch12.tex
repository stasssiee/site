\documentclass[../hchem.tex]{subfiles}
\graphicspath{{\subfix{../figures/}}}
\begin{document}
\chapter{Equilibrium}
The nature of the equilibrium state:
\begin{itemize}
    \item Reactions are reversible.
    \item dynamic - $\rightleftarrows$ indicates that the reaction is proceeding in both the forward and reverse directions.
    \item Equilibrium does not mean nothing is happening.
    \item Once equilibrium is established, the rate of reaction in each direction is equal.
    \item This keeps the concentration of reactants and products constant.
\end{itemize}

\begin{center}
    A + B $\rightarrow$ C + D 
\end{center}
\begin{itemize}
    \item The nature and properties of the equilibrium state are the same, no matter what the direction of approach.
    \item Initially, since there is only A and B present, they react very quickly.
    \item As A and B react, there are less and less of unreacted A and B molecules and so the rate of this reaction slows down.
    \item As the concentrations of C and D build up, they start to react to form A and B.
    \item And eventually the rate at which A and B react equals the rate at which C and D react. At this point equilibrium is established.
\end{itemize}
Note the concentrations of A, B, C, and D are not necessarily equal at equilibrium, but the concentrations are constant at equilibrium.

The equilibrium constant is used to determine equilibrium conditions. It is always temperature dependent.

For the general reaction 
\begin{center}
    aA + bB $\rightleftarrows$ cC + dD 
\end{center}
The equilibrium constant expression is 
\begin{center}
    K = $\frac{[C]^c[D]^d}{[A]^a[B]^b}$
\end{center}
The product concentrations appear in the numerator and the reactant concentration in the denominator.
Each concentration is raised to the power of its stoichiometric coefficient in the balanced equation.

\begin{itemize}
    \item [] indicate concentration 
    \item K$_c$ is for concentration 
    \item K$_p$ is for partial pressure 
\end{itemize}

In equilibrium constant expressions:
\begin{itemize}
    \item Pure solids do not appear in the expression 
    \item Pure liquids do not appear in the expression 
    \item Water as a liquid or gas does not appear in the expression 
\end{itemize}

If you know the concentration of things, just plug them into the K expression and solve.

Changing coefficients:
\begin{itemize}
    \item When the stoichiometric coefficients of a balanced equation are multiplied by some factor, the K value is raised to the power of the multiplication factor.
    \item For the reverse reaction, take the reciprocal of K 
    \item When adding reactions to produce another, multiply the respective K's to determine the K of the final reaction.
\end{itemize}

K$_p$ only applies when all reactants and products are gases. Pressure must be units of atmospheres.

K$_c$ and K$_p$ are not interchangeable!
\begin{center}
    K$_p$ = K$_c$(RT)$^n$
\end{center}
where n is the change in the number of moles of gas from the reactant to product side of the arrow.

Use a RICE table if you are not given all the concentrations.

External factors affecting equilibria:

Le Chatelier's Principle - If you disrupt a system in equilibrium, shifts in direction occur to reestablish equilibrium positions.

Temperature:
\begin{itemize}
    \item Exothermic reactions - heat is a product
    \item Endothermic reactions - heat is a reactant 
    \item The value of K is a constant only at constant T 
    \item If you change T, the value of K changes  
\end{itemize}

Changing reagent and product amounts:
\begin{itemize}
    \item Adding reagent - shift tries to consume additional material by favoring forward reaction 
    \item Removing a reagent - shift tries to replace material by favoring reverse reaction 
    \item Adding product - shift tries to consume the product by favoring reverse reaction 
    \item Removing product - shift tries to replace the product by favoring forward reaction 
    \item Has no effect on the value of K 
\end{itemize}

Pressure:
\begin{itemize}
    \item An increase in pressure favors the side with the least \# of gas moles 
    \item A decrease in pressure favors the side with the most \# of gas moles 
    \item Has no effect on solids and liquids 
    \item Has no effect on the value of K
\end{itemize}

Catalysts:
\begin{itemize}
    \item No effect on the value of K 
    \item But the reaction gets to equilibrium faster!
\end{itemize}

Weak acids only dissociate partly. There is an equilibrium constant associated with weak acid dissociation, K$_a$. It works just like K$_c$.
\begin{center}
    K$_a$ = $\frac{[H_3O^+][A^-]}{[HA]}<1$
\end{center}

The reaction quotient, Q.

For a general reaction 
\begin{center}
    aA + bB $\leftrightarrow$ cC + dD 
\end{center}
the reaction quotient is 
\begin{center}
    Q = $\frac{[C]^c[D]^d}{[A]^a[B]^b}$
\end{center}
Q has the appearance of K but the concentrations are not necessarily at equilibrium.

If $K>Q$, the system is not at equilibrium.
\begin{itemize}
    \item Products are too small and the reactants are too big.
    \item The system will move towards equilibrium by making products and consuming reactants.
\end{itemize}

If K = Q, the system is at equilibrium.

If $K<Q$, the system is not at equilibrium.
\begin{itemize}
    \item Reactants are too small and the products are too big.
    \item The system will move towards equilibrium by making reactants and consuming products.
\end{itemize}
\end{document}