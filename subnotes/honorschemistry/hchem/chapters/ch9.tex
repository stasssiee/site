\documentclass[../hchem.tex]{subfiles}
\graphicspath{{\subfix{../figures/}}}
\begin{document}
\chapter{Gas Laws}
\section{Kinetic Molecular Theory, Temperature, and Pressure}
\begin{itemize}
    \item A gas has no definite shape or volume.
    \item They adapt to the shape and volume of their container.
    \item Ideal Gases are imaginary gases that comply with all the postulates of the Kinetic Molecular Theory.
    \item Gas Laws attempt to explain the behavior of gases under certain conditions.
\end{itemize}

The Kinetic Molecular Theory 
\begin{itemize}
    \item Gases are made up of tiny particles.
    \item Gas particles move randomly, in straight lines in all directions and at various speeds.
    \item The forces of attractions or repulsion between two gas particles are extremely weak or neglibible, except when they coolide.
    \item When gas molecules collide, the collisions are elastic.
    \item The average kinetic energy of a molecule is proportional to the Kelvin temperature. Gases at higher temperatures have higher kinetic energies.
\end{itemize}

Characteristics of Gases
\begin{itemize}
    \item Expansion - gases will expand to fill their containers since they have no definite shape or volume.
    \item Fluidity - gases have the ability to flow and be poured as liquids are.
    \item Low Density - gases have low density because the particles are spread far apart.
    \item Compressibility - gas particles can be made to occupy a smaller space by decreasing the volume of the container.
    \item Diffusion - gases spread out and mix with each other without agitation.
\end{itemize}

Avogadro's Principle
\begin{itemize}
    \item Equal volumes of gases contain equal numbers of moles of those gases if the temperatures and pressures are the same.
    \item The volume occupied by one mole of any gas is 22.4 liters at standard temperature and pressure. This is called the molar volume/
\end{itemize}

Temperature Conversions:
\begin{itemize}
    \item $T_K = T_C +273$
    \item $T_C = T_K-273$
    \item $T_F = (9/5)T_C+32$
    \item $T_C = (5/9)\cdot (T_F-32)$
\end{itemize}

Exercise - Convert $-20\degree$ C to K: (253 K)

\begin{itemize}
    \item Absolute zero is 0 K.
    \item At absolute zero, matter stops moving.
    \item Atoms/molecules in a solid, which usually vibrate, come to a complete stop.
\end{itemize}

Pressure (the force of a gas acting on the walls of its container) is measured in several different units.
\begin{itemize}
    \item atm - atmospheres 
    \item mm Hg - millimeters of mercury 
    \item torr - torr
    \item Pa - pascals
    \item kPa - kilopascals 
    \item psi - pounds (force) per square inch
\end{itemize}

Atmospheric pressure varies from day to day and is measured with a barometer.

1 atm of pressure is equal to 760 mm Hg, 760 torr, 101.325 kPa, 14.7 psi.

Exercise - Convert 800. mm Hg to atm (1.05 atm)
\section{Gas Laws \& Density}
Boyle's Law states that the volume of a gas varies inversely with the pressure if the temperature is held constant.
\begin{itemize}
    \item Boyle discovered that for any given ideal gas, the product of pressure and volume is always an exact constant.
    \item $P\cdot V$ = constant 
    \item So, even if you change the pressure and volume of a gas, the product will still be the same.
    \item $P_1V_1=P_2V_2$
    \item Remember, in Boyle's Law, temperature is held constant.
\end{itemize}

Exercise - A syringe has 10.0 mL of gas inside and the pressure is 1.00 atm. If pressure is applied and the volume decreases to 4.8 mL, what is the final pressure of the gas inside? (2.1 atm)

Charles' Law states that the volume of a gas varies directly with the Kelvin temperature if the pressure is held constant.
\begin{itemize}
    \item Charles discovered that volume divided by temperature is a constant.
    \item V/T = constant 
    \item So, if you change the volume and temperature of a sample of gas, V/T will always be the same number.
    \item $\frac{V_1}{T_1}=\frac{V_2}{T_2}$
    \item Remember, this only applies when pressure is held constant and temperature is in Kelvin.
\end{itemize}

Exercise - To what temperature Kelvin must 7.98 cm$^3$ of oxygen be cooled, to reduce its volume to 5.00 cm$^3$ if it is initially at STP and pressure does not change? (171 K)

Gay-Lussac's Law states that the pressure of a gas varies directly with temperature if the volume is held constant. Just like Charles' Law, the temperature must be in Kelvin.
\begin{itemize}
    \item Gay-Lussac discovered that for any given mass of an ideal gas, the pressure divided by temperature (in Kelvin!) was always a constant.
    \item P/T = constant 
    \item So if you change the pressure and temperature of a gas, the press/temp will still be the same.
    \item $\frac{P_1}{T_1}=\frac{P_2}{T_2}$
    \item Remember, now volume is held constant.
\end{itemize}

Exercise - If you cap a 2 L coke bottle containing air, and the temperature changes from $25\degree$C to $35\degree$C, what is the pressure on the inside wall of the bottle?
Assume the initial atmospheric pressure when you capped the bottle was 728 mmHg. (752 mmHg)

The three laws can be combined into one law that can always be used when conditions are changed. We use this equation to figure out the new 
pressure, temperature, or volume of a gas if the initial conditions are known.
\[\frac{P_1V_1}{T_1}=\frac{P_2V_2}{T_2}\]

Exercise - How much pressure must be applied to 68 L of a gas at STP to reduce its volume by half if the temperature is raised to 20.$\degree$C? (2.1 atm)

The Ideal Gas Law describes the conditions of an ideal gas in terms of pressure, temperature, volume, and the number of moles of a gas.
Ideal gas law does not involve changes in conditions.
\[PV=NRT\]

R is the ``Ideal Gas Constant'' and is equal to 0.08206 L$\cdot$ atm/ mol $\cdot$ K 

Exercise - At which temperature would 0.0828 moles of hydrogen have a pressure of 1.00 atm and a volume of 55.0 L? (8090 K)

Real Gases do have forces of attraction and the molecules do have volume. Real pressure is lowered than what is predicted due to IMFs, especially 
for polar molecules or when hydrogen bonding is present. Real volume is higher than what is predicted due to molecular volume being significant, especially for larger molecules of gas.
Gases act ``most ideally'' at high temperatures and low pressure.

We can relate the molar mass of a gas with density.
\[MM=\frac{DRT}{P}\]

Exercise - If the density of a gas is 1.2 g/L at 745 torr and 20.$\degree$C, what is its molar mass? (29 g/mol)

Dalton's Law of Partial Pressures shows the pressure of a mixture of gases is simply equal to the sum of the partial pressure of each gas.
\[P_T = P_1 + P_2 + P_3 + \dots\]

When you collect a gas by water displacement, the collected gas also contains water vapor. There is more water vapor at higher temperatures.

Exercise - A student collects 89 mL of oxygen gas by bubbling it through water. The pressure reading that day is 103.2 kPa and the temperature is 20.$\degree$C. Determine the number of moles 
of gas collected. At 20.$\degree$C, the partial P of water vapor is 2.3 kPa. (0.0037 mol O$_2$)

Graham's Law of Diffusion shows the rate of diffusion of gases is inversely proportional to their molar masses.
\[\frac{r_A}{r_B}=\sqrt{\frac{MM_B}{MM_A}}\]
where r is the rate (speed) and MM is the molar mass of the gas.

Exericse - A molecule of oxygen gas has an average speed of 12.3 m/s at a given temp and pressure. What is the 
average speed of hydrogen molecules at the same conditions? (49.0 m/s)

\end{document}