\documentclass[../hchem.tex]{subfiles}
\graphicspath{{\subfix{../figures/}}}
\begin{document}
\chapter{Nuclear Chemistry}
Types of Reactions:

Chemical Reactions:
\begin{itemize}
    \item involve valence electrons 
    \item elements are rearranged to form new compounds 
    \item Law of Conservation of Mass obeyed 
    \item relatively little energy can be absorbed or released
\end{itemize}

Nuclear Reactions:
\begin{itemize}
    \item changes occur in the nucleus; involve protons, neutrons, and electrons 
    \item Transmutation occurs (one element turns into another)
    \item Law of Conservation of Mass is not always obeyed; mass can be changed into energy 
    \item lots of energy can be produced 
\end{itemize}

Roentgen discovered x-rays in 1895 as a result of exposing photographic plates to the rays emitted from a cathode ray tube. He found that x-rays could travel through some substances.

Becquerel in around 1900 studied phosphorescence (glowing) in uranium ore when he discovered that the ore also exposed photographic plates.

The Curies further experimented with uranium ore. They isolated uranium as the substance with the unique properties.
They were the first to call uranium ``radioactive'' - meaning a substance that gave off rays of energy from the nucleus.

Radioisotopes are isotopes that have unstable nuclei; they emit radiation to become more stable.

The radiation emitted can be one of several types. The three most common types of radiation are:
\begin{itemize}
    \item alpha, $\alpha$
    \item beta, $\beta$
    \item gamma, $\gamma$
\end{itemize}

Alpha decay occurs when an atom loses an alpha particle (2 protons + 2 neutrons), aka a helium nucleus. The loss of two protons decreases the atomic number by 2
and decreases the mass number by 4. Since the number of protons changes, a new element is formed. Alpha radiation isn't powerful (compared to other types of radioactive decay)
and can be stopped by paper alone.

Beta decay occurs when an atom loses an electron from the nucleus. It is the result of a neutron converting to a proton and electron and the high energy electron is ejected.
The new proton increases the atomic number by 1, which results in a new element. Note that the mass number is unchanged.

Gamma radiation occurs when high energy photons (electromagnetic energy) are ejected from the nucleus.
Usually accompanies other nuclear emissions and represents the energy lost when the nucleus becomes more stable. Does not change the atomic 
number or mass and it is difficult to stop; it requires several cm of lead or several feet of concrete to block.

Relative Strengths of Radiation:
\begin{itemize}
    \item Alpha particles can be stopped by paper 
    \item Beta particles can be stopped by a few mm of aluminum 
    \item Gamma particles can be stopped by a few cm of lead or several feet of concrete 
\end{itemize}

Nuclear Mass Defect - The difference in the calculated mass of an atom (based on C-12) and the measured mass.
Measured mass if always smaller than expected. Lost mass has been converted to energy.

Nuclear Binding Energy - the energy that comes from the lost mass. Holds the nucleus together. Lots of energy produced, according to Einstein's famous equation, E = mc$^2$

Strong Nuclear Force - a force that keeps the protons from repelling each other in the nucleus. Acts over a short distance and overcomes electrostatic repulsion (like charges repel). Strong nuclear force struggles in very large elements.

Stability of Isotopes:
\begin{itemize}
    \item Nuclei that are not stable tend to undergo radioactive decay, a process that emits radiation and particles from the nucleus 
    \item Spontaneous decay is the result of instability in the nucleus of an atom.
\end{itemize}

What determines nuclear stability?
\begin{itemize}
    \item Neutron to proton ratio. If this ratio is close to 1:1 in a low mass (small) element, the isotope is likely to be stable.
    \item Neutron to proton ratio. For larger elements, the stable ratio is 1.5: 1-1.5 neutrons to protons
\end{itemize}

In a graph of protons vs neutrons, the area of the graph within which all stable nuclei are found is called the band of stability.

\begin{itemize}
    \item Nuclei with a high neutron to proton ratio will undergo beta decay.
    \item Nuclei with low ratio will undergo positron emission or electron capture.
    \item Nuclei with high atomic masses tend to undergo alpha decay.
\end{itemize}

Electron Capture:
\begin{itemize}
    \item The nucleus captures an electron from the emission cloud 
    \item The electron combines with a proton to convert into a neutron 
    \item The mass number remains the same and the atomic number decreases since a proton was converted 
\end{itemize}

Positron Emission:
\begin{itemize}
    \item The radioactive decay process in which a positron is emitted from the nucleus.
    \item A positron is a particle with the same mass as an electron but opposite in charge.
    \item The number of protons decreases bu the mass number remains the same.
\end{itemize}

Artifical Nuclear Charges Neutron Bombardment
\begin{itemize}
    \item A neutron enters the nucleus at a high speed and is captured; usually creates a radioactive isotope 
    \item Creates synthetic isotopes useful in medicine and research 
    \item Increases the mass number by one, but the atomic number remains the same.
\end{itemize}

A radioactive decay series is a series of nuclear reactions that begins with an unstable nucleus and ends with the formation of a stable nucleus.

\begin{itemize}
    \item Nuclei that are not stable tend to undergo radioactive decay, a process that emits radiation (gamma radiation) and particles from the nucleus 
    \item Spontaneous decay is the result of instability in the nucleus of an atom 
\end{itemize}

Most elements have some radioactive isotopes (radioisotopes) that occur naturally, but the percentage of radioactive isotopes is so low that we do not consider that element radioactive.

Other elements, such as radon, have a high percentage of radioisotopes, so we consider that element, in general, to be radioactive.

Radioactive isotopes can be created from stable ones by neutron bombardment, which causes instability in the nucleus by disturbing the 1:1 neutron:proton ratio.

All elements heavier than bismuth are naturally radioactive due to their instability. They tend to spontaneously undergo alpha decay to bring them closer to a stable nucleus.

All elements heavier than uranium, atomic number 92, are artifically created and are always radioactive. These are the transuranium elements. They are usually formed by bombarding a heavy atom with neutrons or alpha particles. Scientists have created elements 
up to about 118 protons.

Half life is the amount of time it takes for one half of a radioactive element to change from the parent element to the daughter element (from radioactive element to product element).

Use for formulas
\begin{center}
    m$_t$ = m$_0$(1/2)$^n$ \\ 
    n = $\frac{t}{T}$
\end{center}
where m$_t$ is the final mass, m$_0$ is the original mass, n is the number of half lives, t is the elapsed time and T is the time for one half life.

You can also use the following formula 
\begin{center}
    ln(m$_t$) = -(0.693/T)t + ln(m$_0$)
\end{center}

\begin{itemize}
    \item Fossils are dated by knowing the ratio of parent element to daughter element 
    \item Two things must be known - length of half life and ratio of parent:daughter 
    \item The carbon-14 isotope is used for wood, bones, tissue, anything that used to be alive 
    \item All living things contain carbon, and part of the carbon is C-14 
    \item After something dies, it stops exchanging carbon with the atmosphere and the C-14 begins to decay 
    \item Because we know the half-life of C-14 to be 5730 years, we can measure the amount of C-14 left and calculate how long the organism has been dead.
\end{itemize}

Nuclear Fission:
\begin{itemize}
    \item Fission occurs when a neutron hits a large, unstable nucleus and causes it to break into two smaller nuclei.
    \item Extra neutrons are emitted from this reaction, and those neutrons go on to hit other nearby atoms, causing them to go through fission.
    \item A chain reaction occurs if there are enough atoms present.
\end{itemize}

Critical mass is the amount of a substance that must be present in order for a chain reaction to occur.

If not enough mass is present, the neutrons will escape the sample without hitting other atoms, and there will be no chain reaction. This is called subcritical mass.

In addition to neutrons being released, larget amounts of energy are also released. Nuclear fission is highly exothermic. In nuclear power plants, the energy and chain reaction is controlled and slow.
In an atom bomb, the fission is uncontrolled and very fast.

Parts of a Nuclear Reactor in a Power Plant 
\begin{itemize}
    \item Fuel rods - usually uranium-235 or plutonium, a fissionable isotope 
    \item Control rods - surround the fuel rods and absorb some of the neutrons to control the chain rxn 
    \item Graphite Moderator - slows the neutrons so they can enter the fuel atoms at the most effective rate 
    \item Water heated to steam - same as fossil fuel plant. The exothermic reaction heats the water to steam, which turns a turbine 
    \item Concrete Containment Building - dome shaped building around the nuclear reactor to keep radioactivity inside 
\end{itemize}

Nuclear Fusion 
\begin{itemize}
    \item Fusion is the process by which small nuclei fuse together to form a heavier nucleus 
    \item This is the main process that goes on in our sun and other stars. It is responsible for the light and energy given of - highly exothermic for small atoms 
    \item The primary reaction is two hydrogen isotopes building helium 
\end{itemize}

Fusion produces lots of energy, and its products are generally not radioactive. It is appealing as an energy source. However, to get fusion started, lots of energy is needed to get two repelling hydrogen nuclei to move together.
A thermonuclear reactor is required. Fusion isn't a feasible energy source for now.

Uses of Radioactivity 
\begin{itemize}
    \item Radiation therapy for cancer - radiation targets cancerous cells 
    \item Radiotracers for medical diagnosis and research 
    \item Nuclear Power 
    \item Food preservation - it doesn't make your food radioactive, it kills bacteria and fungus 
    \item Smoke Detectors 
\end{itemize}

Properties of Radioactive Elements 
\begin{itemize}
    \item Radioactive elements will expose photographic film 
    \item Ionize other atoms, causing them to become charged 
    \item Can cause some other substances to fluoresce, or glow in the dark 
    \item Prolonged exposure can cause damage to cells 
\end{itemize}

2 types of radiation damage:

Somatic 
\begin{itemize}
    \item affects nonreproductive body tissue 
    \item Only affects organism during its lifetime 
    \item Causes burns, cancer 
\end{itemize}

Genetic 
\begin{itemize}
    \item Damages sex cells 
    \item Affects chromosomes and therefore offspring 
    \item Causes mutations, lifelong disease, and deformities 
\end{itemize}



\end{document}