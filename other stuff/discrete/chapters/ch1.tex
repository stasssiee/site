\documentclass[../discrete.tex]{subfiles}
\graphicspath{{\subfix{../figures/}}}
\begin{document}
\chapter{Speaking Mathematically}
This chapter will contain 1.1-1.4 from the textbook.

The topics included are Variables, The Language of Sets, The Language of Relations and Functions, and The Language of Graphs.

A variable is a placeholder to talk about something, it has unknown value.

Universal Statement: a property that is true for all elements in a set.

Conditional Statement: If one thing is true, then some other thing is true.

Existential Statement: There is at least one thing for which a property is true.

These can be combined.

A set is a collection of elements.

Notation, $x\in S$ can be said as $x$ is an element of $S$. $x\notin S$ is said as $x$ is not an element in $S$.

Axiom of extension: a set is completely determined by what its elements are, not the order in which the elements are listed or if the elements are listed more than once.

$\mathbb{R}$ is the set of all real numbers, $\mathbb{Z}$ is the set of all integers, $\mathbb{Q}$ is the set of all rational numbers. If the superscript has a $+$ or a $-$ or something similar, it would mean positive/negative. nonneg superscript includes $0$.

The set of all integers is discrete, because they are not continuous like the real number line.

Subset: $A\subseteq B$ iff every element in $A$ is an element in $B$, or if $x\in A$, then $x\in B$.

$A\nsubseteq B$ means there is an element $x$ such that $x\in A$, but $x\notin B$.

A proper subset is if every element in $A$ is in $B$ and there iat at least one element in $B$ that is not in $A$ given two sets $A$ and $B$.

An ordered pair is a set $\{ \{a\}, \{a,b\}\}$. $a\neq b$ means that the sets are distinct so $a$ is the first element and $b$ is the second element. If $a=b$, then the set becomes $\{\{a\}\}$. Ordered pairs are written as $(a,b)$.

$(a,b)=(c,d)$ if $a=c$ and $b=d$.

An ordered $n$-tuple is $(x_1,x_2,\dots,x_n)$ where each of these are not necessarily distinct elements. In general, $x_1=y_1, x_2=y_2,\dots,x_n=y_n\implies (x_1,x_2,\dots,x_n)=(y_1,y_2,\dots,y_n)$.

The Cartesian Product of sets $A_1\times A_2\times \dots \times A_n$ is the set of all $n$-tuples $(a_1,a_2,\dots,a_n)$ where $a_1\in A_1, a_2\in A_2, \dots a_n\in A_n$. 

Basically, $A_1\times A_2 = \{ (a_1,a_2)|a_1\in A_1$ and $a_2\in A_2\}$, which can be generalized to set $A_n$.

A string of length $n$ over set $A$ is an $n$-tuple without paranthesis or commas. The elements are called characters. A no character string is denoted as a null string ($\lambda$).

A relation $R$ from $A$ to $B$ is a subset of $A\times B$. Given $(x,y)\in A\times B$, $x$ is related to $y$ by $R$ by relation $x R y$ iff $(x,y)\in R$. $A$ is the domain and $B$ is the codomain.

In arrow diagrams, an arrow from $x$ to $y$ means $xRy$ or $(x,y)\in R$.

A function $F$ from a set $A$ to set $B$ is a relationship with domain $A$ and codomain $B$ that will satisfy the two conditions:
\begin{enumerate}
    \item For every element $x$ in $A$, there is a $y$ in set $B$ such that $(x,y)\in F$ (basically all elements of $x$ are related to an element in $B$, not necessarily the other way around)
    \item For all elements in $x$ in $A$ and $y$ and $z$ in $B$, if $(x,y)\in F$ and $(x,z)\in F$, then $y=z$. (Basically, $x$ cannot have two codomains, it is one-to-one.)
\end{enumerate}

An edge connects endpoints. A loop is an edge that connects one vertex to itself. Two edges are parallel if the two edges connect to the same vertices.

Two edges that are incident on the same endpoint are called adjacent. A vertex with no edges is called isolated.

A directed graph uses arrows as the edges, they go in a direction kind of. Instead of the edge being associated with a vertex $v$, it is associated with a ordered pair $(v,w)$ as a vertex.

The degree of a vertex is the number of edges that are incident on $v$, with loops being counted twice.

\end{document}