\documentclass[../physics12.tex]{subfiles}
\graphicspath{{\subfix{../figures/}}}
\begin{document}
\chapter{Electrostatics}
\section{Constants and units for chapter 9 through 16}
Permittivity of free space: $\epsilon_0$ = $8.85\times 10^{-12}$ C$^2$/(N$\cdot$m$^2$)

Coulomb constant: $k_e = \frac{1}{4\pi\epsilon_0} = 8.987\times 10^9$ N$\cdot$m$^2$/C$^2$

Permeability of free space: $\mu_0 = 4\pi\times 10^{-7}$ T$\cdot$m/A 

Speed of light in vacuum: $c=3.00\times 10^8$ m/s 

Electron charge: $e=1.604\times 10^{-19}$ C 

Electron mass: $m_e = 9.11\times 10^{-31}$ kg 

Proton mass: $m_p = 1.67\times 10^{-27}$ kg 

Unified atomic mass unit: 1$u$ = 931.5 MeV/c$^2$

Planck's constant: $h=6.63\times 10^{-34}$ J$\cdot$s 

Electron Volt: 1 eV = $1.60\times 10^{-19}$ J 

curie: 1 Ci = $3.70\times 10^{10}$ Bq 

becquerel: 1 Bq = 1 decay/second 

Radiation dose unit: 1 rad = 0.01 J/kg
\section{Formulas}
Coulomb's Law: $F=\frac{k_e |q_1||q_2|}{r^2}$

Definition of electric field: $\vec{E}=\vec{F}/q$

Position in a uniform $E$ field: 
\[ x = x_0 + v_0\Delta t + \frac{1}{2}\frac{qE}{m}\Delta t^2 \]
Velocity $(\Delta t)$ in a uniform $E$ field:
\[ v = v_0 + \frac{qE}{m}\Delta t \]
Velocity $(\Delta x)$ in a uniform $E$ field:
\[ v_2-v_0^2 = \frac{2qE}{m}\Delta x\]

Surface charge density: $\sigma = Q/A$

Electric field of a point charge: $\vec{E}=\left(\frac{k_e q}{r^2}\right)\hat{r}$

Gauss' law: $\varphi_E \equiv EA = Q_{\text{inside}}/\epsilon_0$, where $\epsilon_0 = \frac{1}{4\pi k_e}$

Spherical symmetry: $\varphi_E = E4\pi r^2$

Planar symmetry: $\varphi_E = E2A$ (two sides) or $\varphi_E = EA$ (one side)

Electric field due to an infinite sheet of charge: $E=\frac{\sigma}{2\epsilon_0}$

Relation between electric potential and potential energy: $\Delta V = \frac{\Delta PE}{q}$

Electric potential due to a point charge: $V=\frac{k_e q}{r}$

Relation between electric potential and potential energy: $\Delta V = Ed$

\section{Three Point Charges Problem}
Three point charges are located at the vertices of an equilateral triangle. The charge at the top vertex of the triangle is $-6.3\mu$C. The two charge $q$ at the bottom vertices of the 
triangle are equal. A fourth charge $-7\mu$C is placed below the triangle 0.85 m on its symmetry-axis, and experiences a zero net force from the other three charges. The distance from the top vertex to either of the bottom vertices is 3.7 m.

Find $q$. The value of the Coulomb constant is $8.98755\times 10^9$ N$\cdot$m$^2$/C$^2$. Answer in units of $\mu$C.

\section{Two Hanging Spheres Problem}
Two identical small charged spheres of mass 0.03 kg hang in equilibrium with equal masses. The length of the strings are 0.15 m at an angle of 5$\degree$ with the vertical.

Find the magnitude of the charge on each sphere. The acceleration of gravity is 9.8 m/s$^2$ and the value of Coulomb's constant is $8.98755\times 10^9$ N$\cdot$m$^2$/C$^2$. Answer in units of C.

\section{Two Spheres Problem}
Two identical small metal spheres with $q_1>0$ and $|q_1|>|q_2$ attract each other with a force of magnitude 85.3 mN when separated by a distance of 1.19 m/ The radius of each sphere is 25$\mu$m. The spheres are then brough together until they are touching, enabling the spheres to attain the same final charge $q$.
After the charges on the spheres have come to equilibrium, they spheres are separated so that they are again 1.19 m apart. Now the spheres repel each other with a force of magnitude 17.06 mN.

(a) What is the final charge on the sphere on the right? The value of the Coulomb constant is $8.98755\times 10^9$ N$\cdot$m$^2$/C$^2$. Answer in units of $\mu$C.

(b) What is the initial charge $q_1$ on the first sphere? Answer in units of $\mu$C.

\section{4 Charges in a Square Problem}
Four point charges, each of magnitude 2 $\mu$C, are placed at the corners of a square 10 cm on a side. 

If three of the charges are positive and one is negative, find the magnitude of the force experienced by the negative charge. 
The value of Coulomb's constant is $8.98755\times 10^9$ N$\cdot$m$^2$/C$^2$. Answer in units of N.

\section{Three Charges in a Plane Problem}
Three charges are arranged in the $(x,y)$ plane where there is a 4 nC charge at the point $(0,6)$, there is a -2 nC charge at $(0,0)$ and a -3 nC charge at $(8,0)$.

What is the magnitude of the resulting force (in nano-Newtons) on the -2 nC charge at the origin? The value of the Coulomb constant is $8.98755\times 10^9$ N$\cdot$m$^2$/C$^2$. Answer in units of nN. 

\section{Electrons in a Nickel Problem}
We want to find how much charge is on the electrons in a nickel coin. Follow this method. A nickel coin has a mass of about 5 g.

(a) Find the number of atoms in a nickel coin. Each mole ($6.02\times 10^{23}$ atoms) has a mass of about 58 g. Answer in units of atoms.

(b) Find the number of electrons in the coin. Each nickel atom has 28 electrons/atom. Answer in units of electrons.

(c) Find the magnitude of the charge of all these electrons. Answer in units of C.

\section{Alpha Particle Fired at Nucleus Problem}
In Rutherford's famous scattering experiments (which led to the planetary model of the atom), alpha particles (having charges of +2 $e$ and masses of $6.64\times 10^{-27}$ kg) were 
fired toward a gold nucleus with charge +79 $e$. An alpha particle, initially very far from the gold nucleus, is fired at $2\times 10^7$ m/s directly toward the gold nucleus.

How close the alpha particle get to the gold nucleus before turning around? Assume the gold nucleus remains stationary. The fundamental charge is $1.602\times 10^{-19}$ C and 
the Coulomb constant is $8.98755\times 10^9$ N$\cdot$m$^2$/C$^2$. Answer in units of m.

\section{Four Charges in a Square Problem}
Four charges are fixed at the corners of a square centered at the origin as follows: $q$ at $(-a, +a)$; $2q$ at $(+a, +a)$; $-3q$ at $(+a,-a)$; 
and $6q$ at $(-a,-a)$. A fifth charge $+q$ with mass $m$ is placed at the origin and released from rest.

Find the speed when it is a great distance from the origin, where the potential energy of the fifth charge due to the four point charges is negligible.

\section{Electron Through Electric Field Problem}
An electron traveling at $3\times 10^6$ m/s enters a 0.1 m region with a uniform electric field of 200 N/C. 

(a) Find the magnitude of the acceleration of the electron while in the electric field. The mass of an electron is $9.109\times 10^{-31}$ kg and the 
fundamental charge is $1.602\times 10^{-19}$ C. Answer in units of m/s$^2$.

(b) Find the time it takes the electron to travel through the region of the electric field, assuming it doesn't hit the side walls. Answer in units of s.

(c) What is the magnitude of the vertical displacement $\Delta y$ of the electron while it is in the electric field? Answer in units of m.
\end{document}