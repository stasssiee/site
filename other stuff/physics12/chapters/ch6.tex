\documentclass[../physics12.tex]{subfiles}
\graphicspath{{\subfix{../figures/}}}
\begin{document}
\chapter{Solids and Fluids}
\section{Formulas}
Stress is Force/Area, $F/A$. Strain is $\Delta L/L_0$

Young's Modulus: $\frac{F}{A}=Y\frac{\Delta L}{L}$

Density: $\rho = M/V$; Pressure: $P=F/A$.
\[ P_2 = P_1 + \rho g(y_1-y_2)\]

Water has density of 1000 kg/m$^3$. 1 liter (L) is 1 m$^3$/1000 = 1000 cm$^3$. Atmospheric pressure is $1.013\times 10^5$ Pa.

Pa = N/m$^2$.

Archimedes' Principle: The upward force on any stationary object partially or completely submerged in non-moving water is equal to the weight of the displaced water.

Continuity: $v_1A_1=v_2A_2$

Bernoulli's Law: $P+\frac{1}{2}\rho v^2+\rho gh$ = const

\section{Bulk Modulus Problem}
Find the density of seawater at a depth where the pressure is 160 atm if the density at the surface is 1025 kg/m$^3$. Seawater 
has a bulk modulus of $2.3\times 10^9$ N/m$^2$. Bulk modulus is defined to be 
\[ B \equiv \frac{\rho_0\Delta P}{\Delta \rho}\]
Answer in units of kg/m$^3$.

\section{Block in Oil and Water Problem}
Oil having a density of 930 kg/m$^3$ floats on water. A rectangular block of wood 4 cm high and with a density of 960 kg/m$^3$ floats partly in the oil and partly in the water.
The oil completely covers the block.

How far below the interface between the two liquids is the bottom of the block? Answer in units of m.

\section{Block on Spring Underwater Problem}
A light spring of constant 160 N/m rests vertically on the bottom of a large beaker of water. A 5 kg block of wood of density 650 kg/m$^3$ is connected to the top of the spring 
and the block-spring system is allowed to come to static equilibrium.

What is the elongation $\Delta L$ of the spring? The acceleration of gravity is 9.8 m/s$^2$. Answer in units of cm.

\section{Basic Pressure Problems}
(a) How much pressure is applied to the ground by a 104 kg man who is standing on square stilts that measure 0.05 m on each edge? Answer in units of Pa.

(b) What is this pressure in pounds per square inch? Answer in units of lb/in$^2$.

(c) If a 1-megaton nuclear weapon is exploded at ground level, the peak overpressure (that is, the pressure increase above normal atmospheric pressure) will be 0.2 atm at a distance of 6 km. Atmospheric pressure is $1.013\times 10^5$ Pa. What force due to such an explosion will be exerted on the side of a house with dimensions 4.5 m $\times$ 22 m? Answer in units of N.

\section{Pressure Under the Ocean Problem}
Calculate the depth in the ocean at which the pressure is three times atmospheric pressure. Atmospheric pressure is $1.013\times 10^5$ Pa. The acceleration of gravity is 9.81 m/s$^2$ and the density of sea water is 1025 kg/m$^3$. Answer in units of m.

\section{Object Immersed in Water and Oil Problem}
An object weighing 300 N in air is immersed in water after being tied to a string connected to a balance. The scale now reads 265 N. Immersed in oil, the object appears to weigh 275 N.

(a) Find the density of the object. Answer in units of kg/m$^3$.

(b) Find the density of the oil. Answer in units of kg/m$^3$.

\section{Hailstones on Windshield Problem}
In a 30 s interval, 500 hailstones strike a glass window of area 0.6 m$^2$ at an angle $45\degree$ to the window surface. Each hailstone has a mass of 5 g and speed of 8 m/s.

(a) If the collisions are elastic, find the average force on the window. Answer in units of N.

(b) Find the pressure on the window. Answer in units of N/m$^2$.

\section{Two Scales Buoyancy Problem}
A beaker of mass 1 kg containing 2 kg of water rests on a scale. A 3 kg block of a metallic alloy of density 2700 kg/m$^3$ is suspended from a spring scale and is submerged in the water of density 1000 kg/m$^3$.

(a) What does the hanging scale read? The acceleration of gravity is 9.8 m/s$^2$. Answer in units of N.

(b) What does the lower scale read? Answer in units of N.

\section{Water Squirting from Tank Problem}
A jet of water squirts out horizontally from a hole 1 m from the bottom of the tank and to a point 0.6 m right from the tank.

If the hole has a diameter of 3.5 mm, what is the height of the water above the hole in the tank? Answer in units of cm.

\section{Oil in Horizontal Pipe Problem}
A horizontal pipe of diameter 1 m has a smooth constriction to a section of diameter 0.6 m. The density of oil flowing in the pipe is 821 kg/m$^3$.

If the pressure in the pipe is 8000 N/m$^2$ and in the constricted section is 6000 N/m$^2$, what is the rate at which oil is flowing? Answer in units of m$^3$/s

\section{Fireman Hose Problem}
A fireman standing on a 10 m high ladder operates a water hose with a round nozzle of diameter 2 inch. The lower end of the hose (10 m below the nozzle) is connected to the pump outlet of diameter 3 inch. 
The gauge pressure of the water at the pump is 
\begin{align*}
    P^{\text{(gauge)}}_{\text{pump}}=P^{\text{(abs)}}_{\text{pump}}-P_{\text{atm}}\\
    = 43.2 \text{PSI} = 297.854 \text{kPa}
\end{align*} 

Calculate the speed of the water jet emerging from the nozzle. Assume that water is an incompressible liquid of density 1000 kg/m$^3$ and negligible velocity. The acceleration of gravity is 9.8 m/s$^2$. Answer in units of m/s.
\section{U Tube Problem}
A heavy liquid with a density 13.6 g/cm$^3$ is poured into a U-tube. The left-hand arm of the tube has a cross-sectional area of 10 cm$^2$, and the right-hand arm has a cross-sectional area 
of 5 cm$^2$. A quantity of 100 g of a light liquid with a density 1 g/cm$^3$ is then poured into the right-hand arm.

(a) Determine the height $L$ of the light liquid in the column in the right arm of the U-tube. Answer in units of cm.

(b) If the density of the heavy liquid is 13.6 g/cm$^3$, by what height $h_1$ does the heavy liquid rise in the left arm? Answer in units of cm.

\end{document}