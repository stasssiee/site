\documentclass[../physics12.tex]{subfiles}
\graphicspath{{\subfix{../figures/}}}
\begin{document}
\chapter{Energy}
\section{Formulas}
Work is force times distance. $W=Fd\cos\theta$. Power $P$ is work per time; also $P=Fv$. Mechanical energy is conserved for systems without external forces, and which 
do not generate heat through friction, and do not create or consume chemical energy.

Kinetic Energy: KE = $\frac{1}{2}mv^2$

$W_{\text{net}} = \Delta KE$

$U_{\text{gravity}} = mgh$; $\qquad U_{\text{spring}} = \frac{1}{2}kx^2$

\section{Crate Pulled up Ramp Problem}
A crate with mass 15 kg is pulled by a force (parallel to the incline) up a rough incline. The crate has a initial speed is 1.5 m/s.
The crate is pulled a distance of 7.5 m on the incline by a 150 N force. The angle of the incline is 30$\degree$ and the coefficient of friction is 0.3.
The acceleration due to gravity is 9.8 m/s$^2$.

(a) What is the change in kinetic energy of the crate?
Answer in units of J.

(b) What is the speed of the crate after it is pulled the 7.5 m?
Answer in units of m/s.

\section{Work Done from Two Vectors Problem}
A force $\vec{F}=F_x\hat{i}+F_y\hat{j}$ acts on a particle that undergoes a displacement of 
$\vec{s}=s_x\hat{i}+s_y\hat{j}$ where $F_x = 6$N, $F_y=-2$N, $s_x=3$ m, and $s_y=1$ m.

(a) Find the work done by the force on the particle.
Answer in units of J.

(b) Find the angle between $\vec{F}$ and $\vec{s}$.
Answer in units of $\degree$.

\section{Block on a Wall Problem}
A 5.0 kg block is pushed 3.0 m at a constant velocity up a vertical wall by a constant force applied at an angle of $30.0\degree$ with the horizontal.
The acceleration of gravity is 9.81 m/s$^2$.

If the coefficient of kinetic friction between the block and the wall is 0.30, find 

(a) the work done by the force on the block.
Answer in units of J.

(b) the work done by gravity on the block.
Answer in units of J.

(c) the magnitude of the normal force between the block and the wall.
Answer in units of N.
\section{Car Driving Up Pike's Peak Problem}
The engine of a 2000 kg Mercedes going up Pike's Peak delivers energy to its drive wheel at the rate 100 kW.

Neglecting air resistance, what is the largest speed the car can sustain on the steep Pike's Peak mountain highway, 
where the road is $30\degree$ to the horizontal? The acceleration due to gravity is 10 m/s$^2$.

\section{Block Dragged on Rough Surface Problem}
A 15 kg block is dragged over a rough, horizontal surface by a constant force of 70 N acting at an angle of 30$\degree$ above the horizontal.
The block is displaced 5 m, and the coefficient of kinetic friction is 0.1.

(a) Find the work done by the 70 N force. The acceleration of gravity is 9.8 m/s$^2$. 
Answer in units of J.

(b) Find the magnitude of the work done by the force of friction. 
Answer in units of J.

\end{document}