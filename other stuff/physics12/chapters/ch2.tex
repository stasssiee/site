\documentclass[../physics12.tex]{subfiles}
\graphicspath{{\subfix{../figures/}}}
\begin{document}
\chapter{2D Motion and Newton's Laws}
\section{Formulas}
Speed: Magnitude of velocity (2 dimensions)

Two-dimensional uniformly accelerated motion: In the absence of air resistance, for uniformly accelerated motion in two dimensions, the $x$ and $y$ directions can be treated independently.

Free fall: $y=y_0+v_0t-\frac{1}{2}gt^2$

$v_x=v_{0x}$ $\qquad$ $\Delta x = v_{0x}t$

$\Delta x = v_0 \cos\theta_0 t \qquad v_y = v_{0y}-gt$

$\Delta y = v_{0y}t-\frac{1}{2}gt^2$

$\Delta y = v_0\sin\theta_0 t - \frac{1}{2}gt^2$

$v_y^2 = v_0^2\sin^2\theta_0 - 2g\Delta y$

Range: $\Delta x = \frac{v_0^2\sin2\theta_0}{g}$

Newton's second law: $\vec{F}=m\vec{a}$

Newton's third law: $\vec{F}_{12}+\vec{F}_{21} = 0$

Force due to static friction is less than or equal to $\mu_s N$.

Force due to kinetic friction is $\mu_k N$.

\section{Two Planes Problem}
Two airplanes leave an aiport at the same time. The velocity of the first plane is 700 m/h at a heading of 23.4$\degree$.
The velocity of the second is 620 m/h at a heading of 116$\degree$.

How far apart are they after 2.9 h?
Answer in units of m.

\section{Projectile Final Velocity Problem}
A ball is thrown horizontally from the top of a building 140 m high. The ball strikes the ground 68 m horizontally from the point of release. 

What is the speed of the ball just before it strikes the ground?

\section{Two Snowballs Problem}
One strategy in a snowball fight is to throw a snowball at a high angle over level ground. While your opponent is watching this first snowball, you throw a second snowball at a low angle and time it to arrive at the same time as the first.

(a) Assume both snowballs are thrown with the same initial speed 24 m/s. The first snowball is thrown at an angle of 59$\degree$ above the horizontal. At what angle should you throw the second snowball to make it hit the same point as the first? 
Note that starting and ending heights are the same. The acceleration of gravity is 9.8 m/s$^2$.

(b) How many seconds after the first snowball should you throw the second so that they arrive on target at the same time?
Answer in units of s.

\section{Cheetah and Gazelle Problem}
A cheetah can run at a maximum speed 109 km/h and a gazelle can run at a maximum speed of 73.8 km/h.

(a) If both animals are running at full speed, with the gazelle 90.2 m ahead, how long before the cheetah catches its prey?
Answer in units of s.

The cheetah can maintain its maximum speed for only 7.5 s.

(b) What is the minimum distance the gazelle must be ahead of the cheetah to have a chance of escape? (After 7.5 s the speed of cheetah is less than that of the gazelle.)
Answer in units of m.

\section{Salmon Problem}
Salmon often jump waterfalls to reach their breeding grounds.

Starting downstream, 1.99 m away from a waterfall 0.354 m in height, at what minimum speed must a salmon jumping at an angle of 38.4$\degree$ leave the water to continue upstream?
The acceleration due to gravity is 9.81 m/s$^2$.
Answer in units of m/s.

\section{Block at Rest on Plane Problem}
A block is at rest on an incline where the mass of the block is 10 kg and the angle of the incline is 28$\degree$. The coefficients of static and kinetic friction are $\mu_s = 0.62$ and $\mu_k = 0.53$, respectively.
The acceleration of gravity is 9.8 m/s$^2$.

(a) What is the frictional force acting on the 10 kg mass?
Answer in units of N.

(b) What is the largest angle which the incline can have so that the mass does not slide down the incline?
Answer in units of $\degree$.

(c) What is the acceleration of the block down the incline if the angle of the incline is $39\degree$?
Answer in units of m/s$^2$.

\section{Atwood Machine Problem}
A light, inextensible cord passes over a light, frictionless pulley with a radius of 14 cm. It has a(n) 1.5 kg mass on the left and a(n) 1 kg mass on the right, both hanging freely.
Initially their center of masses are a vertical distance 3.3 m apart.
The acceleration of gravity is 9.8 m/s$^2$.

(a) At what rate are the two masses accelerating when they pass each other?
Answer in units of m/s$^2$.

(b) What is the tension in the cord when they pass each other?
Answer in units of N.

\section{Force Between Two Blocks Problem}
Consider the following system, where there is a force $F = 40$N right on a block of $M = 9$ kg and a smaller block $m=1$ kg that is touching block with mass $M$.

What is the magnitude of the force with which one block acts on the other?

\section{Two Cars Braking Problem}
You are driving at the speed of 30 m/s (67.1224 mph) when suddenly the car in front of you (previously traveling at the same speed) brakes and begins to slow down with the largest 
deceleration possible without skidding. Considering an average human reaction, you press your brakes 0.5 s later.
You also brake and decelerate as rapidly as possible without skidding. Assume that the coefficient of static friction is 0.8 between both cars' wheels and the road.
The acceleration of gravity is 9.8 m/s$^2$.

(a) Calculate the acceleration of the car in front of you when it brakes.
Answer in units of m/s$^2$.

(b) Calculate the braking distance for the car in front of you.
Answer in units of m.

(c) Find the minimum safe distance at which you can follow the car in front of you and avoid hitting it (in the case of emergency braking described here).
Answer in units of m.

\end{document}