\documentclass[../physics12.tex]{subfiles}
\graphicspath{{\subfix{../figures/}}}
\begin{document}
\chapter{Rotational Motion and Gravity}
\section{Formulas}
$\Delta \theta = \theta_f - \theta_i$

Angular velocity: $\omega = \lim_{\Delta t \to 0}\frac{\Delta \theta}{\Delta t}$

Tangential velocity: $v=r\omega$

Angular acceleration: $\alpha = \lim_{\Delta t \to 0}\frac{\Delta \omega}{\Delta t}$

Tangential acceleration: $a=r\alpha$

Const. angular accel: $\omega = \omega_0 + \alpha t$

$\Delta \theta = \omega_0 t + \frac{1}{2}\alpha t^2$

$\omega^2 = \omega_0^2 + 2\alpha\Delta\theta$

Centripetal accel: $a_c = \frac{v^2}{r}$

Gravitation: $F=G\frac{m_1m_2}{r^2}$

Torque: $\tau = rF\sin\theta$

Center of Gravity: $x_{\text{cg}} = \frac{\sum_i m_ix_i}{\sum_i m_1}$

Moment of Intertia: $I=\sum_i m_ir_i^2$

Hoop: $I=MR^2$

Solid Sphere: $I=\frac{2}{5}MR^2$

Thin Spherical Shell: $I=\frac{2}{3}MR^2$

Solid Cylinder: $I=\frac{1}{2}MR^2$

Thin Rod (Center): $I=\frac{1}{12}MR^2$

Thin Rod (End): $I=\frac{1}{3}MR^2$

For statics problems, sum of torques vanishes and vector sum of forces vanishes. Angular momentum is conserved for any system where the total torque vanishes. 
Results can depend on origin of choice.

Torque: $\tau = I\alpha$

Angular Kinetic Energy: KE$_{\text{rot}} = \frac{1}{2}I\omega^2$

Angular Momentum: $L=I\omega$
\section{Rotating Disks Problem}
The speed of a moving bullet can be determined by allowing the bullet to pass through two rotating paper disks mounted a distance 72 cm 
apart on the same axle. From the angular displacement $20.9\degree$ of the two bullet holes in the disks and the rotational speed 
509 rev/min of the disks, we can determine the speed of the bullet.

What is the speed of the bullet?
Answer in units of m/s.

\section{Car Decelerating Problem}
The driver of a car traveling at 31.3 m/s applies the brakes and undergoes a constant deceleration of 1.6 m/s$^2$.

How many revolutions does each tire make before the car comes to a stop, assuming that the car does not skid and that the tires have radii of 0.31 m?
Answer in units of rev.

\section{Testing Car Tires Problem}
To test the performance of its tires, a car travels along a perfectly flat (no banking) circular track of radius 190 m.
The car increases its speed at a uniform rate of 
\[ a_t \equiv \frac{\dd |v|}{\dd t} = 4.94 \text{m/s}^2\]
until the tires start to skid.

If the tires start to skid when the car reaches a speed of 29.7 m/s, what is the coefficient of static friction between the tires and the road?
The acceleration of gravity is 9.8 m/s$^2$.

\section{Coin on Turntable Problem}
A coin is placed 33 cm from the center of a horizontal turntable, initially at rest. The turntable then begins to rotate.
When the speed of the coin is 120 cm/s (rotating at a constant rate), the coin just begins to slip. The acceleration due to gravity is 980 cm/s$^2$.

What is the coefficient of static friction between the coin and the turntable?

\section{Amusement Park Ride Problem}
An amusement park ride consists of a rotating circular platform 7.66 m in diameter from which 10 kg seats are suspended at the end of 3.71 m massless chains.
When the system rotates, the chains make an angle of 33.1$\degree$ with the vertical.
The acceleration of gravity is 9.8 m/s$^2$.

(a) What is the speed of each seat? 
Answer in units of m/s.

(b) If a child of mass 57.2 kg sits in a seat, what is the tension in the chain (for the same angle)?
Answer in units of N.

\section{VCR Tape Problem}
The tape in a videotape cassette has a total length 272 m and can play for 2.4 h. As the tape starts to play, the full reel has an outer radius of 44 mm and an inner radius of 12 mm.
At some point during the play, both reels will have the same angular speed.

What is this common angular speed?
Answer in units of rad/s.

\section{Apollo Spacecraft Problem}
On the way to the moon, the Apollo astronauts reach a point where the Moon's gravitational pull is stronger than that of Earth's.

(a) Find the distance of this point from the center of the Earth. The masses of the Earth and the Moon are $5.98\times 10^{24}$ kg and $7.36\times 10^{22}$ kg, respectively, 
and the distance from the Earth to the Moon is $3.84\times 10^8$ m. 
Answer in units of m.

(b) What would the acceleration of the astronaut be due to the Earth's gravity at this point if the moon was not there? The value of the universal gravitational constant is $6.672\times 10^{-11}$ N$\cdot$m$^2$/kg$^2$.
Answer in units of m/s$^2$.

\section{Car on a Curve Problem}
A highway curves to the left with radius of curvature of 47 m and is banked at $25\degree$ so that cars can take this curve at higher speeds.
Consider a car of mass 909 kg whose tires have a static friction coefficient 0.64 against the pavement.

How fast can the car take this curve without skidding to the outside of the curve? The acceleration due to gravity is 9.8 m/s$^2$.
Answer in units of m/s.

\section{Force from Two Large Masses Problem}
Objects with masses of 149 kg and 424 kg are separated by 0.413 m. A 75 kg mass is placed midway between them.

(a) Find the magnitude of the net gravitational force exerted by the two larger masses on the 75 kg mass. The value of the universal gravitational constant is 
$6.672\times 10^{-11}$ N$\cdot$m$^2$/kg$^2$.
Answer in units of N.

(b) Leaving the distance between the 149 kg and the 424 kg masses fixed, at what distance from the 424 kg mass (other than infinitely remote ones) does the 75 kg mass experience a net force of zero?
Answer in units of m.

\section{Man on a Ladder Problem}
A 17.7 kg person climbs up a uniform ladder with negligible mass. The upper end of the ladder rests on a frictionless wall. 
The bottom of the ladder rests on a floor with a rough surface where the coefficient of static friction is 0.22.
The angle between the horizontal and the ladder is $\theta$. The person wants to climb up the ladder at a distance of 0.91 m along the ladder from the ladder's foot.
The length of the ladder is 2.1 m.

What is the minimum angle $\theta_{min}$ (between the horizontal and the ladder) so that the person can reach a distance of 0.91 m without having the ladder slip?
The acceleration of gravity is 9.8 m/s$^2$.
Answer in units of $\degree$.

\section{Ladder on a Wall Problem}
A ladder rests against a vertical wall. There is no friction between the wall and the ladder. 
The coefficient of static friction between the ladder and the ground is $\mu = 0.547$.

Determine the smallest angle $\theta$ for which the ladder remains stationary. 
Answer in units of $\degree$.

\section{Hammer and Nail Problem}
A hammer of length 28 cm pulls a nail at an angle $33\degree$ 3.63 cm left from the point of contact of the hammer on a horizontal board.

(a) If a force of magnitude 211 N is exerted horizontally 28 cm from the board on the hammer, find the force exerted by the hammer on the nail. (Assume that the force the hammer exerts on the nail is parallel to the nail).
Answer in units of N.

(b) Find the force exerted by the surface on the point of contact with the hammer. Assume that the force the hammer exerts on the nail is parallel ot the nail.
Answer in units of N.

\section{4 Plates Center of Mass Problem}
A square plate is produced by welding together four smaller square plates, each of side $a$.
The weight of each of the four places are 30 N on the bottom left, 60 N on bottom right, 80 N on top left, and 80 N on top right.

(a) Find the $x$-coordinate of the center of gravity (as a multiple of $a$).
Answer in units of $a$.

(b) Find the $y$-coordinate of the center of gravity (as a multiple of $a$).
Answer in units of $a$.

\section{Blocks and Pulley Problem}
An Atwood machine is constructed using two wheels (with the masses concentrated at the rims). The left wheel has a mass of 2.4 kg and radius 22.24 cm.
The right whell has a mass of 2.2 kg and radius 32.91 cm. The hanging mass on the left is 1.91 kg and on the right 1.68 kg.

What is the acceleration of the system? The acceleration of gravity is 9.8 m/s$^2$.
Answer in units of m/s$^2$.

\section{Energy of Merry-Go-Round Problem}
A horizontal 846 N merry-go-round of radius 1.95 m is started from rest by a constant horizontal force of 77.8 N applied tangentially to the merry-go-round.

Find the kinetic energy of the merry-go-round after 3.62 s. The acceleration of gravity is 9.8 m/s$^2$. Assume the merry-go-round is a solid cylinder.
Answer in units of J.

\section{Rolling Basketball Problem}
A regulation basketball has a 29 cm diameter and may be approximated as a thin spherical shell.

How long will it take a basketball starting from rest to roll without slipping 2.9 m down an incline that makes an angle of 51.9$\degree$ with the horizontal?
The acceleration of gravity is 9.81 m/s$^2$.

\section{Dishonest Pan Balance Problem}
Two pans of a balance are 72.6 cm apart. The fulcrum of the balance has been shifted 1.01 cm away from the center by a dishonest shopkeeper.

By what percentage is the true weight of the goods being marked up by the shopkeeper? Assume the balance has negligible mass.
Answer in units of \%.

\section{Pitched Baseball Problem}
The center of mass of a pitched baseball of radius 2.57 cm moves at 50.2 m/s. The ball spins about an axis through its center of mass with an angular speed of 198 rad/s.

Calculate the ratio of the rotational energy to the translational kinetic energy. Treat the ball as a uniform sphere.

\section{Rotating Stool Problem}
A student sits on a rotating stool holding two 4 kg objects. When his arms are extended horizontally, the objects are 1.1 m from the axis of rotation, and he rotates with angular speed of 0.62 rad/sec.
The moment of inertia of the student plus the stool is 7 kg m$^2$ and is assumed to be constant. The student then pulls the objects horizontally to a radius 0.27 m from the rotation axis.

Calculate the final angular speed of the student.
Answer in units of rad/s.

\section{Merry-Go-Round Conservation of Angular Momentum Problem}
A merry-go-round rotates at the rate of 0.12 rev/s with an 83 kg man standing at a point 1.5 m from the axis of rotation.

(a) What is the new angular speed when the man walks to a point 0 m from the center? Consider the merry-go-round is a solid 63 kg cylinder of radius of 1.5 m.
Answer in units of rad/s.

(b) What is the change in kinetic energy due to this movement?
Answer in units of J.

\section{Man Walking in Boat Problem}
A 70.8 kg man sits on the stern of a 4.1 m long boat. The prow of the boat touches the pier, but the boat isn't tied.
The man notices his mistake, stands up and walks to the boat's prow, but by the time he reaches the prow, it's moved 1.75 m away from the pier.

Assuming no water resistance to the boat's motion, calculate the boat's mass (not counting the man). Answer in units of kg.

\end{document}