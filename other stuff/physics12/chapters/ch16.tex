\documentclass[../physics12.tex]{subfiles}
\graphicspath{{\subfix{../figures/}}}
\begin{document}
\chapter{Nuclear Physics}
\section{Formulas}
Check chapter 15 formulas 

\section{Half-Life of Radioactive Substance Problem}
Suppose that you start with 1.23 g of a pure radioactive substance and determine 4 h later that only 0.076875 g of the substance is left undecayed.

What is the half-life of this substance? Answer in units of h.

\section{Radioactive Sample Activity Problem}
A sample of radioactive isotope is found to have an activity of 115 Bq immediately after it is pulled from the reactor that formed the isotope.
Its activity 2 h, 15 min later is measured to be 85.2 Bq.

(a) Find the decay constant of the sample. Answer in units of h$^{-1}$.

(b) Find the half-life of the sample. Answer in units of h.

(c) How many radioactive nuclei were there in the sample initially?

\section{Rubidium Isotope Problem}
The rubidium isotope $^{87}$Rb is a $\beta$ emitter with a half life of $4.9\times 10^{10}$ y that decays into $^{87}$Sr. It is used to determine 
the age of rocks and fossils. Rocks containing the fossils of early animals contain a ratio of $^{87}$Sr to $^{87}$Rb of 0.01.

Assuming that there was no $^{87}$Sr present when the rocks were formed, calculate the age of these fossils. Answer in units of y.

\section{Reduced Activity of Sample Problem}
A 200 mCi sample of a radioactive isotope is purchased by a medical supply house.

If the sample has a half-life of 14 d, how long will it keep before its activity is reduced to 20 mCi?

Answer in units of d.

\section{Carbon Dating Charcoal Problem}
A piece of charcoal used for cooking is found at the remains of an ancient campsite. A 1 kg sample of carbon from the wood 
has an activity of 2000 decays per minute.

Find the age of the charcoal. Living material has an activity of 15 decays/minute per gram of carbon present and the half-life of $^{14}$C is 5730 y.

Answer in units of y.
\end{document}