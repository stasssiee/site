\documentclass[../physics12.tex]{subfiles}
\graphicspath{{\subfix{../figures/}}}
\begin{document}
\chapter{Waves and Sound}
\section{Formulas}
Simple Harmonic Motion: $F(x)=ma=-kx$

$x(t)$ and $v(t)$ are the same as if the mass is traveling around a circle of radius $A$ with velocity $v_{\text{max}}=2\pi A/T = \omega A$, but one keeps track of only motion in the $x$ direction.

\[x = A\cos(2\pi t/T) \qquad v=v_{\text{max}}\sin(2\pi t/T)\]

\[\omega = 2\pi f = 2\pi/T = \sqrt{k/m}\]

Pendulum: $F=-mgs/L \implies \omega = \sqrt{g/L}$
\[ T = 2\pi\sqrt{L/g} \]

Waves: $v=\frac{\lambda}{T}=f\lambda$

Transverse wave on a string: $v=\sqrt{F/\mu}$; $\mu = m/L$
\begin{itemize}
    \item Superposition: Two waves traveling through each other simply add point by point 
    \item Waves reflecting off fixed ends reverse direction and amplitude 
    \item Waves reflecting off free ends reverse direction but not amplitude 
\end{itemize}

Longitudinal waves:

Bulk modulus: $B = -\frac{\Delta P}{\Delta V/V}$
\[ v = \sqrt{B/\rho} \qquad v = \sqrt{Y/\rho} \]
For sound in air: $v\approx 331$ m/s $\sqrt{\frac{T}{273K}}$

For a string fixed at both ends.
\[ f_1 = v\lambda_1 = v/(2L)=\frac{1}{2L}\sqrt{F/\mu} \]
\[ f_n = nf_1 = \frac{n}{2L}\sqrt{F/\mu}, \text{ where } n = 1,2,3,\dots\]

Here $n$ describes the $n$'th harmonic. For a standing wave in an air column closed at one end and open at the other, 
\[ f_n = nf_1 = \frac{nv}{4L}, \text{ where } n = 1,3,5,\dots \]

Sound: $I=\frac{P}{A}, I_0 = 10^{-12}$ W/m$^2$; Threshold of pain: 1 W/m$^2$.
\[ \beta = 10\log_{10}(I/I_0) \]
Spherical waves:
\[ I = P_{\text{ave}}/4\pi r^2 \]

\section{Temperature of Air Problem}
A sound wave has a frequency of 700 Hz in air and a wavelength of 0.5 m.

What is the temperature of the air? Relate the speed of sound in air to temperature in units of Kelvin, but answer in units of Celsius. Assume the velocity of sound at $0\degree$C is 331 m/s. Answer in units of $\deg$ C.

\section{Rock Band Music Problem}
A rock group is playing in a bar. Sound emerging from the door spreads uniformly in all directions.
The intensity level of the music is 80 dB at a distance of 5 m from the door.

At what distance is the music just barely audible to a person with a normal threshold of hearing? Disregard absorption. Answer in units of m.

\section{Intensity of Chorus Problem}
The sound level produced by one singer is 80 dB. 

What would the sound level produced by a chorus of 29 such singers (all singing at the same intensity at approximately the same distance as the original singer)? Answer in units of dB.

\section{String Harmonics Problem}
A cello string vibrates in its fundamental mode with a frequency of 220 1/s. The vibrating segment is 70 cm long and has a mass of 1.2 g.

(a) Find the tension in the string. Answer in units of N.

(b) Find the frequency of the string when it vibrates in three segments. Answer in units of 1/s.

\section{Spring Constant Problem}
A common technique used to measure toe force constant $k$ of a spring is the following:

Hang the spring vertically, then allow a mass $m$ to stretch if a distance $d$ from the equilibrium position under the action of the ``load'' $mg$.

Find the spring constant $k$ if the spring is stretched a distance 55 m by a suspended weight of 39 N. The acceleration of gravity is 9.8 m/s$^2$. Answer in units of N/m.

\section{String Stretched Twice Problem}
A load of 50 N attached to a spring hanging vertically stretches the spring 5 cm. The spring is now placed horizontally on a table and stretched 11 cm.

What force is required to stretch it by this amount? Answer in units of N.

\section{Half of Max Speed SHM Problem}
A particle executes simple harmonic motion with an amplitude of 3 cm. 

At what positive displacement from the midpoint of its motion does its speed equal one half of its maximum speed? Answer in units of cm.

\section{Tension in Phone Cord Problem}
A phone cord is 4 m long. The cord has a mass of 0.2 kg. A transverse wave pulse is produced by plucking one end of the taut cord.
The pulse makes four trips down and back along the cord in 0.8 s.

What is the tension in the cord? Answer in units of N.
\end{document}