\documentclass[10pt,a4paper,oneside]{book}

\title{Lesson 1}
\author{}
\date{}

\usepackage[utf8]{inputenc}
\usepackage[margin=1.0in]{geometry}
\usepackage{amsmath}
\usepackage{amsfonts}
\usepackage{amssymb}
\usepackage{enumitem}                       % custom enum labels
\usepackage{parskip}                        % add vertical paragraph space
\usepackage{tocloft}						% modify toc position
\usepackage{xr}								% cross-references
\usepackage{mathtools}						% Aboxed
\usepackage{empheq}							% box multiple lines
\usepackage{upgreek}
\usepackage{gensymb}
\usepackage{chemformula}
\usepackage{esint}							% oiint
\usepackage{cancel}
\usepackage{tikz}
\usetikzlibrary{calc}
\usepackage{asymptote}
\usepackage[framemethod=TikZ]{mdframed}     % graphics and framed envs
\usepackage[hang,flushmargin]{footmisc}		% remove footnote indentation
\usepackage[hyperfootnotes=false,
					hidelinks]{hyperref}	% create clickable table of contents
\usepackage{cancel}

\newcommand{\tarc}{\mbox{\large$\frown$}}
\newcommand{\arc}[1]{\stackrel{\tarc}{#1}}
%\newcommand{\degree}{^{\circ}}
\newcommand{\blank}{\_\_\_\_\_\_}

\DeclareMathOperator\cis{cis}
\DeclareMathOperator\Arg{Arg}


\renewcommand{\familydefault}{\sfdefault}  	% sans serifs text
\setlength{\parindent}{0pt}                	% no paragraph indentation

% region TITLES
%\setbox0=\hbox{\Huge{\textbf{\textsf{\courseid: }}}}
\setlength{\cftbeforetoctitleskip}{0em}
\setlength{\cftaftertoctitleskip}{1em}
%\renewcommand{\contentsname}{\hangindent=\wd0 \strut \courseid: \coursetitle \\ \medskip {\professor, \campus, \semester}}

% region COMMANDS
\newcommand{\ds}{\displaystyle}
\newcommand{\pfn}[1]{\textrm{#1}}	% enables new functions
\newcommand{\mbf}[1]{\mathbf{#1}}	% mathbf
\newcommand{\C}{\mathbb{C}}			% fancy C
\newcommand{\R}{\mathbb{R}}			% fancy R
\newcommand{\Q}{\mathbb{Q}}			% fancy Q
\newcommand{\Z}{\mathbb{Z}}			% fancy Z
\newcommand{\N}{\mathbb{N}}			% fancy N
\newcommand{\dd}{\mathrm{d}}
\newcommand{\from}{\leftarrow}
\newcommand{\qed}{$\square$}
\newcommand{\ex}{\textit{Exercise }}
% endregion

\let\footnoterule\relax
\newcommand\blfootnote[1]{%
  \begingroup
  \renewcommand\thefootnote{}\footnote{#1}%
  \addtocounter{footnote}{-1}%
  \endgroup
}

\usepackage{titlesec}
\titleformat{\chapter}{\Huge\normalfont\bfseries}{\thechapter}{1em}{}
\titlespacing*{\chapter}{0em}{0em}{2em}
% endregion

% region ENVIRONMENTS
\newcounter{theo}[chapter]\setcounter{theo}{0}
\newcommand{\numTheo}{\arabic{chapter}.\arabic{theo}}
\newcommand{\mdftheo}[3]{
	\mdfsetup{
		frametitle={
			\tikz[baseline=(current bounding box.east),outer sep=0pt]
			\node[anchor=east,rectangle,fill=#3]
			{\ifstrempty{#2}{\strut #1~\numTheo}{\strut #1~\numTheo:~#2}};
		},
		innertopmargin=4pt,linecolor=#3,linewidth=2pt,
		frametitleaboveskip=\dimexpr-\ht\strutbox\relax,
		skipabove=11pt,skipbelow=0pt
	}
}
\newcommand{\mdfnontheo}[3]{
	\mdfsetup{
		frametitle={
			\tikz[baseline=(current bounding box.east),outer sep=0pt]
			\node[anchor=east,rectangle,fill=#3]
			{\ifstrempty{#2}{\strut #1}{\strut #1:~#2}};
		},
		innertopmargin=4pt,linecolor=#3,linewidth=2pt,
		frametitleaboveskip=\dimexpr-\ht\strutbox\relax,
		skipabove=11pt,skipbelow=0pt
	}
}
\newcommand{\mdfproof}[1]{
	\mdfsetup{
		skipabove=11pt,skipbelow=0pt,
		innertopmargin=4pt,innerbottommargin=4pt,
		topline=false,rightline=false,
		linecolor=#1,linewidth=2pt
	}
}


\newenvironment{theorem}[1][]{
	\refstepcounter{theo}
	\mdftheo{Theorem}{#1}{red!25}
	\begin{mdframed}[]\relax
}{\end{mdframed}}

\newenvironment{lemma}[1][]{
	\refstepcounter{theo}
	\mdftheo{Lemma}{#1}{red!15}
	\begin{mdframed}[]\relax
}{\end{mdframed}}

\newenvironment{corollary}[1][]{
	\refstepcounter{theo}
	\mdftheo{Corollary}{#1}{red!15}
	\begin{mdframed}[]\relax
}{\end{mdframed}}

\newenvironment{definition}[1][]{
	\mdfnontheo{Definition}{#1}{blue!20}
	\begin{mdframed}[]\relax
}{\end{mdframed}}

\newenvironment{exercise}[1][]{
	\mdfproof{black!15}
	\textit{Exercise. }
}

\newenvironment{proof}[1][]{
	\mdfproof{black!15}
	\begin{mdframed}[]\relax
\textit{Proof. }}{\qed \end{mdframed}}

\newenvironment{claim}[1][]{
	\mdfproof{black!15}
	\begin{mdframed}[]\relax
\textit{Claim. }}{\end{mdframed}}

\newenvironment{example}[1][]{
	\mdfnontheo{Example}{#1}{yellow!40}
	\begin{mdframed}[]\relax
}{\end{mdframed}}

\newenvironment{summary}[1][]{
	\mdfnontheo{Summary}{#1}{green!70!black!30}
	\begin{mdframed}[]\relax
}{\end{mdframed}}
% endregion

\let\oldsum\sum
\renewcommand{\sum}{\oldsum\limits}

\let\oldlim\lim
\renewcommand{\lim}{\oldlim\limits}

\let\oldprod\prod
\renewcommand{\prod}{\oldprod\limits}

\let\oldbigcup\bigcup
\renewcommand{\bigcup}{\oldbigcup\limits}

\let\oldbigcap\bigcap
\renewcommand{\bigcap}{\oldbigcap\limits}

\let\oldmax\max
\renewcommand{\max}{\oldmax\limits}

\let\oldmin\min
\renewcommand{\min}{\oldmin\limits}

\let\oldsup\sup
\renewcommand{\sup}{\oldsup\limits}

\let\oldinf\inf
\renewcommand{\inf}{\oldinf\limits}
\graphicspath{{figures/}}
\usepackage{subfiles}

\begin{document}
\maketitle

The topic for this lesson is on \textbf{sets}.

Roughly speaking, a set is a collection of objects. The objects can be anything: numbers, functions, other sets, a combination of these, or nothing at all.

All that matters is what objects are in the set, order is unimportant.

There might be a finite number of objects in the set, that makes a finite set. Otherwise, it is an infinite set. The objects in the set are called the elements or members of the set.

There are two basic ways that one can describe a set. The first is to list its elements such as 
\begin{center}
    $A = \{2,9,22\}$
\end{center}

This is a set with three elements and the simplest way to define or describe a set: By listing the elements inside of curly braces and separate the elements by commas.

As we said earlier, order doesn't matter so $A=\{9,22,2\} = \{2,9,22\}$.

Also each element can only be in the set once, so $B=\{3,6,3\}$ is not legal as a set.

Sometimes it's impractical to list a big set, so we can use ellipses if the pattern is clear. For example
we can describe $\{1,2,3,\dots,99,100\}$ reasonably.

If a set is infinite, then we can list the elements using ellipses, such as $\{1,2,3,\dots\}$. Just make sure the pattern is clear.

Another way to describe a set is to provide a property that describes the elements such as 
\begin{center}
    $B = \{x|x \text{ is an integer} \}$
\end{center}

You should read the vertical bar as ``such that''. 

Another way we can write this is on the interval of real numbers such as 
\begin{center}
    $\{x|x \text{ is a real number and } 2<x\leq 3\}$
\end{center}

We can even describe a set even if we don't know what the elements are such as 
\begin{center}
    $\{y|y \text{ is a real number and } 2y^4-y^3+6y^2-11y+12=0\}$
\end{center}

If an object $x$ is an element of $S$ we write this as $x\in S$. If $x$ is not an element of $S$, we write $x\notin S$. For any object $x$ and any set of $S$, one of these is true.

There is also an empety set denoted by $\emptyset$. This is the set with no elements.

For example one of these sets is 
\begin{center}
    $\{x|x \text{ is a real number and } x^2<0\}$
\end{center}

If $S$ is a finite set, when we let $\#(S)$ denote the number of elements of $S$. This is the cardinality of $S$.

For example if $S=\{2,4,9,11\}$ then $\#S = 4$.

If $S$ is an infinite set, then we cannot define $\#(S)$ in the scope of this class.

A set $A$ is called a subset of $B$ if every element of $A$ is also an element of $B$. The notation for this is $A\subseteq B$.

If $A$ is a subset of $B$, $B$ is a superset of $B$.

Note that every set is a subset of itself, so to notate a proper subset, one where the subset is not equal to the larger set, we notate this as $A\subset B$.

Note some properties of subsets:
\begin{itemize}
    \item Every set if a subset of itself 
    \item The empety set if a subset of any set 
    \item If $A,B$ are two sets such that $A\subseteq B$ and $B\subseteq A$, then $A=B$.
    \item If $A$ and $B$ are any two sets, then we cannot have both $A\subset B$ and $B\subset A$.
\end{itemize}

Just as we can perform operations on numbers, we can also perform operations on sets.

\begin{definition}
    The union $A\cup B$ of two sets $A$ and $B$ is the set of all objects that are elements of $A$ or of $B$.

    This is also written as 
    \[ A \cup B = \{ x|x \in A \text { or } x\in B\}\]
\end{definition}

\begin{definition}
    The intersection $A\cap B$ of two sets $A$ and $B$ is the set of all objects that are both elements of both $A$ and $B$.

    This is also written as 
    \[ A \ cap B = \{ x|x \in A \text{ and } x \in B \}\]
\end{definition}

\begin{definition}
    If $A\cap B = \emptyset$, then we say that $A$ and $B$ are disjoint.
\end{definition}

\begin{definition}
    The set difference of two sets $A$ and $B$ is 
    \[ A\setminus B = \{x \in A | x\notin B\}\]
\end{definition}

\subsubsection*{Homework}

\begin{enumerate}
    \item Consider the following sets:
    \[ A = \{1,2,3,4,5\} \qquad B = \{2,3,4\} \qquad C = \{3,\{4,5\}\}\]
    \begin{itemize}
        \item Is $A\subseteq A$? Is $A\subset A$?
        \item Is $B\subseteq A$? Is $B\subset A$?
        \item Is $C\subseteq A$? Is $C\subset A$?
        \item Is $4\in B$? Is $4\in C$?
        \item List all of the subsets of $B$. How many are there?
    \end{itemize}

    \item Suppose $A$ and $B$ are sets such that $A\subseteq B$.
    \begin{itemize}
        \item What is $A\cup B$?
        \item What is $A\cap B$?
    \end{itemize}

    \item Show that if $A\subseteq B$ and $A\subseteq C$, then $A\subseteq(B\cap C)$. Show that the same statement is not true if we replace every ``$\subseteq$'' with ``$\subset$''.
    
    \item Prove that $A\subseteq A$ for any set $A$.

    \item Show that, for any sets $A$, $B$, $C$,
    \[A\cup (B\setminus C) = (A\cup B)\setminus (C\setminus A)\]
\end{enumerate}


\end{document}