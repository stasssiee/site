\documentclass[10pt,a4paper,oneside]{book}

\title{Lesson 2}
\author{}
\date{}

\usepackage[utf8]{inputenc}
\usepackage[margin=1.0in]{geometry}
\usepackage{amsmath}
\usepackage{amsfonts}
\usepackage{amssymb}
\usepackage{enumitem}                       % custom enum labels
\usepackage{parskip}                        % add vertical paragraph space
\usepackage{tocloft}						% modify toc position
\usepackage{xr}								% cross-references
\usepackage{mathtools}						% Aboxed
\usepackage{empheq}							% box multiple lines
\usepackage{upgreek}
\usepackage{gensymb}
\usepackage{chemformula}
\usepackage{esint}							% oiint
\usepackage{cancel}
\usepackage{tikz}
\usetikzlibrary{calc}
\usepackage{asymptote}
\usepackage[framemethod=TikZ]{mdframed}     % graphics and framed envs
\usepackage[hang,flushmargin]{footmisc}		% remove footnote indentation
\usepackage[hyperfootnotes=false,
					hidelinks]{hyperref}	% create clickable table of contents
\usepackage{cancel}

\newcommand{\tarc}{\mbox{\large$\frown$}}
\newcommand{\arc}[1]{\stackrel{\tarc}{#1}}
%\newcommand{\degree}{^{\circ}}
\newcommand{\blank}{\_\_\_\_\_\_}

\DeclareMathOperator\cis{cis}
\DeclareMathOperator\Arg{Arg}


\renewcommand{\familydefault}{\sfdefault}  	% sans serifs text
\setlength{\parindent}{0pt}                	% no paragraph indentation

% region TITLES
%\setbox0=\hbox{\Huge{\textbf{\textsf{\courseid: }}}}
\setlength{\cftbeforetoctitleskip}{0em}
\setlength{\cftaftertoctitleskip}{1em}
%\renewcommand{\contentsname}{\hangindent=\wd0 \strut \courseid: \coursetitle \\ \medskip {\professor, \campus, \semester}}

% region COMMANDS
\newcommand{\ds}{\displaystyle}
\newcommand{\pfn}[1]{\textrm{#1}}	% enables new functions
\newcommand{\mbf}[1]{\mathbf{#1}}	% mathbf
\newcommand{\C}{\mathbb{C}}			% fancy C
\newcommand{\R}{\mathbb{R}}			% fancy R
\newcommand{\Q}{\mathbb{Q}}			% fancy Q
\newcommand{\Z}{\mathbb{Z}}			% fancy Z
\newcommand{\N}{\mathbb{N}}			% fancy N
\newcommand{\dd}{\mathrm{d}}
\newcommand{\from}{\leftarrow}
\newcommand{\qed}{$\square$}
\newcommand{\ex}{\textit{Exercise }}
% endregion

\let\footnoterule\relax
\newcommand\blfootnote[1]{%
  \begingroup
  \renewcommand\thefootnote{}\footnote{#1}%
  \addtocounter{footnote}{-1}%
  \endgroup
}

\usepackage{titlesec}
\titleformat{\chapter}{\Huge\normalfont\bfseries}{\thechapter}{1em}{}
\titlespacing*{\chapter}{0em}{0em}{2em}
% endregion

% region ENVIRONMENTS
\newcounter{theo}[chapter]\setcounter{theo}{0}
\newcommand{\numTheo}{\arabic{chapter}.\arabic{theo}}
\newcommand{\mdftheo}[3]{
	\mdfsetup{
		frametitle={
			\tikz[baseline=(current bounding box.east),outer sep=0pt]
			\node[anchor=east,rectangle,fill=#3]
			{\ifstrempty{#2}{\strut #1~\numTheo}{\strut #1~\numTheo:~#2}};
		},
		innertopmargin=4pt,linecolor=#3,linewidth=2pt,
		frametitleaboveskip=\dimexpr-\ht\strutbox\relax,
		skipabove=11pt,skipbelow=0pt
	}
}
\newcommand{\mdfnontheo}[3]{
	\mdfsetup{
		frametitle={
			\tikz[baseline=(current bounding box.east),outer sep=0pt]
			\node[anchor=east,rectangle,fill=#3]
			{\ifstrempty{#2}{\strut #1}{\strut #1:~#2}};
		},
		innertopmargin=4pt,linecolor=#3,linewidth=2pt,
		frametitleaboveskip=\dimexpr-\ht\strutbox\relax,
		skipabove=11pt,skipbelow=0pt
	}
}
\newcommand{\mdfproof}[1]{
	\mdfsetup{
		skipabove=11pt,skipbelow=0pt,
		innertopmargin=4pt,innerbottommargin=4pt,
		topline=false,rightline=false,
		linecolor=#1,linewidth=2pt
	}
}


\newenvironment{theorem}[1][]{
	\refstepcounter{theo}
	\mdftheo{Theorem}{#1}{red!25}
	\begin{mdframed}[]\relax
}{\end{mdframed}}

\newenvironment{lemma}[1][]{
	\refstepcounter{theo}
	\mdftheo{Lemma}{#1}{red!15}
	\begin{mdframed}[]\relax
}{\end{mdframed}}

\newenvironment{corollary}[1][]{
	\refstepcounter{theo}
	\mdftheo{Corollary}{#1}{red!15}
	\begin{mdframed}[]\relax
}{\end{mdframed}}

\newenvironment{definition}[1][]{
	\mdfnontheo{Definition}{#1}{blue!20}
	\begin{mdframed}[]\relax
}{\end{mdframed}}

\newenvironment{exercise}[1][]{
	\mdfproof{black!15}
	\textit{Exercise. }
}

\newenvironment{proof}[1][]{
	\mdfproof{black!15}
	\begin{mdframed}[]\relax
\textit{Proof. }}{\qed \end{mdframed}}

\newenvironment{claim}[1][]{
	\mdfproof{black!15}
	\begin{mdframed}[]\relax
\textit{Claim. }}{\end{mdframed}}

\newenvironment{example}[1][]{
	\mdfnontheo{Example}{#1}{yellow!40}
	\begin{mdframed}[]\relax
}{\end{mdframed}}

\newenvironment{summary}[1][]{
	\mdfnontheo{Summary}{#1}{green!70!black!30}
	\begin{mdframed}[]\relax
}{\end{mdframed}}
% endregion
\graphicspath{{figures/}}
\usepackage{subfiles}

\begin{document}
\maketitle

There are some special sets of numbers.

There is the set of positive integers
\begin{center}
    $\{1,2,3,4,\dots\}$
\end{center}

However, we cannot ignore $0$ or the negative numbers, so we have $\mathbb{Z}$, the entire set of integers.
\begin{center}
    $\mathbb{Z}=\{\dots,-4,-3,-2,-1,0,1,2,3,4,\dots\}$
\end{center}

Addition in $\mathbb{Z}$ gives some nice axioms
\begin{itemize}
    \item \textbf{commutativity}: for any $a,b\in \mathbb{Z}$, $a+b=b+a$
    \item \textbf{associativity}: for any $a,b,c \in \mathbb{Z}$, $(a+b)+c=a+(b+c)$
    \item $\mathbb{Z}$ has an additive identity element, namely $0$, for any $a\in \mathbb{Z}, a+0=0+a=a$
    \item $\mathbb{Z}$ has additive inverses: for any $a\in \mathbb{Z}$, there exists $b\in \mathbb{Z}$ such that $a+b=b+a=0$. (Of course, $b=-a$)
\end{itemize}

We know that $\mathbb{Z}$ has multiplication that is also commutative, associative, and has an identity element, $1$.

We also have that 
\begin{center}
    for any $a,b,c\in \mathbb{Z}$, $a(b+c)=(ab)+(ac)$
\end{center}

We move to the rational numbers due to not having multiplicative inverses.
\begin{center}
    $\mathbb{Q}=\{\frac{m}{n}|m,b\in\mathbb{Z} \text{ and } n\neq 0\}$
\end{center}

The set $\mathbb{Q}$ together with the operations of addition and multiplication is what is called a field, and these operation satisfy all the nice properties we want them to have. However there is something missing from $\mathbb{Q}$.

For example there is no number $x\in \mathbb{Q}$ such that $x^2=2$, since, yes in $\mathbb{R}$ we have a solution of $\sqrt{2}$, but in $\mathbb{Q}$ this does not exist.

\begin{definition}
    Let $S$ be a subset of $\mathbb{R}$.

    \begin{itemize}
        \item A number $x\in \mathbb{R}$ is called an upper bound for $S$ if for all $y\in S$, we have $y\leq x$. If an upper bound for $S$ exists, we say that $S$ is bounded above.
        \item A number $x\in \mathbb{R}$ is called a lower bound for $S$ if for all $y\in S$, we have $y\geq x$. If a lower bound for $S$ exists, we say that $S$ is bounded below.
        \item If $S$ has an upper bound and a lower bound, we say that $S$ is bounded.
    \end{itemize}
\end{definition}

There are some special bounds.
\begin{definition}
    Let $S$ be a subset of $\mathbb{R}$.

    \begin{itemize}
        \item A number $x\in \mathbb{R}$ is called a least upper bound of $S$ if $x$ is an upper bound for $S$ and if for every $z\in \mathbb{R}$ such that $z$ is an upper bound of $S$, we have $x\leq z$.
        \item A number $x\in \mathbb{R}$ is called a greatest lower bound of $S$ if $x$ is a lower bound for $S$ and if for every $z\in \mathbb{R}$ such that $z$ is a lower bound of $S$, we have $x\geq z$.
    \end{itemize}
\end{definition}

In other words, an upper bound of a subset $S\subseteq \mathbb{R}$ is any number greater than or equal to all of the numbers in $S$.
If we can find a smallest possible upper bound for $S$, then this number is the least upper bound of $S$. Similarly, a lower bound of $S$ is a number that is less than or equal to all of the numbers in $S$.
If we can find a greatest possible lower bound for $S$, then this number is the greatest lower bound of $S$.

The most important subsets of $\mathbb{R}$ that we will be regularly using are intervals:
\begin{definition}
    A subset $I$ of $\mathbb{R}$ is called an interval if, for any $a,b \in I$ and $x\in \mathbb{R}$ such that $a\leq x\leq b$, we have $x\in I$.
\end{definition}

Informally, an interval $I$ is a subset of $\mathbb{R}$ without any ``holes''. If $a$ and $b$ are in $I$, then all the numbers between $a$ and $b$ are also in $I$. There are two types of intervals commonly used.
\begin{definition}
    Let $a\leq b$ be any two real numbers. The open interval $(a,b)$ is defined as the set 
    \[ (a,b)=\{x\in\mathbb{R}|a<x<b\} \]
    The closed interval $[a,b]$ is defined as the set
    \[ [a,b] = \{x\in \mathbb{R}|a\leq x\leq b\}\] 
\end{definition}

There are also half-open intervals.

We have one more definition
\begin{definition}
    Let $S$ be a subset of $\mathbb{R}$. If $\text{sup} S\in S$, we say that $\text{sup} S$ is the maximal element of $S$ (or simply the maximum of $S$). 
    If $\text{inf} \in S$, we say that $\text{inf} S$ is the minimal element of $S$ (or simply the minimum of $S$).
\end{definition}


\subsubsection*{Homework}

\begin{enumerate}
    \item Show that there is no number $x\in \mathbb{Q}$ such that $x^2=2$.
    \item 
    \begin{itemize}
        \item Show that the intersection of any two intervals is an interval.
        \item Show that the union of any two intervals need not be an interval.
    \end{itemize}
    \item If $A,B$ are bounded intervals with $A\cap B\neq \emptyset$, then show that $\text{sup}(A\cap B)=\min\{\text{sup}A,\text{sup}B\}$.
\end{enumerate}


\end{document}