\documentclass[../maths.tex]{subfiles}
\graphicspath{{\subfix{../figures/}}}
\begin{document}
\chapter{Algebra \& Functions}
\section{Indices}
\subsection*{Subsets of Real Numbers}
\subsubsection*{Introducing Subsets of Real Numbers}
Natural numbers are represented by $\mathbb{N}$. They are just the counting numbers - like $1,2,3,4,5,6,\dots$. This does not include 0 or negative numbers.

Integers are represented by $\mathbb{Z}$. This includes all the natural numbers and also includes $0,-1,-2,-3,\dots$. It is twice the size of natural numbers plus a zero.

Rational numbers are represented by $\mathbb{Q}$. This would include $\frac{1}{2}, \frac{2}{3}, -\frac{3}{4}, -\frac{5}{7}, -\frac{7}{2}$ along with the natural numbers and integers.

The real numbers are represented by $\mathbb{R}$. This includes everything above, along with things such as $\sqrt{2}, \sqrt{3}, \pi, e$.

The complex numbers are based on if we allowed to square root $-1$. We define this as $i$. The complex numbers will include things such as $2i, 3+i$.
\subsection*{The Laws of Indices}
\subsubsection*{The Laws of Indices}

We should know that $x^2 = x \times x$, and $x^3 = x \times x \times x$. The index tells us how many times we are multiplying $x$ by itself.

When we put the $x$ as $x^2$, we can see that $x^2\times x^2 = x\times x\times x\times x = x^4$ or $(x^2)^2$.

As we can see, when multiplying $x^p \times x^q = x^{p+q}$. 

Also when we have $(x^p)^q = x^{pq}$. Of course we know that $pq = qp$, and ad can also see that $(x^q)^p = (x^p)^q$.

Now let's imagine what we have $x^5 \div x^3 = \frac{x \times x\times x\times x\times x}{x\times x\times x} = x\times x = x^2$.

When we are dividing, then $x^p \div x^q = x^{p-q}$.

Let's say we have $x^{3.5}$. As long as the power is a rational number (in this case $3.5=\frac{7}{2}$), then we can have an idea on what it is. We can write $x^{\frac{7}{2}}$ as $x^{\frac{1}{2}\times 7}$.
This is the same now as $(x^{\frac{1}{2}})^7$.

This shows us our next rule - $x^{\frac{1}{p}} = \sqrt[p]{x}$.

So the above equation can be written as $(\sqrt{x})^7$.

Now let's also consider $x^0$. If you think about writing this as $x^{2-2}$, this equals $\frac{x^2}{x^2} = 1$.

Therefore, $x^0=1$.

Now we can look at $x^{-1} = x^{4-5} = \frac{x^4}{x^5}$. So from this we get $\frac{x\times x\times x\times x}{x\times x\times x\times x\times x} = \frac{1}{x}$.

This means that $x^{-1} = \frac{1}{x}$.

We have the rule then that $x^{-p} = \frac{1}{x^p}$.

\subsubsection*{Examples of Negative Indices}
\ex $2^{-3} = $

\ex $3^{-4} = $

\ex $5^{-2} = $

\ex $\left(\frac{1}{4}\right)^{-2} = $

\ex $\left(\frac{2}{3}\right)^{-3} = $

\subsubsection*{Examples of Positive Rational Indices}

\ex $36{\frac{1}{2}} = $

\ex $81^{\frac{1}{4}} = $

\ex $\left(\frac{1}{8}\right)^{\frac{1}{3}} = $

\ex $25^{\frac{3}{2}} = $

\ex $\left(\frac{8}{27}\right)^{\frac{2}{3}} = $

\subsubsection*{Examples of Negative Rational Indices}

\ex $8^{-\frac{1}{3}} = $

\ex $16^{-\frac{3}{4}} = $

\ex $4^{-\frac{5}{2}} = $

\ex $\left(\frac{36}{49}\right)^{-\frac{1}{2}} = $

\ex $\left(\frac{10000}{16}\right)^{-\frac{5}{4}} = $

\subsubsection*{More Complicated Examples}

\begin{example}
    $2^3\times 8^{-\frac{5}{3}} \times \frac{1}{\sqrt{2}} = 2^k$. Find $k$.

    For this problem, you want to write everything in terms of $2$ to the power of something. We can rewrite this equation as 
    \[ 2^3 \times (2^3)^{-\frac{5}{3}} \times 2^{-\frac{1}{2}} \]

    So this can be rewritten as $2^3\times 2^{-5} \times 2^{-\frac{1}{2}}$, and using laws of indices, we can see that this is equivalent to 
    $2^{-\frac{5}{2}}$. So $k = -\frac{5}{2}$.
\end{example}
\ex Write $\frac{x^2y^5}{\sqrt{x}}\div \frac{x^{\frac{3}{2}}}{y^7}$ as a product of powers of $x$ and $y$.

\subsubsection*{Examples of Simplifying Expressions}

\ex $5a^3b^2c \times 6a^8bc^{-3} = $

\ex $(60a^4b^2c) \div (12a^8b^5c^{-4}) = $

\ex $\frac{(3x)^3\times (2x^3)^4}{(6x^8)^2} = $

\subsubsection*{Write in the form of $2^k$}
\smallbreak
\begin{example}
    Write $\frac{\sqrt{2}}{4^3}$ in the form $2^k$.

    We can rewrite $\sqrt{2} = 2^{\frac{1}{2}}$ and $4^3 = (2^2)^3 = 2^6$. So now we have $\frac{2^{\frac{1}{2}}}{2^6} = 2^{-\frac{11}{2}}$.
\end{example}
\ex Write $8^4 \times \frac{2}{\sqrt[3]{16}}$ in terms of $2^k$.

\subsubsection*{Write in the form of $3^k$}
\smallbreak
\begin{example}
    Write $\sqrt[3]{3} \times \sqrt[3]{9}$ in terms of $3^k$.

    We can rewrite this as $3^{\frac{1}{3}} \times (3^2)^{\frac{1}{3}} = 3^{\frac{1}{3}}\times 3^{\frac{2}{3}} = 3^1$.
\end{example}

\ex Write $\frac{\sqrt[5]{27}}{\sqrt{3}}\times 81$ in terms of $3^k$.

\subsubsection*{Write in the form of $4^k$}
\smallbreak
\begin{example}
    Write $\frac{16}{\sqrt[4]{5}}$ in terms of $4^k$.

    This can be rewritten as $\frac{4^2}{4^{\frac{1}{5}}}$, so this is equivalent to $4^{\frac{9}{5}}$.
\end{example}
\ex Rewrite $2\times \sqrt[3]{16} \times \sqrt[5]{64}$ in terms of $4^k$.

\subsubsection*{Write in the form $5^k$}
\smallbreak
\begin{example}
    Rewrite $\frac{125}{\sqrt[3]{25}}\times \sqrt{5}$ in terms of $5^k$.

    We first start off with $\frac{5^3}{(5^2)^{\frac{1}{3}}}\times 5^{\frac{1}{2}}$. 

    This is equal to $5^{\frac{7}{3}}\times 5^{\frac{1}{2}} = 5^{\frac{17}{6}}$.
\end{example}
\ex Rewrite $\frac{\sqrt[3]{50}}{\sqrt{625}}\times \sqrt[3]{12.5}$ in terms of $5^k$.

\section{Surds}
\subsection*{Simplifying Surds}
\subsection*{Rationalising the Denominator}
\subsection*{Problem Solving}
\section{Quadratics}
\subsection*{The Difference of Two Squares}
\subsection*{Factorising Quadratics}
\subsection*{Sketching Quadratics from Factorised Form}
\subsection*{Completing the Square}
\subsection*{Sketching Quadratics from Completed Square Form}
\subsection*{Solving Quadratics}
\subsection*{Using the Discriminant}
\subsection*{Using the Quadratic Formula}
\subsection*{Sketching Quadratics Using the Quadratic Formula}
\subsection*{Sketching Quadratic Using a Calculator}
\subsection*{Using Quadratic Methods for Solving}
\section{Simultaneous Equations}
\subsection*{The Elimination Method}
\subsection*{The Substitution Method}
\subsection*{Further Simultaneous Equations}
\section{Inequalities}
\subsection*{Introducing Inequalities, Set Notation and Interval Notation}
\subsection*{Linear Inequalities}
\subsection*{Quadratic Inequalities}
\subsection*{Discriminant Inequalities}
\subsection*{More Inequalities}
\subsection*{Double and Triple Inequalities}
\subsection*{Representing Inequalities Graphically}
\section{Polynomials \& Rational Expressions}
\subsection*{Introducing Polynomials}
\subsection*{Polynomial Division}
\subsection*{The Factor Theorem}
\subsection*{Simplifying Algebraic Fractions}
\subsection*{Adding and Subtracting Algebraic Fractions}
\subsection*{Simplifying using Polynomial Division}
\section{Graphs \& Proportion}
\subsection*{Sketching Polynomials}
\subsection*{Sketching Cubics from Factorised Form}
\subsection*{Sketching Quartics from Factorised Form}
\subsection*{The Modulus Function}
\subsection*{Reciprocal Graphs}
\subsection*{Finding Points of Intersection}
\subsection*{Direct and Inverse Proportion}
\section{Functions}
\subsection*{Introducing Functions}
\subsection*{The Domain and Range of a Function}
\subsection*{One-to-One, Many-to-One, One-to-Many, Many-to-Many}
\subsection*{Restricting the Domain}
\subsection*{Even \& Odd Functions}
\subsection*{Composite Functions}
\subsection*{Inverse Functions}
\subsection*{Set Notation}
\subsection*{Interval Notation}
\section{Graph Transformations}
\subsection*{Introducing Transformations}
\subsection*{Translations}
\subsection*{Stretches}
\subsection*{Reflections}
\subsection*{Examples of Transformations}
\subsection*{Combining Transformations}
\section{Algebraic Fractions}
\subsection*{Algebraic Fractions}
\subsection*{Partial Fractions}
\subsection*{Partial Fractions: Repeated Factors}
\subsection*{Partial Fractions: Extensions}
\subsection*{Partial Fractions: with Binomial Expansion \& Integration}
\section{Modelling}
\end{document}