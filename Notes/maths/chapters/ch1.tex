\documentclass[../maths.tex]{subfiles}
\graphicspath{{\subfix{../figures/}}}
\begin{document}
\chapter{Proof}
\subsection*{Introduction to Proof}
\subsubsection*{Introduction to Proof}
In this section we will working with these topics:
\begin{itemize}
    \item Consequence and Equivalence
    \item Proof by Exhaustion
    \item Proof by Deduction
    \item Disproof by Counter-Example
    \item Proof by Contradiction
\end{itemize}

\subsubsection*{Introducing Consequence and Equivalence}

When we look at consequence, we essentially say that ``$a$ implies $b$'', or:
\[a\rightarrow b\]

If the arrow points the other way, we say that ``$b$ implies $a$'', or:
\[a\leftarrow b\]

Let's say that statement $a$ states that $p$ is a prime number $>2$.

Let's say that statement $b$ states that $p$ is an odd number.

For these statements, we see that $a$ does imply $b$, so we can write that
\[a\rightarrow b\]
The other way however does not work, since because $p$ is an odd number, it does not imply that $p$ is a prime number.

However, if this was true, we can write that $a$ implies $b$ and $b$ implies $a$, or:
\[a\leftrightarrow b\]
which is sometimes written as ``$a$ if and only $b$'' or ``$a$ iff $b$''.

Let's show a logical equivalence. Let $a$ be the statement $n^2$ is odd and $b$ be the statement $n$ is odd.

We know that when $n^2$ is odd, that $n$ is odd when we list out the odd squared numbers. We can see the converse is true as well in this statement since every time a number $n$ is squared, we are given an odd number, therefore:
\[a\leftrightarrow b\]

\subsubsection*{Consequence and Equivalence Examples}
Let's give some examples where we determine whether one of the statements implies the other statement.

Given that an object is a cube and an object has six faces. If an object is a cube, it definitely has six faces. Therefore, 
The object is a cube $\implies$ The the object has six faces. The opposite is not true, because it can be a cuboid, for example.

Given $x=29$ and $x>10$, then $x=29 \implies x>10$. The opposite is not true, since there are many more values where $x>10$.

Given $x^3=x$ and $x=-1$. We need to find the solutions of $x^3=x$ first. By subtracting and obtaining $x^3-x=0$, we can factor this to $x(x^2-1)=0$. Then we have $x(x-1)(x+1)=0$, and the solution of this equation are $0,1$, and $-1$.
Therefore $x^3=x$ does not imply $x=-1$. However, going the other way, $x=-1 \implies x^3=x$.

Given $n$ is a positive integer greater than 1, we are given the statements that $n$ is a prime number and $n$ has exactly two factors.
$n$ always has two factors if it is prime, then $n$ is a prime number $\implies$ $n$ has exactly two factors. If $n$ has exactly two factors, then it must be prime, so we can see that 
$n$ has exactly two factors $\implies$ $n$ is a prime number, so $n$ is a prime number $\leftrightarrow$ $n$ has exactly two factors.
\subsection*{Proof by Exhaustion}
\subsubsection*{Introducing Proof by Exhaustion}
Proof by Exhaustion is trying all possible variations to prove a statement is true.

We are going to prove a conjecture, which is a statement that we believe to be correct but needs to be proved.

The conjecture is ``97 is a prime number''. To show this, we need to show that 97 has two factors, 1 and itself.

Let's try some numbers. 

$97\div 2$ is 48.5, clearly 2 is not a factor of 97. $97\div 3$ is $32.\overline{3}$. Therefore, 3 is not a factor either.
We wouldn't need to try 4 since 2 already isn't a factor. Let's try 5. $97\div 5$ is 19.4, so 5 is also not a factor of 97. We don't need to try 6 since 3 and 2 are both not factors of 97.
Now we try 7. $97\div 7=13.85\dots$, so 7 is not a factor either. It's clear we are just working through all the prime numbers now.

We don't need to go further than this because when we square root 97, we will get a number a little less than 10. Because the square root of 97 is a little less than 10, 
when we go beyond 10, if we are to find any factor above 10, then there would have to have been a factor less than 10 to multiply with to make 97.

In other words, because there were no factors below the square root of 97, this implies there are no factors larger than the square root of 97, indicating that 97 is a prime number.

\subsubsection*{Proof by Exhaustion Examples}
Let's do three examples.

\begin{itemize}
    \item No square number ends in an 8
    
    This problem looks at squaring each unit digit. If a number ends in a 1, the square one gets will end in a 1 as well.
    If the number ends in a 2, and I square it, then this number will end with a 4. If the number ends with a 3, the number will end with a 9.
    If the number ends with a 4, the squared number will end with a 6. If the number ends with a 5, the squared number will end with a 5. 
    If the number ends with a 6, the squared number will end with a 6. If the number ends with a 7, the squared number will end with a 9.
    If the number ends with a 8, the squared number will end with a 4. If the number ends with a 9, the squared number will end with a 1.
    If the number ends with a 0, the squared number will end with up with a 0.
    
    As we can see, there are no numbers that can have a unit digit of 8.
\end{itemize}

\begin{itemize}
    \item If $n$ is an integer and $2\leq n\leq 7$, then $A=n^2+2$ is not divisible by 4.
    
    To show this, lets consider all values of $n$. 

    $\begin{array}{|c|c|c|}
    \hline 
    n & n^2+2 & \text{divisible by 4?} \\
    \hline 
    2 & 6 & \text{no} \\
    3 & 11 & \text{no} \\
    4 & 18 & \text{no} \\
    5 & 27 & \text{no} \\ 
    6 & 38 & \text{no} \\
    7 & 51 & \text{no} \\
    \hline
    \end{array}$
    
    so in none of these cases, none of these values of $A$ are divisible by 4 and we have gone through every single part of this and show that this is never divisible by 4.
\end{itemize}

\begin{itemize}
    \item Every integer that is a perfect cube is either a multiple of 9, is 1 more than a multiple of 9, or is 1 less than a multiple of 9.
    
    The first statement says that $n=3k$, that the number is a multiple of 3, or $n=3k-1$, one less than a multiple of three, or $n=3k-2$, a number is two less than a multiple of 3.

    Let's start by cubing. $n^3=27k^3$. Because 27 is a multiple of 9, $k$ is an integer and $n^3$ is a multiple of 9. 

    Let's look at $n=3k-1$. $n^3=27k^3-27k^2+9k-1$. If we factor a 9 out, we get $9(3k^3-3k^2+k)-1$. This is clearly 1 less than a multiple of 9.

    Now let's look at $n=3k-2$. $n^3=27k^3-54k^2+36k-8$. If I write the 8 as a $-9+1$, we can factor out the $9$ and get $9(3k^3-6k^2+4k-1)+1$, or one more than a multiple of 9.
\end{itemize}
\subsection*{Proof by Deduction}
\subsubsection*{Introduction Proof by Deduction}
Proof by deduction is all about going through a logical sequence of arguments where you will start with something you know to be true, and subsequently, the next thing is true, etc, until the conjecture is true.

Conjecture: ``The sum of any two consecutive odd numbers is a multiple of 4.''

We can start with an odd number $2n+1$, since $2n$ is always an even number, so adding 1 will make it odd. If we are looking for the next consecutive odd number, then we can see this as $2n+3$.
The conjecture talks about the sum of the consecutive odd numbers. Adding them together, we get $4n+4$, which factors to $4(n+1)$, which is always a multiple of 4.


\subsubsection*{Proof by Deduction Example}
\begin{example}
    For any four consecutive integers, the difference between the prodcut of the last two and the product of the first two of these numbers is equal to their sum.

    Let's first label four consecutive integers as $n, n+1, n+2, n+3$. We have to find the product of the last two and the product of the last two and to find the difference between the two things.

    Therefore, we are finding $(n+2)(n+3)-n(n+1)$. Expanding this, we get $n^2+5n+6-n^2-n$. Simplifying, we get $4n+6$. 

    Adding the consecutive integers, we have $n+n+1+n+2+n+3=4n+6$. We have shown that the difference between the products of the last two and the first two is the same as the sum of the four numbers.
\end{example}
\pagebreak
\begin{example}
    $k^3-k$ is divisible by 6 for all integers $k>1$.

    First we can factor $k^3-k$ to $k(k^2-1)$. We can factor this further as $k(k-1)(k+1)$. Now if we write this in a slightly different order, as $(k-1)(k)(k+1)$. What we have here is the product of three consecutive integers. At least one of these integers therefore will be an even integer, so $k^3-k$ is divisible by 2.

    Now because we have three consecutive integers, precisely one of them will be a multiple of 3 because since $k>1$, there will always be a number that is divisible by 3 when consecutively counting. Therefore $k^3-k$ is also divisible by 3.

    Because $k^3-k$ is divisible by 2 and 3, then it is divisible by 6.
\end{example}
\subsection*{Disprove by Counter-Example}
\subsubsection*{Introducing Disproof by Counter Example}
Sometimes we are asked to find a single example where a conjecture fails.

Let's start with the conjecture ``The value of $n^2+n+11$ is prime for all integers $n>0$''

When $n=11$, we can see that $11^2+11+11$ which is equal to $11(13)$ which is evidently not prime.

\subsubsection*{Disproof by Counter Example Examples}
\begin{example}
    If $x^2>x$, then $x>1$. 

    When we plug in $x=-2$, we can see that $4>-2$, but $-2$ is not greater than 1.
\end{example}

\begin{example}
    If $n$ is prime, then $n^2+n+1$ is prime.

    When we plug in $n=7$, we get $n^2+n+1=57$, which is not prime, so this conjecture fails.
\end{example}

\begin{example}
    The sum of $n$ consecutive integers is divisible by $n$ (where $n$ is a positive integer).

    We can easily disprove this in one example. $1+2+3+4=10$, which is not divisible by 4.
\end{example}
\end{document}