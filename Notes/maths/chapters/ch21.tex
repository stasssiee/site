\documentclass[../maths.tex]{subfiles}
\graphicspath{{\subfix{../figures/}}}
\begin{document}
\chapter{Complex Numbers}
\section{Introducing Complex Numbers}
\subsection*{Introducing Complex Numbers}
\subsection*{Solving Polynomial Equations with Real Coefficients}
\section{Working with Complex Numbers}
\subsection*{Real and Imaginary Parts}
\subsection*{Working with Complex Numbers}
\section{Complex Conjugates}
\subsection*{The Complex Conjugate}
\subsection*{Complex Conjugate Pairs}
\section{Introducing the Argand Diagram}
\section{Introducing Modulus-Argument Form}
\subsection*{Introducing the Modulus and Argument}
\subsection*{Modulus-Argument Form}
\section{Multiply and Divide in Modulus-Argument Form}
\section{Loci with Argand Diagrams}
\subsection*{Circles}
\subsection*{Perpendicular Bisectors}
\subsection*{Loci Problems with Circles \& Perpendicular Bisectors}
\subsection*{Half-Lines}
\subsection*{Loci Problems with Circles, Perpendicular Bisectors and Half-Lines}
\section{De Moivre's Theorem}
\subsection*{Introducing De Moivre's Theorem}
\subsection*{Expansions of \texorpdfstring{$\cos(n\theta)$}{cos(n*theta)} and \texorpdfstring{$\sin(n\theta)$}{sin(n*theta)}}
\section{\texorpdfstring{$z=re^{(i\theta)}$}{z=re to the power of i*theta}}
\subsection*{Introducing \texorpdfstring{$z=re^{i\theta}$}{z=re to the power of i*theta}}
\subsection*{Summing Series}
\section{ nth Roots of Unity}
\subsection*{The nth Roots of Unity}
\subsection*{The nth Roots of any Complex Number}
\section{ Geometrical Problems}
\end{document}