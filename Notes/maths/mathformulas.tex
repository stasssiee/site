\documentclass[10pt,a4paper,oneside]{book}

\title{Math Formula Sheet}
\author{anastasia}

\usepackage[utf8]{inputenc}
\usepackage[margin=1.0in]{geometry}
\usepackage{amsmath}
\usepackage{amsfonts}
\usepackage{amssymb}
\usepackage{enumitem}                       % custom enum labels
\usepackage{parskip}                        % add vertical paragraph space
\usepackage{tocloft}						% modify toc position
\usepackage{xr}								% cross-references
\usepackage{mathtools}						% Aboxed
\usepackage{empheq}							% box multiple lines
\usepackage{upgreek}
\usepackage{gensymb}
\usepackage{chemformula}
\usepackage{esint}							% oiint
\usepackage{cancel}
\usepackage{tikz}
\usetikzlibrary{calc}
\usepackage{asymptote}
\usepackage[framemethod=TikZ]{mdframed}     % graphics and framed envs
\usepackage[hang,flushmargin]{footmisc}		% remove footnote indentation
\usepackage[hyperfootnotes=false,
					hidelinks]{hyperref}	% create clickable table of contents
\usepackage{cancel}

\newcommand{\tarc}{\mbox{\large$\frown$}}
\newcommand{\arc}[1]{\stackrel{\tarc}{#1}}
%\newcommand{\degree}{^{\circ}}
\newcommand{\blank}{\_\_\_\_\_\_}

\DeclareMathOperator\cis{cis}
\DeclareMathOperator\Arg{Arg}


\renewcommand{\familydefault}{\sfdefault}  	% sans serifs text
\setlength{\parindent}{0pt}                	% no paragraph indentation

% region TITLES
%\setbox0=\hbox{\Huge{\textbf{\textsf{\courseid: }}}}
\setlength{\cftbeforetoctitleskip}{0em}
\setlength{\cftaftertoctitleskip}{1em}
%\renewcommand{\contentsname}{\hangindent=\wd0 \strut \courseid: \coursetitle \\ \medskip {\professor, \campus, \semester}}

% region COMMANDS
\newcommand{\ds}{\displaystyle}
\newcommand{\pfn}[1]{\textrm{#1}}	% enables new functions
\newcommand{\mbf}[1]{\mathbf{#1}}	% mathbf
\newcommand{\C}{\mathbb{C}}			% fancy C
\newcommand{\R}{\mathbb{R}}			% fancy R
\newcommand{\Q}{\mathbb{Q}}			% fancy Q
\newcommand{\Z}{\mathbb{Z}}			% fancy Z
\newcommand{\N}{\mathbb{N}}			% fancy N
\newcommand{\dd}{\mathrm{d}}
\newcommand{\from}{\leftarrow}
\newcommand{\qed}{$\square$}
\newcommand{\ex}{\textit{Exercise }}
% endregion

\let\footnoterule\relax
\newcommand\blfootnote[1]{%
  \begingroup
  \renewcommand\thefootnote{}\footnote{#1}%
  \addtocounter{footnote}{-1}%
  \endgroup
}

\usepackage{titlesec}
\titleformat{\chapter}{\Huge\normalfont\bfseries}{\thechapter}{1em}{}
\titlespacing*{\chapter}{0em}{0em}{2em}
% endregion

% region ENVIRONMENTS
\newcounter{theo}[chapter]\setcounter{theo}{0}
\newcommand{\numTheo}{\arabic{chapter}.\arabic{theo}}
\newcommand{\mdftheo}[3]{
	\mdfsetup{
		frametitle={
			\tikz[baseline=(current bounding box.east),outer sep=0pt]
			\node[anchor=east,rectangle,fill=#3]
			{\ifstrempty{#2}{\strut #1~\numTheo}{\strut #1~\numTheo:~#2}};
		},
		innertopmargin=4pt,linecolor=#3,linewidth=2pt,
		frametitleaboveskip=\dimexpr-\ht\strutbox\relax,
		skipabove=11pt,skipbelow=0pt
	}
}
\newcommand{\mdfnontheo}[3]{
	\mdfsetup{
		frametitle={
			\tikz[baseline=(current bounding box.east),outer sep=0pt]
			\node[anchor=east,rectangle,fill=#3]
			{\ifstrempty{#2}{\strut #1}{\strut #1:~#2}};
		},
		innertopmargin=4pt,linecolor=#3,linewidth=2pt,
		frametitleaboveskip=\dimexpr-\ht\strutbox\relax,
		skipabove=11pt,skipbelow=0pt
	}
}
\newcommand{\mdfproof}[1]{
	\mdfsetup{
		skipabove=11pt,skipbelow=0pt,
		innertopmargin=4pt,innerbottommargin=4pt,
		topline=false,rightline=false,
		linecolor=#1,linewidth=2pt
	}
}


\newenvironment{theorem}[1][]{
	\refstepcounter{theo}
	\mdftheo{Theorem}{#1}{red!25}
	\begin{mdframed}[]\relax
}{\end{mdframed}}

\newenvironment{lemma}[1][]{
	\refstepcounter{theo}
	\mdftheo{Lemma}{#1}{red!15}
	\begin{mdframed}[]\relax
}{\end{mdframed}}

\newenvironment{corollary}[1][]{
	\refstepcounter{theo}
	\mdftheo{Corollary}{#1}{red!15}
	\begin{mdframed}[]\relax
}{\end{mdframed}}

\newenvironment{definition}[1][]{
	\mdfnontheo{Definition}{#1}{blue!20}
	\begin{mdframed}[]\relax
}{\end{mdframed}}

\newenvironment{exercise}[1][]{
	\mdfproof{black!15}
	\textit{Exercise. }
}

\newenvironment{proof}[1][]{
	\mdfproof{black!15}
	\begin{mdframed}[]\relax
\textit{Proof. }}{\qed \end{mdframed}}

\newenvironment{claim}[1][]{
	\mdfproof{black!15}
	\begin{mdframed}[]\relax
\textit{Claim. }}{\end{mdframed}}

\newenvironment{example}[1][]{
	\mdfnontheo{Example}{#1}{yellow!40}
	\begin{mdframed}[]\relax
}{\end{mdframed}}

\newenvironment{summary}[1][]{
	\mdfnontheo{Summary}{#1}{green!70!black!30}
	\begin{mdframed}[]\relax
}{\end{mdframed}}
% endregion

\let\oldsum\sum
\renewcommand{\sum}{\oldsum\limits}

\let\oldlim\lim
\renewcommand{\lim}{\oldlim\limits}

\let\oldprod\prod
\renewcommand{\prod}{\oldprod\limits}

\let\oldbigcup\bigcup
\renewcommand{\bigcup}{\oldbigcup\limits}

\let\oldbigcap\bigcap
\renewcommand{\bigcap}{\oldbigcap\limits}

\let\oldmax\max
\renewcommand{\max}{\oldmax\limits}

\let\oldmin\min
\renewcommand{\min}{\oldmin\limits}

\let\oldsup\sup
\renewcommand{\sup}{\oldsup\limits}

\let\oldinf\inf
\renewcommand{\inf}{\oldinf\limits}
\graphicspath{{figures/}}

\begin{document}
\maketitle
\section*{Shapes}
Area of a Triangle: $\frac{1}{2} \times$ base $\times$ height 

Area of a Parallelogram: base $\times$ height 

Area of a Rectangle: length $\times$ width 

Area of a Trapezoid: $\frac{1}{2}$(sum of parallel sides)$\times$ height 

Circumference \& Area: Circle: $c=2\pi r, A=\pi r^2$

Cuboid Surface Area: $SA=2xy+2xz+2yz$, where $x$, $y$, and $z$ are side lengths 

Cuboid Volume: $V=xyz$, where $x$, $y$, and $z$ are side lengths 

Cylinder Surface Area: $SA=2\pi rh+2\pi r^2$. Note: Curved Part: $2\pi rh$

Cylinder Volume: $V=\pi r^2 h$

Cone Surface Area: $SA=\pi r l +\pi r^2$. Note Curved part: $\pi r l$, where $l$ is slant length 

Cone Volume: $V=\frac{1}{3}\pi r^2 h$

Sphere Surface Area: $SA=4\pi r^2$. Note: Hemisphere = $2\pi r^2+\pi r^2 = 3\pi r^2$

Sphere Volume: $V=\frac{4}{3}\pi r^3$. Note: Hemisphere = $\frac{2}{3}\pi r^3$

Prism Volume: $V$= Area of cross section $\times$ height 

Pyramid Volume: $V=\frac{1}{3}\times$ base area $\times h$

\section*{Indices}
Multiplication:
\begin{itemize}
    \item $x^a \times x^b = x^{a+b}$
    \item $(x^a)^b=x^{ab}$
    \item $(cx^ay^b)^d=c^dx^{ad}y^{bd}$
\end{itemize}

Division: $x^a\div x^b = \frac{x^a}{x^b}=x^{a-b}$

Negative Powers: $x^{-n}=\frac{1}{x^n}$

Fractions:
\begin{itemize}
    \item $\left(\frac{x}{y}\right)^n = \frac{x^n}{y^n}$
    \item $\left(\frac{x}{y}\right)^{-n}=\frac{y^n}{x^n}$
\end{itemize}

Rational Powers: $a^{\frac{n}{m}} = (a^{\frac{1}{m}})^n = (\sqrt[m]{a})^n = (a^n)^{\frac{1}{m}}=\sqrt[m]{a^n}$

\section*{Series}

Arithmetic sequence: $n$th term - $u_n = a+(n-1)d$ where $a$ is the first term and $d$ is the common difference.

Arithmetic sequence: sum of $n$ terms - $S_n=\frac{n}{2}[2a+(n-1)d]=\frac{n}{2}(a+l)$ where $a$ is the first term, $d$ is the common difference and $l$ is the last term.

Geometric sequence: $n$th term - $u_n=ar^{n-1}$ where $a$ is the first term and $r$ is the common ratio.

Geometric sequence: sum of $n$ terms - $S_n=\frac{a(1-r^n)}{1-r}=\frac{a(r^n-1)}{r-1},r\neq 1$ where $a$ is the first term and $r$ is the common ratio 

Geometric sequence: sum to infinity - $S_{\infty}=\frac{a}{1-r}, |r|<1$, where $a$ is the first term and $r$ is the common ratio 

Compound interest: $FV=PV\left(1+\frac{r}{100k}\right)^{kt}$ where $FV$ is the future value, $PV$ is the present value, $t$ is the no. of years, $r$ is the nominal annual interest rate, and $k$ is the no. of compounding periods per year 

Binomial Theorem: integer powers - $(a+b)^n=a^n+\binom{n}{1}a^{n-1}b+\cdots+\binom{n}{r}a^{n-r}b^r+\dots+b^n$

Binomial Theorem: Fractional \& Negative powers = $(a+b)^n=a^n\left(1+n(\frac{b}{a})+\frac{n(n-1)}{2!}\frac{b}{a}^2+\cdots\right)$

Binomial Coefficient: $\binom{n}{r}=nc_r=\frac{n!}{(n-r)!r!}$

\section*{Geometry}
Straight Line: Equation (gradient means slope), Parallel $\implies$ same slope. Perpendicular $\implies$ ``flip fraction and change the sign'' (slopes multiply to make $-1$)
\begin{itemize}
    \item Slope intercept form: $y=mx+c$
    \item General form: $ax+by+d=0$
    \item Point slope form: $y-y_1=m(x-x_1)$
\end{itemize}

Straight Line: Gradient - $m=\frac{y_2-y_1}{x_2-x_1}$

Distance between 2 points $(x_1,y_1),(x_2,y_2)$: $\sqrt{(x_2-x_1)^2+(y_2-y_1)^2}$

Coordinates of midpoint of $(x_1,y_1),(x_2,y_2)$: $\left(\frac{x_1+x_2}{2},\frac{y_1+y_2}{2}\right)$

Circles: $(x-a)^2+(y-b)^2=r^2$, where the centre is $(a,b)$ the radius is $r$.

\section*{Quadratics}
Quadratic Function: Solutions to $ax^2+bx+c=0$ - $x=\frac{-b\pm \sqrt{b^2-4ac}}{2a}, a\neq 0$

Quadratic Function: Axis of Symmetry - $f(x)=x^2+bx+c \implies x=-\frac{b}{2a}$

Quadratic Function: Discriminant - $\Delta = b^2-4ac$
\begin{itemize}
    \item $>0$ (2 real distinct roots)
    \item = 0 (2 real repeated/double roots)
    \item $<0$ (no real roots)
\end{itemize}

Completing The Square $ax^2\pm bx+c=0$ - $a\left(x\pm\frac{b}{2a}\right)^2+c-\frac{b^2}{4a}$

Max/Min Value: $c-\frac{b^2}{4a}$

Exponential and Logarithmic Functions
\begin{itemize}
    \item $a^x=e^{x\ln a}$
    \item $\log_a a^x = x = a^{\log_a x}$
\end{itemize}
where, $a,x>0, a\neq 1$

Exponential and Logarithm Rules 
\begin{itemize}
    \item $c\log_a b = \log_a b^c$
    \item $\log_a b = c \implies a^c=b, a,b>0, a\neq 1$
    \item $\log_a b + \log_a c = \log_a bc$
    \item $\log_a b - \log_a c = \log_a \frac{b}{c}$
    \item $\log_a b = \frac{\log_c b}{\log_c a}$
    \item Solving a power of $x$: log both sides if 2 terms or use substitution if 3 terms 
    \item Solving an exponential: $\ln$ both sides 
    \item Solving a logarithm: raise $e$ both side or write as $\log_e$ as proceed as usual for $\log$
\end{itemize}

\section*{Trigonometry}
Sine Rule: 
\begin{itemize}
    \item Finding a side: $\frac{a}{\sin A}=\frac{b}{\sin B}=\frac{c}{\sin C}$
    \item Finding an angle: $\frac{\sin A}{a}=\frac{\sin B}{b}=\frac{\sin C}{c}$
\end{itemize}

Cosine Rule:
\begin{itemize}
    \item Finding a side: $a^2=b^2+c^2-2bc\cos A$
    \item Finding an angle: $A=\cos^{-1}\left(\frac{b^2+c^2-a^2}{2bc}\right)$
\end{itemize}

Area of Triangle: $\frac{1}{2}ab\sin C$

Degrees $\rightleftarrows$ radians
\begin{itemize}
    \item Degrees to radians: $\times \frac{\pi}{180}$
    \item Radians to degrees: $\times \frac{180}{\pi}$
\end{itemize}

Length of an arc: $\frac{\theta}{360}\times 2\pi r$ (degrees) or $r\theta$ (radians)

Area of a sector: $\frac{\theta}{360}\times \pi r^2$ (degrees) or $\frac{1}{2}r^2\theta$ (radians)

Small Angle Approximations
\begin{itemize}
    \item $\sin\theta \approx \theta$
    \item $\cos\theta \approx 1-\frac{\theta^2}{2}$
    \item $\tan\theta \approx \theta$
\end{itemize}

Pythagorean identity 1: $\sin^2 x + \cos^2 x = 1$

Pythagorean identity 2: $1+\tan^2 x = \sec^2 x$

Pythagorean identity 3: $1+\cot^2 x = \csc^2 x$

Confunction
\begin{itemize}
    \item $\cos x = \sin(90-x)$
    \item $\sin x = \cos(90-x)$
\end{itemize}

Identity of $\tan x$: $\tan{x}=\frac{\sin x}{\cos x}$

Reciprocal:
\begin{itemize}
    \item $\sec{x}=\frac{1}{\cos x}$
    \item $\csc x = \frac{1}{\sin x}$
    \item $\cot x = \frac{1}{\tan x}$
\end{itemize}

Double Angle 
\begin{itemize}
    \item $\sin 2x = 2\sin x\cos x$
    \item $\cos 2x = \cos^2 x-\sin^2 x = 2\cos^2x -1 \implies \cos^2x = \frac{\cos 2x+1}{2} = 1-2\sin^2\theta \implies \sin^2 x = \frac{1-\cos 2x}{2}$
    \item $\tan 2x = \frac{2\tan x}{1-\tan^2 x}$
\end{itemize}

Half Angle 
\begin{itemize}
    \item $\sin \frac{x}{2}=\pm \sqrt{\frac{1-\cos x}{2}}$
    \item $\cos \frac{x}{2}=\pm \sqrt{\frac{\cos x+1}{2}}$
    \item $\tan \frac{x}{2}=\pm \sqrt{\frac{1-\cos x}{1+\cos x}}=\frac{1-\cos x}{\sin x}=\frac{\sin x}{1+\cos x}$
\end{itemize}

Compound Angle 
\begin{itemize}
    \item $\sin(A\pm B)=\sin A\cos B\pm \cos A\sin B$
    \item $\cos(A\pm B)=\cos A\cos B\mp \sin A\sin B$
    \item $\tan(A\pm B)=\frac{\tan A\pm \tan B}{1\mp \tan A\tan B}$
\end{itemize}

Factor Formula: sum to product (Note: For product $t$ to sum rearrange andd let $\frac{A+B}{2}$ and $\frac{A-B}{2}$ equal your given angles and solve for $A$ and $B$ simultaneously)
\begin{itemize}
    \item $\sin A+\sin B\equiv 2\sin\left(\frac{A+B}{2}\right)\cos\left(\frac{A-B}{2}\right)$
    \item $\sin A-\sin B\equiv 2\cos\left(\frac{A+B}{2}\right)\sin\left(\frac{A-B}{2}\right)$
    \item $\cos A+\cos B\equiv 2\cos\left(\frac{A+B}{2}\right)\cos\left(\frac{A-B}{2}\right)$
    \item $\cos A-\cos B \equiv -2\sin\left(\frac{A+B}{2}\right)\sin\left(\frac{A-B}{2}\right)$
\end{itemize}

If $t=\tan \frac{1}{2}x \implies \sin x = \frac{2t}{1+t^2}$ and $\cos x = \frac{1-t^2}{1+t^2}$

\section*{Vectors}
Notations: vector = $\textbf{a}$, $\underline{a}$, $\overrightarrow{OA}$. Distance = $OA$

Vector Form: $a\textbf{i}+b\textbf{j}+c\textbf{k}\equiv \begin{pmatrix}
    a\\b\\c  
\end{pmatrix}$

Properties:
\begin{itemize}
    \item $\begin{pmatrix}
        a\\b\\c
    \end{pmatrix}\pm \begin{pmatrix}
        d\\e\\f 
    \end{pmatrix}=\begin{pmatrix}
        a \pm d \\
        b \pm e \\
        c \pm f 
    \end{pmatrix}$

    \item $\lambda\begin{pmatrix}
        a\\b\\c 
    \end{pmatrix}=\begin{pmatrix}
        \lambda a\\\lambda b\\\lambda c 
    \end{pmatrix}$

    \item $\begin{pmatrix}
        a\\b\\c
    \end{pmatrix}\cdot \begin{pmatrix}
        d\\e\\f
    \end{pmatrix}=ad+be+cf$
\end{itemize}

Magnitude of a vector: $\left|\begin{pmatrix}
    a\\b\\c 
\end{pmatrix}\right| = \sqrt{a^2+b^2+c^2}$

Unit Vector: Unit vector of $\begin{pmatrix}
    a\\b\\c 
\end{pmatrix}=\frac{1}{\sqrt{a^2+b^2+c^2}}\begin{pmatrix}
    a\\b\\c
\end{pmatrix}$

Parallel means vectors are a multiple of each other.

Perpendicular means scalar product equals zero.

Midpoint of $\begin{pmatrix}
    a\\b\\c
\end{pmatrix}$ and $\begin{pmatrix}
    d\\e\\f
\end{pmatrix}$: $\left(\frac{a+d}{2},\frac{b+e}{2},\frac{c+f}{2}\right)$

Scalar Product: $\begin{pmatrix}
    a\\b\\c 
\end{pmatrix}\cdot \begin{pmatrix}
    d\\e\\f
\end{pmatrix}=\left|\begin{pmatrix}
    a\\b\\c
\end{pmatrix}\right|\left|\begin{pmatrix}
    d\\e\\f
\end{pmatrix}\right|\cos\theta$, where $\theta$ is the angle between $\begin{pmatrix}
    a\\b\\c
\end{pmatrix}$ and $\begin{pmatrix}
    d\\e\\f
\end{pmatrix}$

Vector Product: Note: $\theta$ is the angle between $\begin{pmatrix}
    a\\b\\c
\end{pmatrix}$ and $\begin{pmatrix}
    d\\e\\f
\end{pmatrix}$:

$\begin{pmatrix}
    a\\b\\c
\end{pmatrix}\times \begin{pmatrix}
    d\\e\\f
\end{pmatrix}=\begin{pmatrix}
    bf-ec\\
    -(af-cd)\\
    ae-bd
\end{pmatrix}$ or 
$\left|
\begin{pmatrix}
    a\\b\\c
\end{pmatrix}\times \begin{pmatrix}
    d\\e\\f
\end{pmatrix}
\right|=\left|\begin{pmatrix}
    a\\b\\c
\end{pmatrix}\right| \left|\begin{pmatrix}
    d\\e\\f
\end{pmatrix}\right| \sin\theta$

Angle Between 2 vectors (This is just a re-arrangement of above): $\theta=\cos^{-1}\left(\frac{\begin{pmatrix} a\\b\\c\end{pmatrix}\cdot \begin{pmatrix}
d\\e\\f 
\end{pmatrix}}{\left|\begin{pmatrix}
a\\b\\c
\end{pmatrix}\right|\left|\begin{pmatrix}
    d\\e\\f
    \end{pmatrix}\right|}\right)$

Vector Equation of a line: To find this we need: point and direction (if given 2 points find the directions and use either point) 
$r=\begin{pmatrix}
    a\\b\\c
\end{pmatrix}+\lambda \begin{pmatrix}
    d\\e\\f
\end{pmatrix}$, $\begin{pmatrix}
    a\\b\\c
\end{pmatrix}$ is position, and $\begin{pmatrix}
    d\\e\\f
\end{pmatrix}$ is direction (parallel to)

Cartesian Equation of a line: $\frac{x-a}{d}=\frac{y-b}{e}=\frac{z-c}{f}$

Parametric form of a line: $x=a\lambda d$, $y=b+\lambda e$, $z=c+\lambda f$

Equation of a plane: $\textbf{r}\cdot\textbf{n}=\begin{pmatrix}
    a\\b\\c
\end{pmatrix}\cdot \textbf{n}$, where $n$ is the normal vector 

Vector Equation of a plane: To find this we need: a point in plane and perp direction. If not given perp direction take the cross product of 2 direction vectors. Remember to find a direction we subtract 2 position vectors;
$\begin{pmatrix}
    x\\y\\z
\end{pmatrix}=\begin{pmatrix}
    a\\b\\c
\end{pmatrix}+\lambda \begin{pmatrix}
    d\\e\\f
\end{pmatrix}+ \mu\begin{pmatrix}
    r\\s\\t
\end{pmatrix}$, where $\begin{pmatrix}
    a\\b\\c
\end{pmatrix}$ is position, $\begin{pmatrix}
    d\\e\\f
\end{pmatrix}$ and $\begin{pmatrix}
    r\\s\\t
\end{pmatrix}$ are directions (parallel to)

Cartesian Equation of a plane: $ax+by+cz=d$, where $d$ is the distance form origin to plane and $\begin{pmatrix}
    a\\b\\c
\end{pmatrix}$ is the direction vector (perpendicular to )

Area of a Parallelogram: $A=\left|\begin{pmatrix}
    a\\b\\c
\end{pmatrix}\times \begin{pmatrix}
    d\\e\\f
\end{pmatrix}\right|$ where $\begin{pmatrix}
    a\\b\\c
\end{pmatrix}$ and $\begin{pmatrix}
    d\\e\\f
\end{pmatrix}$ form 2 adjacent sides of a parallelogram 

Perp Distance between point and plane from $(\alpha, \beta, \gamma)$ to $ax+by+cz=d$: $\frac{|a(\alpha)+b(\beta)+c(\gamma)+d}{\sqrt{a^2+b^2+c^2}}$

Scalar Product Properties:
\begin{itemize}
    \item $0\cdot \textbf{a}=\textbf{a}$
    \item $\textbf{a}\cdot \textbf{b}=\textbf{b}\cdot \textbf{a}$
    \item $(-\textbf{a})\cdot \textbf{b}=-(\textbf{a}\cdot \textbf{b})$
    \item $(k\textbf{a})\cdot\textbf{b}=k(\textbf{a}\cdot \textbf{b})$
    \item $\textbf{a}\cdot (\textbf{b}+\textbf{c})=\textbf{a}\cdot\textbf{b}+\textbf{a}\cdot\textbf{c}$
    \item If $a$ and $b$ are parallel: $\textbf{a}\cdot\textbf{b}=|\textbf{a}||\textbf{b}|$, moreover $\textbf{a}\cdot\textbf{a}=|\textbf{a}|^2$
\end{itemize}

Cross Product Properties:
\begin{itemize}
    \item $\textbf{a}\times \textbf{a}=0$
    \item $\textbf{a}\times 0 = 0\times \textbf{a}=0$
    \item $\lambda(\textbf{a}\times \textbf{b})=(\lambda\textbf{a})\times \textbf{b}=\textbf{a}\times (\lambda\textbf{b})$
    \item $\textbf{a}\times (\textbf{b}+\textbf{c})=(\textbf{a}\times \textbf{b})+(\textbf{a}\times \textbf{c})$
    \item $\textbf{a}\times \textbf{b}=-(\textbf{b}\times \textbf{a})$
    \item $\textbf{b}(\textbf{c}\times \textbf{a})=\textbf{c}(\textbf{a}\times \textbf{b})$
\end{itemize}

\section*{Probability}
Mean:
\begin{itemize}
    \item If no frequency: $\overline{x}=\frac{\sum x}{n}$
    \item If frequency, $\overline{x}=\frac{\sum fx}{\sum f}$
\end{itemize}

Variance:
\begin{itemize}
    \item If no frequency: $\sigma^2 = \frac{\sum x^2}{n}-\overline{x}^2=\frac{\sum(x-\mu)^2}{n}$
    \item If frequency: $\sigma^2 \frac{\sum fx^2}{\sum f}-\overline{x}^2=\frac{\sum f(x-\mu)^2}{\sum f}$
    Note: can also use the formula $\frac{s_{xx}}{n}$
\end{itemize}

Standard Deviation: $\sigma=\sqrt{\text{variance}}$

$s_{xx}$: $\sum (x_i-\overline{x})^2=\sum x_i^2-\frac{(\sum x_i)^2}{n}$

Probability of event $A$: $P(A)=\frac{n(A)}{n(U)}=\frac{\text{number of favourable outcomes}}{\text{number of possible outcomes}}$

Complementary Events: $P(A')=1-P(A)$ i.e. probabilities add up to 1

Combined Events (Addition Rule): $P(A\cup B)=P(A)+P(B)-P(A\cap B)$

Mutually Exclusive Events: $P(A\cap B)=0$ Addition rule becomes: $P(A\cup B)=P(A)+P(B)$

Independent Events: $P(A\cap B)=P(A)P(B)$, Addition rule becomes: $P(A\cup B)=P(A)+P(B)-P(A)P(B)$. To find whether independent: Find $P(A), P(B)$ and $P(A\cap B)$ and see whether the former 2 multiply to make the latter or show that $P(A|B)=P(A)$

Conditional ``$A$ given $B$'': $P(A|B)=\frac{P(A\cap B)}{P(B)}$. If independent, $P(A|B)=P(A)$

Bayes Theorem: $P(A|B)=\frac{P(B|A)P(A)}{P{B|A}P(A)+P(B|A')P(A')}$

Binomial Distribution (binompd (=), binomcd ($\leq$))
\begin{itemize}
    \item $x \sim B(n,p)$
    \item $E(X)$ = Mean = $np$, Var($X$) = $np(1-p)$
    \item $P(X=x)=\binom{n}{x}p^x(1-p)^x$
\end{itemize}

Normal Distribution (normcd (given $x$, want prob), invnorm (given prob, want $x$))
\begin{itemize}
    \item $x\sim N(\mu,\sigma^2)$
    \item Standardised variable $z=\frac{x-\mu}{\sigma}$
\end{itemize}

Interquartile Range: IQR = $Q_3-Q_1$

Outliers: Ant=y values $>UQ+1.5(IQR)$ or $<LQ-1.5(IQR)$

\section*{Mechanics}
SUVAT (5 formulas)
\begin{itemize}
    \item $v=u+at$ 
    \item $s=vt-\frac{1}{2}at^2$
    \item $s=\left(\frac{u+v}{2}\right)t$
    \item $s=ut+\frac{1}{2}at^2$
    \item $v^2=u^2+2as$
\end{itemize}

Centres of Mass for Uniform Bodies 
\begin{itemize}
    \item Triangular Lamina: $\frac{2}{3}$ along median from vertex 
    \item Circular arc: radius $r$, angle at centre $2\alpha = \frac{r\sin\alpha}{\alpha}$ from centre 
    \item Sector of circle, radius $r$, angle at centre $2\alpha: \frac{2r\sin\alpha}{3\alpha}$ from centre 
    \item Solid hemisphere, radius $r:\frac{3}{8}r$ from centre 
    \item Hemispherical Shell, radius $r: \frac{1}{2}r$ from centre 
    \item Solid cone or pyramid of height $h: \frac{1}{4}h$ above the base on the line from centre to base of vertex 
    \item Solid cone or pyramid of height $h: \frac{3}{4}h$ from the vertex 
    \item Conical shell of height $h: \frac{1}{4}h$ above the base on the line from centre to base of vertex 
\end{itemize}

Motion in a circle 
\begin{itemize}
    \item Transverse velocity: $v=r\dot{theta}$
    \item Transverse acceleration: $\dot{v}=r\ddot{\theta}$
    \item Radial acceleration: $-r\dot{\theta}^2=-\frac{v^2}{r}$ Note: Mag = $r\dot{\theta}^2$ or $\frac{v^2}{r}$
\end{itemize}

Motion of a projectile: Equation of a trajectory: $y=x\tan\theta - \frac{gx^2}{2v^2\cos^2\theta}$

Elastic Strings and Springs: $F$ is force needed to extend or compress, $T$ is tension, $x$ is length of extension/compression, $k$ is the stiffness constant (spring constant measured in N/m), $\lambda$ is the modulus of elasticity (spring modulus) measured in Newtons, and $l$ is the natural length of the spring.
\begin{itemize}
    \item Hooke's Law: $F=-kx$
    \item Tension in elastic spring/string: $T=\frac{\lambda}{l}x=\frac{\lambda x}{l}$
\end{itemize}

Energy: Note if answer is negative then means a loss 
\begin{itemize}
    \item Kinetic Energy: $\frac{1}{2}mv^2$
    \item Gravitational Potential Energy: $mgh$
    \item Elastic Potential Energy: $\frac{\lambda x^2}{2l}$
    \item Change in Kinetic Energy: $\frac{1}{2}m_1v_1^2-\frac{1}{2}m_2v_2^2$
    \item Change in potential energy: $m_1gh_1-m_2gh_2$
\end{itemize}

Work Done: $W$ is the work done, $F$ is the magnitude of the force, $d$ is the distance moved in the direction of the force, $\theta$ is the angle between the force and the displacement, and total energy is kinetic + potential + elastic 
\begin{itemize}
    \item If work done against an opposing force: Final total energy = Initial total energy - work done against force where work is done against opposing force is $W$
    \item $W=Fd\cos\theta$ total energy lost (aka work energy principle) and potential energy (if talking about against gravity)
    \item Note: Total energy lost = change in total energy = change in kinetic + change in potential 
    \item If no work done against an opposing force, final total energy = initial total energy 
\end{itemize}

\section*{Induction Template}
Let $P_n$ be the proposition$\dots$:
\begin{itemize}
    \item Let $n=1$, Plug in $n=1$ to both the LHS and RHS. Show that LHS = RHS $\implies P_1$ true 
    \item Assume $n=k$ true i.e $P_k$ true: replace $n$ with $k$. There is nothing to prove here, we just assume this to be true 
    \item Let $n=k+1$: Replace $n$ with $k+1$. Usually only need work on the LHS side by simplifying and using assumed $P_k$ step to show that what we get for LHS is equal to RHS (sometimes we may need to work on the RHS also) $\implies P_{k+1}$ is true 
    \item So $P_k$ true $\implies P_{k+1}$ true and $P_1$ true then $P_2,P_3,P_4,\dots$ true $\therefore$ true for all $n\in \dots$.  
\end{itemize}



\section*{Calculus}
Turning/Stationary Points (Max/Min): Solve $\frac{\dd y}{\dd x}=0$

Proving whether Max/Min: If $\frac{\dd^2 y}{\dd x^2}>0$ min and $\frac{\dd^2 y}{\dd x^2}<0$ max. Or can do sign change test for $\frac{\dd y}{\dd x}$ using number line 

Points of Inflection: solve $\frac{\dd^2 y}{\dd x^2}=0$

Increasing/Decreasing (use number line to solve)
\begin{itemize}
    \item To find where increasing: solve $\frac{\dd y}{\dd x}>0$
    \item To find where decreasing: solve $\frac{\dd y}{\dd x}<0$
\end{itemize}

Convex/Concave (use number line to solve)
\begin{itemize}
    \item To find where concave up/convex: solve $\frac{\dd^2 y}{\dd x^2}>0$
    \item To find where concave down/concave: solve $\frac{\dd^2 y}{\dd x^2}<0$
\end{itemize}

Tangents and Normals: $y-y_1=m(x-x_1)$. Differentiate to get $m$ (tangent means $\parallel$, Normal means $\perp$)

Implicit: ``every time we differentiate a $y$ we write $\frac{\dd y}{\dd x}$''

Area Between
\begin{itemize}
    \item curve \& $x$-axis: $\int_{x=a}^{x=b}y\dd x$
    \item curve \& $y$-axis: $\int_{y=a}^{y=b}x\dd y$
    \item Between 2 curves: $\int_{x=a}^{x=b}$(top curve - bottom curve)$\dd x$
\end{itemize}
Remember to split up if separate areas.

Kinematics:
\begin{itemize}
    \item Distance = $\int_{t_1}^{t_2}|v(t)|\dd t$
    \item Displacement = $\int_{t_1}^{t_2}v(t)\dd t$
    \item Velocity: $\int_{t_1}^{t_2}a(t)\dd t$ or $\frac{\dd s}{\dd t}$
    \item Acceleration = $\frac{\dd v}{\dd t}=\frac{\dd^2 s}{\dd t^2}$
\end{itemize}

Differentiation 1st Principles: $\frac{\dd y}{\dd x}=f'(x)=\lim_{h\to 0}\frac{f(x+h)-f(x)}{h}$

Chain Rule: $y=g(u), u=f(x)\implies \frac{\dd y}{\dd x}=\frac{\dd y}{\dd u}\times \frac{\dd u}{\dd x}$

Product Rule: $y=uv \implies \frac{\dd y}{\dd x}=u\frac{\dd v}{\dd x}+v\frac{\dd u}{\dd x}$

Quotient Rule: $y=\frac{u}{v}\implies \frac{\dd y}{\dd x}=\frac{v\frac{\dd u}{\dd x}-u\frac{\dd v}{\dd x}}{v^2}$

Derivatives:
\begin{itemize}
    \item $x^n\implies nx^{n-1}$
    \item $(f(x))^n\implies n(f(x))^{n-1}f'(x)$
    \item $\ln(f(x))\implies \frac{f'(x)}{f(x)}$
    \item $\sin f(x) \implies f'(x)\cos f(x)$
    \item $\cos f(x) \implies -f'(x)\sin f(x)$
    \item $e^{f(x)}\implies f'(x)e^{f(x)}$
    \item $a^{f(x)}\implies f'(x)a^{f(x)}\ln a$
    \item $\tan f(x) \implies f'(x)\sec^2 f(x)$
    \item $\sec f(x) \implies f'(x)\sec f(x) \tan f(x)$
    \item $\csc f(x) \implies -f'(x)\csc f(x)\cot f(x)$
    \item $\cot f(x) \implies -f'(x)\csc^2 f(x)$
    \item $\sin^{-1}f(x)\implies \frac{f'(x)}{\sqrt{1-(f(x))^2}}$
    \item $\cos^{-1}f(x)\implies -\frac{f'(x)}{\sqrt{1-(f(x))^2}}$
    \item $\tan^{-1}f(x)\implies \frac{f'(x)}{1+(f(x))^2}$
    \item $\sec^{-1}f(x)\implies \frac{f'(x)}{f(x)\sqrt{(f(x))^2-1}}$
    \item $\csc^{-1}f(x)\implies -\frac{f'(x)}{f(x)\sqrt{(f(x))^2-1}}$
    \item $\cot^{-1}f(x)\implies -\frac{f'(x)}{1+(f(x))^2}$
\end{itemize}

Integrals:
\begin{itemize}
    \item $\int x^n \dd x = \frac{x^{n+1}}{n+1}+c, n\neq -1$
    \item $\int \frac{1}{kx}\dd x = \frac{1}{k}\ln |X| + c$
    \item $\int \sin kx \dd x = -\frac{1}{k}\cos kx + c$
    \item $\int \cos kx \dd x = \frac{1}{k}\sin kx + c$
    \item $\int e^{kx}\dd x = \frac{1}{k}e^{kx}+c$
    \item $\int a^{kx}\dd x = \frac{1}{k\ln a}a^{kx}+c$
    \item $\int\sec^2 kx \dd x = \frac{1}{k}\tan kx + c$
    \item $\int \sec kx \tan kx \dd x = \frac{1}{k}\sec kx +c$
    \item $\int \csc kx \cot kx \dd x = -\frac{1}{k}\csc kx +c$
    \item $\int \csc^2 kx \dd x = -\frac{1}{k}\cot kx + c$
    \item $\int \sec kx \dd x = \frac{1}{k}\ln|\sec kx + \tan kx|+c$
    \item $\int \csc kx \dd x = -\frac{1}{k}\ln |\csc kx + \cot kx|+c$
    \item $\int \frac{1}{\sqrt{a^2-(bx)^2}}\dd x = \frac{1}{b}\sin^{-1}\left(\frac{bx}{a}\right)+c$
    \item $\int -\frac{1}{\sqrt{a^2-(bx)^2}}\dd x = \frac{1}{b}\cos^{-1}\left(\frac{bx}{a}\right)+c$
    \item $\int \frac{1}{a^2+(bx)^2}\dd x = \frac{1}{ab}\tan^{-1}\left(\frac{bx}{a}\right)+c$
\end{itemize}

Integration by parts: $\int u \frac{\dd u}{\dd x}\dd x = uv-\int v\frac{\dd u}{\dd x}\dd x$

Trapezium Rule: $h=\frac{b-a}{\text{number of strips}}$: $\frac{h}{2}[y_0+2(y_1+y_2+y_3+y_4+\cdots)+y_n]$. Simply put, $\frac{1}{2}h[\text{1st}y + 2(\text{middle y's}+\text{last y})]$

Newton Raphson: For solving $f(x)=0: x_{n+1}=x_n-\frac{f(x_n)}{f'(x_n)}$

\section*{Functions}
Inverse: Replace $f(x)$ with $y$, swap $x$ \& $y$, solve for $y$

Composite: $fg(x)$ means plug $g(x)$ into $f(x)$

Odd and Even Functions
\begin{itemize}
    \item Even: $f(-x)=f(x)$
    \item Odd: $f(-x)=-f(x)$
\end{itemize}

Transformations: $af(bx+c)+d$ ``anything in a bracket affects $x$ and does the opposite''.
\begin{itemize}
    \item $a$ is vertical stretch sf $a$
    \item $b$ is horizontal stretch sf $\frac{1}{b}$
    \item $c$ is translation $c$ units $x$ direction 
    \item $d$ is translation $d$ units $y$ direction 
    \item $f(-x)$ reflection in $y$ axis 
    \item $-f(x)$ reflection in $x$ axis 
\end{itemize}

Linear: $y=mx+b$
\begin{itemize}
    \item Domain: $x\in \mathbb{R}$
    \item Range: $y\in \mathbb{R}$
\end{itemize}

Quadratic: $y=\pm a(bx+c)^2+d$
\begin{itemize}
    \item Domain: $x\in \mathbb{R}$
    \item Range: $y\geq d$ if min, $y\leq d$ if max 
\end{itemize}

Exponential: $y=ae^{bx+c}+d$
\begin{itemize}
    \item Domain: $x\in \mathbb{R}$ (Hint: power of exp can be anything, so no restriction)
    \item Range: $y>d$ if $a>0$, $y<d$ if $a<0$ (Hint: exp can't be zero)
    \item Asymptote: $y=d$
\end{itemize}

Logarithm: $y=a\ln (bx+c)+d$
\begin{itemize}
    \item Domain: $x>-\frac{c}{d}$ (Hint: $\ln$ can't take a neg number so $bx+c>0$)
    \item Range: $y\in \mathbb{R}$
    \item Asymptote: $x=-\frac{c}{b}$
\end{itemize}

Root: $y=a\sqrt{bx+c}+d$
\begin{itemize}
    \item Domain: $x\geq -\frac{c}{b}$ (Hint: underneath root must be positive so $bx+c\geq 0$)
    \item Range: $y\geq d$ if $a>0$ and $y\leq d$ if $a<0$
\end{itemize}

Modulus: $y=a|bx+c|+d$
\begin{itemize}
    \item Domain: $x\in \mathbb{R}$
    \item Range: $y\geq d$ if $a>0$ and $y\leq d$ if $a<0$
\end{itemize}
Note: Definition of $|x|=\begin{cases}
    x \qquad x\geq 0 \\ 
    -x \qquad x<0 
\end{cases}$

Rational: $y=\frac{ax+b}{cx+d}+e$
\begin{itemize}
    \item Domain: $x\in \mathbb{R}, x\neq -\frac{d}{e}$ (hint: denom $\neq$ 0)
    \item Range: $y\in \mathbb{R}, y\neq\frac{a}{c}+e$
    \item Asymptotes: $x=-\frac{d}{c}, y=\frac{a}{c}+e$
    Note: often $a$ and or $e$ are zero
\end{itemize}

Trigonometry: $y=a\sin(bx+c)+d$, $y=a\cos(bx+c)+d$
\begin{itemize}
    \item Domain: $x\in \mathbb{R}$
    \item Range: $-a+d\leq y\leq a+d$
    Note: If asked to find values of $a$, $b$, $c$, $d$:
    \item $a$ = amplitude = $\frac{\text{max} y - \text{min} y}{2}$
    \item $b=\frac{2\pi}{\text{period}}$ or $\frac{360}{\text{period}}$
    \item $d$ = principal axis = $\frac{\text{max} y + \text{min}y}{2}$
    \item $c$ = phase shift (plug in point to find) 
\end{itemize}

Inverse trig: $y=\sin^{-1}x$
\begin{itemize}
    \item Domain: $-1\leq x\leq 1$
    \item Range: $-\frac{\pi}{2}\leq x\leq \frac{\pi}{2}$
\end{itemize}

Inverse trig: $y=\cos^{-1}x$
\begin{itemize}
    \item Domain: $-1\leq x\leq 1$
    \item Range: $0\leq x\leq \pi$
\end{itemize}

Inverse trig: $y=\tan^{-1}x$
\begin{itemize}
    \item Domain: $-\infty \leq x \leq \infty$
    \item Range: $-\frac{\pi}{2}<x<\frac{\pi}{2}$
\end{itemize}

\section*{Summation Results}
Properties:
\begin{itemize}
    \item Can take constants out: $\sum^n_{k=1}ca_k = c\sum^n_{k=1}a_k$
    \item Can split up: $\sum^n_{k=1}(a_k\pm b_k) = \sum^n_{k=1}a_k \pm \sum^n_{k=1}b_k$
    \item Inclusive-Exclusive principle: $\sum^n_{k=m}(a_k)=\sum^n_{k=1}(a_k)-\sum^{m-1}_{k=1}(a_k)$
\end{itemize}

Results:
\begin{itemize}
    \item $\sum^n_{i=1}1=n$
    \item $\sum^n_{i=1}c=cn$
    \item $\sum^n_{i=1}i=\frac{n(n+1)}{2}$
    \item $\sum^n_{i=1}i^2 = \frac{n(n+1)(2n+1)}{6}$
    \item $\sum^n_{i=1}i^3 = \frac{n^2(n+1)^2}{4}$
\end{itemize}

\section*{Matrix Transformations}
Reflection in the line $y=(\tan\theta)x$: $\begin{pmatrix}
    \cos 2\theta & \sin 2\theta \\ 
    \sin 2\theta & -\cos 2\theta 
\end{pmatrix}$

Horizontal stretch by scale factor $k$: $\begin{pmatrix}
    k & 0\\
    0 & 1
\end{pmatrix}$

Vertical stretch by scale factor $k$: $\begin{pmatrix}
    1 & 0\\
    0 & k 
\end{pmatrix}$

Enlargement by scale factor $k$ centre $(0,0)$: $\begin{pmatrix}
    k & 0\\
    0 & k 
\end{pmatrix}$

Anti-clockwise rotation of angle $\theta$ about origin: $\begin{pmatrix}
    \cos\theta & -\sin\theta \\ 
    \sin\theta & \cos\theta 
\end{pmatrix}, \theta > 0$

Clockwise rotation of angle $\theta$ bout origin: $\begin{pmatrix}
    \cos\theta & \sin\theta \\ 
    -\sin\theta & \cos\theta 
\end{pmatrix}, \theta>0$

\section*{Matrices}
Determinant:
\begin{itemize}
    \item $2\times 2: \left|\begin{pmatrix}
        a & b \\
        c & d
    \end{pmatrix}  \right| = ad-bc$

    \item $3\times 3: \left| 
\begin{pmatrix}
    a & b & c \\
    d & e & f \\
    g & h & i 
\end{pmatrix}
    \right| = a \begin{vmatrix}
        e & f \\ 
        h & i 
    \end{vmatrix}-b\begin{vmatrix}
        d & f \\ 
        g & i
    \end{vmatrix} +c \begin{vmatrix}
        d & e \\
        g & h
    \end{vmatrix}= a(ei-fh)-b(di-fg)+c(dh-eg)$
\end{itemize}

Inverse: $\frac{1}{\text{determinant}}\times \text{adjugate}$
\begin{itemize}
    \item $2\times 2: \begin{pmatrix}
        a & b \\
        c & d 
    \end{pmatrix}^{-1}= \frac{1}{ad-bc}\begin{pmatrix}
        d & -c \\ 
        -b & a 
    \end{pmatrix}$

    \item $3\times 3: \begin{pmatrix}
        a & b & c \\ 
        d & e & f \\ 
        g & h & i
    \end{pmatrix}^{-1}$
\end{itemize}
To find adjugate:
\begin{enumerate}
    \item Find matrix of minors (cross of corresponding row and column for each element and the determinant of each remaining part forms the new element)
    \item Find matrix of cofactors (get the correct signs)
    \item Transpose all the elements. In other words swap their positions over the diagonal (the diagonal stays the same)
\end{enumerate}

3 types of solutions for systems of linear equations 
\begin{itemize}
    \item Consistent/unique - one solution (a point). To find unknowns - use basic elimination or solve $\det \neq 0$ (if unknowns on LHS)
    \item Consistent/non-unique - infinite solutions ($\det = 0$). To find unknowns - Use basic elimination with 2 pairs of the same variable eliminated and look to get $0=0$
    \item Inconsistent/non unique - no sol ($\det = 0$). To find unknowns - use basic elimination with 2 pairs of the same variable eliminated and look to get an inconsistency
\end{itemize}

\section*{Roots}
Quadratic:
\begin{itemize}
    \item form: $(x-\alpha)(x-\beta)$, $\alpha$ and $\beta$ are the roots 
    \item form: $x^2+bx+c$. Sum roots = $-b=\alpha + \beta$. product roots = $c=\alpha\beta$
    \item form: $ax^2+bx+c$. Sum: $-\frac{b}{a}=\alpha$, product: $\frac{c}{a}=c=\alpha\beta$
\end{itemize}

Cubic:
\begin{itemize}
    \item form: $(x-\alpha)(x-\beta)(x-\gamma)$, $\alpha$, $\beta$, $\gamma$ are the roots 
    \item form: $ax^3+bx^2+cx+d=0$, sum: $\alpha+\beta+\gamma=-\frac{b}{a}$, product: $\alpha\beta\gamma = -\frac{d}{a}$, sum of all possible products of pairs of roots: $\alpha\beta + \beta\gamma + \alpha\gamma = \frac{c}{a}$
\end{itemize}

Quartic:
\begin{itemize}
    \item form: $(x-\alpha)(x-\beta)(x-\gamma)(x-\sigma)$, $\alpha$, $\beta$, $\gamma$, $\sigma$ are the roots 
    \item form: $ax^4+bx^3+cx^2+dx+e=0$, sum: $\alpha+\beta+\gamma+\sigma = -\frac{b}{a}$, product: $\alpha\beta\gamma\sigma = \frac{e}{a}$, sum of all possible products of pairs of roots: $\alpha\beta + \alpha\gamma + \alpha\sigma + \beta\gamma + \beta\sigma + \gamma\sigma = \frac{c}{a}$, sum of all possible products of triples of roots: $\alpha\beta\gamma + \beta\gamma\sigma + \gamma\alpha\sigma + \sigma\alpha\beta = -\frac{d}{a}$
\end{itemize}

Form: $\sum^n_{i=0}a_ix^i=0$
\begin{itemize}
    \item Sum = $\frac{-a_{n-1}}{a_n}$
    \item Product = $\frac{(-1)^na_0}{a_n}$
\end{itemize}

\section*{Polar Coordinates}
Area of Sector: $\frac{1}{2}\int r^2\dd \theta$

\section*{Complex Numbers}
Definition: $\sqrt{-1}=i, i^2=-1$

Cartesian Form: $z=a+bi$

Modulus Argument Form 
\begin{itemize}
    \item $z=r(\cos\theta + i\sin\theta)=r\cis \theta$
    \item $r=\sqrt{a^2+b^2}\arg = \theta = \tan^{-1}\left(\left|\frac{b}{a}\right|\right)$
\end{itemize}
Then draw the angle $\theta$ in the quadrant where the complex number $a+bi$ lies. Read off $\theta$ by starting on the positive $x$ axis (like when you solve for trig using the CAST diagram).
Remember that $-\pi \leq \theta < \pi$. Or you can use the following to get $\theta$:
\begin{itemize}
    \item If Quadrant 1 or 4, $\theta = \tan^{-1}\left(\frac{b}{a}\right)$
    \item If Quadrant 2, $\theta = \tan^{-1}\left(\frac{b}{a}\right)+\pi$
    \item If Quadrant 3, $\theta = \tan^{-1}\left(\frac{b}{a}\right)-\pi$
\end{itemize}

Eulers Form: $z=re^{i\theta}$

De Moivre's Theorem: $z^n=r^n(\cos n\theta + i\sin n\theta)=r^n \cis n\theta$

Roots of $z^n=1$: $z=e^{\frac{2\pi ki}{n}}$, for $k=0,1,2,\dots,n-1$.

\section*{Linear Algebra}
Eigenvalues: Set characteristic polynomial which is $\det(A-\lambda I)$ or $\det(\lambda-AI)$ equal to zero, where $A$ = matrix and $I$ = identity matrix. Solve this for eigenvalue.

Eigenvectors: $\textbf{Av}=\lambda\textbf{v}$, where $\lambda$ = eigenvector \& $\text{v}$ = eigenvector. $Av-\lambda\textbf{v}=0$ i.e. $A\begin{pmatrix}
    x\\y
\end{pmatrix} =\lambda\begin{pmatrix}
    x\\y
\end{pmatrix}$. Solve with value of $\lambda$ above. Quick way: Put $\lambda$ into $(A-\lambda I)$ and you'll obviously have a matrix. Multiply this matrix by $\begin{pmatrix}
    x \\ y
\end{pmatrix}$ and set equal to 0. Solve for $x$ in terms of $y$.

\section*{Conics}
Ellipse:
\begin{itemize}
    \item Standard form: $\frac{x^2}{a^2}+\frac{y^2}{b^2}=1$
    \item Parametric Form: $(a\cos\theta, b\sin\theta)$
    \item Eccentricity: $e<1$, $b^2=a^2(1-e^2)$
    \item Foci: $(\pm ae,0)$
    \item Directrices: $x=\pm \frac{a}{e}$
    \item Asymptotes: none 
\end{itemize}

Parabola: 
\begin{itemize}
    \item Standard Form: $y^2=4ax$
    \item Parametric Form: $(at^2,2at)$
    \item Eccentricity: $e=1$
    \item Foci: $(a,0)$
    \item Directrices: $x=-a$
    \item Asymptotes: none 
\end{itemize}

Hyperbola:
\begin{itemize}
    \item Standard Form: $\frac{x^2}{a^2}-\frac{y^2}{b^2}=1$
    \item Parametric Form: $(a\sec\theta, b\tan\theta)$, $(\pm a\cosh \theta, b\sinh \theta)$
    \item Eccentricity: $e>1$, $b^2=a^2(e^2-1)$
    \item Foci: $(\pm ae,0)$
    \item Directrices: $x=\pm \frac{a}{e}$
    \item Asymptotes: $\frac{x}{a}=\pm \frac{y}{b}$
\end{itemize}

Rectangular Hyperbola:
\begin{itemize}
    \item Standard Form: $xy=c^2$
    \item Parametric Form: $\left(ct,\frac{c}{t}\right)$
    \item Eccentricity: $e=\sqrt{2}$
    \item Foci: $(\pm \sqrt{2}c, \pm\sqrt{2}c)$
    \item Directrices: $x+y=\pm \sqrt{2}c$
    \item Asymptotes: $x=0, y=0$
\end{itemize}

\section*{Groups}
Order of a group: Number of elements in the group 

Order of an element: Least positive integer $n$ such that $g^n=e$ (how many times before you get $e$ in modular arithmetic). If $g\in G$ then $g\dots g=e$ i.e. $g^n=e$. 
If no $n$ such that $x^n=e$ then we say element has infinite order.

Definition: A set $G$ with a binary operation * on $G$ such that 
\begin{enumerate}
    \item $G$ is associative 
    \item $G$ is closed under * 
    \item $G$ has an identity element (usually denoted $e$)
    \item Each element of $G$ has an inverse 
\end{enumerate}

\section*{Hyperbolics}
Definitions:
\begin{itemize}
    \item $\sinh x = \frac{e^x-e^{-x}}{2}$
    \item $\cosh x = \frac{e^x+e^{-x}}{2}$
    \item $\tanh x = \frac{\sinh x}{\cosh x} = \frac{e^x-e^{-x}}{e^x+e^{-x}}$
    \item $\text{csch} x = \frac{1}{\sinh x} = \frac{2}{e^x-e^{-x}}$
    \item $\text{sech} x = \frac{1}{\cosh x} = \frac{2}{e^x+e^{-x}}$
    \item $\coth x = \frac{1}{\tanh x} = \frac{e^x+e^{-x}}{e^x-e^{-x}}$
\end{itemize}

Identities:
\begin{itemize}
    \item $\cosh^2 - \sinh^2 x = 1$
    \item $\tanh^2 x + \text{sech}^2 x = 1$
    \item $\coth^2 x - \text{csch}^2 x = 1$
    \item $\tanh x = \frac{\sinh x}{\cosh x}$
    \item $\sinh 2x = 2\sinh x \cosh x$
    \item $\cosh 2x = \cosh^2 x + \sinh^2 x$
\end{itemize}

Inverse: 
\begin{itemize}
    \item $\text{sech}x = \cosh^{-1}x = \ln(x+\sqrt{x^2-1}), x\geq 1$
    \item $\text{csch}x = \sinh^{-1}x = \ln(x+\sqrt{x^2+1})$
    \item $\coth x = \tanh^{-1}x = \frac{1}{2}\ln\left(\frac{1+x}{1-x}\right), |x|<1$
\end{itemize}

\section*{Number Theory}
Fermat's Theorem: $a^p \equiv a \pmod{p}$ is $p$ is prime and $a$ is any integer 



\end{document}