\documentclass[../linalg.tex]{subfiles}
\graphicspath{{\subfix{../figures/}}}
\begin{document}
\chapter{Linear Equations in Linear Algebra}
\section{Systems of Linear Equations}
A linear equation in the variables $x_1,x_2,\dots,x_n$ is an equation that can be written in the form $a_1x_1+a_2x_2+\dots+a_nx_n=b$ where $b$ 
and the coefficients $a_1,a_2,\dots,a_n$ are real or complex numbers.

Example: $4x_1-5x_2+2=x_1, x_2=2(6^{1/2}-x_1)+x_3$ are both linear. Not linear examples are $4x_1-5x_2=x_1x_2, x_2=2(x_1^{1/2})-6$, and $2x_1^{-1}+\sin x_2 = 0$.

Systems of Linear Equations: an $(m\times n)$ system of linear equations is a system of $m$ linear equations with $n$ unknowns.

Example: the $2\times 3$ system of equations below has a solution $x_1=5, x_2=6.5$, and $x_3=3$:
\begin{align*}
    2x_1-x_2+1.5x_3=8\\
    x_1-4x_3=-7
\end{align*}

Two linear systems are called equivalent if they have the same solution set.

A system of linear equations can have: infinitely many solutions, no solution, or a unique solution. Coincident lines have infinitely many solutions, parallel lines have no solution, and intersecting lines have a unique solution.

Matrix Notation:

Let's say we have $x_1-2x_2+x_3=0, 2x_2-8x_3=8$, and $5x_1-5x_3=10$.

The coefficient matrix is 
$\begin{bmatrix}
1 & -2 & 1 \\
0 & 2 & -8 \\
5 & 0 & -5
\end{bmatrix}$ and the augmented matrix is $\begin{bmatrix}
    1 & -2 & 1 & 0 \\
    0 & 2 & -8 & 8\\
    5 & 0 & -5 & 10
\end{bmatrix}$. The augmented matrix is $3\times 4$ (3 rows, 4 columns).

To solve a linear system: if one of the following elementary operations is applied to a system of linear equations, the resulting system is equivalent, 
that is the resulting system has the same set of solutions as the original:
\begin{enumerate}
    \item interchange two equations 
    \item multiply an equation by a non-zero scalar 
    \item add a constant multiple of one equation to another 
\end{enumerate}

Let's use the system 
\begin{align*}
x_1-2x_2+x_3 =0 \\
2x_2 - 8x_3 = 8 \\
5x_1-5x_3=10
\end{align*}

So using row operations and rearranging the rows, we can do $R_1 \leftrightarrow R_2$ to swap the first and second row. Now we can multiply the second line 
by $1/2$ so $1/2R_2$. The next step is to do $R_2+R_1$. 

The matrix that results from this is $\begin{bmatrix}
    1 & -2 & 1 & 0 \\
    0 & 2 & -8 & 8 \\ 
    5 & 0 & -5 & 10 
\end{bmatrix}$

Doing the operations $R_3-5R_1$, $R_3-5R_2$, and $1/2R_2$ and $1/30R_3$, we end up getting $\begin{bmatrix}
    1 & -2 & 1 & 0 \\
    0 & 1 & -4 & 4 \\ 
    0 & 0 & 1 & -1
\end{bmatrix}$, so from this we have the equations 
\begin{align*}
    x_1-2x_2+x_3 =0 \\
    x_2-4x_3=4\\
    x_3=-1
\end{align*}, so the solution set is $(1,0,-1)$. Since a solution exists, the system is consistent.

Let's see if this one is consistent.
\begin{align*}
    x_2-4x_3 =8 \\
    2x_1-3x_2+2x_3 = 1\\
    4x_1-8x_2+12x_3 = 1
\end{align*}

Doing the row operations $R_1\leftrightarrow R_2$, $R_3-2R_1$, $R_3+2R_2$ we get that $0=15$, so this is inconsistent.

This last example has infinitely many solutions:
\begin{align*}
    x_1-2x_2-x_3=-2 \\
    2x_1+x_2+3x_3 = 1\\
    -3x_1+x_2-2x_3 = 1
\end{align*}

Doing the row operations $R_2-2R_1$ and  $R+3+3R_1$, then $R_3+R_2$ and then $1/5R_2$ results in $\begin{bmatrix}
    1 & -2 & 1 & -2 \\
    0 & 1 & 1 & 1 \\
    0 & 0 & 0 & 0
\end{bmatrix}$

Solving this, we have $x_3$ with no restrictions, it is a ``free parameter'' and $x_2=1-x_3$ and $x_1=-x_3$ so $x_1$ and $x_2$ are parameterized by $x_3$.

\section{Row Reduction and Echelon Forms}
Leading entry of a row: the first (counting from left to right) non-zero entry (in a nonzero row)

Echelon Form: upper right-hand stair-case, triangle 
\begin{enumerate}
    \item all rows that consist entirely of zeros are grouped together at the bottom of the matrix 
    \item the first (counting from left to right) non-zero entry in the $(i+1)$st row must appear in a column to the right of the first non-zero entry in the $i$th row.
    \item all entries in a column below a leading entry are zeros 
\end{enumerate}

Reduced Echelon Form: an echelon Form matrix that also has the following properties:
\begin{enumerate}
    \item the leading entry in each nonzero row is 1 
    \item each leading one is the only nonzero entry in its column
\end{enumerate}

A pivot point: a location in a matrix $A$ that corresponds to a leading 1 in a reduced echelon form of $A$. A pivot column is a column of $A$ that contains a pivot position.

Steps to solving a system of linear equations:
\begin{enumerate}
    \item begin with the leftmost nonzero column. This is the pivot column, the pivot position is at the top.
    \item select a non zero entry in the pivot column as a pivot. If necessary, interchange rows to move this entry into the pivot position.
    \item Use row replacement operations to create zeros in all positions below the pivot.
    \item Cover (or ignore) the row containing the pivot position and cover all rows (if any) above it. Apply previous steps to the sub matrix that remains. Repeat until there are no more nonzero rows to modify.
    \item Beginning with the rightomst pivot, create zeros above each pivot. Make each pivot equal to 1 by scaling.
\end{enumerate}

\begin{example}
    Determine the existence and uniqueness of the solution to 

    \begin{align*}
    3x_2-6x_3+6x_4+4x_5=-5\\ 
    3x_1-7x_2+8x_3-5x_4+8x_5 = 9\\
    3x_1-9x_2+12x_3-9x_4+6x_5=15
    \end{align*}

    We can find that $x_5=4$ , and $x_3,x_4$ are of infinite number of solutions. 
\end{example}

\begin{example}
    Find the general solution of the linear system whose augmented matrix has been reduced to 
    $\begin{bmatrix}
        1 & 6 & 2 & -5 & -2 & -4 \\
        0 & 0 & 2 & -8 & -1 & 3 \\
        0 & 0 & 0 & 0& 1 & 7
    \end{bmatrix}$

    We have that $x_5=7$, $x_3=1/2(3+8x_4+7)$ and $x_1=-4-6x_2-2(1/2)(3+8x_4+7)+14$
\end{example}

\begin{theorem}[Existence and Uniqueness]
    A linear system is consistent if and only if the right most column of the augmented matrix is not a pivot column, that is if and only if an echelon form of the augmented matrix 
    has no row of the form $[0,\dots, 0 b]$ with $b$ nonzero. If the system if consistent in the solution contains either a unique solution when there are no free variables or infinitely many solutions when there is at least one free variable.
\end{theorem}

\section{Vector Equations}
Vectors in $\textbf{R}^2$: a matrix with only 1 column is called a vector. The set of all vectors with 2 entries is $\textbf{R}^2$

$\vec{u}=\begin{bmatrix}3\\ -1\end{bmatrix}$ for example.

The vectors are odered pairs of real numbers:

$\vec{u}_1=\begin{bmatrix}
    3\\-1
\end{bmatrix}=(3,-1)\neq (-1,3)=\begin{bmatrix}
    -1\\3
\end{bmatrix}$

Vector addition: $(x_1,y_1)+(x_2,y_2)=(x_1+x_2,y_1+y_2)$

Geometric interpretation: parallelogram law.

$(2,4)+(3,1)=(2+3,4+1)=(5,5)$. We can create a parallelogram (google for review).

Let's say we subtract, $-a$ has the same magnitude as $a$ but points in the opposite direction in this case. $(1,3)-(2,1)=(-1,2)$. Drawing the line from the origin to the tip of the vectors give you what you get algebraically.

Vector multiplication: scalar multiplication: $a(x,y)=(ax,ay)$ where $a$ is a scalar.

Geometric interpretation: $a(x,y,z)$ points in the same direction as $(x,y,z)$ but is scaled by a factor of $a$.

\begin{example}
    Let $\vec{u}=\begin{bmatrix}
        1\\ -2
    \end{bmatrix}$ and $\vec{v} = \begin{bmatrix}
        2\\-5
    \end{bmatrix}$ Find $4\vec{u}$, $-3\vec{v}$ and $4\vec{u}+(-3)\vec{v}$/

    $4\vec{u}=(4,-8)$, $-3\vec{v}=(-6,15)$ and $4\vec{u}+(-3)\vec{v}=(-2,7)$
\end{example}

Vectors start at the origin and have magnitude and direction.

Representing vectors in $\textbf{R}^3$. We add the $z$-axis.

Vectors in $\textbf{R}^n$ we have that $\vec{u}=\begin{bmatrix}
    u_1\\ u_2 \\ \vdots \\u_n 
\end{bmatrix}=(u_1,u_2,\dots,u_n)$ where $u_1,u_2,\dots \in \mathbb{R}$.

The $\vec{0}=\begin{bmatrix}
    0\\0\\ \vdots\\0
\end{bmatrix}$ is the zero vector.

Algebratic Properties of $\textbf{R}^n$: for all $\textbf{u}, \textbf{v}, \textbf{w}$ in $\textbf{R}^n$ and all scalars $c$ and $d$:
\begin{itemize}
    \item $\textbf{u}+\textbf{v}=\textbf{v}+\textbf{u}$
    \item $(\textbf{u}+\textbf{v})+\textbf{w}=\textbf{u}+(\textbf{v}+\textbf{w})$
    \item $\textbf{u}+0=0+\textbf{u}=\textbf{u}$
    \item $\textbf{u}+(-\textbf{u})=0$
    \item $c(\textbf{u}+\textbf{v})=c\textbf{u}+c\textbf{v}$
    \item $(c+d)\textbf{u}=c\textbf{u}+d\textbf{u}$
    \item $c(d\textbf{u})=(cd)\textbf{u}$
    \item $1\textbf{u}=\textbf{u}$
\end{itemize}

Linear Combinations: given vectors $\textbf{v}_1,\textbf{v}_2,\dots,\textbf{v}_p$ in $\textbf{R}^n$ and scalars 
$c_1,c_2,\dots, c_p$, the vector $\textbf{y}=c_1\textbf{v}_1+c_2\textbf{v}_2+\dots+c_p\textbf{v}_p$ is called a linear combination of 
$\textbf{v}_1,\textbf{v}_2,\dots \textbf{v}_p$ with weights $c_1,c_2,\dots, c_p$.

For example, if we have $3^{1/2}\textbf{v}_1+\textbf{v}_2$ this can be written as $\vec{y}=\sqrt{3}\vec{v_1}+\vec{v_2}$ with $c_1=\sqrt{3}$ and $c_2=1$.

\begin{example}
    If $\vec{a_1}=(1,-2,-5), \vec{a_2}=(2,5,6)$, and $\vec{a_3}=(7,4,-3)$ then determine if $\vec{b}$ can be written as a linear combination of $\vec{a_1}$ and $\vec{a_2}$. That is determine if there exists weights $x_1,x_2$ such that $x_1\vec{a_1}+x_2\vec{a_2}=\vec{b}$.

    Using elementary row operations, we can determine that $x_1=3$, $x_2=2$ which is the linear combination of $\vec{a_1}$ and $\vec{a_2}$.

\end{example}

A vector equation $x_1\textbf{a}_1+x_2\textbf{a}_2+\dots x_n\textbf{x}_n=\textbf{b}$ has the same solution set as the linear system with augmented matrix 
$[\textbf{a}_1 \textbf{a}_2 \dots \textbf{a}_n \textbf{b}]$. In particular $\textbf{b}$ can be generated by a linear combination of $\textbf{a}_1,\textbf{a}_2,\dots\textbf{a}_n$ if and only if there exists a solution to the linear system corresponding to the matrix $[\textbf{a}_1 \textbf{a}_2 \dots \textbf{a}_n \textbf{b}]$

Span: if $\textbf{v}_1,\textbf{v}_2,\dots \textbf{v}_p$ are in $\textbf{R}^n$ then the set of all linear combinations of $\textbf{v}_1,\textbf{v}_2,\dots,\textbf{v}_p$ is denoted Span$\{\textbf{v}_1,\textbf{v}_2\dots\textbf{v}_p\}$ and is called the subset of 
$\textbf{R}^n$ spanned by $\textbf{v}_1,\textbf{v}_2,\dots \textbf{v}_p$. That is, Span$\{\textbf{v}_1,\textbf{v}_2\dots\textbf{v}_n\}$ is the collection of all vectors 
that can be written in the form: $c_1\textbf{v}_1+c_2\textbf{v}_2+\dots+c_p\textbf{v}_p$ with $c_1,c_2,\dots,c_p$ scalars.

Asking if a vector $\textbf{b}$ is in Span$\{\textbf{v}_1,\textbf{v}_2\dots\textbf{v}_p\}$ amounts to asking whether the vector equation $x_1\textbf{v}_1+x_2\textbf{v}_2+\dots +x_n\textbf{v}_p=\textbf{b}$ has a solution,
or equivalently whether the linear system with augmented matrix $[\textbf{v}_1 \textbf{v}_2 \textbf{v}_p \textbf{b}]$ has a solution.

Note Span$\{\textbf{v}_1,\textbf{v}_2\dots\textbf{v}_p\}$ contains every scalar multiple of $\textbf{v}_1$. 

The span of a single vector is a line. The span of 2 linearly independent vectors is a plane (not scalar multiples of each other). 

\section{The Matrix Equation Ax = b}
\section{Solution Sets of Linear Systems}
\section{Applications of Linear Systems}
\section{Linear Independence}
\section{Introduction to Linear Transformations}
\section{The Matrix of a Linear Transformation}
\section{Linear Models in Business, Science, and Engineering}


\end{document}