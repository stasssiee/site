\documentclass[../linalg.tex]{subfiles}
\graphicspath{{\subfix{../figures/}}}
\begin{document}
\chapter{Vector Spaces}
\section{Vector Spaces and Subspaces}
A vector space is a nonempety set $V$ of objects, called vectors, on which are defined two operations called addition and scalar multiplication to the 10 axioms listed below. The axioms must hold for all vectors $\textbf{u}$, $\textbf{v}$, and $\textbf{w}$ in $V$ and for all scalars $a$ and $b$.

Closure Properties:
\begin{itemize}
    \item $\textbf{u}+\textbf{v}$ is a vector in $V$
    \item $a\textbf{v}$ is a vector in $V$
\end{itemize}

Properties of Addition:
\begin{itemize}
    \item commutative: $\textbf{u}+\textbf{v}=\textbf{v}+\textbf{u}$
    \item associative: $\textbf{u}+(\textbf{v}+\textbf{w})=(\textbf{u}+\textbf{v})+\textbf{w}$
    \item additive identity: there is a vector $\textbf{0}$ in $V$ such that $\textbf{v}+\textbf{0}=\textbf{v}$ for all $\textbf{v}$ in $V$
    \item additive inverse: given a vector $\textbf{v}$ in $V$, there is a vector $-\textbf{v}$ such that $\textbf{v}+(-\textbf{v})=\textbf{0}$
\end{itemize}

Properties of Scalar Multiplication:
\begin{itemize}
    \item associative: $a(b\textbf{v})=(ab)\textbf{v}$
    \item distributive: $a(\textbf{u}+\textbf{v})=a\textbf{u}+a\textbf{v}$
    \item distributive: $(a+b)\textbf{v}=a\textbf{v}+b\textbf{v}$
    \item multiplicative identity: $1\textbf{v}=\textbf{v}$ for all $\textbf{v}$ in $V$
\end{itemize}

A subspace of a vector space, $V$, is a subset $H$ of $V$ that has three properties:
\begin{enumerate}
    \item the zero vector of $V$ is $H$
    \item $H$ is closed under vector addition 
    \item $H$ is closed under scalar multiplication
\end{enumerate}

A subspace $H$ of $V$ itself is a vector space. The other properties of a vector space are ``inherited'' since $H$ is a subset of $V$.

\begin{theorem}
    If $\textbf{v}_1,\textbf{v}_2,\dots,\textbf{v}_p$ are in a vector space $V$, then Span$\{\textbf{v}_1,\textbf{v}_2,\dots,\textbf{v}_p\}$ is a subspace of $V$.
\end{theorem}

\section{Null Spaces, Column Spaces, Row Spaces, and Linear Transformations}
The Null Space of a $m\times n$ matrix $A$, denoted Nul $A$, is the set of solutions of the homogeneous equation $A\textbf{x}=\textbf{0}$. In set notation: Nul $A$ = $\{\textbf{x}:\textbf{x}$ is in $\textbf{R}^n$ and $A\textbf{x}=\textbf{0}\}$.
\begin{theorem}
    The null space of an $m\times n$ matrix $A$ is a subspace of $\textbf{R}^n$. Equivalently, the set of all solutions to a system $A\textbf{x}=\textbf{0}$ of $m$ homogeneous linear equations in $n$ unknown is a subspace of $\textbf{R}^n$.
\end{theorem}

The Column Space of a $m\times n$ matrix $A$, denoted Col $A$, is the set of all linear combinations of the columns of $A$. If $A=[\textbf{a}_1,\textbf{a}_2,\dots, \textbf{a}_n]$ then Col $A$ = Span$\{\textbf{a}_1,\textbf{a}_2,\dots,\textbf{a}_n\}$.

\begin{theorem}
    The column space of a $m\times n$ matrix $A$ is a subspace of $\textbf{R}^m$.
\end{theorem}

The Row Space of an $m\times n$ matrix $A$, denoted Row $A$, is the set of all linear combinations of the row vectors. Each row has $n$ entries, so the Row $A$ is a subspace of $\textbf{R}^n$.

The null and column space are related. They are very different: When $A$ Is not square, the column space and the null space exist entirely in different ``universes''. For an $m\times n$ matrix, the Column Space is in $m$-dimensional space, where the Null space is in $n$-dimensional space.

Kernel and Range if a Linear Transformation
\begin{itemize}
    \item a Linear Transformation $T$ from a vector space $V$ to be vector space $W$ is a rule that assigns each vector $\textbf{x}$ in $V$ a unique vector $T(\textbf{x})$ in $W$ such that $T(\textbf{u}+\textbf{v})=T(\textbf{u})+T(\textbf{v})$ and 
    $T(c\textbf{u})=cT(\textbf{u})$ for all $\textbf{u},\textbf{v}$ in $V$ and $c$ in $\textbf{R}$.
    \item The Kernel of $T$ is the set of all $\textbf{u}$ in $V$ such that $T(\textbf{u})=\textbf{0}$ 
    \item The Range of $T$ is the set of all vectors in $W$ of the form $T(\textbf{x})$ for some $\textbf{x}$ in $V$
    \item The kernel and range of $T$ are subspaces of $V$
\end{itemize}



\section{Linearly Independent Sets; Bases}
\section{Coordinate Systems}
\section{The Dimension of a Vector Space}
\section{Change of Basis}


\end{document}