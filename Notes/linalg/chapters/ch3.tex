\documentclass[../linalg.tex]{subfiles}
\graphicspath{{\subfix{../figures/}}}
\begin{document}
\chapter{Determinants}
\section{Introduction to Determinants}
Definition of the Determinant: given an $n\times n$ matrix $A=[a_{ij}]$, the determinant is det A = $a_{11}a_{22}-a_{12}a_{21}$.

For $n\geq 2$, the determinant of an $A$ is the sum of $n$ terms of the form $+/-a_{1j}A_{1j}$ with the plus and minus signs alternating, where the entries $a_{11},a_{12},\dots, a_{1n}$ are from the first row of $A$. 

The $(i,j)$-cofactor = $C_{ij}=(-1)^{i+j}$ det A$_{ij}$ s.t. det A = $a_{11}C_{11}+a_{12}+C_{12}+\dots + a_{1n}+C_{1n}$.

\begin{theorem}
    The determinant of an $n\times n$ matrix $A$ can be computed by a cofactor expansion across any row or down any column/

    The expansion across the ith row using the cofactors is: det A = $a_{i1}C_{i1}+a_{i2}C_{i2}+\dots +a_{in}C{in}$

    The expansion across the jth column is: det A = $a_{1j}C_{1j}+a_{2j}C_{2j}+\dots + n_{nj}C_{nj}$.
\end{theorem}

This theorem is helpful for computing determinants of a matrix that contains many zeros. For example, if 1 row contains many zeros, than a cofactor expansion across that row will be easier to calculate.

\begin{theorem}
    If $a$ is a triangular matrix, then det A is the product of the entries on the main diagonal of $A$.
\end{theorem}

\section{Properties of Determinants}
``The secret of determinants lies in how they change when row operations are performed''

\begin{theorem}
    Let $A$ be a square matrix.

    \begin{itemize}
        \item A multiple of one row of $A$ is added to another row to produce a matrix $B$, then det $B$ = det $A$
        \item if two rows of $A$ are interchanged to produce $B$ then det $B$ = -det $A$
        \item if 1 row of $A$ is multiplied by $k$ to produce $B$, then det $B$ = $k$det $A$
    \end{itemize}

    we can use a strategy to reduce a matrix to echelon form and then use the fact that the determinant of a triangular matrix is the product of its diagonal entries.
\end{theorem}

\begin{theorem}
    A square matrix $A$ is invertible if and only if det $A$ is not zero. If $A$ is an $n\times n$ matrix, then det $A^T$ = det $A$.
\end{theorem}

\begin{theorem}
    If $A$ and $B$ are $n\times n$ matrices, then det $AB$ = (det $A$)(det $B$)
\end{theorem}

\section{Cramer's Rule, Volume, and Linear Transformations}
Cramer's Rule: Cramers rule is needed for a variety of theoretical calculations. However the formula is inefficient for hand calculations except for $2\times 2$.

For any $n\times n$ matrix $A$ and any $\textbf{b}$ in $\textbf{R}^n$ let $A_1(\textbf{b})$ be the matrix obtained from $A$ by replacing the ith column by $\textbf{b}$, $A_i(\textbf{b})= [\textbf{a}_1 \textbf{a}_2 \dots \textbf{a}_{i-1} \textbf{b} \textbf{a}_{i+1} \dots \textbf{a}_n]$

\begin{theorem}
    Let $A$ be an intertible invertible $n\times n$ matrix. For any $\textbf{b}$ in $\textbf{R}^n$, the unique solution $\textbf{x}$ of $A\textbf{x}=\textbf{b}$ has entries given by: $\textbf{x}_i$ = (det $A_i(\textbf{B})$/(det $A$)) $i=1,2,\dots,n$.
\end{theorem}

Application to engineering: A number of important engineering problems can be analyzed by Laplace transformations. This approach converts an appropriate system of linear differential equations into a 
system of linear algebraic equations whose coefficient involve a parameter.

Formula for $A^{-1}$: Cramer's Rule leads to a general formula for the inverse of an $n\times n$ matrix.
\begin{theorem}
    Let $A$ be an invertible $n\times n$ matrix. Then $A^{-1}$ is given by:
    (Google This)
\end{theorem}

Determinants as Area of Volume
\begin{theorem}
    If $A$ is a $2\times 2$ matrix, the area of the parallelogram determined by the columns of $A$ is |det A|. If $A$ is a $3\times 3$ matrix, the volume of the paralellepiped determined by the columns is |det A|.
\end{theorem}

\begin{theorem}
    Let $T:\textbf{R}^2 \rightarrow \textbf{R}^2$ be the linear the transformation obtained by a $2\times 2$ matrix $A$. If $S$ is a parallelogram in $\textbf{R}^2$ then the area of $T(S)$ is equal to |det A| times the area of $S$.

    If $T$ is determined by a $3\times 3$ matrix $A$, and $S$ is a paralellepiped in $\textbf{R}^3$, then the volume of $T(S)$ is equal to the area of |det $A$| times the volume of $S$.
\end{theorem}

\end{document}