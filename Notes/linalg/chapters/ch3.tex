\documentclass[../linalg.tex]{subfiles}
\graphicspath{{\subfix{../figures/}}}
\begin{document}
\chapter{Determinants}
\section{Introduction to Determinants}
Definition of the Determinant: given an $n\times n$ matrix $A=[a_{ij}]$, the determinant is det A = $a_{11}a_{22}-a_{12}a_{21}$.

For $n\geq 2$, the determinant of an $A$ is the sum of $n$ terms of the form $+/-a_{1j}A_{1j}$ with the plus and minus signs alternating, where the entries $a_{11},a_{12},\dots, a_{1n}$ are from the first row of $A$. 

The $(i,j)$-cofactor = $C_{ij}=(-1)^{i+j}$ det A$_{ij}$ s.t. det A = $a_{11}C_{11}+a_{12}+C_{12}+\dots + a_{1n}+C_{1n}$.

\begin{theorem}
    The determinant of an $n\times n$ matrix $A$ can be computed by a cofactor expansion across any row or down any column/

    The expansion across the ith row using the cofactors is: det A = $a_{i1}C_{i1}+a_{i2}C_{i2}+\dots +a_{in}C{in}$

    The expansion across the jth column is: det A = $a_{1j}C_{1j}+a_{2j}C_{2j}+\dots + n_{nj}C_{nj}$.
\end{theorem}

This theorem is helpful for computing determinants of a matrix that contains many zeros. For example, if 1 row contains many zeros, than a cofactor expansion across that row will be easier to calculate.

\begin{theorem}
    If $a$ is a triangular matrix, then det A is the product of the entries on the main diagonal of $A$.
\end{theorem}

\section{Properties of Determinants}
\section{Cramer's Rule, Volume, and Linear Transformations}


\end{document}