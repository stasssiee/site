\documentclass[../hchem.tex]{subfiles}
\graphicspath{{\subfix{../figures/}}}
\begin{document}
\chapter{Reactions}
\textit{I told a chemistry joke. There was no reaction.}

\section{Balancing Equations}
When chemical reactions occur:
\begin{itemize}
    \item Bonds are broken and new bonds form.
    \item Energy is produced or absorbed.
    \item New compounds are formed, or compounds decompose to their elements.
    \item The Law of Conservation of Mass is obeyed.
\end{itemize}

Symbols in Equations:
\begin{itemize}
    \item yield $\rightarrow$
    \item Sometimes a reaction occurs then stops 
    
    \item reversible reaction $\leftrightarrow$
    \item Sometimes the reaction goes back and forth between product and reactant.
    
    \item solid or precipitate (s)
    \item gas (g)
    \item liquid (l)
    \item water solution (aq)
    
    \item heat $\Delta$
\end{itemize}

A catalyst is a chemical that speeds up a reaction, but is not actually used up.

\ex Put the reaction Ca(OH)$_2$ (s) $\rightarrow$ CaO (s) + H$_2$O (l) in words. 

Balancing Chemical Equations is necessary so that the correct amount of reactants can be determined and the amounts of the products can be predicted.

It also satisfies the law of conservation of mass.

How to balance a chemical reaction:
\begin{itemize}
    \item Write the equation with the correct formulas and symbols.
    \item Add coefficients to the formulas to make the number of atoms of each element on both sides of the equation the same. A coefficient is a whole number before a chemical formula.
    \item You may not add coefficients to the middle of a formula.
    \item You may not change the subscript of a correctly written formula.
\end{itemize}

\ex Balance FeS + HCl $\rightarrow$ FeCl$_2$ + H$_2$S 

\ex Write the balanced equation for lithium chlorate decomposing into lithium chloride and oxygen gas. 

\section{Synthesis \& Decomposition}
Synthesis Reactions - more than one reactant and only one product.

Decomposition Reactions - one reactant and more than one product.

\textbf{Synthesis Rules}
\begin{itemize}
    \item Reaction between 2 nonmetals produces a common covalent compound.
    \item Reaction of a metal and a nonmetal produces an ionic compound.
    \item Reaction of a metal oxide and water produces a metal hydroxide.
    \item Reaction of a metal oxide with carbon dioxide produces a metal carbonate.
    \item Reaction of a metal chloride with oxygen produces a metal chlorate.
    \item Reaction of a nonmetal oxide with water produces an acid in solution.
\end{itemize}

\ex Write the equation for carbon burning.

\textbf{Decomposition Rules}
\begin{itemize}
    \item Decomposition of a binary compound produces two elements.
    \item Decomposition of a metal carbonate produces a metal oxide and CO$_2$
    \item Decomposition of a metal hydroxide produces a metal oxide and water 
    \item Decomposition of a metal chlorate produces a metal chloride and oxygen 
    \item Decomposition of an oxyacid produces a nonmetal oxide and water. The oxidation number of the nonmetal remains the same.
\end{itemize}

\ex Write the equation for sodium carbonate decomposing when heated.

\section{Single Replacement, Double Replacement, \& Combustion}
Single replacement is when an element replaces another element in a compound.

An element will replace another element in a compound if the lone element is more reactive than the element in the compound.

\textbf{Single Replacement Rules}
\begin{itemize}
    \item Replacement of a metal by a more reactive metal.
    \item Replacement of hydrogen in water by a group 1 metal produces a metal hydroxide and H$_2$
    \item Replacement of hydrogen in water by a group 2 metal produces a metal oxide and H$_2$
    \item Replacement of hydrogen in an acid by a metal more active than H. Metal replaces hydrogen as if it were a metal.
    \item Replacement of a nonmetal (usually a halogen) in a compound by a more reactive nonmetal.
\end{itemize}

\ex Write the equation of Fluorine + Potassium Bromide

Double replacement reactions occur when elements in two compounds exchange places to make two new compounds.

These reactions occur between ions in aqueous solutions and produce at least one of the following - a precipitate, a gas, or water

If a product is insoluble, it is a precipitate.

Note that hydrogen sulfide is a gas.

H$_2$CO$_3$ decomposes to carbon dioxide and water and H$_2$SO$_3$ decomposes to sulfur dioxide and water.

Ammonium hydroxide decomposes to form ammonia gas (NH$_3$) and water.

\ex Write the equation for Sodium bicarbonate + hydrochloric acid 

The burning of a hydrocarbon in O$_2$ to produce heat is combustion.

When hydrocarbons burn in excess oxygen, the products are always carbon dioxide and water.

If there is too little oxygen, carbon monoxide is produced. Carbon monoxide is highly toxic!
\section{Reaction Rates}
Reversible reactions - some reactions continue until the products being to react and form the reactants again.

Equilibrium - the point in a reaction when the rate of the forward reaction is equal to the rate of the reverse reaction.

Reaction rate is how fast a chemical reaction will occur.

Molecular collisions are necessary for two substances to react. Many factors affect how often molecules collide, and therefore affect the reation rate.

There are several factors that affect the speed of a reaction:
\begin{itemize}
    \item Temperature of reactants 
    \item Concentration of reactants 
    \item Presence of a catalyst or an inhibitor 
    \item Surface area of reactants 
\end{itemize}

Raising the temperature of a substance causes its molecules to move faster. Faster molecules will collide 
more often, increasing the speed of the reaction. Therefore, higher temperature results in a faster reaction.

Increasing the concentration results in more reactants in a given space, so you will have more collisions per unit time.

A catalyst is a substance that helps molecules come together. It is not used up in a reaction, it just speeds the reaction.

An inhibitor prevents molecules from reacting with each other, thus slowing the reaction rate.

Reactions depend on collisions. The more surface area on which collisions can occur, the faster the reaction.

Some reactions would never happen unless energy is added to the system.
\section{Redox Reactions}
Redox reactions are reactions in which elements' oxidation numbers (charges) change due to moving electrons.

Redox stands for reduction-oxidation reactions. Electrons move from one atom to another or from one ion to another. This means the oxidation numbers of elements change from the reactant to the product side of an equation.

Many types of reactions classify as redox. This isn't a totally separate type of reaction.

Oxidation is loss of electrons and reduction is gain of electrons. 

Loss of electrons means the charge goes up and gain of electrons means that the charge goes down.

The element that is oxidized comes from the reactant that is the reducing agent, and the element that is reduced 
comes from the reactant that is the oxidizing agent.

\ex What is being oxidized, reduced, and the reducing agent, and the oxidizing agent in 2Na + Cl$_2$ $\rightarrow$ 2NaCl 
\section{Net Ionic Equations}
A net ionic equation does not show the ions that don't change (i.e do not include the ions that stay aqueous)

Steps for Writing Net Ionic Equations:
\begin{enumerate}
    \item Write the balanced equation with all states labeled. 
    \item Split any aqueous ionic or strong acids into ions.
    \item Cancel out any ions that appear on both sides of the arrow.
\end{enumerate}

Note that the diatomic elements in aqueous form are no longer diatomic.

\ex Write the net ionic equation for a solution of lead(II) nitrate is added to hydrochloric acid. 

\section*{Chapter Problems}
\begin{enumerate}
    \item Balance the equation H$_2$ + O$_2$ $\rightarrow$ H$_2$O
    \item Write a balanced equation for ammonium bicarbonate yielding ammonia, water, and carbon dioxide 
    \item Identify the reaction of magnesium oxide plus water as synthesis or decomposition and complete and balance the equation 
    \item Make up a decomposition equation 
    \item Use the activity series to determine if the single replacement reaction for mercury and nickel (II) carbonate occurs. If it occurs, write the equation and balance, otherwise write no reaction.
    \item Write the balanced equation for lead (II) nitrate + sodium chloride 
    \item Determine the type of reaction for propane (C$_3$H$_8$) oxygen and write the balanced equation for this reaction 
    \item Determine the type of reaction for sodium and water and write the balanced equation.
    \item Write the net ionic equation for chlorine gas reacting with a solution of potassium iodide.
    \item Write the full ionic equation for a solution of copper (II) chloride reacting with a solution of lead (II) acetate.
\end{enumerate}
\end{document}