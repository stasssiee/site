\documentclass[../hchem.tex]{subfiles}
\graphicspath{{\subfix{../figures/}}}
\begin{document}
\chapter{Acids and Bases}
\section{Acids \& Bases}
Acids and bases are two types of chemicals that have special properties.

They react with each other in a certain way - but they are not exact opposites of each other.

Acids:
\begin{itemize}
    \item Acids are substance that dissociate in water to form hydrogen ions, which combine with water to form hydronium ions.
    \item The presence of these ions are what give a substance acidic properties.
    \item The H$^+$ can come directly from the acid molecule or from the acid's interaction with water. Either way, the chemical yields H$^+$.
\end{itemize}

Physical and Chemical Properties of Acids:
\begin{itemize}
    \item Taste sour 
    \item Produce hydrogen gas when they react with metals 
    \item React with carbonates to form carbon dioxide and water.
    \item Turns litmus paper red.
    \item Electrolytes.
\end{itemize}

Strong acids dissociate nearly 100\% and weak acids dissociate very little.

Polyprotic acids are acids that have more than one ionizable hydrogen. The hydrogens are dissociated one at a time.

Bases:
\begin{itemize}
    \item Bases are substances that react in water to form hydroxide ions.
    \item The presence of hydroxide is what gives bases their basic properties.
    \item The OH$^-$ can come from the basic molecule or from an interaction with water.
    
    Properties:
    \begin{itemize}
        \item Taste bitter 
        \item Feel slippery
        \item Turns litmus paper blue 
        \item Electrolytes in water solution 
        \item Can damage tissue if strong.
    \end{itemize}
\end{itemize}

Acids and bases react with each other to produce neutral products.

Acid + Base $\rightarrow$ Water + Salt 

Arrhenius Definition:
\begin{itemize}
    \item Arrhenius acids contain hydrogen ions.
    \item Arrhenius bases contain hydroxide ions.
\end{itemize}

Bronsted-Lowry Acids \& Bases 
\begin{itemize}
    \item Bronsted-Lowry acids are any substances that donate hydrogen ions (which is a proton, so called ``proton donors''.)
    \item Bronsted-Lowry bases are any substances that can accept hydrogen ions (from an acid or water - ``proton acceptors'')
\end{itemize}

\begin{itemize}
    \item Lewis acids are electron pair acceptors.
    \item Lewis bases are electron pair donors.
\end{itemize}

Recall that when an acid dissolves in water, it donates a H$^+$ ion to a water molecule, forming hydronium.

The reverse reaction is also an acid/base reaction. In the reverse reaction, we call them conjugates.

We've seen water act like an acid and a bse.

This is called amphoteric. Several substances can act as acid or base, depending on what it is paired with.

The pH scale is used to indicate how acidic or basic a solution is.
\begin{itemize}
    \item pH = 7 = neutral 
    \item pH $<$ 7 = acidic 
    \item pH $>$ 7 = basic 
\end{itemize}

Because H$^+$ ions make something acidic, the more H$^+$ ions present, the more acidic the solution is.

The molar concentrations of H$^+$ are usually small, written in scientific notation, and hard to compare.

The pH scale, 0-14, is easier to interpret.

pH = -log[H$^+$]

A change of one pH unit represents a tenfold change in H$^+$ concentration.

The pOH is a similar scale that mirrors the pH and [H$^+$] relationship in the pH scale.

pOH = -log[OH$^-$]

pH + pOH = 14 for any given solution. 

If you know the pH or pOH, you can figure out the ion concentrations by working backwards.

[H$^+$] = 10$^{-\text{pH}}$ and [OH$^-$] = 10$^{-\text{pOH}}$

Substances that change color as pH changes are acid base indicators.
\section{Titrations}
The purpose of titrations are to determine the concentration of a solution.

To determine the concentration, if the unknown is the base, you titrate it with an acid of known concentration.

You titrate until the unknown is neutralized as indicated by a change in pH which will change the color of an indicator.

Moles H$^+$ will equal moles OH$^-$.

Neutralization is a reaction of an acid and a base to produce a salt and water.
\begin{itemize}
    \item Water is produced from a union of a H$^+$ from the acid and an OH$^-$ from the base.
    \item Salt: ionic compound formed from the positive ion of an aqueous base and the negative ion of an aqueous acid. The salt is not always NaCl.
    \item Equal amounts of H$^+$ and OH$^-$ will neutralize completely.
    \item Salts produced from neutralization may or may not be neutral in solution.
\end{itemize}

The titration formula:
\begin{center}
    $V_aM_a$(\#H$^+$) = $V_bM_b$(\#OH$^-$)
\end{center}

Where V is volume, M is molarity, and \#H$^+$ or \#OH$^-$ is the \# of hydrogen or hydroxide ions in the chemical formula of the acid or base.

Titration: lab procedure in which a carefully measured solution of known concentration is slowly added to a known volume of a second solution to experimentally determine its concentration.

Equivalence point: point of neutralization where [H$^+$] = [OH$^-$]

Endpoint: point where the color change of the indicator occurs.

Performing a titration:
\begin{itemize}
    \item Buret: a measuring instrument with very small volume increments used to perform titrations.
    \item Before beginning a titration, rinse the buret with 5 mL of distilled water. Discard the water.
    \item Then rinse the buret with \~ 5 mL of the known concentration solution. Discard the rinse solution.
    \item THEN fill the buret with the known concentration solution.
    \item Record your initial volume and begin the titration.
    \item The buret showhs how much solution has been emptied. 
    \item Be careful not to overshoot the end point.
\end{itemize}
\section{Molar Mass through Titrations}
Titrations are a method to determine many things:
\begin{itemize}
    \item The molarity of an acid or base 
    \item The molar mass of an acid or base 
    \item The K$_a$ or K$_b$ value of an acid or base.
\end{itemize}

If you know bot hte mass and number of moles of any substance, you can determine its molar mass by dividing the grams by moles.
\section{Acid-Base Equilibrium: Ka \& Kb}
The general reaction for a weak acid is 
\begin{center}
    HA + H$_2$O $\leftrightarrow$ H$_3$O$^+$ + A$^-$
\end{center}

Often we leave out the water part and just show:
\begin{center}
    HA $\leftrightarrow$ H$^+$ + A$^-$
\end{center}

The general equilibrium constant expression for weak acids is shown below:
\begin{center}
    K$_a$ = $\frac{[H^+][A^-]}{[HA]}<1$
\end{center}

The general equation for weak base reactions is 
\begin{center}
    B + H$_2$O $\leftrightarrow$ HB$^+$ + OH$^-$
\end{center}

The general equilibrium constant expression for weak bases is shown below:
\begin{center}
    K$_b$ = $\frac{[HB^+][OH^-]}{[B]}<1$
\end{center}

Note that the concentrations that you substitute in are at equilibrium, meaning after the acid/base has dissociated to its maximum extent. Notice that 
water is not included in the K expression even if it is in the ionization/dissociation equation.

The K value can help us determine how much of the acid or base will ionize, called ``x''. We use RICE tables to organize our calculations and information.
\begin{itemize}
    \item Reaction 
    \item Initial Amount 
    \item Change 
    \item Equilibrium 
\end{itemize}
There are two basic types of acid/base equilibria problems:
\begin{itemize}
    \item You need to calculate K$_a$ or K$_b$ (you have x)
    \item You need to calculate x (you have K$_a$ or K$_b$)
\end{itemize}

Exercise - The weak acid hydrogen fluoride can be used to etch glass. In a 0.25 M solution of HF, the fluoride ion concentration is found to be 0.012 M. Find the K$_a$ value for HF. ($6.0\times 10^{-4}$)
\end{document}