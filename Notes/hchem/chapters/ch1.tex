\documentclass[../hchem.tex]{subfiles}
\graphicspath{{\subfix{../figures/}}}
\begin{document}
\chapter{Nature of Science}
\textit{Chemistry Theme Songs? Sodium-Sodium-Sodium-Sodium Sodium-Sodium-Sodium-Sodium Batman}

\section{Lab Safety \& Equipment}
\begin{itemize}
    \item Goggles must be worn over your eyes at all times! Wearing safety goggles correctly is required to protect your eyes during laboratory investigations.
    \item Your clothing should cover your legs - shorts are not appropriate for the laboratory. Lab aprons can be used to protect good clothing. No loose clothing! It can dip into chemicals or fall into a flame and catch fire.
    \item Sandals and open-toed shoes do not protect your feet from broken glass that is frequently found in the lab. Also, leather shoes protect your feet from chemical spills, shoes do not.
    \item Dangling hair can fall into the Bunsen burner and catch fire or can fall into a chemical solution.
    \item Do not apply cosmetics, eat, or drink in the lab - these activities are ways by which you can accidentally ingest harmful chemicals.
    \item Do not taste chemicals. Do not small any chemicals directly. If you smell chemicals, use your hand to waft vapors to your nose.
    \item Heat test tubes at an angle and away from you and others.
    \item Handle hot glassware with the appropriate tongs.
    \item Never work along in the lab - in the case of a problem, you may need another person to prevent injury or even save your life!
    \item Don't assume you dispose waste down the sink. Dispose of all waste materials according your instructional procedure.
    \item Never remove chemicals from the laboratory.
    \item Wash your hands with soap and water before leaving - this rule applies even if you have been wearing gloves!
    \item Report any accidents or unsafe conditions immediately!
    \item Remember that the lab is a place for serious work! Careless behavior may endanger yourself and others and will not be tolerated!
\end{itemize}

\textbf{Know the safety equipment and how to use the following safety equipment.}
\begin{itemize}
    \item Eye wash fountain 
    \item Safety shower
    \item Fire extinguisher
    \item Emergency exits 
\end{itemize}

\textbf{NFPA Chemical Hazard Label} 
\begin{itemize}
    \item Blue - Health 
    \item Red - Flammability
    \item Yellow - Reactivity (Stability)
    \item White - Special 
\end{itemize}

\textbf{Hazard Ratings}
\begin{itemize}
    \item 4 - Severe 
    \item 3 - Serious 
    \item 2 - Dangerous 
    \item 1 - Minor 
    \item 0 - Slight 
\end{itemize}

\textbf{MSDS} 
\begin{itemize}
    \item Material Safety Data Sheet (now often just called Safety Data Sheet, SDS)
    \item On file for all purchased chemicals.
    \item Includes all information shown on a chemical label and more.
\end{itemize}

\textbf{Lab Equipment} 
\begin{itemize}
    \item Beakers hold liquids. They don't precisely measure.
    \item Test tubes hold small amounts of liquid.
    \item Erlenmeyer flasks are used to hold liquids and swirl mixtures.
    \item Test tube racks hold test tubes.
    \item Bunsen burners heat with intensity.
    \item Hot plates heat at a wide variety of temperatures, from low to high.
    \item Plastic pipettes transfer small, approximate amounts. Not for measuring.
    \item Volumetric flasks are used for making solutions of a specific volume. They only have one line for measuring.
    \item Beaker tongs are used to pick up a hot beaker.
    \item Test tube tongs are used to hold one test tube.
    \item Crucible tongs are used to pick up a crucible or hold something in flame.
    \item Ring stand \& rings are used for holding items over flame for a long period of time or filtering.
    \item Wire gauzes are used to put hot beakers on to prevent shattering.
    \item Balances are used to measure the mass of an object.
    \item Glass pipettes measure small amounts of liquid by suction.
    \item Graduated cylinders measure the volume of a liquid.
\end{itemize}

\ex List the Laboratory Safety DOs and DON'Ts 

\ex What are four pieces of safety equipment and how do you use them?

\ex What does a Safety Data Sheet (SDS) tell you?

\ex Which lab equipment measures chemicals?

\section{Matter, Energy, \& Change}
Chemistry is the science that investigates structures and properties of matter.
\begin{itemize}
    \item Matter - anything composed of atoms
    \item Mass - a measure of how much matter is in an object 
    \item Weight - measure of gravity's pull on matter 
    \item Volume - measure of how much space is taken up
\end{itemize}

\ex What is the difference between mass and weight, and what instruments are used to measure mass and weight?

\textbf{There are two types of data}
\begin{itemize}
    \item Qualitative (qualities)
    \item Quantitative (quantities)
\end{itemize}

\textbf{Graphs} 
\begin{itemize}
    \item Independent Variable - the one that is controlled or consistent; found on the x-axis 
    \item Dependent Variable - the result; found on the y-axis 
\end{itemize}

\textbf{Measurable Properties}
\begin{itemize}
    \item Extensive - property that depends on HOW MUCH matter you have 
    \item Intensive - property that is INDEPENDENT of the amount of matter 
\end{itemize}

\textbf{Physical and Chemical Properties}
\begin{itemize}
    \item A physical property can be observed without a chemical change occurring.
    \item A chemical property can be observed only when a chemical change occurs.
    In physical changes:
    \begin{itemize}
        \item atoms are not rearranged into new substances 
        \item include all changes of state 
        \item changes in size, shape, or dissolving 
    \end{itemize}

    In chemical changes:
    \begin{itemize}
        \item bonds are broken between atoms and new bonds are formed to make new substances.
        \item Chemical changes are usually more interesting than physical changes 
    \end{itemize}
\end{itemize}

\textbf{Four Indicators of a Chemical Change}
\begin{enumerate}
    \item Energy change - heat or light is produced, or a decrease in temperature occurs 
    \begin{itemize}
        \item Exothermic - gives off heat, feels hot 
        \item Endothermic - absorbs heat, feels cool 
    \end{itemize}
    \item Production or evolution of a gas 
    \item Precipitate - a solid is formed when two liquids are mixed together
    \begin{itemize}
        \item The clue that a precipitate has formed is that the liquid turns cloudy, it could be any color.
    \end{itemize}
    \item Color Change 
\end{enumerate}

\ex Is tarnishing a physical of chemical change/property?

\ex Is breaking a physical of chemical change/property?

\ex Is combustion a physical of chemical change/property?

\textbf{Classification of Matter}
\begin{itemize}
    \item Mixture: two or more pure substances that can be separated by physical changes.
    \item Homogeneous Mixture: two or more pure substances mixed evenly. When you look at it, you can't see separate parts.
    \item Heterogeneous Mixture: two or more pure substances mixed unevenly.
    \item Element: one of the 118 pure substances that cannot be separated by chemical change or physical change. Represented by a symbol on the periodic table.
    \item Allotrope: same element with different bonding of atoms (different properties)
    \item Compound: made from atoms that are chemically bonded together. Can be separated by chemical change, but not physical change. Represented by a formula.
\end{itemize}

\ex How is bonding different than mixing?

The Law of Definite Proportions (sometimes called Law of Constant Composition) states that all samples of a compound 
contain the same elements in the same proportion.

The Law of Multiple Proportions states that if elements combine to make more than one compound, the masses will be small, whole number ratios.

The Law of Conservation of Mass states that matter cannot be created or destroyed in any type of change. What you start with is what you end up with, just in a different form.

The Law of Conservation of Energy states that energy cannot be created or destroyed (but it can change forms).

\textbf{The Periodic Table}
\begin{itemize}
    \item Find the zig-zag line.
    \item Metals are to the left of the zig-zag line (except for H)
    \item Non-metals are to the right of the zig-zag line
    \item Elements touching the line are called metalloids
    \item The vertical columns are called groups (or families)
    \item The horizontal rows are called periods.
\end{itemize}

\section{Measurement}
\textbf{Accuracy vs. Precision}
\begin{itemize}
    \item Accuracy - how close a measurement is to the accepted value 
    \item Precision - how close a series of measurements are to each other 
\end{itemize}
Accuracy is correctness, precise is consistency.

Percent error indicates the accuracy of a measurement 
\[ \%error= \left|\frac{accepted-experimental}{accepted}\right|\times 100 \]

\begin{example}
    Juan calculated the density of aluminum three times: $2.75$ g/cm$^3$, $2.68$ g/cm$^3$, and $2.84$ g/cm$^3$. 

    Aluminum has a density of $2.70$ g/cm$^3$. Calculate the average percent error for the three trials.

    The percent error for the trials are $1.85\%$, $0.74\%$ and $5.19\%$ respectively. The average of the three is $2.59\%$.
\end{example}

\ex Suppose you calculate your semester grade in chemistry as 90.1, but you receive a grade 
of 89.4 on your report card. What is your percent error? 

\ex On a bathroom scale, a person always weighs 2.5 lbs less than on the scale at the doctor's office. What is the 
percent error of the bathroom scale if the person's actual weight is 125 pounds?

\textbf{Significant Figures}
\begin{itemize}
    \item Indicate accuracy of a measurement.
    \item Sig figs in a measurement include the known digits plus a final estimated digit.
    \item It is important to be honest when reporting a measurement so that is does not appear to be more accurate than the equipment used to make the measurement.
\end{itemize}

\textbf{Counting Sig Figs}

Count all numbers except
\begin{itemize}
    \item Leading zeroes 
    \item Trailing zeros without a decimal point 
\end{itemize}

\textbf{Rules for Counting Sig Figs}
\begin{itemize}
    \item All nonzero digits are significant 
    \item Sandwiched zeroes are significant 
    \item Zeroes at the beginning are never significant
    \item Zeroes at the end are significant only if you can see the decimal point 
\end{itemize}
Note:
\begin{itemize}
    \item Non significant does mean unaccounted for 
    \item Sig Figs keep track of the accuracy of our measurements 
\end{itemize}

\ex Count the number of sig figs in each number 
\begin{enumerate}
    \item 98
    \item 0.0098000
    \item 980.0
\end{enumerate}

\textbf{Scientific Notation}
Converting into scientific notation:
\begin{itemize}
    \item Move decimal until there's 1 digit to its left. The places moved is the exponent.
    \item A number greater than 1 gets a positive exponent and a number less than 1 gets a negative exponent.
\end{itemize}

\ex Write in scientific notation and keep the same number of significant figures:
\begin{enumerate}
    \item 0.00007
    \item 422000.
\end{enumerate}

\ex Write in standard notation and keep the same number of significant figures:
\begin{enumerate}
    \item 3.1 $\times 10^4$
    \item 1.00 $\times 10^2$
\end{enumerate}

\textbf{Mathematical Operations with Sig Figs}
\begin{itemize}
    \item When combining measurements with differing degrees of accuracy and precision, the 
    accuracy of the final answer can be no greater than the least accurate measurement.
    \item This principle can be translated into simple rules for mathematical operations.
    \item Remember the order of operations and always include units in your answer if units are given in the problem.
\end{itemize}

When adding or subtracting, the answer cannot go beyond the last significant place of the least precise measurement.

When multiplying or dividing, the \# with the fewest sig figs determines the \# of sig figs in the answer.

Exact numbers do not limit the number of significant figures.

\ex 
\begin{enumerate}
    \item 150.0 grams + 0.507 grams
    \item 98.0 grams $\div$ 2.33 liters
\end{enumerate}

Tips: 
\begin{itemize}
    \item Determine which rule you are dealing with first! Add/Sub = least decimal places. Mult/Div = least number of sig figs.
\end{itemize}

\textbf{Density}
\begin{itemize}
    \item Density is the measure of how much mass is contained in a given unit of volume.
    \item It depends on what the matter is, not how much you have.
    \item Density is an intensive property.
\end{itemize}

Density depends on two things:
\begin{enumerate}
    \item How tightly packed the atoms are 
    \item What kind of atoms they are 
\end{enumerate}

Density is calculated with the formula 
\begin{center}
    $D = \frac{m}{V}$.
    
    This can be arranged to solve for mass or volume.
\end{center}

\ex Use algebra to rearrange the density formula to solve for volume.

When working density problems, use the following steps:
\begin{enumerate}
    \item Write the correct formula you'll be using 
    \item Substitute in the correct values with units 
    \item Work the problem with your calculator and give the answer with the correct number of sig figs and correct units 
\end{enumerate}

\ex A metal cylinder is placed into a graduated cylinder with a 24.0 mL of water. After the cylinder is added, the volume of water rised to 30.4 mL. 
The density of the cylinder is known to be 8.9 g/mL. What is the mass of the cylinder?

\ex A metal cylinder has a diameter of 4.4 cm and a height of 10.5 cm. If the cylinder is silver, which has a density of 10.5 g/cm$^3$, what is its mass? The volume of a cylinder is $\pi r^2h$. (Use 3.14 for $pi$.)

\textbf{Proportions}
In a direct proportion, the relationship should be linear.

In an inverse proportion, the relationship will be non-linear and decreasing.

\section{Dimensional Analysis}
First, off the metric system!

S.I. or metric units are: Mass in grams (g), Length in meters (m), Volume in liters (L)

Prefixes to know: kilo = 1000, centi = 1/100, milli = 1/1000

Memorize these conversions!
\begin{itemize}
    \item 1 kg = 1000 g 
    \item 1 g = 100 cg 
    \item 1 g = 1000 mg 
    \item 1 km = 1000 m 
    \item 1 m = 100 cm 
    \item 1 m = 1000 mm 
    \item 1 cm = 10 mm 
    \item 1 L = 1000 mL 
\end{itemize}

Dimensional analysis is the method that chemists (and other scientists) use to solve conversion problems.

\ex  
\begin{enumerate}
    \item Convert 23.9 km to m 
    \item If 1 inch = 2.54 cm, convert 3.00 cm to inches
    \item Convert 25 inches to cm
    \item If 1 gallon = 4.1 L, convert 2.5 gal to L
\end{enumerate}

\ex  
\begin{enumerate}
    \item What is the length of a football field in cm if there are 2.54 cm in an inch and 36 inches in a yard?
    \item Diamonds are measured in units called a carat. One carat equals 200 mg. If a diamond is 0.600 carat, what is the mass of the diamond in ounces?
\end{enumerate}

\section*{Chapter Problems}
\begin{enumerate}
    \item How many significant figures are in 1.003?
    \item Write $0.00007$ in scientific notation, keeping the same number of significant figures.
    \item Write $1.00\times 10^2$ in floating decimal notation (standard notation), keeping the same number of significant figures.
    \item Round 0.003008 to 3 significant figures.
    \item Calculate 45.0 cm - 9.2 cm and round to the correct number of significant figures and include the correct units.
    \item Round $400\times 600$ to the correct number of significant figures.
    \item If candy bars are 3 for one dollar, how much money will you need to buy 46 candy bars?
    \item What volume will be occupied by 7.0 kg of helium if 4.003 g of helium occupies 22.4 L?
    \item Twenty five paper clips are dropped into a graduated cylinder and the water level rises from 10.8 mL to 12.2 mL. If the density of the paper clips is 7.87 g/mL, what is the mass of the 25 paper clips? What is the mass of one paper clip?
    \item A block of wood with a density of 0.548 g/cm$^3$ has a mass of 34.49 g. If two dimensions of the block are 2.5 cm and 7.8 cm, what is the 3rd dimension?
\end{enumerate}


\end{document}