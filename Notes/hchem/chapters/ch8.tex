\documentclass[../hchem.tex]{subfiles}
\graphicspath{{\subfix{../figures/}}}
\begin{document}
\chapter{States of Matter}
\textit{You has mass, you have volume. You matter.}

There are three states of matter.

Solid:
\begin{itemize}
    \item Matter that has both a definite shape and definite volume.
    \item Molecules or atoms are very close together and can only vibrate a little.
    \item They do not move past each other.
\end{itemize}

Liquid:
\begin{itemize}
    \item Matter that has a distinct volume but no specific shape.
    \item Molecules or atoms are close together but have the ability to slide across one another very easily.
\end{itemize}

Gas:
\begin{itemize}
    \item Matter that has no fixed volume or shape. It conforms to the volume and shape of its container.
    \item Its molecules or atoms are very far apart from each other and move very fast.
\end{itemize}

Compression - forcing a substance into a smaller volume.
\begin{itemize}
    \item Gases are very compressible because of their empty space.
    \item Liquids have very little compressibility.
    \item Solids have almost no compressibility.
\end{itemize}

Density Comparison:
\begin{itemize}
    \item If you consider the solid, liquid, and gas state of one particular substance, this rule holds true in most cases:
    \item Solid is more dense than liquid and liquid is more dense than gas.
\end{itemize}

Two Types of Solids:

Crystalline Solids
\begin{itemize}
    \item molecules are packed together in a predictable way. THey are arranged in an orderly, geometric, three dimensional 
    structure. The smallest repeating part of a crystalline structure is called a unit cell.

    Atomic Solids 
    \item Unit particles are atoms.
    \item Noble gases are atomic solids when they are cooled to solid state. Usually very soft because they have weak IMFs.
    
    Molecular Solids 
    \item Units are molecules, held together by weak IMFs. Low melting points.
    
    Covalent Network Solids 
    \item Form a 3-D covalent network, very strong. High melting points.
    
    Ionic Solids 
    \item Crystal lattice is formed from alternating cations and anions.
    \item High melting point and hardness.
    \item Always solids at room temperature.
    
    Metallic Solids 
    \item Atoms are surrounded by mobile valence electrons.
    \item Malleable, ductile, conductors.
\end{itemize}

Amorphous Solids:
\begin{itemize}
    \item particles are not arranged in a regular repeating manner.
    \item Amorphous means ``without shape''
\end{itemize}

Liquids:
\begin{itemize}
    \item Fluidity - liquids have the ability to flow 
    \item Viscosity - the measure of the resistance of a liquid to flow.
    \item Liquids with big, complex molecules tend to be very viscous.
    \item Viscosity decreases with increasing temperature.
\end{itemize}

Buoyancy:
\begin{itemize}
    \item The upward force a liquid exerts on an object.
\end{itemize}

Phase Changes - matter can change from one phase to another by adding or removing energy. There are six phase changes.

Phase Changes That Require Energy
\begin{itemize}
    \item Melting - solid changing to liquid.
    \item Vaporization - liquid to gas, occurs when molecules have enough energy to escape the pull of the other molecules.
    \item Sublimation - solid changing directly into gas 
\end{itemize}

Phase Changes That Release Energy 
\begin{itemize}
    \item Condensation - gas to liquid.
    \item Freezing - liquid to solid, achieved by removing heat.
    \item Deposition - gas directly to solid - achieved by removing heat.
\end{itemize}

Boiling is heating a liquid to the temperature at which all molecules have enough energy to escape and vaporize. Evaporation is the vaporization of surface molecules; very slow. This does not occur at high temperatures.

A phase diagram shows what phase a substance will be in at a certain temperature and pressure. Pressure is usually measured in atmospheres.

Triple point - the point on a phase diagram that shows the temperature and pressure combination at which three phases of a substance can coexist.

Critical point - temperature and pressure combination above which a vapor cannot be liquefied under any circumstances.

When energy/heat is added to or removed from a substance, two things could happen: temperature changes or phase change.

How do you know how much energy is needed for a change?

First off.
\begin{itemize}
    \item Q for heating/adding energy is always positive.
    \item Q for cooling/releasing energy is always negative.
\end{itemize}

For a single phase, use the formula $q=mc\Delta T$, where q is heat in Joules, m is mass in grams, c is the specific heat, and $\Delta T$ is the change in temperature.

For a single temperature, use the formula $g= mol\cdot\Delta H$.

The heat needed to melt or freeze is the latent heat of fusion and the heat needed to boil/condense is the latent heat of vaporization.

\ex How much energy is needed to convert 153 grams of ice at $-15\degree$ C to steam at $125\degree$ C? The molar mass of water is 18.016 g/mol.

\section*{Chapter Problems}
\begin{enumerate}
    \item Which liquid is more viscous at room temperature, water of molasses? Explain your reasoning.
    \item How is it possible that a pile of snow can slowly shrink even on days when the temperature never rises above the freezing point?
    \item How much heat is required to warm 225 g of ice from $-46.8^{\circ}$C to $0.0^{\circ}$C, melt the ice, warm the water from $0.0^{\circ}$C to $100.0^{\circ}$C, boil the water, and heat the steam to $173.0^{\circ}$C?
\end{enumerate}

\end{document}