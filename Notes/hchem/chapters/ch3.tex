\documentclass[../hchem.tex]{subfiles}
\graphicspath{{\subfix{../figures/}}}
\begin{document}
\chapter{Periodicity}
\textit{How often do I like jokes about chemistry? Periodically.}

\section{Introduction to Periodic Table \& Activity}
\textbf{Dmitri Mendeleev}
\begin{itemize}
    \item Many people had arranged the known elements of their day, and Dmitri Mendeleev arranged them by increasing atomic mass.
\end{itemize}
In 1869 when created, he left gaps and predicted some elements that had yet to be discovered. Later, they were and they fit into his table perfectly.

\textbf{Henry Moseley}
\begin{itemize}
    \item Moseley's periodic table was similar, but he arranged them in order of increasing atomic number, not mass 
    \item Remember that atomic number is the same as number of protons.
    \item This is the periodic table we use today.
\end{itemize}

\textbf{Mass vs. Number}
\begin{itemize}
    \item Increasing atomic mass and atomic number are not exactly the same 
    \item On our modern periodic table (Moseley's) there are a few atomic masses out of order. That's okay because we organize it by atomic number.
\end{itemize}

\textbf{Modern Periodic Law}
\begin{itemize}
    \item States that the physical and chemical properties of elements repeat when they are arranged by increasing atomic number.
\end{itemize}

\textbf{Classification of Elements}
\begin{itemize}
    \item The zig-zag line divides periodic table into two parts.
    \item Left of zig-zag line are metals.
    \item Right of zig-zag line are nonmetals.
    \item The elements touching the line are metalloids. 
\end{itemize}

\textbf{Properties of Metals}
\begin{itemize}
    \item Usually silver-gray in color, except gold \& copper 
    \item Solid at room temperature, except mercury
    \item Lustrous or shiny appearance 
    \item Malleable
    \item Ductile
    \item Good conductors
    \item Usually react with acids 
    \item High melting points
\end{itemize}

\textbf{Properties of Nonmetals}
\begin{itemize}
    \item Dull
    \item Brittle (nonmalleable)
    \item Poor conductors of heat and electricity
    \item Usually no reaction with acids 
    \item Gases, liquids, or low-melting-point solids 
\end{itemize}

\textbf{Properties of Metalloids}
\begin{itemize}
    \item All elements touching zig-zag line, except aluminum which is a metal 
    \item Exhibit properties of both metals and nonmetals 
    \item Not good conductors alone
\end{itemize}

The metals can be divided up into smaller groups.

\textbf{Alkali Metals}
\begin{itemize}
    \item Group 1 of the periodic table 
    \item Have one valence electron
    \item Very reactive
    \item Form +1 ions 
    \item The exception in group 1 is hydrogen, which is not an alkali metal 
\end{itemize}

\textbf{Alkaline Earth Metals}
\begin{itemize}
    \item Group 2
    \item Two valence electrons
    \item Form +2 ions 
    \item Less reactive than group 1
\end{itemize}

Blocks:
\begin{itemize}
    \item The periodic table is divide up into four blocks, the s block, the p block, the d block, and the f block, based on electron arrangement.
    \item The s-block is all elements in groups 1 and 2.
    \item Groups 3-12 have transition metals and are called the d-block. They do not follow patterns as well as groups 1, 2, and 13-18. The number of valence electrons are harder to predict and they can have a variety of charges.
    \item Al, Ga, In, Sn, Tl, Pb, Bi are sometimes called ``poor metals'' because they don't have perfectly metallic properties 
    \item Metalloids are the elements touching the zig-zag line, except aluminum which is a metal. These are commonly used in electronics as a semiconductor
\end{itemize}

Rare Earth Elements:
\begin{itemize}
    \item The Lanthanide and Actinide series 
    \item The Lanthanide series is part of Period 6
    \item The Actinide series is part of Period 7
    \item These are found in the f-block and are also called rare earth elements 
\end{itemize}

There are also a few groups of elements that are nonmetals.

\textbf{Halogens}
\begin{itemize}
    \item Group 17 of the table 
    \item Have 7 valence electrons
    \item Form -1 ions
    \item Very reactive, especially with the alkali metals.
\end{itemize}

\textbf{Noble Gases (Inert Gases)}
\begin{itemize}
    \item Group 18 of the PT.
    \item Octet of valence electrons (full valence shell)
    \item Tend not to form ions 
    \item Inert (do not react)
\end{itemize}

\textbf{p-block}
\begin{itemize}
    \item Groups 13-18 are called the p block 
    \item The p-block has a few metals: Al, Ga, In, Sn, Tl, Pb, Bi, Po 
    \item The p-block also contains metalloids and nonmetals 
\end{itemize}

Once you know which group an element is in, the number of valence electrons that element has is predictable.

Once you know which group an element is in, the charge of the ion that element forms is likewise predictable.

\ex Calcium is in which block?

\ex Uranium is in which block? 

\ex Silicon is in which block? 

\section{Periodic Trends}
Periodic Trends are patterns that appear on the periodic table.

\textbf{4 factors that cause the trends}
\begin{itemize}
    \item Nuclear Pull (Z) - the number of protons 
    \begin{itemize}
        \item The protons pull on the outer electrons. The more protons, the more pull exerted by the nucleus on the outer electrons.
    \end{itemize}

    \ex Which of the following elements has the most nuclear pull? Carbon or Fluorine?

    \item Electron repulsion - size of the e$^-$ cloud.
    \begin{itemize}
        \item The more electrons in an atom's electron cloud, the more they are pushed away from each other, making a bigger cloud.
    \end{itemize}

    \item Shielding electrons - all inner e$^-$ shield the valence electrons from nuclear pull 
    \begin{itemize}
        \item Electrons on the inner shells feel the nuclear pull stronger than the valence electrons, which are farther from the nucleus 
    \end{itemize}

    \item Z$_{\text{eff}}$ - the ``effective'' nuclear pull on outer electrons. This takes into account the shielding electrons which are taking most of the force.
\end{itemize}

\textbf{Atomic Radius Trend}

Atomic radius increases down a column because the valence electrons are in a farther energy level and 
decrease across a period because the nuclear pull is increasing and pulling the energy levels in.

\textbf{Ionic Size}

Metals ions are smaller than their atoms because metal ions lose electrons causing electron repulsion and smaller size.

Nonmetal ions are larger than their atoms because they are gaining electrons, causing more electron repulsion, and larger size.

\textbf{Ionization Energy}

The energy needed to pull an electron from an atom.

The greater the ionization energy, the more difficult it is to remove an electron.

This decreases down a group because there are more shielding electrons, so it takes less energy to ``steal'' and electron. This increases across a period because the nuclear 
pull on those electrons is increased with no extra shielding, so its takes more energy to get the electrons away.

\textbf{Electronegativity}

The ability of an atom to take an electron from another atom.

This decreases down a group because there are more electrons to shield the nucleus. This increases across a period because of increased Z.

\textbf{Electron Affinity}

The energy change that occurs when an atom acquires an electron.

Most atoms give off energy when gaining an electron, the more attracted an atom is to the new electron, the more energy released.

Therefore, the trend correlates with electronegativity.

$\textbf{Z}_{\text{eff}}$

The effective nuclear charge - the nuclear pull as felt by the valence electrons.

Equal to the number of protons in the nucleus minus the number of electrons that are between the nucleus and the valence electrons.

No change down a group, because even though nuclear pull has increased, you have more shielding e$^-$'s.

Increases across a period because nuclear pull is increasing and no additional shielding.

\textbf{Reactivity}

Most reactive corners of the PT are lower left and upper right.

This is because metals tend to donate electrons to obtain their octet. The most reactive metals are therefore the ones with the lowest ionization energy.

Nonmetals tend to gain electrons to obtain their octet. The most reactive nonmetals are on the upper right because they have the highest electronegativity.

\ex Which is the smallest atom? Na, Li, or Be?

\ex Which has the highest electronegativity? As, Sn, or S?

\ex In the following pairs, which have the larger atomic radius? Mg or Ba, Cu or Cu$^{2+}$, S or S$^{2-}$. 

\ex In the following pairs, which has the higher ionization energy? Li or Cs, Ca or Br.

\section*{Chapter Problems}
\begin{enumerate}
    \item Explain why it takes more energy to remove the second electron from a lithium atom than it does to remove the first electron from a lithium atom.
    \item How does the ionic radius of a nonmetal compare with its atomic radius? Explain why the change in radius occurs.
    \item Experiments show that the electronegativity of phosphorus is 2.5, and the electronegativity of chlorine 3.5. Explain the difference in electronegativity using principles of atomic structure.
    \item Explain how ionization energy changes as you move left to right across a period using principles of atomic structure. 
\end{enumerate}

\end{document}