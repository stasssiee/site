\documentclass[../hchem.tex]{subfiles}
\graphicspath{{\subfix{../figures/}}}
\begin{document}
\chapter{Atomic Structure and Energy of Electrons}
\section{Atomic Theory \& Structure}
\textbf{Theories vs. Laws}
\begin{itemize}
    \item A theory is an explanation based on many observations.
    \item A law is a fact of nature that is observed so often it is accepted as truth.
    \item Theories EXPLAIN laws
\end{itemize}
\begin{itemize}
    \item Both a scientific theory and scientific law are accepted to be true by the scientific community as a whole.
    \item A theory is like a car. Components of it can be changed or improved upon, without changing the overall truth of the theory as a whole.
\end{itemize}

\textbf{What is atomic theory?}
\begin{itemize}
    \item The idea that matter is made up of atoms, the smallest pieces of matter.
    \item Over the years, atomic theory has evolved and changed to better explain scientific observations about atoms.
\end{itemize}

\begin{itemize}
    \item Ancient Greeks believed all matter was made up of four basic elements: fire, earth, water and air.
    \item Democritus 
    \begin{itemize}
        \item Greek philosopher
        \item Idea of `democracy'
        \item Idea of `atomos'
        \begin{itemize}
            \item Atomos = `indivisible'
            \item `Atom' is derived 
        \end{itemize}
        \item No experiments to support idea
    \end{itemize}
\end{itemize}
Democritus's model of the atom consisted of a solid and indestructable atom with no protons, electrons, or neutrons.

\begin{itemize}
    \item Lavoisier 18th century 
    \item Proposed the law of conservation of mass/matter.
    \item Observed that the mass of the reactants equaled the mass of the products in a chemical reaction.
\end{itemize}

\begin{itemize}
    \item Proust - Proposed the law of definite proportions for compounds.
\end{itemize}

\textbf{Dalton's Atomic Theory}
\begin{itemize}
    \item All matter is made of tiny indivisible particles called atoms.
    \item Atoms of the same element are identical, those of different elements are different.
    \item Atoms of different elements combine in whole number ratios to form compounds.
    \item Chemical reactions involve the rearrangement of atoms. No new atoms are created or destroyed.
\end{itemize}

\textbf{Thomson}
\begin{itemize}
    \item J.J. Thomson - English physicist. 1897
    \item Made a piece of equipment called a cathode ray tube. It is a vacuum tube - all the air has been pumped out.
    \item Thomson's Model - Plum Pudding Model (also called Chocolate Chip Cookie Model)
    \begin{itemize}
        \item Atoms are composed of charged particles (subatomic particles).
        \item The particles that were attracted to the positive plate were negative.
        \begin{itemize}
            \item These were called ``electrons''
            \item Protons were discovered the same way.
        \end{itemize}
    \end{itemize}
\end{itemize}

\textbf{Rutherford 1895}
\begin{itemize}
    \item Experiment: Gold Foil Experiment 
    \item Most particles pass through, but some are bounced back towards the source.
    \item Model: Rutherford explained that atoms must be mostly empty space with a small, concentrated center of positive charge.
\end{itemize}

\textbf{Chadwick}
\begin{itemize}
    \item Discovered the neutron.
    \begin{itemize}
        \item Neutron is a subatomic particle roughly the size of a proton (large compared to electrons).
    \end{itemize}
\end{itemize}

\textbf{Bohr}
\begin{itemize}
    \item Model: proposed the ``electron cloud'' in which electrons orbit at a given distance from the nucleus.
    \item Small orbits = low energy 
    \item Big orbits = high energy 
\end{itemize}

\textbf{Quantum Mechanical Model}\\
Modern atomic theory describes the electronic structure of the atom as the probability of finding electrons within certain regions of space (orbitals).

\textbf{Modern View}
\begin{itemize}
    \item The atom is mostly empty space
    \item Two regions 
    \begin{itemize}
        \item Nucleus 
        \begin{itemize}
            \item protons and neutrons 
        \end{itemize}
        \item Electron cloud 
        \begin{itemize}
            \item region where you might find an electron 
        \end{itemize}
    \end{itemize}
\end{itemize}

\section{Structure of Atom \& Isotopes}
\textbf{Major Parts of the Atom}
\begin{itemize}
    \item Nucleus: dense, central part of the atom 
    \item Protons and neutrons are found in the nucleus
    
    \item Electron cloud: large area outside of the nucleus 
    \item Electrons occupy the electron cloud
\end{itemize}

Protons are located in the nucleus with positive charge and have a large relative size.

Nuetrons are located in the nucleus with 0 charge and with a large relative size.

Electrons are located in the electron cloud, have a negative charge and have a tiny relative size.

\textbf{Atoms and the Periodic Table}
\begin{itemize}
    \item Atomic Number - the whole number in an element's box on the periodic table.
    \begin{itemize}
        \item Atomic \# = \# protons = \# electrons 
        \item The atomic number determines an element's identity!
    \end{itemize}
\end{itemize}

Exercise - An atom has 24 protons. What element is it? (chromium)

\begin{itemize}
    \item Mass Number - the sum of the protons and neutrons 
    \item This number isn't on the periodic table, because the number of neutrons can vary (these are called isotopes)
    
    \item Atomic Mass - the decimal number on the periodic table. The weighted average mass of all isotopes of that element. 
    
    \item Isotopes - atoms of the same element that have different mass numbers.
    \item This means the number of protons is the same, and the number of neutrons is different.
\end{itemize}

\textbf{Isotopes of Hydrogen}
\begin{itemize}
    \item Protium - 1 proton, 1 electron, mass number of 1 
    \item Deuterium - 1 proton, 1 neutron, 1 electron, mass number of 2 
    \item Tritium - 1 proton, 2 neutrons, 1 electron, mass number of 3
\end{itemize}

\textbf{How to write isotopes}
\begin{itemize}
    \item Method 1: Subscript/Superscript Method 
    \item The atomic \# is your subscript (below) and the mass \# is the superscript (above), both on the left side of the symbol
    
    \item Method 2: Hyphen-notation method 
    \item This symbol is written, then hyphen, then mass \#
\end{itemize}

Exercise - given ruthenium and the super/sub method of $^{101}_44$Ru, 
write the atomic number, mass number, number of protons, neutrons, and electrons and the hyphen method for this element.

Answer: atomic number - 44, mass number - 101, protons - 44, neutrons - 57, electrons - 44, hyphen method - Ru-101
\section{Average Atomic Mass}
\begin{itemize}
    \item Atoms can't be easily measured in grams because they are so small.
    \item Scientists devised ``atomic mass units'' - a carbon-12 isotope is 12.000000 amu's.
\end{itemize}

\textbf{Average Atomic Mass}
\begin{itemize}
    \item A different kind of average - a ``weighted'' average.
    \item This means that we take into account the abundance of each isotope found in nature.
\end{itemize}

Formula to memorize:
\begin{center}
    [(mass)(abundance)+(mass)(abundance)+(mass)(abundance)]/100.000
\end{center}
\begin{itemize}
    \item That's for 3 isotopes. Use the (mass)(abundance) for as many isotopes as there are.
    \item The 100 won't limit sig figs in your answer. Your answer is limited by whichever mass or abundance has the fewest sig figs.
\end{itemize}

Exercise - Argon has three isotopes with the following percent abundances: Ar-36 with a mass of 
35.968 amu and an abundance of 0.3337\%. Ar-38 with a mass of 37.963 amu and an abundance of 0.063\%.
Ar-40 with a mass of 39.962 amu and an abundance of 99.600\%. Calculate the average atomic mass. (40. amu)

Exercise - The atomic weight of gallium is 69.72 amu. The masses of naturally occurring isotopes are 68.92 amu for Ga-69 
and 70.92 amu for Ga-71. Calculate the percent abundance of each isotope. (Ga-71: 40\%, Ga-69: 60\%)
\section{Moles}
\begin{itemize}
    \item A mole is the amount of substance that contains the same number of atoms as 12 grams of Carbon-12.
    \item It is a counting unit just like a dozen.
    \item A mole is $6.02\times 10^{23}$ of something.
\end{itemize}

\begin{itemize}
    \item $6.02\times10^{23}$ is called ``Avogadro's Number'' because Amedeo Avogadro discovered it.
    \item 1 mole of any element has a mass (in grams) equal to its average atomic mass.
\end{itemize}

Exercise - 1 mole of potassium has a mass of \_\_\_\_\_ g. (39.10)

\textbf{Molar Mass}
\begin{itemize}
    \item When we write out the average atomic mass in ``grams'' we call this the molar mass - it is literally the mass of one mole.
\end{itemize}

Exercise - What is the molar mass of fluorine? (19.00 grams)

\textbf{Conversions}

1.0000 mole of any substance equals $6.02\times10^{23}$ atoms of that element equals molar mass in grams of that element.

To do a molar conversion problem:
\begin{itemize}
    \item Do dimensional analysis.
    \item Start with what you're given.
    \item Bring that unit down and over.
    \item Put the unit you want on top.
    \item Fill in the numbers.
    \begin{itemize}
        \item Put a ``1'' in front of moles in a conversion.
        \item Put ``$6.022\times10^{23}$'' in front of atoms in a conversion.
        \item Put the molar mass in front of grams in a conversion.
    \end{itemize}
\end{itemize}

Exercise - How many atoms are in 55.4 grams of lithium? (4.81$\times10^{24}$ atoms)

Exercise - What is the mass in grams of $3.011\times10^{23}$ atoms of iron? (27.93 g Fe)

Exercise - How many atoms are in 8.43 moles of nickel? ($5.07\times 10^{24}$ atoms Ni)

Exercise - How many atoms are in $1.00\times10^{-10}$ grams of gold? ($3.06\times 10^{11}$ atom Au)
\section{Electron Configuration}
\textbf{Energy Level}
\begin{itemize}
    \item The region surrounding the nucleus where an electron is likely to be found.
    \item Think of rungs on a ladder, fixed levels with space in between.
    \item Sublevel - smaller part of an energy level indicated by letters (1s, 2s, 4d, etc.)
    \item Orbital - smaller part of a sublevel, each orbital holds 2 electrons, moving in opposite direction... (4 possible shapes)
\end{itemize}

\begin{itemize}
    \item ``Electron configuration'' describes the location of electrons in a given atom. This determines how an element behaves chemically, and thus is the core of chemistry.
\end{itemize}

We'll learn three ways to show electron configuration 
\begin{itemize}
    \item Orbital Notation 
    \item Electron Configuration 
    \item Lewis Dot Structures
\end{itemize}

Aufbau Principle - electrons enter orbitals of lowest energy first. Low energy orbitals are closer to the nucleus.

Pauli Exclusion Principle - no two electrons can be in the same orbital moving the same way. Each electron is unique.

Hund's Rule - when electrons are filling up orbitals of equal energy (say for instance 3 orbitals, which is 6 electrons), one electron enters 
each orbital until they're half-filled with electrons spinning in the same direction, then they fill with the opposite-spin electrons 

Orbital Notation
\begin{itemize}
    \item Numbers represent energy levels and letters represent sublevels 
    \item Lines represent 1 orbital each (can also use boxes)
\end{itemize}

Electron configuration notation 
\begin{itemize}
    \item Write coefficient \& letter for each energy sublevel.
    \item Superscript (number on top) shows \# of electrons at that sublevel.
    \item This method simply takes less space.
\end{itemize}

Shorthand Notation
\begin{itemize}
    \item If you had to show the electron configuration for bismuth, it would be long. There is a way to shorten what you have to write.
    \item Use the symbol for the noble gas before the element you are using and put it in brackets. That represents all the electrons up until that noble gas. Then continue with the rest of the electron configuration.
\end{itemize}

\textbf{f-block issues}
\begin{itemize}
    \item Period 6
    \begin{itemize}
        \item f-block includes elements La to Yb
        \item d-block includes elements Lu to Hg 
    \end{itemize}
    \item Period 7
    \begin{itemize}
        \item f-block includes Ac to No 
        \item d-block includes Lr to Uub 
    \end{itemize}
\end{itemize}

Lewis Dot Notation 
\begin{itemize}
    \item Lewis Dot diagrams show electrons available for bonding. These are the outermost electrons (valence electrons).
    \item Valence electrons are the total electrons in the last energy level (highest coefficient).
    \item Notice that electrons do not pair up until all four sides have one electron already.
\end{itemize}

Exercise - Halogens have how many valence electrons? (7)

Exercise - Copper is part of which block? (d)

Exercise - Which group contains the alkaline earth metals? (2)

Exercise - Which block do the lanthanide and actinide series belong to? (f)
\section{Ion Electron Configurations}
\begin{itemize}
    \item How do positive ions (cations) form?
    Atoms (typically metals) lose electrons.

    \item How to negative ions (anions) form?
    Atoms (typically nonmetals) gain electrons. 
\end{itemize}

When representative elements (s and p block) become ions, they take on the electron 
configuration of the nearest noble gas. This gives them 8 valence electrons.

Exercise - Write the electron configuration for the nitrogen atom (1s$^2$2s$^2$2p$^3$)

Exercise - Which noble gas does the nitrogen ion mimic? (neon)

\begin{itemize}
    \item Transition metals (Groups 3-12) often have variable charges 
    \item Use these guidelines to help figure out their electron configuration
    \begin{itemize}
        \item Transition elements usually lose their s and p electrons first. 
        \item Completely full, half-full, or empty sublevels are stable.
        \item Electrons can move from s-sublevels to d-sublevels if it makes the atom more stable.
    \end{itemize}
\end{itemize}

Exercise - What is the electron configuration for a copper atom? 
\begin{itemize}
    \item cuprous, Cu$^+$ - 1s$^2$2s$^2$2p$^6$3s$^2$3p$^6$3d$^{10}$
    \item cupric, Cu$^{2+}$ - 1s$^2$2s$^2$2p$^6$3s$^2$3p$^6$3d$^9$
\end{itemize}

\textbf{Memorizing Monatomic Ions}
\begin{itemize}
    \item Monatomic cations - attach ``ion'' to the element name 
    \item Monatomic anions - change the element ending to ``-ide''
    \item The systematic name just uses a Roman numeral to indicate the charge. Used for transition metals (variably charged)
\end{itemize}

\textbf{Memorizing Polyatomic Ions}
\begin{itemize}
    \item Help with formulas
    \begin{itemize}
        \item Does the polyatomic ion contain an element in ``the elbow''?
        \item If so, the ion ``-ate'' 3 oxygen atoms 
        \item If not, the ion ``-ate'' 4 oxygen atoms 
        \item ``-ite'' ions contain 1 less oxygen atom than ``-ate''
        \item ``hypo-x-ite'' ions have 1 less oxygen atom than ``-ite''
        \item ``per-x-ate'' ions have 1 more oxygen atom than ``-ate''
    \end{itemize}
    \item Help with charges 
    \begin{itemize}
        \item There are only two polyatomic cations; the rest are anions 
        \item If the polyatomic ion contains oxygen, look at what group the other element is in 
        \begin{itemize}
            \item If it's an even \# group, the ion charge is even 
            \item If it's an odd \# group, the ion charge is odd 
        \end{itemize}
    \end{itemize}
\end{itemize}

\section{EM Spectrum}
\textbf{The Wave-Particle Theory}
\begin{itemize}
    \item A theory that attempts to explain how electrons can behave in two different ways
    \begin{itemize}
        \item as waves (like light)
        \item as particles (like a ball)
    \end{itemize}
\end{itemize}

First, we will look at wave behavior.

Light consists of electromagnetic waves that travel $3.00\times 10^8$ m/s.
\begin{itemize}
    \item That's 670,616,629 miles per hour! 
    \item This is the ``speed of light'', also known as ``$c$''
\end{itemize}

\textbf{Electromagnetic Waves}
\begin{itemize}
    \item The electromagnetic (EM) spectrum is a series of waves that have different wavelengths.
    \item Visible light is small portion of the EM spectrum, with mid-energy.
    \item EM waves are also called radiation.
\end{itemize}

\textbf{EM Wave Characteristics}
\begin{itemize}
    \item Amplitude - height from origin to crest
    \item Frequency - number of cycles that pass a given point in a given amount of time 
    \begin{itemize}
        \item Measured in Hertz (Hz)
        \item 1 Hz = 1 wave passes per second 
        \item 1 Hz = 1/s = s$^{-1}$
        \item Symbol is nu, $\nu$
    \end{itemize}
    \item Wavelength - distance between crests of a wave 
    \begin{itemize}
        \item Symbol is lambda, $\lambda$
    \end{itemize} 
\end{itemize}

All EM Waves move at the speed of light 
\[c=\lambda \nu\]
As wavelength increases, frequency decreases. They are inversely proportional.

Exercise - What is the wavelength of a wave with a frequency of 7600 Hz? (39000 m)

Important conversions:
\begin{itemize}
    \item 1 m = $1\times10^9$ nm 
    \item 1 MHz = $1\times 10^6$ Hz 
\end{itemize}

Exercise - What is the frequency of a wave with a wavelength of 467 nm? ($6.42\times10^{14}$ Hz)

\begin{itemize}
    \item The visible spectrum of ROYGBIV is continuous; there are no breaks and the colors blend together.
    \item White light is a combination of all colors of light. A prism breaks up white light into the separate colors so we can see them.
    \item Each color has a definite frequency and wavelength. 
    \begin{itemize}
        \item The speed these colors of light are traveling never changes; it's always the speed of light, $c$
    \end{itemize}
\end{itemize}

Low energy colors have a long wavelength and low frequency, while high energy colors have a short wavelength and high frequency.

\begin{itemize}
    \item Remember that electrons occupy energy levels.
    \item When electrons are in the lowest energy level, they are said to be in their ground state 
    \item It is possible for electrons to jump from ground state to a higher energy level (called excited state) by absorbing energy.
    \item When electrons lose energy they will fall back down to their ground state and release energy, and some of it is released as waves we can see - LIGHT!
    
    \item With many electrons jumping to energy levels and falling back, many different shades of light are released and blended. 
    \item We can use a prism to separate the light to see the individual shades.
    \item This is called an atomic emission spectrum.
\end{itemize}

\textbf{Types of Spectra}
\begin{itemize}
    \item Continuous Spectrum - no breaks 
    \item Atomic Emission Spectrum - a lot of black space, aka ``bright line'' spectrum 
    \item Absorption Spectrum - small dark regions, aka ``dark line'' spectrum 
\end{itemize}

Spectroscopy is the science of producing atomic spectra and studying them.

\textbf{Particle Model}
\begin{itemize}
    \item The idea that light can act as a particle 
    \item Particles of light are called photons, or quanta (plural for quantum)
    \item A quantum behaves like a particle, and can move other matter 
\end{itemize}

\textbf{The Photoelectric Effect}
\begin{itemize}
    \item The particle model was needed to explain why when you shine a high energy light on some metals, electrons are ejected (moved) from the metal 
\end{itemize}

\begin{itemize}
    \item Einstein proposed in 1905 that light can behave as both a wave and a particle.
    \item He defined a photon as a particle of electromagnetic radiation with no mass that carries a quantum of energy.
    \item For this, he won the Nobel Prize.
\end{itemize}

\begin{itemize}
    \item The energy contained in a photon (a quantum) depends on its frequency 
    \[E_{\text{photon}}=h\nu\]
    $E$ = energy in joules [J]\\
    $h$ = Planck's constnat = $6.626\times 10^{-34}$ J$\cdot$s 
    $\nu$ = frequency (nu), [Hz]
    \item According to Planck, matter can emit or absorb energy only in whole quanta (1$h\nu$, 2$h\nu$, etc.)
\end{itemize}

Exercise - Calculate the frequency of a photon with $7.2\times10^{-34}$ J of energy. (1.1 Hz)

Exercise - Calculate the wavelength of a photon with $5.32\times 10^{-33}$ J of energy. ($3.74\times10^7$ m)


\end{document}