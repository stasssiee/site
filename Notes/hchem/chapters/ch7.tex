\documentclass[../hchem.tex]{subfiles}
\graphicspath{{\subfix{../figures/}}}
\begin{document}
\chapter{VSEPR/IMFs}

\section{Types of Bonding}
Chemical components are formed by the joining of two or more atoms. When atoms bond, their 
valence electrons are redistributed in ways that make the atoms more stable. The way the electrons are redistributed depends on the type of bond formed.

A chemical bond is a mutual attraction between the nuclei and valence electrons of different atoms that binds atoms together.

Ionic bonds are the result of the electrical attraction between positive and negative ions.

The ions are formed because atoms completely give up their electrons to other atoms.

\textbf{Ionic Bonds}
\begin{itemize}
    \item These bonds usually occur between a metal and a nonmetal, creating an ionic compound, also known as a salt.
    \item Both atoms end up with an octet of electrons in their valence shell.
    \item Salts are neutral because they have an equal positive and negative charge.
    \item Metals lose electrons and nonmetals gain electrons in an ionic compound.
\end{itemize}

\textbf{Covalent Compounds}
\begin{itemize}
    \item These bonds are the result of the sharing of electron pairs between two atoms.
    \item In a covalent bond, the electrons are ``owned'' by the two bonded electrons.
\end{itemize}

Covalent bonds usually occur between two nonmetals and results in individual molecules.

\textbf{Metallic Bonding}
\begin{itemize}
    \item In pure metals or alloys, there are usually vacant valence orbitals. The vacant orbitals overlap from one atom to another, allowing the outermost electrons to roam freely throughout the entire metal.
    \item These are called delocalized electrons. These mobile electrons, a ``sea of electrons'', move throughout the entire metal.
    \item Metallic bonds are the result of the attraction between metal nuclei and the surrounding sea of electrons.
\end{itemize}

Exercise - What type of bonding is present in carbon dioxide? (covalent)

\section{Bonding}
A chemical bond is an attractive force between atoms or ions that binds them together as a unit. Bonds form in order to decrease potential energy and increase stability.

What is Chemical Bonding?
\begin{itemize}
    \item A chemical bond is formed when electrons are shared or given between two or more atoms.
    \item The electrons involved are only the outermost electrons - the valence electrons.
    \item Chemical Bond - a link between atoms that holds them together.
\end{itemize}

Keeping Track of Electrons:
\begin{itemize}
    \item The electrons responsible for the chemical properties of atoms are those in the outer energy level.
    \item Valence electrons - The s and p electrons that are in the highest energy level.
    \item Core (or shielding) electrons - those in the energy levels below.
\end{itemize}

Remember atoms in the same column have the same outer electron configuration and have the same number of valence electrons.

In the s block, the number of valence electrons is the group number, in the d block the number of valence electrons 
varies and isn't always predictable, and in the p block the number of valence electrons is the group number minus 10.

Exercise - How many valence electrons does phosphorus have? (5)

Atoms typically bond to form an octet in their valence level. All atoms want this stability. This is also called ``noble gas configuration''.

When an atom gains or loses electrons, it is an ion. Loss of electrons is a cation and is positively charged. Gain of electrons is an anion and is negatively charged.

Intramolecular bonds hold atoms to atoms - they are your ionic, covalent, and metallic bonds.

Intermolecular Bonds hold two or more molecules/ions together. They are your hydrogen, dipole-dipole, ion-dipole, and london dispersion forces.

\textbf{Ionic Bond}
\begin{itemize}
    \item An ionic bond is formed when electrons are transferred from one atom to another. This creates positive and negative ions.
    \item When one or more electrons is transferred, you get both a positive and negative ion.
    \item Since they have opposite charges, they are attracted to one another. This is called an ``electrostatic attraction''.
    \item Ionic bonds typically form with a metal and a nonmetal.
    \item Ionic substances are sometimes caleld salts. 
    \item Overall, salts are neutral. They have equal amounts of positive and negative charge.
\end{itemize}

What are ionic compounds?
\begin{itemize}
    \item Because of their valence electron structure, metals lose their electrons, and nonmetals gain electrons.
    \item This is why metal ions have a positive charge and nonmetal ions have a negative charge.
\end{itemize}

Formula Unit 
\begin{itemize}
    \item A formula that tells the ratio of ions in an ionic compound.
    \item The smallest part of an ionic compound that still has the composition of the compound.
\end{itemize}

Lewis Dot notation can be use to visualize ionic compounds and how they form.

Properties of Ionic Compounds
\begin{itemize}
    \item High melting points/boiling points - it takes a lot of energy to break strong bonds.
    \item Hard, brittle solids 
    \item Many are soluble inw ater 
    \item When dissolved, free ions float and conduct electricity 
    \item Form crystalline solids 
\end{itemize}

Do they conduct?
\begin{itemize}
    \item Conducting electricity is allowing charges to move.
    \item In a solid, the ions are locked in place - ionic solids are insulators.
    \item When melted, the ions can move around.
    \item Melted ionic compounds conduct.
    \item Dissolved in water they can conduct.
\end{itemize}

In order for electrons to be transferred, one element must be much more electronegative than the other. They must have an 
EN difference of more than 1.7. In general, most combinations of metal+nonmetal will have this great $\Delta$EN.

Covalent bonds are a bond that results from the sharing of electrons. They are made of molecules instead of crystal lattice and usually occur between two nonmetals.
When two atoms do not have a big $\Delta$EN, they will share electrons. There are varying degrees of how electrons can be shared.
\begin{itemize}
    \item Shared equally: nonpolar covalent bond. The EN values are almost equal, a difference less than 0.5
    \item Shared unequally: polar covalent bond. The EN values are not equal, but not different enough to form an ionic bond.
\end{itemize}

In nonpolar covalent bonds, electrons are shared equally, the molecule overall is neutral.

In polar covalent bonds, electrons are not shared equally. The more electronegative atom attracts the electrons more, forming a partially negative region of the atom. The less electronegative atom becomes partially positive.

Properties of Covalent Bonds
\begin{itemize}
    \item No ions, no charges, do not conduct electricity.
    \item Weak attraction between molecules.
    \item Usually liquids or gases at room temperature.
    \item If solid, have low melting points.
    \item Amorphous Solid - do not have a regular/repeating pattern.
\end{itemize}

Most bonds are a blend of ionic and covalent characteristics. Difference in electronegativity determines bond type.

Metallic bonds occur between metal atoms. Bonding due to a ``sea of electrons'' - electrons that are not bound 
to one specific atom, they are able to move around the substance from atom to atom. Accounts for properties of metals and metal alloys.

Metals are 
\begin{itemize}
    \item Malleable 
    \item Ductile 
    \item Good at conducting heat and electricity 
\end{itemize}

Properties are due to the free-floating electrons.

Exercise - What type of bond is CH$_4$ (nonpolar covalent)

Lewis Structures:
\begin{itemize}
    \item Lewis structures can be drawn for both ionically and covalently bonded compounds.
    \item Just keep in mind ionically bonded salts will contain ions.
    \item Covalently bonded molecules will show shared electrons.
\end{itemize}

Types of Covalent Bonds:
\begin{itemize}
    \item Single Bond - one pair of electrons is shared; represented by a single line drawn between two atoms.
    \item Double Bond - two pairs of electrons shared; represented by two lines drawn connecting the two atoms.
    \item Triple Bond - three pairs of electrons shared; represented by three lines drawn connecting the two atoms. 
\end{itemize}

Multiple Bonds: usually formed by C, N, O, P, S

Triple bonds are stronger than double bonds and double bonds are stronger than single bonds.
It takes more energy to break a double bond than a single bond, and more energy to break a triple bond than a double or single bond.

Multiple bonds increase the electron density between two nuclei. As the electron density increases, the repulsion 
between the two nuclei decreases. An increase in electron density also increases the attraction each nucleus has for the additional 
bonding electron pairs. The nuclei move closer together and the bond length is shorter for a double bond than a single bond.

Predicting the Arrangement of Atoms within a molecule:
\begin{itemize}
    \item H is always a terminal atom. H is ALWAYS connected to only one other atom.
    \item The element with the lowest electronegativity is the central atom in the molecule. Put other atoms around the central atom.
    \item Find the total \# of valence electrons by adding up group \#'s of the elements. For ions add electrons for negative charges and subtract electrons for positive charges. Divide by two to get the number of electron pairs available to go around.
    \item Use a pair of electrons to connect each terminal atom to the central atom.
\end{itemize}

Usually central atoms will have 4 things around them, so spread atoms at 90 degree angles.

\begin{itemize}
    \item Place lone pairs about teach terminal atom to satisfy the octet rule.
    \item Left over pairs are assigned to the central atom. If the central atom is from the 3rd of higher period, it can accommodate more than four electrons pairs.
    \item If the central atom is not yet surrounded by four electron pairs, convert one or more terminal atom lone pairs to 
    pi bonds pairs. Not all elements form pi bonds! Only C, N, O, P, and S.
\end{itemize}

Remember, only C, N, O, P, S are able to form multiple bonds.

Exceptions to the octet Rule:
\begin{itemize}
    \item Electron Deficient: less than 8 electrons 
    \begin{itemize}
        \item Hydrogen: 2 in outer energy level 
        \item Boron: 6 in outer energy level 
        \item Beryllium: 4 in outer energy level 
    \end{itemize}
    \item Exceed Octet: more than 8
    \begin{itemize}
        \item anything in 3rd period or heavier 
        \item because d-orbitals are available and add extras to the middle atom.
    \end{itemize}
\end{itemize}

Often times there is more than one possible way for atoms to bond together in a given molecule.

VSEPR:
\begin{itemize}
    \item Valence Shell Electron Pair Repulsion 
    \item We've already discussed this - areas of electrons around a central atom tend to spread out to reduce electrostatic repulsion 
    \item Can be used to predict 3-D shape of molecules.
\end{itemize}

Areas of Electron Density:

Bonded electrons or unbonded electrons (lone pairs). These areas spread as far apart from each other as possible.

Molecules are nonpolar if they have only one kind of terminal atom and no lone electron pairs on the center atom. Molecules 
are polar if they have more than one kind of terminal atom or at least one lone electron pair on the central atom.

Hybridization is the mixing of different types of atomic orbitals to produce a set of equivalent hybrid orbitals. For assigning hybridization, we tell what type of orbitals are mixed.

We will use the following key now for describing shapes: ``A'' represents the central atom, ``X'' represents the atoms attached to the central atom, and ``E'' represents a lone pair of electrons on the central atom.

\begin{itemize}
    \item 2 bonding regions, 0 lone pairs - AX$_2$ - linear - usually nonpolar - hybridization: sp - bond angle: 180$\degree$
    \item 3 bonding regions, 0 lone pairs - AX$_3$ - trigonal planar - usually nonpolar - hybridization: sp$^2$ - bond angle: 120$\degree$
    \item 2 bonding regions, 1 lone pair - AX$_2$E - bent - always polar - hybridization: sp$^2$ - bond angle: $<120\degree$
    \item 4 bonding regions, 0 lone pairs - AX$_4$ - tetrahedral - usually nonpolar - hybridization: sp$^3$ - bond angle: $109.5\degree$
    \item 3 bonding regions, 1 lone pair - AX$_3$E - trigonal pyramidal - always polar - hybridization: sp$^3$ - bond angle: $107\degree$
    \item 2 bonding regions, 2 lone pairs - AX$_2$E$_2$ - bent - polar - hybridization: sp$^3$ - bond angle: $104.5\degree$
    \item 5 bonding regions, 0 lone pairs - AX$_5$ - trigonal bipyramidal - usually nonpolar - hybridization: under debate - bond angle: $90\degree, 120\degree, 180\degree$
    \item 4 bonding regions, 1 lone pair - AX$_4$E - seesaw - polar - hybridization: under debate - bond angles: $<90\degree, <120\degree, <180\degree$
    \item 3 bonding regions, 2 lone pairs - AX$_3$E$_2$ - T-shaped - polar - hybridization: under debate - bond angles: $<90\degree$
    \item 2 bonding regions, 3 lone pairs - AX$_2$E$_3$ - linear - polar - hybridization: under debate - bond angle: $180\degree$
    \item 6 bonding regions, 0 lone pairs - AX$_6$ - octahedral - usually nonpolar - hybridization: under debate - bond angles: $90\degree, 180\degree$
    \item 5 bonding regions, 1 lone pair - AX$_5$E - square pyramidal - polar - hybridization: under debate - bond angles: $<90\degree, <180\degree$
    \item 4 bonding regions, 2 lone pairs - AX$_4$E$_2$ - square planar - polar - hybridization: under debate - bond angles: $90\degree$
    \item 3 bonding reginos, 3 lone pairs - AX$_3$E$_3$ - T-shaped - polar - hybridization: under debate - bond angles: $<90\degree$
\end{itemize}

Hybridization: The mixing of different types of atomic orbitals to produce a set of equivalent hybrid orbitals.

Polarity - bonds can be polar while the molecule isn't and vice versa.

Molecular Polarity - if a central atom has no lone pairs of electrons and all surrounding bonds are identical, then the molecule is nonpolar.
Even though a molecule might have polar bonds within it, if those polar bonds cancel each other out, the molecule is nonpolar.

Molecules that have lone pairs of electrons on the central atom and.or different types of terminal atoms attached to the central atom are considered polar molecules. The charge is unevenly distributed throughout the molecule.

Exercise - Is O$_3$ polar or nonpolar. (polar)

There are many types of bonds that hold that hold molecules and molecules or molecules and ions together. These forces are incredibly important.

London Dispersion Forces 
\begin{itemize}
    \item These are the forces that exist among non-ionic and non-polar substances. 
    \item They exist among noble gases and nonpolar molecules.
    \item These forces are weak.
\end{itemize}

They are caused by an instantaneous dipole formation in which electron cloud becomes asymmetrical, and the molecules are slightly attractive to each other.
This is the weakest intra- and intermolecular forces.

When you are comparing two substances that both have dispersion forces:
\begin{itemize}
    \item The substance with more electrons has stronger dispersion forces since its electron cloud is larger and more polarizable.
\end{itemize}
Heavy noble gases have stronger dispersion forces than lighter noble gases.

Dipole-dipole forces 
\begin{itemize}
    \item These forces exist between molecules that have permanent dipole moments.
    \item Look for molecules with lone electron pairs on the central atom or different types of terminal atoms.
    \item These molecules have an uneven distribution of charge and therefore have attraction to each other.
\end{itemize}
Dipole-dipole forces are stronger than dispersion forces since the polarity in these molecules is permanent.

Hydrogen bonding is a special subset of dipole-dipole forces that exist only in H-N, H-O, and H-F.

This results in a partially positive pole and partially negative pole.

\begin{itemize}
    \item When two or more water molecules are near each other, the weak positive hydrogen atom of one molecule will be attracted to the weak negative oxygen atom of the other molecule.
    \item This attraction between molecules is called hydrogen bonding.
    \item After many hydrogen bonds are formed, you have a weak force holding all the water molecules to each other.
    \item Hydrogen bonding is the reason water freezes into ice crystals of a certain repeating shape.
    \item Remember that hydrogen bonding can occur between any molecules that contain O-H, N-H, F-H bonds.
\end{itemize}
Hydrogen bonds have a high boiling point. It takes a large amount of energy to boil water into vapor because of the bonds holding the molecules together.
The bonds must be broken in order for the liquid to change to a gas.

Hydrogen bonds also have high surface tension.

Water molecules are attracted to the glass because glass molecules are also polar, but not attracted to nonpolar plastic, so there is no meniscus in a plastic graduated cylinder.

Water can be drawn up into a thing glass tube with no effort because of the attraction between the water and glass molecules.

Surface tension can be decreased by adding a surfactant - this type of substance interferes with hydrogen bonding.

The Last IMF - ion-dipole forces.
\begin{itemize}
    \item Attraction that helps ionic compounds dissolve in a polar substance.
    \item Think of salt water.
\end{itemize}

Exercise - What IMF is present in CI$_4$? (LDFs)

\end{document}