\documentclass[../hchem.tex]{subfiles}
\graphicspath{{\subfix{../figures/}}}
\begin{document}
\chapter{Thermochemistry}
\textit{If I were a Rapper, I'd be MC Delta T}

\section{Enthalpy, Enthalpy of Reactions, Spontaneity}
Thermochemistry is the study of the energy occurring during chemical reactions and phase changes.

System vs. Surroundings - the part of the universe we are studying vs. everything else.

Endothermic is adding heat to the system and exothermic is removing heat from the system.

Enthalpy means the energy in a system at constant pressure. H is the symbol used for enthalpy. We can never fully measure the enthalpy of a system,
but we can measure changes in enthalpy.
\begin{center}
    $\Delta$H = change in enthalpy = H$_{\text{products}}$ - H$_{\text{reactants}}$
\end{center}

Exothermic  
\begin{itemize}
    \item System loses energy 
    \item Energy listed as a product 
    \item $\Delta$H is negative 
\end{itemize}

Endothermic
\begin{itemize}
    \item System gains energy 
    \item Energy listed as a reactant 
    \item $\Delta$H is positive 
\end{itemize}

A reaction pathway shows the change in energy during a chemical reaction.

In an energy diagram, the energy needed is called the activation energy.

A thermochemical reaction includes a chemical reaction and the corresponding $\Delta$H.

Combustion reactions are very common and release a lot of energy.

The energy change of combustion reactions is usually given as the ``Standard Enthalpy of Combustion'', or $\Delta$H$^{\circ}_{\text{comb}}$

The value is in the unit ``kJ/mol''.

\begin{itemize}
    \item A spontaneous process is one that occurs without any outside intervention.
    \item A nonspontaneous process is one that occurs due to some intervention.
\end{itemize}

In general, exothermic reactions are spontaneous and endothermic reactions are nonspontaneous.
There are some exceptions, and we will why later.

\section{Hess's Law}
Energy change in an overall chemical reaction is equal to the sum of the energy changes in the individual reactions comprising it.

The law is based on the principle of conservation of energy and the path independence of energy changes.

Hess's law can be used to predict energy changes that are not easily measured.

Hess's Law - energy change in an overall chemical reaction is equal to the sum of the energy changes in the individual reactions comprising it.

How to Solve Hess's Law Problems:
\begin{enumerate}
    \item First decide how to rearrange equations so reactions and products are on appropriate sides of the arrows.
    \item If equations had to be reversed, reverse the sign of $\Delta$H.
    \item If equations had be multipled to get a correct coefficient, multiply the $\Delta$H by this coefficient.
    \item Check to ensure that everything cancels out to give you the exact equation you want.
    \item It is often helpful to being your work backwards from the answer you want!
\end{enumerate}
\section{Big Mama Equation}
$\Delta$H$_f^\degree$ is the enthalpy of formation.

It is the production of one mole of compound from its elements in their standard states.

It is zero for elements in their standard states.

Hess's law can be summarized into a similar equation:
\begin{center}
    $\Delta$H$_{\text{rxn}}^\degree$ = $\sum n\Delta H_{f (products)}^\degree - \sum n\Delta H_{f (reactants)}^\degree$
\end{center}

You must still multiply $\Delta$H by the coefficient, n. The reactants are subtracted because they are equations that had to be reversed using Hess's Law.

Notes:
\begin{itemize}
    \item Sometimes all values are not found in the table of thermodynamic data. For most substances it is impossible to go into a lab and directly synthesize a compound from its free elements.
    \item The heat of formation for these substances must be found by working backwards from its heat of combustion.
\end{itemize}
\section{Reaction Spontaneity, Energy \& Heat Transfer}
Entropy, S, of a system is the randomness or disorder of that system. The universe wants more entropy.

Reaction which increase entropy are more likely to occur than ones in which the entropy decreases.

Reactions that decrease entropy are less likely to occur than ones in which the entropy increases.

Entropy is measured in J/K.
\begin{itemize}
    \item Entropy changes associated with changes in state can be predicted.
    \item The dissolving of a gas in a solvent always results in a decrease in entropy.
    \item The dissolving of a solid (or liquid) in a solvent always results in a increase in entropy.
    \item The entropy of a system increases when the number of gaseous product moles is greater than the number of gaseous reactant moles.
    \item An increase in the temperature of a substance is always accompanied by an increase in the random motion of its particles.
    \begin{itemize}
        \item Recall that the kinetic energy of molecules increases with temperature.
        \item Increased kinetic energy means faster movements, meaning more disorder.
    \end{itemize}
\end{itemize}

We can calculate entropy changes using the Big Mama Equation!

We've said that these changes are favorable/spontaneous:
\begin{itemize}
    \item Exothermic 
    \item Increases in entropy 
\end{itemize}

\textbf{Free Energy}
\begin{itemize}
    \item Gibbs free energy, G, is energy that is available to do work. It is a combined function of entropy and enthalpy.
    \begin{center}
        $\Delta$G = $\Delta$H - T$\Delta$S 
    \end{center}
    \item When $\Delta$G for a reaction is negative, the reaction is spontaneous
    \item When $\Delta$G for a reaction is positive, the reaction is non-spontaneous
\end{itemize}

Another way to say a reaction is ``spontaneous'' is by saying it is ``thermodynamically favorable''. A non-spontaneous reaction would be 
``thermodynamically unfavorable''. Big Mama can also be used for Gibbs free energy!

Types of Energy 
\begin{itemize}
    \item Potential - stored energy 
    \item Kinetic - energy of motion 
    \item Chemical - form of potential energy related to the structural arrangement of atoms or molecules 
    \item Thermal - form of kinetic energy resulting from the motion of particles and is transferred as heat 
\end{itemize}

The Law of Conservation of Energy states that the total amount of energy in a closed system remains constant. Energy cannot be created nor destroyed.
In a closed system, energy can only change form.

Heat:
\begin{itemize}
    \item Energy flows from something warm to something cool 
    \item A hotter substance gives energy to a cooler one 
    \item When heat is transferred, there is a change in the energy within the substance 
    \item There are three types of heat transfer - conduction, convection, and radiation 
\end{itemize}

Conduction:
\begin{itemize}
    \item The direct transfer of heat through contact 
    \item Occurs in solids, liquids, and gases. When 2 objects at different T's are in contact with each other, KE is exchanged.
\end{itemize}

Convection:
\begin{itemize}
    \item Transfer of heat by the motion of fluids 
    \item Occurs in liquids and gases 
\end{itemize}

Radiation:
\begin{itemize}
    \item Radiation is heat that is transferred through electromagnetic waves.
    \item Radiation is the only form of heat transfer that does not require matter, and can move through space.
\end{itemize}
\section{Specific Heat}
Different substances have different capacities for storing energy.

Specific heat is the amount of heat needed to raise the temperature of 1 g of a substance by 1$\degree$ C.

\begin{itemize}
    \item Conductors are materials that transfer heat easily and quickly. Metals are the best conductors of heat because they have a low specific heat.
    \item Insulators do not conduct heat easily. They have a high specific heat.
\end{itemize}

Water has a uniquely high specific heat compared to other substances.

There are moderate climates around lakes \& oceans.
\begin{itemize}
    \item The body of water absorbs heat from air on hot days and release it back into the air at night or on cold days.
\end{itemize}

Water is sprayed on citrus fruit to protect it from freezing. As the water freezes, it releases heat, which warms the fruit.

Specific heat can be calculated.
\begin{center}
    q = mc$\Delta$T 
\end{center}

where q is the heat energy, m is the mass of the substance, c is the specific heat capacity, and $\Delta$T is the change in temperature.

Specific heat, c, is a physical property. The higher a substance's specific heat, the more heat it takes to raise its temperature, so the longer it takes to heat it up or cool it off.

Energy is the capacity of doing work or supplying heat.

Heat is a form of energy that transfers from one object to another because of a temperature difference between them. Heat always flows from a warmer object to a cooler object.

Energy units: calorie [cal] or joule [J]

A calorie is a quantity of heat needed to raise the temperature of 1 g of water 1$\degree$C.

\begin{itemize}
    \item System - the part of the universe on which you focus your attention.
    \item Surroundings - everything else in the universe 
\end{itemize}

The Law of Conservation of Energy states that in any chemical or physical process, energy is neither created nor destroyed.

To measure energy changes 
\begin{itemize}
    \item We use a device called a calorimeter 
    \item The reaction in the cup either absorbs or releases energy causing the water temperature to rise or fall 
    \item The temperature change of the surrounding water tells you whether energy is absorbed or released.
\end{itemize}

\section*{Chapter Problems}
\begin{enumerate}
    \item A 4.50 g nugget of pure gold absorbed 276 J of heat. What was the final temperature of the gold if the initial temperature was $25.0^{\circ}$ C?
    \item Draw a potential energy diagram to represent the following reaction. A set of reactants begin with 120 kJ of energy. The activation energy is 60 kJ. The $\Delta H$ of the reaction is 30 kJ.
    \item Pure liquid acetic acid is made from the reaction between methanol and carbon monoxide, as seen below. If you produce 1.00 L of acetic acid (which has a mass of 1044 grams), how much energy is evolved?
    \begin{center}
        CH$_3$OH (l) + CO (g) $\rightarrow$ CH$_3$COOH (l) $\qquad$ $\Delta H$ = -355.9 kJ
    \end{center}
    \item Calculate $\Delta H$ for the reaction: 4NH$_3$ (g) + 5O$_2$ (g) $\rightarrow$ 4NO (g) + 6H$_2$O (g), from the following data.
    \begin{center}
        N$_2$ (g) + O$_2$ (g) $\rightarrow$ 2NO (g) $\qquad$ $\Delta H = -180.5$ kJ\\
        N$_2$ (g) + 3H$_2$ (g) $\rightarrow$ 2NH$_3$ (g) $\qquad$ $\Delta H = -91.8$ kJ\\
        2H$_2$ (g) + O$_2$ (g) $\rightarrow$ 2H$_2$O (g) $\qquad$ $\Delta H = -483.6$ kJ
    \end{center}
    \item Use a standard enthalpies of formation table to determine the change of enthalpy for the following reaction 
    \begin{center}
        2NO (g) + O$_2$ (g) $\rightarrow$ 2NO$_2$ (g)
    \end{center}
    \item Use the table of standard free energies for the following reaction 
    \begin{center}
        2SO$_2$ (g) + O$_2$ (g) $\rightarrow$ 2SO$_3$ (g)
    \end{center}
    \item Solve for $\Delta G$ of the reaction 
    \begin{center}
        2H$_2$O$_2$ (l) $\rightarrow$ 2H$_2$O (l) + O$_2$ (g)
    \end{center}
    Is the reaction spontaneous or not? Assume standard temperature.
    \item Does the following reaction shown an increase or decrease in entropy?
    \begin{center}
        Ag$^+$ (aq) + Cl$^-$ (aq) $\rightarrow$ AgCl (s)
    \end{center}
    \item The conditions in which $\Delta G$ is always negative when $\Delta H$ is \blank and $\Delta S$ is \blank.
\end{enumerate}

\end{document}