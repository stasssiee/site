\documentclass[../hchem.tex]{subfiles}
\graphicspath{{\subfix{../figures/}}}
\begin{document}
\chapter{Solutions}
\section{Solutions, Colloids, Suspensions, Electrolytes \& Solubility}
\begin{itemize}
    \item Mixtures that are mostly liquid can be classified as one of three different things - 
    \begin{itemize}
        \item Solutions (homogeneous mixtures)
        \item Same makeup throughout - evenly mixes on the molecular level 
    \end{itemize}

    \item Heterogeneous Mixtures
    \begin{itemize}
        \item Different makeup in different parts of the mixture.
        \item Includes colloids and suspensions.
    \end{itemize}
    \item Simple tests can help determine what type of mixture you have.
\end{itemize}

Solutions are made up of two parts -
\begin{itemize}
    \item Solute 
    \item Solvent
\end{itemize}

The solute dissolves into the solvent.

Properties of solutions:
\begin{itemize}
    \item Clear, but not necessarily colorless.
    \item Solute particles are ions or molecules wit ha size less than a nonmeter - very small.
    \item Particles cannot be seen even with a microscope, and the mixture doesn't scatter light.
    \item Cannot be separated by filtering, settling, or centrifuging.
    \item Solutions can be separated by evaporation.
    \item Can conduct electricity if an ionic compound is dissolved. Do not conduct if covalent/molecular compounds are dissolved.
\end{itemize}

Colloids:
\begin{itemize}
    \item A heterogeneous mixture. Some can appear homogeneous with just your eyes, but the particles dispersed throughout the liquid can be seen with a microscope.
    \item The particles in a colloid are bigger than the particles dispersed in a solution.
    \item Can appear clear, slightly cloudy, or very cloudy.
    \item They scatter light. This is called the Tyndall Effect.
    \item Dispersed particles are about 10-100 times bigger than the particles dissolved in solutions.
    \item Will not conduct electricity.
    \item Will not separate into separate parts by settling, standing, or filtering.
    \item Can be separated by centrifuge or heating, depends on the specific colloid.
\end{itemize}

Suspensions:
\begin{itemize}
    \item Large particles dispersed in liquid - can be seen with a light microscope and sometimes the naked eye.
    \item Cloudy when shaken, but the dispersed particles settle upon standing.
    \item Can speed separation by filtering or centrifuging.
    \item Will not conduct electricity.
\end{itemize}

Exercise - a mixture doesn't leave residue on paper. (solution, colloid)

The solvent is what is doing the dissolving. It is usually water because it is a non-ionic polar molecule that can dissolve anything else that is non-ionic or polar. Water can also dissolve most ionic compounds due to its hydrogen bonding.

The solute is what is dissolved in something else.

Electrolytes:
\begin{itemize}
    \item Some solutions can conduct electricity. These are called electrolytes.
    \item Either ionic compounds or strong acids can act like electrolytes.
    \item Dissociated ions carry an electric current.
\end{itemize}

There are three categories of strong electrolytes: strong acids, strong bases, and soluble salts.

The strong acids are - HCl, HBr, HI, H$_2$SO$_4$, HNO$_3$, HClO$_3$, and HClO$_4$.

Strong bases are hydroxides of group I and heavy group II metals (Car, Sr, Ba)

The soluble salts are compounds with ions of NO$_3^-$, Group I, NH$_4^+$, C$_2$H$_3$O$_2^-$, ClO$_4^-$, ClO$_3^-$ with no exceptions.
Cl$^-$, Br$^-$ and I$^-$ are soluble except with Pb$^{+2}$, Ag$^+$, and Hg$_2^{+2}$.
SO$_4^{-2}$ is soluble except with Pb$^{+2}$, Ag$^+$, Hg$_2^{+2}$, Ca$^{+2}$, Sr$^{+2}$, and Ba$^{+2}$.

Solubility Factors:

First we need to know if something will dissolve.
\begin{itemize}
    \item Miscible - two substances that are miscible are soluble together and will mix.
    \item Immiscible - two insoluble liquids; they will not mix together.
\end{itemize}

\begin{itemize}
    \item Polar things will dissolve in other polar things.
    \item Ionic substances will dissolve in water if the ionic bonds aren't too strong.
    \item Polar and Nonpolar substances will not dissolve together.
\end{itemize}

The process of dissolving: Solvation
\begin{enumerate}
    \item Solvent is attracted to the solute.
    \item Solvent particles surround the solute particles and pull them into solution.
\end{enumerate}

Factors affecting rate of solution:
\begin{itemize}
    \item Surface area - more contact between solute and solvent increases rate of solution.
    \item Agitation - mixing the mixture causes more contact between solute \& solvent, increases rate of solution.
    \item Temperature - solvent particles are moving faster at higher temperatures, increases rate of solution.
\end{itemize}

Sometimes substances will not fully mix together no matter how much work is done. It all has to do with IMFs. Similar IMFs will dissolve together.

Solubility is the amount of solute that will dissolve in a given amount of solvent.

The rules for this are different for solids and gases.

Temperature
\begin{itemize}
    \item Higher temperatures make gases less soluble in liquid
    \item Most solids are more soluble in liquid 
\end{itemize}

Pressure 
\begin{itemize}
    \item Higher pressure above a solution will increase the solubility of a gas. This relationship is known as Henry's Law 
    \item have no effect on solid solubility
\end{itemize}

Solubility is a physical property and is the amount of a substance that can be dissolved in a liquid.

On a solubility graph, if the amount is on the line, it is called saturated.

Saturation of a solution is sort of like saturation of a sponge. There are only so many holes in a sponge to hold a liquid, and when it is full there is simply no more room.

A saturated solution is when no more solute can dissolve.

Unsaturated solution is when more solute could dissolve. Any point below the line is unsaturated.

Supersaturated is when more solute is dissolved than normally could be. Any disturbance or seed crystal will cause the excess to precipitate out. Supersaturated means 
you are above the line and all solute is dissolved. It's more likely that you actually have a saturated solution with some undissoved solute present.
\section{Units of Concentration}
A concentrated solution has a relatively high amount of solute.

A dilute solution has a relatively low amount of solute.

What concentration units are most useful in chemistry?
\begin{itemize}
    \item Molarity = moles solute/liters solution 
    \item Molality - moles solute/kg solvent 
    \item \% Mass = grams part/grams total $\times$ 100 
    \item Mole Fraction = moles part/moles total 
\end{itemize}

Molarity 
\begin{itemize}
    \item The molarity of a solution is a measure of how many moles of the solute are present for each liter of solution.
    \item The liters of solution is not necessarily how much water was added. The final volume is usually more than the amount of pure water added since the solute adds to the volume.
    \item M = moles solute/Liters solution 
\end{itemize}

Higher molarity values mean more concentrated.

Exercise - How many grams of HCl would be necessary to create 2.00 L of a 4.0 M solution? (290 g)

Molality is the concentration calculated by dividing the number of moles of solute by kilograms of solvent use to dissolve.
This is useful for colligative properties - something we'll get to later.

m = mol solute/kg solvent 

Exercise - A solution contains 5.3 grams of carbon dioxide dissolved in 450. grams of water. What is the molality of the solution? (0.27 m)

The molarity or molality of a solution does not tell you whether it is strong or weak. It tells you whether it is concentrated or dilute.

Mole fraction for one component of a solution is moles of component/total moles of all components. Mol fraction has no units.

Exercise - You mix 30. grams of lithium chloride into 100.0 grams of water. What is the mole fraction of lithium chloride? (0.11)

\% mass is grams of component/total mass of mixture $\times$ 100.

Exercise - 60.0 grams of dextrose are added to 200. grams of water. What is the \% mass of dextrose? What is the \% of water? (23.1\% and 76.9\%)

You can't always find the solution concentration you need. It's common to have to dilute a more concentrated solution to create what you want.
The dilution formula is 
\[M_1V_1 = M_2V_2\]
where M is molarity and V is volume.
\section{Colligative Properties}
Colligative properties of solutions are properties that depend on the amount of solute particles.

Colligative properties only occur when a nonvolatile solute is added. Nonvolatile means it doesn't evaporate easily, so the solutes are usually solid.

4 important properties
\begin{itemize}
    \item Freezing point of solution is lowered 
    \item Boiling point of solution is raised 
    \item Vapor pressure of solution is lowered 
    \item Osmotic pressure is raised 
\end{itemize}

Freezing point can be lowered by adding a nonvolatile solute to water, and the ``depression'',
or specific number by which the freezing point is lowered, can be calculated.

The formula for this is 
\[\Delta T = K_f mi\]
Where, $\Delta T$ is the change in freezing point, K is the molal freezing point constant, m is molality, and i is the number of particles from one ``formula'' of the solute.

Boiling point is increased by adding a nonvolatile solution, and the ``elevation'', or specific number by which the BP is elevated, can be calculated.
\[\Delta T = K_b mi\]
where $K_b$ is the molal boiling point constant, and the other variables are defined above.

Vapor pressure lowering - the greater the number of solute particles in a solution, the lower the vapor pressure of the liquid solvent.
The resulting vapor pressure is equal to the vapor pressure of the pure solvent times the mol fraction of the solvent.
\[P = P_{solvent}\cdot X_{solvent}\]

Osmotic pressure is the pressure required to stop osmosis from happening. Osmosis is the transfer from a dilute solution 
to a more concentrated one - osmosis is trying to balance out concentrations. Osmotic pressure increases with increased solute in a solution.
\[P_{os}V=nRT\]


When we previously mentioned that not all ionic compounds dissolve, there is a catch to this.
They do dissolve a little, but most of the ionic compound stays in solid form, a bit will dissolve into aqueous ions.
An ``equilibrium'' is reached between the solid and aqueous phases.

In order to determine how much actually dissociates, we have to be able to write what is called an ``equilibrium expression''.

Note:
\begin{itemize}
    \item square brackets indicate concentration 
    \item Ion charges are included inside the brackets 
    \item Coefficients become exponents 
    \item Solids and liquids have a concentration of 1
\end{itemize}

Exercise - Write the K$_{sp}$ expression for Lead(II) fluoride. ([Pb]$^{+2}$[F$^-$]$^2$)

We can use the K$_{sp}$ expression to determine how much of the substance will actually dissolve.

Math problems with equilibrium will almost always involve ``RICE'' tables - 
\begin{itemize}
    \item Reaction 
    \item Initial Concentration
    \item Change in concentration
    \item Equilibrium concentration
\end{itemize}

Exercise - What is the solubility of a saturated lead(II) chloride solution? K$_{sp}$ for lead(II) chloride is $1.17\times 10^{-5}$. ($1.43\times 10^{-2}$ M)
\end{document}