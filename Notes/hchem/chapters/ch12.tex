\documentclass[../hchem.tex]{subfiles}
\graphicspath{{\subfix{../figures/}}}
\begin{document}
\chapter{Equilibrium}
\textit{Equilibrium Constant. Que?}
The nature of the equilibrium state:
\begin{itemize}
    \item Reactions are reversible.
    \item dynamic - $\rightleftarrows$ indicates that the reaction is proceeding in both the forward and reverse directions.
    \item Equilibrium does not mean nothing is happening.
    \item Once equilibrium is established, the rate of reaction in each direction is equal.
    \item This keeps the concentration of reactants and products constant.
\end{itemize}

\begin{center}
    A + B $\rightarrow$ C + D 
\end{center}
\begin{itemize}
    \item The nature and properties of the equilibrium state are the same, no matter what the direction of approach.
    \item Initially, since there is only A and B present, they react very quickly.
    \item As A and B react, there are less and less of unreacted A and B molecules and so the rate of this reaction slows down.
    \item As the concentrations of C and D build up, they start to react to form A and B.
    \item And eventually the rate at which A and B react equals the rate at which C and D react. At this point equilibrium is established.
\end{itemize}
Note the concentrations of A, B, C, and D are not necessarily equal at equilibrium, but the concentrations are constant at equilibrium.

The equilibrium constant is used to determine equilibrium conditions. It is always temperature dependent.

For the general reaction 
\begin{center}
    aA + bB $\rightleftarrows$ cC + dD 
\end{center}
The equilibrium constant expression is 
\begin{center}
    K = $\frac{[C]^c[D]^d}{[A]^a[B]^b}$
\end{center}
The product concentrations appear in the numerator and the reactant concentration in the denominator.
Each concentration is raised to the power of its stoichiometric coefficient in the balanced equation.

\begin{itemize}
    \item [] indicate concentration 
    \item K$_c$ is for concentration 
    \item K$_p$ is for partial pressure 
\end{itemize}

In equilibrium constant expressions:
\begin{itemize}
    \item Pure solids do not appear in the expression 
    \item Pure liquids do not appear in the expression 
    \item Water as a liquid or gas does not appear in the expression 
\end{itemize}

If you know the concentration of things, just plug them into the K expression and solve.

Changing coefficients:
\begin{itemize}
    \item When the stoichiometric coefficients of a balanced equation are multiplied by some factor, the K value is raised to the power of the multiplication factor.
    \item For the reverse reaction, take the reciprocal of K 
    \item When adding reactions to produce another, multiply the respective K's to determine the K of the final reaction.
\end{itemize}

K$_p$ only applies when all reactants and products are gases. Pressure must be units of atmospheres.

K$_c$ and K$_p$ are not interchangeable!
\begin{center}
    K$_p$ = K$_c$(RT)$^n$
\end{center}
where n is the change in the number of moles of gas from the reactant to product side of the arrow.

Use a RICE table if you are not given all the concentrations.

External factors affecting equilibria:

Le Chatelier's Principle - If you disrupt a system in equilibrium, shifts in direction occur to reestablish equilibrium positions.

Temperature:
\begin{itemize}
    \item Exothermic reactions - heat is a product
    \item Endothermic reactions - heat is a reactant 
    \item The value of K is a constant only at constant T 
    \item If you change T, the value of K changes  
\end{itemize}

Changing reagent and product amounts:
\begin{itemize}
    \item Adding reagent - shift tries to consume additional material by favoring forward reaction 
    \item Removing a reagent - shift tries to replace material by favoring reverse reaction 
    \item Adding product - shift tries to consume the product by favoring reverse reaction 
    \item Removing product - shift tries to replace the product by favoring forward reaction 
    \item Has no effect on the value of K 
\end{itemize}

Pressure:
\begin{itemize}
    \item An increase in pressure favors the side with the least \# of gas moles 
    \item A decrease in pressure favors the side with the most \# of gas moles 
    \item Has no effect on solids and liquids 
    \item Has no effect on the value of K
\end{itemize}

Catalysts:
\begin{itemize}
    \item No effect on the value of K 
    \item But the reaction gets to equilibrium faster!
\end{itemize}

Weak acids only dissociate partly. There is an equilibrium constant associated with weak acid dissociation, K$_a$. It works just like K$_c$.
\begin{center}
    K$_a$ = $\frac{[H_3O^+][A^-]}{[HA]}<1$
\end{center}

The reaction quotient, Q.

For a general reaction 
\begin{center}
    aA + bB $\leftrightarrow$ cC + dD 
\end{center}
the reaction quotient is 
\begin{center}
    Q = $\frac{[C]^c[D]^d}{[A]^a[B]^b}$
\end{center}
Q has the appearance of K but the concentrations are not necessarily at equilibrium.

If $K>Q$, the system is not at equilibrium.
\begin{itemize}
    \item Products are too small and the reactants are too big.
    \item The system will move towards equilibrium by making products and consuming reactants.
\end{itemize}

If K = Q, the system is at equilibrium.

If $K<Q$, the system is not at equilibrium.
\begin{itemize}
    \item Reactants are too small and the products are too big.
    \item The system will move towards equilibrium by making reactants and consuming products.
\end{itemize}

\section*{Chapter Problems}
\begin{enumerate}
    \item K$_{\text{eq}}$ for the reaction 2A + B $\rightleftarrows$ 2C is 8.0 Find the concentration of C when the concentration of A is $5.00\times 10^{-4}$ M and the concentration of B is $2.50\times 10^{-4}$ M.
    \item For the reaction 
    \begin{center}
        4NH$_3$ (g) + 3O$_2$ (g) $\leftrightharpoons$ 2N$_2$ (g) + 6H$_2$O (l)
    \end{center}
    How will the concentration of each chemical be affected byd ecreasing the volume of the container?

    \item Bromine and chlorine react to produce bromine monochloride according to the equation below. K$_\text{c}$ = 36.0 under the conditions of the experiment.
    \begin{center}
        Br$_2$ (g) + Cl$_2$ (g) $\leftrightarrow$ 2BrCl (g)
    \end{center}
    If 0.180 moles of bromine gas and 0.180 moles of chlorine gas are introduced into a 3.0 Liter flask and allowed to come to equilibrium, what is the equilibrium concentration of the bromine monochloride? How many moles of bromine monochloride are produced?
    \item When 2.0 mol of carbon disulfide and 4.0 mol of chlorine are placed in a 1.0 Liter flask, the following equilibrium system results. At equilibrium, the flask is found to contain 0.30 mol of carbon tetrachloride. 
    What quantities of the other components are present in this equilibrium mixture? What is the equilibrium constant at this temperature?
    \begin{center}
        CS$_2$ (g) + 3Cl$_2$ (g) $\leftrightarrows$ S$_2$Cl$_2$ (g) + CCl$_4$ (g)
    \end{center}
    \item Suppose you dissolved benzoic acid, C$_6$H$_5$COOH, in water to make a 0.15 M solution. K$_\text{a}$ for benzoic acid = $6.3\times 10^{-5}$ at $25^{\circ}$ C. Solve for the concentration of benzoic acid, the concentration of hydronium ion, the concentration of benzoate anion, and the pH.
    \item Calculate the hydronium ion concentration and the pH of a 0.50 M solution of HNO$_2$. For HNO$_2$, K$_{\text{a}}$ = $4.5\times 10^{-4}$.
    \item At $60.2^{\circ}$ C, the equilibrium constant for the reaction, N$_2$O$_4$ (g) $\leftrightarrow$ 2NO$_2$ (g), is $8.75\times 10^2$. At this temperature, a vessel contains dinitrogen tetroxide at a concentration of $1.72\times 10^{-2}$ M at equilibrium. What concentration of nitrogen dioxide does it contain?
    \item If you wished to maximize the products in the following reaction, would you run the reaction at a high pressure or at a low pressure?
    \begin{center}
        CaCO$_3$ (s) $\leftrightharpoons$ CaO (s) + CO$_2$ (g)
    \end{center}
    \item At $500^{\circ}$ C, the equilibrium constant for the following reaction is 0.080. What is the value of Q if [NH$_3$] = 0.059 M, [N$_2$] = 0.59 M and [H$_2$] = 0.42 M? How will this reaction proceed?
    \begin{center}
        N$_2$ (g) + 3H$_2$ (g) $\leftrightarrows$ 2NH$_3$ (g)
    \end{center}
    \item When 54.0 grams of ammonia is placed in an evacuated 3.50 liter rigid container and heated to 550. K, 12.75\% of the gas has dissociated. What is the K$_{\text{p}}$ value?
    \item Write the equilibrium expression for the following reaction: Dinitrogen pentoxide gas decomposes into nitrogen dioxide gas and oxygen gas.
    \item For the chemical equilibrium A + 2B $\leftrightharpoons$ 2C, the value of the equilibrium constant, K, is 10. What is the value of the equilibrium constant for the following reaction: 4C $\leftrightharpoons$ 2A + 4B?
\end{enumerate}
\end{document}