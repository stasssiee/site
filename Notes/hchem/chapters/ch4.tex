\documentclass[../hchem.tex]{subfiles}
\graphicspath{{\subfix{../figures/}}}
\begin{document}
\chapter{Bonding and Compounds}
\section{Types of Bonds Overview}
Chemical compounds are formed by the joining of two or more atoms. When atoms bond, their 
valence electrons are redistributed in ways that make the atoms more stable. The way the electrons are redistributed 
depends on the type of bond formed.

A chemical bond is a mutual attraction between the nuclei and valence electrons of different atoms that binds atoms together.

\textbf{Ionic Bonds}
\begin{itemize}
    \item These bonds are the result of the electrical attraction between positive ions and negative ions.
    \item The ions are formed because atoms completely give up their electrons to other atoms.
\end{itemize}

Ionic Bonding Process:
\begin{enumerate}
    \item In an ionic bond, electrons are transferred from one atom to another.
    \item The transfer creates a positive ion and a negative ion.
    \item Cations and anions are attracted to each other due to the electrostatic attraction between positive and negative ions, so they are bound together.
\end{enumerate}

\begin{itemize}
    \item These bonds usually occur between a metal and a nonmetal, creating an ionic compound, also known as a salt.
    \item Both ions end up with an octet of electrons in their valence shell.
    \item Salts are neutral because they have an equal positive and negative charge.
    \item Metals lose electrons and nonmetals gain electrons in an ionic compound.
\end{itemize}

\textbf{Covalent Bonds}
\begin{itemize}
    \item These bonds are the result of the sharing of electron pairs between two atoms.
    \item In a covalent bond, the electrons are ``owned'' by both of the two bonded atoms.
\end{itemize}

The Covalent Bonding Process:
\begin{enumerate}
    \item Covalent bonds are the result of sharing electrons between two atoms.
    \item Because the atoms must stay together to share, molecules are formed.
    \item The molecules are neutral because they have the same number of protons and electrons.
\end{enumerate}

\begin{itemize}
    \item Covalent bonds usually occur between two nonmetals.
    \item Covalent bonding results in individual molecules.
\end{itemize}

\textbf{Metallic Bonding}
\begin{itemize}
    \item In pure metals or alloys, there are usually vacant valence orbitals. The vacant orbitals overlap from one atom to another, allowing the outermost electrons to roam freely throughout the entire metal.
    \item These are called delocalized electrons. These mobile electrons, a ``sea of electrons'', move throughout the entire metal.
    \item Metallic bonds are a result of the attraction between metal nuclei and the surrounding sea of electrons.
\end{itemize}

The Metallic Bonding Process:
\begin{enumerate}
    \item Metal atoms have overlapping empty orbitals.
    \item Each metal atom loses its valence electrons to roam freely throughout the metal.
    \item The metal is held together because the free-floating electrons and positive metal cores are attracted to each other.
\end{enumerate}

Exercise - What type of bonding is present in phosphorus decoxide? (Covalent)

\section{Ionic Nomenclature}
\textbf{3 main types of compounds}
\begin{itemize}
    \item Type 1 - Ionic Compounds 
    \begin{itemize}
        \item A positive ion and a negative ion.
        \item A metal and a nonmetal.
        \item A metal and a polyatomic ion.
    \end{itemize}
    \item Type 2 - Covalent Compounds
    \begin{itemize}
        \item Made of more than one nonmetal atom 
    \end{itemize}
    \item Type 3 - Acids 
    \begin{itemize}
        \item Made up of positive hydrogen ions paired with negative ions.
        \item These compounds appear to be covalent but behave like ionic compounds.
    \end{itemize}
\end{itemize}

Exercise - What type of compound is Copper (II) bromide? (Ionic)

\textbf{Ionic Formulas and Nomenclature}
\begin{itemize}
    \item Chemical formulas show the number of atoms in a compound.
    \item ``Nomenclature'' is a naming system.
    \item The ``nomenclature'' system for people is first and last names.
    \item The typical ``nomenclature'' system for marriages is that the man keeps his last name and the bride changes her last name.
\end{itemize}

\textbf{How to name ionic compounds}
\begin{enumerate}
    \item Make sure the compound is ionic by looking for a metal.
    \item Name the cation and then the anion. Remember that nonmetal monatomic ions end in ``-ide''.
    \item Write the ion symbols with their charges.
    \item Cross the charges over and take the absolute value. These numbers become the subscripts. Reduce the subscripts if they are divisible by an integer.
\end{enumerate}

Exercise - Zinc nitride (Zn$_3$N$_2$)

If a polyatomic ion needs a subscript, put parenthesis around the polyatomic ion to 
show that more than one polyatomic ion is present.

Exercise - Tin(IV) Sulfate (Sn(SO$_4$)$_2$)

To name an ionic compound, name the positive ion then name the negative ion. Easy peasy. Remember nonmetal ions end in -ide.

Exercise - Zn(OH)$_2$ (zinc hydroxide)

\textbf{Transition Metal Names}
\begin{itemize}
    \item If the metal has more than one possible charge you must know which one it is.
    \item Do a reverse cross of the subscripts to determine the charge of the metal.
\end{itemize}

Exercise - NiPO$_4$ (nickel(III) phosphate)
\section{Covalent \& Acid Nomenclature}
Covalent compounds contain only nonmetals (also called molecular compounds).

To name a covalent compound
\begin{itemize}
    \item name the first element 
    \item then name the second one and change its ending to -ide 
    \item Use prefixes to show how many atoms of each element you have.
\end{itemize}

Exercise - Tetraphosphorus decasulfide (P$_4$S$_{10}$)

Exercise - PCl$_5$ (phosphorus pentachloride)

\begin{itemize}
    \item Acids are an important class of hydrogen-containing compounds and are named in a special way.
    \item Acids are defined as substances whose molecules produce hydrogen ions when dissolved in water.
    \item When we encounter acids, it will be written with H as the first element.
\end{itemize}

\textbf{Writing and Naming Acids}
\begin{itemize}
    \item composed of an anion connected to enough H$^+$ ions to totally neutralize or balance the anion's charge. Criss-cross! 
    \item name of an acid is related to the name of its anion... 
    Three acid naming rules:
    \begin{itemize}
        \item If the anion ends with ide, the acid is hydro\blank ic acid.
        \item If the anion ends with ate, the acid is \blank ic acid.
        \item If the anion ends with ite, the acid is \blank ous acid.
    \end{itemize}
\end{itemize}

Exercise - HCN (hydrocyanic acid)

Exercise - Dichromic Acid (H$_2$Cr$_2$O$_7$)

\section{Mole Problems}
1 mol is $6.022\times 10^{23}$ of anything

\textbf{Molar Mass}
The molar mass concept works the same way with compounds as it did with pure elements.
You simply add the molar mass of each atom within the formula.

We call this ``molar mass'' for molecular compounds and ``formula mass'' for ionic compounds.
Generally, though, we use the term molar mass for atomic mass in grams of any compound, ionic or covalent, or any element.

Exercise - Calculate the molar mass of H$_2$O (18.016 g)

Exercise - Calculate the molar mass of calcium chloride. (110.98 g)

If parenthesis appear ina  formula, the number outside the parenthesis multiplies by every 
atom inside the parenthesis, just like a coefficient in math.

Exercise - Calculate the molar mass of Ca(NO$_3$)$_2$ (164.10 g)

Exercise - How many atoms of each element are in aluminum carbonate. (Al: 2, C: 3, O: 9)

Exercise - Calculate the molar mass of zinc nitrate. (189.41 g)

Exercise - How many moles are equal to $5.06\times 10^{23}$ molecules of Br$_2$? (0.840 mol Br$_2$)

Exercise - How many moles are equal to $3.905\times 10^{23}$ formula units of calcium hydroxide? (0.6485 mol Ca(OH)$_2$)

Exercise - What is the mass of 0.7880 moles of calcium cyanide? (72.59 g Ca(CN)$_2$)

Exercise - How many nitric acid molecules are in 4.20 g of HNO$_3$? ($4.01\times10^{22}$ mole HNO$_3$)

\section{Percent Composition}
Percent composition of a compound:
\begin{center}
    \% composition = $\frac{\text{\# of atoms of element $\times$ (MM of element)}}{\text{MM of compound}}\times 100$

    where ``MM'' is molar mass 
\end{center}

You must know the CORRECT formula of the compound to calculate percent composition.

Exercise - What is the \% copper in copper(II) carbonate? (51.43\%)

Round percent composition answers to 4 sig figs.

Exercise - Calculate the \% composition of nitrogen in ammonium nitrate? (35.00 \%)

Hydrates are ionic compounds that can trap water in their crystalline structure when they form. The water is part of the structure, and it is a definite ratio of the compound.

Exercise - Write the formula for magnesium sulfate heptahydrate. (MgSO$_4\cdot7$H$_2$O)

Anhydrous compounds have no water in their crystalline structure.

To calculate the \% water in the hydrate, use the same formula as before, but water is the part on top.

Exercise - Calculate the \% water in copper(II) sulfate pentahydrate. (36.09 \%)

Exercise - Calculate the \% water in magnesium sulfate heptahydrate. (51.17 \%)
\section{Empirical \& Molecular Formulas}
Questions typically look like this: 
\begin{itemize}
    \item You are given the percent composition of a compound.
    \item You determine the formula based on the percentages.
\end{itemize}

Let's rhyme to solve these:
\begin{itemize}
    \item Percent to mass 
    \item Mass to mole 
    \item Divide by least 
    \item Multiply 'til whole 
\end{itemize}

Exercise - What is the formula of a compound that is 25.9\% nitrogen and 74.1\% oxygen? (N$_2$O$_5$)

So far you have calculated the simplest formula. We will now take this further. The empirical formula is the just the 
lowest possible ratio. The molecular formula, the actual makeup of a molecule, may be different.

If possible, you want to give the molecular formula. It is more descriptive of the actual molecular makeup. To do this,
you must know the molar mass of the molecule.

Note: Ionic compounds never have molecular formulas, since the definition of the formula of an ionic compound 
is the lowest possible ratio. Only molecular, or covalent compounds, can have a molecular formula.

How to determine molecular formula:
\begin{enumerate}
    \item Divide the true molar mass by the empirical formula's molar mass to get an integer.
    \item Multiply the subscripts of the empirical formula by this integer.
\end{enumerate}

Exercise - A compound has the empirical formula CH. The molar mass of the compound is 78.110 g. What is the molecular formula of the compound? (C$_6$H$_6$)
\section{Oxidation Numbers}
``Oxidation numbers'' are an accounting system used to keep track of electrons in a chemical reaction.

The oxidation state of a free element is 0.

The oxidation state for a monatomic ion is equal to its charge.

The algebraic sum of the oxidation numbers of all the atoms in a compound must be zero.

Similarly, the algebraic sum of the oxidation numbers of all the atoms in a polyatomic ion must equal the charge of the polyatomic ion.

Really useful rules:
\begin{itemize}
    \item In compounds, the more electronegative element is always negative.
    \item In compounds, hydrogen is usually +1, unless it is bonded to a metal. In that case it is a hydride and the number is -1.
    \item In compounds, oxygen is usually -2. However, if it is a peroxide, it is -1. If it is bonded to fluorine, oxygen will be +2. This is rare.
    \item The oxidation number for alkali metals in compounds is always +1. The oxidation number for alkaline earth metals in compounds is always +2.
\end{itemize}

Oxidation numbers do not have to be the same ones found on the periodic table. In fact, they will not always be whole numbers! 
Rule 3 cannot be violated! Remember, oxidation numbers are just an accounting system for keeping track of electrons.

Exercise - I$_2$ (Iodine: 0)

Exercise - MnO$_4^-$ (Manganese: +7, Oxygen: -2)


\end{document}