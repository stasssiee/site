\documentclass[../diffeq.tex]{subfiles}
\graphicspath{{\subfix{../figures/}}}
\begin{document}
\chapter{First-Order Differential Equations}
\section{Separable Equations}
\begin{definition}
    If the right-hand side of the equation 
    \[ \frac{\dd y}{\dd x}=f(x,y) \]
    can be expressed as a function $g(x)$ that depends only on $x$ times a function $p(y)$ that depends only on $y$, then the differential equation is called separable.
\end{definition}

To solve the equation 
\[ \frac{\dd y}{\dd x}=g(x)p(y) \]
multiply by $\dd x$ and by $h(y)=1/p(y)$ to obtain 
\[h(y)\dd y=g(x)\dd x \]
Then integrate both sides and you end up getting $H(y)=G(x)+C$, where we have merged the two constants of integration into a single symbol $C$.
The last equation gives an implicit solution to the differential equation.

\begin{example}
    Solve the nonlinear equation 
    \[\frac{\dd y}{\dd x}=\frac{x-5}{y^2}\]

    This can be rewritten as $y^2\dd y = (x-5)\dd x$. Integrating both sides results in $\frac{y^3}{3}=\frac{x^2}{2}-5x+C$.

    To get the explicit form just solve for $y$, which is trivial.
\end{example}

\begin{example}
    Solve the initial value problem 
    \[ \frac{\dd y}{\dd x}=\frac{y-1}{x+3} \qquad y(-1)=0 \]

    Doing Calc BC stuff gives us $y=1-\frac{1}{2}(x+3)$.
\end{example}

Be careful because you can be losing solutions. Ok bye!

\section{Linear Equations}
Remember a linear first-order equation is an equation that can be expressed in the form 
\[ a_1(x)\frac{\dd y}{\dd x}+a_0 y=b(x) \]
where $a_1(x)$, $a_0(x)$, and $b(x)$ depend only on the independent variable $x$, not on $y$.

Method for solving linear equation:
\begin{itemize}
    \item Write the equation in the standard form 
    \[ \frac{\dd y}{\dd x}+P(x)y=Q(x) \]
    \item Calculate the integrating factor $\mu(x)$ by the formula 
    \[ \mu(x)=\exp \left[\int P(x)\dd x \right] \]
    \item Multiply the equation in standard form by $\mu(x)$ and, recalling that the left-hand side is just $\frac{\dd}{\dd x}[\mu(x)y]$, obtain 
    \[ \mu(x)\frac{\dd y}{\dd x}+P(x)\mu(x)y=\mu(x)Q(x) \]
    \[ \frac{\dd}{\dd x}[\mu(x)y]=\mu(x)Q(x)\]
    \item Integrate the last equation and solve for $y$ by dividing by $\mu(x)$ to obtain.
\end{itemize}

\begin{example}
    Find the general solution to 
    \[ \frac{1}{x}{\dd y}{\dd x}-\frac{2y}{x^2}=x\cos x \qquad x>0 \]

    We have $\frac{\dd y}{\dd x}-\frac{2}{x}y=x^2\cos x$.

    The integrating factor $\mu(x)=e^{\int P(x)\dd x}$ which in this case is $e^{-2\int \frac{1}{x}\dd x}$ and this is equivalent to $\frac{1}{x^2}$.

    Using this we can multiply through in standard form then we have $\frac{1}{x^2}\frac{\dd y}{\dd x}-\frac{2}{x^3}y=\cos x$.

    The left side is just $\frac{\dd}{\dd x}\left(\frac{1}{x^2}\cdot\right) = \cos x$.

    Integrating and solving for $y$ we get that $y=x^2\sin x+Cx^2$.
\end{example}

\begin{example}
    For the initial value problem 
    \[ y' + y = \sqrt{1+\cos^2 x}\qquad y(1)=4 \]
    find the value of $y(2)$.

    Our $P(x)$ is $1$ here, so $\mu = e^x$.

    So the equation after multiplying through by it gives us that $\mu y' + \mu y = \mu \sqrt{1+\cos^2 x}$, or $e^xy'+e^xy=e^x\sqrt{1+\cos^2 x}$.

    This is equivalent to basically $\frac{\dd}{\dd x}(e^xy)=e^x\sqrt{1+\cos^2 x}$.

    This is $e^x y = \int e^x\sqrt{1+\cos^2 x}\dd x$.

    Using a calculator $y(2)=2.127$.
\end{example}

\begin{theorem}[Existence and Uniqueness of Solution]
    Suppose $P(x)$ and $Q(x)$ are continuous on an interval $(a,b)$ that contains the point $x_0$. Then for any choice of initial value $y_0$, there exists a unique solution $y(x)$ on $(a,b)$ to the initial value problem 
    \[ \frac{\dd y}{\dd x}+P(x)y=Qx \qquad y(x_0)=y_0 \]
    In fact the solution is given for a suitable value of $C$.
\end{theorem}

\section{Exact Equations}
\begin{definition}[Exact Differential Form]
    The differential form $M(x,y)\dd x + N(x,y)\dd y$ is said to be exact in a rectangle $R$ is there is a function $F(x,y)$ such that 
    \[ \frac{\partial F}{\partial x}(x,y)=M(x,y) \qquad \text{and} \qquad \frac{\partial F}{\partial y}(x,y)=N(x,y)\]

    for all $(x,y)$ in $R$. That is, the total differential of $F(x,y)$ satisfies 
    \[ \dd F(x,y)=M(x,y)\dd x+N(x,y)\dd y \]
    If $M(x,y)\dd x + N(x,y)\dd y$ is an exact differential form, then the equation 
    \[ M(x,y)\dd x + N(x,y)\dd y =0 \]
    is called an exact equation.
\end{definition}

\begin{theorem}[Test for Exactness]
    Suppose the first partial derivatives of $M(x,y)$ and $N(x,y)$ are continuous in a rectangle $R$. Then 
    \[ M(x,y)\dd x+ N(x,y)\dd y =0 \]
    is an exact equation in $R$ if and only if the compatibility condition 
    \[ \frac{\partial M}{\partial y}(x,y)=\frac{\partial N}{\partial x}(x,y) \]
    holds for all $(x,y)$ in $R$.
\end{theorem}

\pagebreak
\begin{example}
    Solve the differential equation 
    \[ \frac{\dd y}{\dd x}=-\frac{2xy^2+1}{2x^2y}\]

    Ok so this is not separable or linear, so we use exactness.

    \begin{align*}
        \frac{\dd y}{\dd x}+\frac{2xy^2+1}{2x^2y}=0 \\ 
        \dd y + \frac{2xy^2+1}{2x^2y}\dd x = 0\\
        \frac{2xy^2+1}{2x^2y}\dd x + 1 \dd y = 0 \\ 
    \end{align*}

    This is the same form we want. 

    Another form we can get is $(2xy^2+1)\dd x + 2x^2y\dd y =0$.

    Another form we can get is $1\dd x+\frac{2x^2y}{2xy^2+1}\dd y =0$.

    We are now looking for a $F(x,y)=c$ and we know this is true when $\frac{\partial m}{\partial y}=\frac{\partial n}{\partial x}$.

    So the second one of these is probably the best, so we now have $m=2xy^2+1$ and $n=2x^2y$. 

    Doing the partial of $m$ with respect to $y$ we get $4xy$ and the partial of $n$ with respect to $x$ is $4xy$ and these are the same.

    Let $F(x,y)=x^2y^2+x=C$. The partial of this function with respect to $x$ is $2xy^2+1$ and the partial of this function with respect to $y$ is $2x^2y$ and this is the same as previous.
\end{example}

Method for Solving Exact Equations:
\begin{itemize}
    \item If $M\dd x+N\dd y =0$ is exact, then $\partial F/\partial x=M$. Integrate this last equation with respect to $x$ to get 
    \[ F(x,y)=\int M(x,y)\dd x + g(y)\]
    \item To determine $g(y)$, take the partial derivative with respect to $y$ of both sides of the aboev equation and substitute $N$ for $\partial F/\partial y$. We can now solve for $g'(y)$.
    \item Integrate $g'(y)$ to obtain $g(y)$ up to a numerical constant. Substituting $g(y)$ into the equation from step 1 gives $F(x,y)$
    \item The solution to $M\dd x+N\dd y = 0$ is given implicitly by 
    \[ F(x,y)=C \]
\end{itemize}
(Alternatively, starting with $\partial F/\partial y=N$, the implicit solution can be foudn by first integrating with respect to $y$.)

\pagebreak
\begin{example}
    Solve 
    \[ (2xy-\sec^2 x)\dd x + (x^2+2y)\dd y = 0 \]

    Let $m$ be the first term and $n$ be the second term, and the partial derivatives of these are the same, so they are exact.

    Let $F(x,y)=\int 2xy-\sec^2 x \dd x$. When we integrate this, we get $yx^2-\tan x$. In this case, the constant is anything with $y$, so the integral is equivalent to 
    $yx^2-\tan x + g(y)$.

    Now we take the $\frac{\partial F}{\partial y} = x^2-0+g'(y)$. These two are $n$ so $x^2+2y=x^2+g'(y)$, so solving for $g(y)$ we get that this is equal to $y^2+C$.

    So $F(x,y)=xy^2-\tan x + y^2 = C$.
\end{example}

\ex Solve $(1+e^xy+xe^xy)\dd x + (xe^x+2)\dd y = 0$.

Solution: $x+xye^x+2y=C$.

\begin{example}
    Solve 
    \[ (x+3x^3\sin y)\dd x+(x^4\cos y)\dd y = 0 \]

    Doing the partials originally makes them not equal to each other.
    
    We can get this to exact form by multiplying through by $x^{-1}$. When we do this we get $(1+3x^2\sin y)\dd x+x^3\cos y \dd y = 0$ and the partials of these are the same.

    $x^{-1}$ is called an integrating factor.

    Integrating $m$ with respect to $x$, we get that $F(x,y)=\int 1+3x^2\sin y\dd x = x+\sin y\cdot x^3 + g(y)$.

    Doing the partial of $F$ with respect to $y$, we get $\frac{\partial F}{\partial y}=x^3\cos y = 0 + x^3\cos y + g'(y)$, and this gets that $g(y)=C$.

    So the answer is $x+x^3\sin y = C$.
\end{example}

\section{Special Integrating Factors}
\begin{definition}
    If the equation 
    \[ M(x,y)\dd x + N(x,y)\dd y = 0\]
    is not exact, but the equation 
    \[ \mu(x,y)M(x,y)\dd x + \mu(x,y)N(x,y)\dd y = 0 \]
    which results from multiplying the first equation by the function $\mu(x,y)$, is exact, then $\mu(x,y)$ is called an integrating factor of the first equation.
\end{definition}

\pagebreak
\begin{theorem}[Special Integrating Factors]
    If $(\partial M/\partial y-\partial N/\partial x)/N$ is continuous and depends only on $x$, then 
    \[ \mu(x)=\exp \left[\int \left(\frac{\partial M/\partial y-\partial N/\partial x}{N}\right)\dd x\right] \]
    is an integrating factor for an equation. If $(\partial N/\partial x - \partial M/\partial y)/M$ is continuous and depends only on $y$, then 
    \[ \mu(y)=\exp \left[\int \left(\frac{\partial M/\partial x-\partial N/\partial y}{M}\right)\dd y\right] \]
    is an integrating factor for the same equation.
\end{theorem}

Method for Finding Special Integrating Factors:

If $M\dd x + N\dd y = 0$ is neither separable nor linear, compute $\partial M/\partial y$ and $\partial N/\partial x$. If $\partial M/\partial y = \partial N/\partial x$, then the equation is exact. If it is not exact, consider 
\[ \frac{\partial M/\partial y-\partial N/\partial x}{N}\]
If this is a function of just $x$, then an integrating factor is given by the formula above of $\mu(x)$. If not consider 
\[ \frac{\partial N/\partial x - \partial M/\partial y}{M} \]
If this is a function of just $y$< then an integrating factor is given by above of $\mu(y)$.

\begin{example}
    Solve $(2x^2+y)\dd x + (x^2y-x)\dd y = 0$

    When we do the partials, we get that $1\neq 2xy-1$.

    So lets look at $\frac{\partial m/\partial y - \partial n/\partial x}{N}$, which is $\frac{1-(2xy-1)}{x^2y-x} = \frac{-2}{x}$ which is just a function of $x$. So we have that $\mu = e^{\int -\frac{2}{x}\dd x}$, so we don't have to look at the one in terms of $y$.

    Doing the integral of all this gives us that $e^{-2\ln x} = x^{-2}$. So when we multiply through by $x^{-2}$, we get that $(2+x^{-2}y) \dd x + (y-x^{-1})\dd y =0$.

    The partials are equal to each other, so this equation is now exact.

    Now we find $F(x,y)$ by integrating $m$, so $\int (2+x^{-2}y)\dd x = 2x+-x^{-1}y+g(y)=F(x)$

    Now we differentiation with respect to $y$ so $\frac{\partial F}{\partial y}=y-x^{-1}=-x^{-1}+g'(y)$, so $g(y)=\frac{y^2}{2}$

    The solution is therefore $2x-x^{-1}y+\frac{y^2}{2}=C$.
\end{example}

\section{Substitutions and Transformations}
Substitution Procedure:
\begin{itemize}
    \item Identify the type of equation and determine the appropriate substitution or transformation 
    \item Rewrite the original equation in terms of new variables 
    \item Solve the transformed equation 
    \item Express the solution in terms of the original variables 
\end{itemize}

\pagebreak
\begin{definition}[Homogeneous Equation]
    If the right-hand side of the equation
    \[ \frac{\dd y}{\dd x} = f(x,y) \]
    can be expressed as a function of the ratio $y/x$ alone, then we say the equation is homogeneous.
\end{definition}

To solve a homogeneous equation, use the substitution $v=\frac{y}{x}$; $\frac{\dd y}{\dd x} = v+x\frac{\dd v}{\dd x}$ to transform the equation into a separable equation.

\begin{example}
    Solve $(xy+y^2+x^2)\dd x - x^2\dd y = 0$.

    Solving for $\frac{\dd y}{\dd x}$ we get that this is equal to $\frac{-x^2-y^2-xy}{-x^2}$ and this simplifies to $1+\left(\frac{y}{x}\right)^2 + \frac{y}{x}$.

    This is equivalent to $v+x\frac{\dd v}{\dd x} = 1+v^2+v$. We end up getting that $\frac{\dd v}{\dd x}=\frac{v^2+1}{x}$ and this can be done by separation. The solution is $y=x\tan(\ln|x|+C)$ after solving.
\end{example}

To solve an equation of the form $\frac{\dd y}{\dd x}=G(ax+by)$, use the substitution $z=ax+by$ to transform the equation into a separable equation.
\begin{example}
    Solve $\frac{\dd y}{\dd x}=y-x-1+(x-y+2)^{-1}$

    First we have $\frac{\dd y}{\dd x}=-(x-y)-1+(x-y+2)^{-1}$

    So substituting with $z=x-y$, we have that $\frac{\dd z}=1-\frac{\dd z}{\dd x}$. Knowing this, the equation is equal to $1-\frac{\dd z}{\dd x}=-z-1+(z+2)^{-1}$. From this this simplifies to 
    $\frac{\dd z}{\dd x}=z+2-(z+2)^{-1}$. 

    So now we write this into a separable equation with $\frac{(z+2)\dd z}{(z+2)^2-1} = \dd x$.

    Separating by parts and substituting gives $(x-y+2)^2=ce^{2x}+1$
\end{example}

\begin{definition}[Bernoulli Equation]
    A first-order equation that can be written in the form 
    \[ \frac{\dd y}{\dd x}+P(x)y=Q(x)y^n \]
    where $P(x)$ and $Q(x)$ are continuous on the interval $(a,b)$ and $n$ is a real number, is called a Bernoulli equation.
\end{definition}

To solve a Bernoulli equation use the substitution $v=y^{1-n}$ to transform the equation into a linear equation.

\pagebreak
\begin{example}
    Solve $\frac{\dd y}{\dd x}-5y=-\frac{5}{2}xy^3$.

    From above, we have $v=y^{-2}$ and $\frac{\dd v}{\dd x}=-2y^{-3}\frac{\dd u}{\dd x}$ and $-\frac{1}{2}\frac{\dd v}{\dd x}=y^{-3}\frac{\dd y}{\dd x}$.

    So multiplying through by $y^{-3}$ and substituting, we get that $-\frac{1}{2}\frac{\dd v}{\dd x}-5v=-\frac{5}{2}x$.

    This is equal to $\frac{\dd v}{\dd x}+10v=5x$. The integrating factor here is $\mu = e^{\int P(x)\dd x}$, which is $e^{10x}$ in this case.

    Multiplying through by $\mu$, we get that $e^{10x}\frac{\dd v}{\dd x}+10ve^{10x}=5xe^{10}x$ and the LHS should be equal to $\frac{\dd}{\dd x}(e^{10x}v)=5xe^{10x}$.

    Using elementary integration techniques the answer is $y^{-2}=\frac{x}{2}-\frac{1}{20}+Ce^{-10x}$.
\end{example}
\end{document}