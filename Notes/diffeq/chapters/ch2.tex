\documentclass[../diffeq.tex]{subfiles}
\graphicspath{{\subfix{../figures/}}}
\begin{document}
\chapter{First-Order Differential Equations}
\section{Separable Equations}
\begin{definition}
    If the right-hand side of the equation 
    \[ \frac{\dd y}{\dd x}=f(x,y) \]
    can be expressed as a function $g(x)$ that depends only on $x$ times a function $p(y)$ that depends only on $y$, then the differential equation is called separable.
\end{definition}

To solve the equation 
\[ \frac{\dd y}{\dd x}=g(x)p(y) \]
multiply by $\dd x$ and by $h(y)=1/p(y)$ to obtain 
\[h(y)\dd y=g(x)\dd x \]
Then integrate both sides and you end up getting $H(y)=G(x)+C$, where we have merged the two constants of integration into a single symbol $C$.
The last equation gives an implicit solution to the differential equation.

\begin{example}
    Solve the nonlinear equation 
    \[\frac{\dd y}{\dd x}=\frac{x-5}{y^2}\]

    This can be rewritten as $y^2\dd y = (x-5)\dd x$. Integrating both sides results in $\frac{y^3}{3}=\frac{x^2}{2}-5x+C$.

    To get the explicit form just solve for $y$, which is trivial.
\end{example}

\begin{example}
    Solve the initial value problem 
    \[ \frac{\dd y}{\dd x}=\frac{y-1}{x+3} \qquad y(-1)=0 \]

    Doing Calc BC stuff gives us $y=1-\frac{1}{2}(x+3)$.
\end{example}

Be careful because you can be losing solutions. Ok bye!

\section{Linear Equations}
\section{Exact Equations}
\section{Special Integrating Factors}
\section{Substitutions and Transformations}

\end{document}