\documentclass[../diffeq.tex]{subfiles}
\graphicspath{{\subfix{../figures/}}}
\begin{document}
\chapter{Introduction to Systems}

\section{Differential Operators and the Elimination Method for Systems}
$y'(t)= \frac{\dd y}{\dd t}=\frac{\dd}{\dd t}y$ is used to emphasize that the derivative of a function $y$ is the result of operation on the function $y$ with the differentiation operator 
$\frac{\dd}{\dd t}$. Similarly $y''(t)=\frac{\dd^2 y}{\dd t^2}=\frac{\dd}{\dd t}\frac{\dd}{\dd t}y$. Commonly the symbol $D$ is used instead of $\frac{d}{\dd t}$.

$y''+4y'+3y=0$ is represented by $(D^2+4D+3)[y]=0$.

We use the convention that the operator ``product'' is a composition when it operates on functions.
\ex Show that the operator $(D+1)(D+3)$ is the same as $D^2+4D+3$.

\ex Show that $(D+3t)D$ is not the same as $D(D+3t)$.

Because the ``algebra'' of differential operators follows the same rules as the algebra of polynomials, we can adapt the elimination method used to solve algebraic operations to solve any system of 
linear differential equations with constant coefficients.

\pagebreak
\begin{example}
    Solve the system 
    \begin{align*}
        x'(t)=3x(t)-4y(t)+1 \\
        y'(t) = 4x(t)-7y(t)+10t
    \end{align*}

    First we can solve this by writing this in differential operator form:
    \begin{align*}
        (D-3)x+4y = 1\\
        -4x + (D+7)y = 10t 
    \end{align*}

    By eliminating this using algebra, we get that $(D^2+4D-5)y=14-30t$ (this is essentially $y''+4y-5y=14-30t$).

    The first step to solving this nonhomogeneous equation is to solve the homogeneous equation.

    The homogeneous equation results in giving us that $y_h = c_1e^{-5t}+c_2e^t$. 

    We can guess the particular solution as $y_p=At+B$, so $y'_p =A$ and $y''_p=0$. So we have that $0+4A-5(At+B)=14-30t$. From this, we have $4A-5B=14$ and $-5A=-30$, so $A=6$ and $B=2$.

    So the general solution for $y$ is $y=C_1e^{-5t}+C_2e^t+6t+2$.

    Now we need to find the function $x$.

    We can find $x$ from elimination once again.

    Using the same methods above, we have $x=k_1e^{-5t}+k_2e^t+8t+5$. 

    We should only end up with two constants, so we need to find the relationship between the constants of $y$ and $x$.

    So using the derivative of $x$ we get $-5k_1e^{-5t}+k_2e^t+8=3k_1e^{-5t}+3k_2e^t+24t+15-4C_1e^{-5t}-4C_2e^t-24t-8+1$.

    Simplifying we get $-5k_1e^{-5t}-k_2e^t = (3k_1-4C_1)e^{-5t}+(3k_2-4C_2)e^t$. So we have that $-5k_1=3k_1-4C_1$ and $k_2=3k_2-4C_2$, so $C_1=2k_1$ and $C_2=\frac{1}{2}k_2$.

    
\end{example}

To find a general solution for the system 
\begin{align*}
    L_1[x]+L_2[y]=f_1 \\
    L_3[x]+L_4[y]=f_2    
\end{align*}
where $L_1$, $L_2$, $L_3$, and $L_4$ are polynomials in $D=\dd / \dd t$.
\begin{itemize}
    \item Make sure that the system is written in operator form.
    \item Eliminate one of the variables, say, $y$, and solve the resulting equation for $x(t)$. If the system is degenerate, stop! A separate analysis is required to determine whether or not there are solutions.
    \item (Shortcut) If possible, use the system to derive an equation that involves $y(t)$ but not its derivatives (Otherwise go to the next step). Substitute the found expression for $x(t)$ into this equation to get a formula for $y(t)$. The expressions for $x(t)$, $y(t)$ give the desired general solution.
    \item Eliminate $x$ from the system and solve for $y(t)$. Solving for $y(t)$ gives more constants - in fact, twice as many as needed.
    \item Remove the extra constants by substituting the expressions for $x(t)$ and $y(t)$ into one or both of the equations in the system. Write the expressions for $x(t)$ and $y(t)$ in terms of the remaining constants.
\end{itemize}

\pagebreak
\begin{example}
    Find a general solution to 
    \begin{align*}
        x''(t)+y'(t)-x(t)+y(t)=-1 \\ 
        x'(t)+y'(t)-x(t)=t^2
    \end{align*}

    Subtracting the two, we get that $x''-x'+y=-1-t^2$.

    We will now solve for $x$. In differential operator form we have $(D^2-1)x+(D+1)y=-1$ and $(D-1)x+Dy=t^2$.

    Eliminating for $x$ gives the $x_h=C_1e^t+C_2te^t+C_3e^{-t}$.

    The particular solution will be in the form $At^2+Bt+C$. The first derivative of this is $2At+B$ and the second derivative is $2A$.

    Plugging this in gives us that $(D^2-2D+1)(D+1)x=-2t-t^2$. This is equal to $(D^3-D^2-D+1)x=-2t-t^2$.

    This is $2-2A-2At-B+At^2+Bt+C=-2t-t^2$. From this we see that $A=-1, B=-4, C=-6$.

    The particular solution $x_p=-t^2-4t-6$.

    Now since we have $x=c_1e^t+C_2te^t+C_3e^{-t}-t^2-4t-6$, we can solve this for the $y$.

    Plugging the general solution in gives us $y=-c_1e^t-c_2e^t-c_2e^t-c_2te^t-c_1e^{-t}+2+c_1e^t+c_2e^t+c_2te^t-c_3e^{-t}-2t-4-1-t^2$.

    Simplifying this gives $y=-c_2e^t-2c_3e^{-t}-t^2-2t-3$.
\end{example}



\section{Solving Systems and Higher-Order Equations Numerically}
If equations have variable coefficients, the solution process is limited. The solutions can be expressed as infinite series which can be very laborious (with the exception of the Cauchy-Euler equation).
Fortunately all cases, constant and variable coefficient, nonlinear and higher order equations and systems can be addressed numerically.

We will express differential equations as a system in normal form and used the basic Euler method for computer solution that can be ``vectorized'' to apply to such systems.

A system of $m$ differential equations in the $m$ unknown functions $x_1(t),x_2(t),\dots,x_m(t)$ expressed as 
\begin{align*}
    x'_1(t)=f_1(t,x_1,x_2,\dots, x_m)\\
    x'_2(t)=f_2(t,x_1,x_2,\dots,x_m)\\
    \vdots \\ 
    x'_m(t)=f_m(t,x_1,x_2,\dots,x_m)
\end{align*}
is said to be in normal form. It can be expressed in vector form as $x'=t(t,x)$.

A single higher-order equation can always be converted to an equivalent system of first-order equations. To convert an $m$th-order differential equation 
\[ y^{(m)}(t)=f(t,y,y',\dots,y^{(m-1)}) \]
into a first-order system, introduce additional unknowns, the sequence of derivatives of $y$:
\[ x_1(t)=y(t), x_2(t)=y'(t),\dots, x_m(t)=y^{(m-1)}(t) \]

We obtain this system 
\begin{align*}
    x'_1(t)=y'(t)=x_2(t) \\
    x'_2(t)=y''(t)=x_3(t) \\
    \vdots \\
    x'_{m-1}(t)=y^{(m-1)}(t)=x_m(t) \\
    x'_m(t) = y^{(m)}(t)=f(t,x_1,x_2,\dots,x_m)
\end{align*}

\begin{example}
    Convert the initial value problem 
    \[ y''(t)+3ty'(t)+y(t)^2=\sin t \qquad y(0)=1, y'(0)=5 \]
    into an initial value problem for a system in normal form.

    We have $x_1=y, x'_1=y', x_2 = y'$ and $x'_2=y''$.

    In normal form, we would have $y''=\sin t - 3ty'+y^2$ which is what $x'_2$ and $y''$ are equal to. Note that this ia function of $f(t,y,y')$.

    We also have $x'_1=y'=x_2$ and we also know that $\sin t - 3tx_2+x_1^2 = x'_2$.

    We now have a system where $x'_1=x_2, x'_2=\sin t - 3tx_2+x_1^2$ where we know $x_1(0)=1$ and $x_2(0)=5$.
\end{example}

\end{document}