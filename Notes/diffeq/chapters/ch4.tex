\documentclass[../diffeq.tex]{subfiles}
\graphicspath{{\subfix{../figures/}}}
\begin{document}
\chapter{Linear Second-Order Equations}
\section{Introduction: The Mass-Spring Oscillator}
A damped mass-spring oscillator consists of a mass $m$ attached to a spring fixed at one end. Devise a differential equation that governs the motion of this oscillator, taking into account 
the forces acting on it due to the spring elasticity, damping friction, and possible external influences.

Newton's second law - force equals mass times acceleration ($F=ma$) - is the most commonly encountered differential equation. It is an ordinary differential equation of the second order 
since acceleration is the second derivative of position ($y$) with respect to time ($a=\dd^2 y/\dd t^2$).

If the spring is unstretched and the inertial mass $m$ is still, the system is at equilibrium. We stretch the coordinate $y$ of the mass by its displacement from the equilibrium position.

When the mass $m$ is displaced from equilibrium, the spring is stretched or compressed and it exerts a force that resists the displacement. For most springs this force is directly proportional to the displacement $y$ 
and is given by Hooke's law.
\[ F_{\text{spring}}= -ky \]
where the positive constant $k$ is known as the stiffness (spring constant) and the negative sign reflects the opposing nature of the force. Hooke's law is only valid for sufficiently small displacements.

Usually all mechanical systems also experience friction. For vibrational motion this force is usually modeled accurately by a term proportional to velocity:
\[ F_{\text{friction}}=-b\frac{\dd y}{\dd t}=-by' \]
where $b\geq 0$ is the damping coefficient and the negative sign reflects the opposing nature of the force.

The other forces on the oscillator are usually regarded as external to the system. Although they may be gravitational, electrical, or magnetic, commonly the most important external forces are transmitted 
to the mass by shaking the supports holding the system. For now we refer to the combined external forces by a single known function $F_{\text{ext}}(t)$. Newton's law provides the differential equation for the mass-spring oscillator:
\[ my''=-ky-by'+F_{\text{ext}}(t)\] or \[ my''+by'+ky=F_{\text{ext}}(t) \]

\begin{example}
    Verify that if $b=0$ and $F_{\text{ext}}=0$, that the above equation has a solution of the form $y(t)=\cos(\omega t)$ and the angular frequency $\omega$ increases with $k$ and decreases with $m$.

    The differential equation is $my''+by'+ky=F_{\text{ext}}$ and that $my''+ky=0$.

    Since we are given what $y(t)$ is, taking the derivative of the differential equation gives that $-m\omega^2\cos(\omega t)+k\cos(\omega t)=0$, and solving for $\omega$, we get that $\omega=\sqrt{\frac{k}{m}}$.

    If $k$ increases, $\omega$ increases, and if $m$ increases, $\omega$ decreases.
\end{example}

\begin{example}
    Verify that the exponentially damped sinusoid given by $y(t)=e^{-3t}\cos 4t$ is a solution to the above differential equation if $F_{\text{ext}}=0, m=1, k=25$, and $b=6$.

    From the differential equation $my''+by'+ky=F_{\text{ext}}$, we can plug in stuff and we get that $y''+by'+25y=0$.

    Since we are given $y(t)$ we can find the derivatives and substitute. What we are given is quite long, but essentially it cancels out to $0=0$, which means it is a solution to the system.
\end{example}

\ex Verify that the simple exponential function $y(t)=e^{-5t}$ is a solution to the above differential equation if $F_{\text{ext}}=0$, $m=1$, $k=25$, and $b=10$.

Sometimes the external force will make the system look somewhat erratic. There are many real world examples where the external force must defintely be taken into account.

\section{Homogeneous Linear Equations: The General Solution}
A second-order constant-coefficient differential equation has the form 
\[ ay''+by'+cy=f(t) \qquad (a\neq 0) \]
A homogeneous second-order constant-coefficient differential equation is the special case with $f(t)=0$.
\[ ay''+by'+cy=0 \qquad (a\neq 0) \]
A solution of this equation has the form $y=e^{rt}$. The resulting equation $ar^2+br+c=0$ is called the auxiliary equation (or characteristic equation) associated with the homogeneous equation.

\begin{example}
    Find a pair of solutions to 
    \[ y''+5y'-6y=0\]

    Plugging in $y=e^{rt}$, we get that $e^{rt}(r^2+5r-6)$ after substituting. Solving the quadratic $r^2+5r-6 =0$, we get that $r=-6$ and $r=1$, which is $y=e^{-6t}$ and $y=e^t$.
\end{example}

Note that the zero function, $y(t)=0$ is always a solution to an equation above (figure this out later). In addition when we have a pair of solutions $y_1(t)$ and $y_2(t)$, we can construct an infinite number of other solutions by forming linear combinations:
\[ y(t)=c_1y_1(t)+c_2y_2(t) \]
for any choice of the constants $c_1$ and $c_2$. This is a two-paramter solution form since there are two unknown constants. To find a specific solution, two initial conditions are needed.

\begin{example}
    Solve the initial value problem 
    \[ y''+2y'-y=0\qquad y(0)=0, \qquad y'(0)=-1 \]

    Doing the default substitution, our auxiliary equation is $r^2+2r-1=0$. 

    The solutions from this are $r=-1+\sqrt{2}$ and $r=-1-\sqrt{2}$. Remember $y=e^{rt}$. Using the initial conditions, we get that $c_1=-\frac{\sqrt{2}}{4}$ and $c_2=\frac{\sqrt{2}}{4}$.
\end{example}

\begin{theorem}
    For any real numbers $a(\neq 0),b,c,t_0,Y_0$, and $Y_1$, there exists a unique soultion to the initial value problem. 
    \[ ay''+by'+cy=0 \qquad y(t_0)=Y_0 \qquad y'(t_0)=y_1 \]
    The solution is valid for all $t$ in $(-\infty,\infty)$
\end{theorem}

\begin{definition}
    A pair of functions $y_1(t)$ and $y_2(t)$ is said to be linearly independent on the interval $t$ if and only if neither of them is a constant multiple of the other on all of $t$. We say that 
    $y_1$ and $y_2$ are linearly dependent on $t$ if one of them is a constant multiple of the other on all of $t$.
\end{definition}

\begin{theorem}
    If $y_1(t)$ and $y_2(t)$ are any two solutions to the differential equation that are linearly independent on $(-\infty,\infty)$, then unique constants $c_1$ and $c_2$ can always be found so that $c_1y_1(t)+c_2y_2(t)$ satisfies the initial value problem on $(-\infty,\infty)$.
\end{theorem}

\begin{definition}[Wronksian]
    Suppose each of the functions $f_1(x),f_2(x),\dots f_n(x)$ possess at least $n-1$ derivatives. 
    
    The determinant \[
        W(f_1, f_2, \dots, f_n) =
        \begin{vmatrix}
        f_1 & f_2 & \dots & f_n \\
        f_1' & f_2' & \dots & f_n' \\
        \vdots & \vdots & \ddots & \vdots \\
        f_1^{(n-1)} & f_2^{(n-1)} & \dots & f_n^{(n-1)}
        \end{vmatrix}
        \]
        is called the Wronksian of the functions.
\end{definition}

\begin{theorem}
    Let $y_1,y_2,\dots,y_n$ be $n$ solutions of the homogeneous linear $n$th-order differential equation on an interval $I$. Then the set of solutions is linearly independent on $I$ 
    if and only if $W(y_1,y_2,\dots,y_n)\neq 0$ for every $x$ in the interval.
\end{theorem}

Distinct real roots: If the auxiliary equation has distinct real roots $r_1$ and $r_2$, then both $y_1(t)=e^{rt}$ and $y_2(t)=e^{rt}$ are solutions to the above differential equation and $y(t)=e_1e^{rt}+e_2e^{rt}$ is a general solution.

Repeated root: if the auxiliary equation has a repeated root $r$, then both $y_1(t)=e^{rt}$ and $y_2(t)=te^{rt}$ are solutions to the differential equation and $y(t)=e_1e^{rt}+e_2te^{rt}$ is a general solution.

A homogeneous linear $n$th-order equation has a general solution of the form 
\[ y(t)=c_1y_1(t)+c_2y_2(t)+\dots+c_ny_n(t) \]
where the individual solutions $y_i(t)$ are linearly independent, i.e. no $y_i(t)$ is expressible as a linear combination of the others.

\section{Auxiliary Equations with Complex Roots}
\section{Nonhomogeneous Equations: the Method of Undetermined Coefficients}
\section{The Superposition Principle and Undetermined Coefficients Revisited}
\section{Variation of Parameters}
\section{Variable-Coefficient Equations}

\end{document}