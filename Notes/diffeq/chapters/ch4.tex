\documentclass[../diffeq.tex]{subfiles}
\graphicspath{{\subfix{../figures/}}}
\begin{document}
\chapter{Linear Second-Order Equations}
\section{Introduction: The Mass-Spring Oscillator}
A damped mass-spring oscillator consists of a mass $m$ attached to a spring fixed at one end. Devise a differential equation that governs the motion of this oscillator, taking into account 
the forces acting on it due to the spring elasticity, damping friction, and possible external influences.

Newton's second law - force equals mass times acceleration ($F=ma$) - is the most commonly encountered differential equation. It is an ordinary differential equation of the second order 
since acceleration is the second derivative of position ($y$) with respect to time ($a=\dd^2 y/\dd t^2$).

If the spring is unstretched and the inertial mass $m$ is still, the system is at equilibrium. We stretch the coordinate $y$ of the mass by its displacement from the equilibrium position.

When the mass $m$ is displaced from equilibrium, the spring is stretched or compressed and it exerts a force that resists the displacement. For most springs this force is directly proportional to the displacement $y$ 
and is given by Hooke's law.
\[ F_{\text{spring}}= -ky \]
where the positive constant $k$ is known as the stiffness (spring constant) and the negative sign reflects the opposing nature of the force. Hooke's law is only valid for sufficiently small displacements.

Usually all mechanical systems also experience friction. For vibrational motion this force is usually modeled accurately by a term proportional to velocity:
\[ F_{\text{friction}}=-b\frac{\dd y}{\dd t}=-by' \]
where $b\geq 0$ is the damping coefficient and the negative sign reflects the opposing nature of the force.

The other forces on the oscillator are usually regarded as external to the system. Although they may be gravitational, electrical, or magnetic, commonly the most important external forces are transmitted 
to the mass by shaking the supports holding the system. For now we refer to the combined external forces by a single known function $F_{\text{ext}}(t)$. Newton's law provides the differential equation for the mass-spring oscillator:
\[ my''=-ky-by'+F_{\text{ext}}(t)\] or \[ my''+by'+ky=F_{\text{ext}}(t) \]

\begin{example}
    Verify that if $b=0$ and $F_{\text{ext}}=0$, that the above equation has a solution of the form $y(t)=\cos(\omega t)$ and the angular frequency $\omega$ increases with $k$ and decreases with $m$.

    The differential equation is $my''+by'+ky=F_{\text{ext}}$ and that $my''+ky=0$.

    Since we are given what $y(t)$ is, taking the derivative of the differential equation gives that $-m\omega^2\cos(\omega t)+k\cos(\omega t)=0$, and solving for $\omega$, we get that $\omega=\sqrt{\frac{k}{m}}$.

    If $k$ increases, $\omega$ increases, and if $m$ increases, $\omega$ decreases.
\end{example}

\begin{example}
    Verify that the exponentially damped sinusoid given by $y(t)=e^{-3t}\cos 4t$ is a solution to the above differential equation if $F_{\text{ext}}=0, m=1, k=25$, and $b=6$.

    From the differential equation $my''+by'+ky=F_{\text{ext}}$, we can plug in stuff and we get that $y''+by'+25y=0$.

    Since we are given $y(t)$ we can find the derivatives and substitute. What we are given is quite long, but essentially it cancels out to $0=0$, which means it is a solution to the system.
\end{example}

\ex Verify that the simple exponential function $y(t)=e^{-5t}$ is a solution to the above differential equation if $F_{\text{ext}}=0$, $m=1$, $k=25$, and $b=10$.

Sometimes the external force will make the system look somewhat erratic. There are many real world examples where the external force must defintely be taken into account.

\section{Homogeneous Linear Equations: The General Solution}
A second-order constant-coefficient differential equation has the form 
\[ ay''+by'+cy=f(t) \qquad (a\neq 0) \]
A homogeneous second-order constant-coefficient differential equation is the special case with $f(t)=0$.
\[ ay''+by'+cy=0 \qquad (a\neq 0) \]
A solution of this equation has the form $y=e^{rt}$. The resulting equation $ar^2+br+c=0$ is called the auxiliary equation (or characteristic equation) associated with the homogeneous equation.

\begin{example}
    Find a pair of solutions to 
    \[ y''+5y'-6y=0\]

    Plugging in $y=e^{rt}$, we get that $e^{rt}(r^2+5r-6)$ after substituting. Solving the quadratic $r^2+5r-6 =0$, we get that $r=-6$ and $r=1$, which is $y=e^{-6t}$ and $y=e^t$.
\end{example}

Note that the zero function, $y(t)=0$ is always a solution to an equation above (figure this out later). In addition when we have a pair of solutions $y_1(t)$ and $y_2(t)$, we can construct an infinite number of other solutions by forming linear combinations:
\[ y(t)=c_1y_1(t)+c_2y_2(t) \]
for any choice of the constants $c_1$ and $c_2$. This is a two-paramter solution form since there are two unknown constants. To find a specific solution, two initial conditions are needed.

\begin{example}
    Solve the initial value problem 
    \[ y''+2y'-y=0\qquad y(0)=0, \qquad y'(0)=-1 \]

    Doing the default substitution, our auxiliary equation is $r^2+2r-1=0$. 

    The solutions from this are $r=-1+\sqrt{2}$ and $r=-1-\sqrt{2}$. Remember $y=e^{rt}$. Using the initial conditions, we get that $c_1=-\frac{\sqrt{2}}{4}$ and $c_2=\frac{\sqrt{2}}{4}$.
\end{example}

\begin{theorem}
    For any real numbers $a(\neq 0),b,c,t_0,Y_0$, and $Y_1$, there exists a unique soultion to the initial value problem. 
    \[ ay''+by'+cy=0 \qquad y(t_0)=Y_0 \qquad y'(t_0)=y_1 \]
    The solution is valid for all $t$ in $(-\infty,\infty)$
\end{theorem}

\begin{definition}
    A pair of functions $y_1(t)$ and $y_2(t)$ is said to be linearly independent on the interval $t$ if and only if neither of them is a constant multiple of the other on all of $t$. We say that 
    $y_1$ and $y_2$ are linearly dependent on $t$ if one of them is a constant multiple of the other on all of $t$.
\end{definition}

\begin{theorem}
    If $y_1(t)$ and $y_2(t)$ are any two solutions to the differential equation that are linearly independent on $(-\infty,\infty)$, then unique constants $c_1$ and $c_2$ can always be found so that $c_1y_1(t)+c_2y_2(t)$ satisfies the initial value problem on $(-\infty,\infty)$.
\end{theorem}

\begin{definition}[Wronksian]
    Suppose each of the functions $f_1(x),f_2(x),\dots f_n(x)$ possess at least $n-1$ derivatives. 
    
    The determinant \[
        W(f_1, f_2, \dots, f_n) =
        \begin{vmatrix}
        f_1 & f_2 & \dots & f_n \\
        f_1' & f_2' & \dots & f_n' \\
        \vdots & \vdots & \ddots & \vdots \\
        f_1^{(n-1)} & f_2^{(n-1)} & \dots & f_n^{(n-1)}
        \end{vmatrix}
        \]
        is called the Wronksian of the functions.
\end{definition}

\begin{theorem}
    Let $y_1,y_2,\dots,y_n$ be $n$ solutions of the homogeneous linear $n$th-order differential equation on an interval $I$. Then the set of solutions is linearly independent on $I$ 
    if and only if $W(y_1,y_2,\dots,y_n)\neq 0$ for every $x$ in the interval.
\end{theorem}

Distinct real roots: If the auxiliary equation has distinct real roots $r_1$ and $r_2$, then both $y_1(t)=e^{rt}$ and $y_2(t)=e^{rt}$ are solutions to the above differential equation and $y(t)=e_1e^{rt}+e_2e^{rt}$ is a general solution.

Repeated root: if the auxiliary equation has a repeated root $r$, then both $y_1(t)=e^{rt}$ and $y_2(t)=te^{rt}$ are solutions to the differential equation and $y(t)=e_1e^{rt}+e_2te^{rt}$ is a general solution.

A homogeneous linear $n$th-order equation has a general solution of the form 
\[ y(t)=c_1y_1(t)+c_2y_2(t)+\dots+c_ny_n(t) \]
where the individual solutions $y_i(t)$ are linearly independent, i.e. no $y_i(t)$ is expressible as a linear combination of the others.

\section{Auxiliary Equations with Complex Roots}
The simple harmonic equation $y''+y=0$ so called because of its relation to the fundamental vibration of a musical tone, has as solutions $y_1(t)=\cos t$ and $y_2(t)=\sin t$.

When $b^2-4ac<0$, the roots of the auxiliary equation $ar^2+br+c=0$ associated with the homogeneous equation $ay''+by'+cy=0$ are the complex conjugate numbers $r_1=\alpha + i\beta$ and $r_2=\alpha-i\beta$ where $\alpha=-\frac{b}{2a}$ and $\beta = \frac{\sqrt{4ac-b^2}}{2a}$

Combing the solutions $e^{r_1t}$ and $e^{r_2t}$ with Euler's formula $e^{i\theta}=\cos\theta + i\sin\theta$, yields complex function solutions 
\[ e^{(\alpha+i\beta)t}=e^{at}(\cos\beta t+i\sin\beta t) \text{ and } e^{(\alpha-i\beta)t}=e^{at}(\cos\beta t-i\sin\beta t) \]

\begin{example}
    Solve the initial value problem $y''+2y'+2y=0$ given $y(0)=0$ and $y'(0)=2$.

    Using the auxiliary form of the equation we have $r^2+2r+2=0$. From the quadratic formula, $r=-1\pm i$, the two roots are $r_1=-1+i$ and $r_2=-1-i$.

    The solution is therefore $y_1=e^{(-1+i)t}$ and $y_2=e^{(-1-i)t}$

    From the form given from euler's formula earlier, $y_1=e^{-t}(\cos t+i\sin t)$ and $y_2=e^{-t}(\cos t-i\sin t)$, so our general solution is $y=c_1e^{-t}(\cos t+i\sin t)+c_2e^{-t}(\cos t-i\sin t)$.

    Plugging the initial conditions, we get that $0=c_1+c_2$.

    The derivative of the general solution is $y'=c_1e^{-t}(-\sin t+i\cos t)+(\cos t+i\sin t)\cdot c_1(-1)e^{-t}+c_2e^{-t}(-\sin t-i\cos t)+(\cos t-i\sin t)\cdot c_2(-1)e^{-t}$. Plugging in the initial conditions gives $2=c_1i-c_1-c_2i-c_2$.

    Factoring we get $2=c_1(i-1)+c_2(-i-1)$. Using some substitution $c_2=i$ and $c_1=-i$. Plug this in the general solution to solve.
\end{example}

If the auxiliary equation has complex conjugate roots $a\pm i\beta$, then two linearly independent solutions to the equation are 
\[ e^{\alpha t}\cos \beta t \text{ and } e^{\alpha t}\sin \beta t\]
and a general solution is 
\[ y(t)=c_1e^{\alpha t}\cos \beta t + c_2e^{\alpha t}\sin \beta t \]
where $c_1$ and $c_2$ are arbitrary constants.

\begin{example}
    Find a general solution to $y''+2y'+4y=0$.

    Using the auxiliary equation the roots are $-1\pm \sqrt{3}i$. Using what was given above, the general solution is $y=c_1e^{-t}\cos(\sqrt{3}t)+c_2e^{-t}\sin(\sqrt{3}t)$.
\end{example}

\begin{example}
    Newton's second law implies the position $y(t)$ of the mass $m$ is governed by the second-order differential equation $my''(t)+by'(t)+ky(t)=0$ where the terms are physically identified as $my$ being interial, $by$ is damping and $ky$ is stiffness.
    Determine the equation of motion for a spring system when $m=36$ kg, $b$ = 12 kg/sec (which is equivalent to 12 N - sec/m), $k=37$ kg/sec$^2$, $y(0)=0.7$ m and $y'(0) = 0.1$ m/sec. Also find $y(10)$, the displacement after 10 sec.

    The differential equation is $36y''+12y'+37$, so the roots are $-\frac{1}{6}\pm i$.

    Doing the methods explained above, the solution is $y=.7e^{-t/6}\cos t + \frac{13}{60}e^{-t/6}\sin t$, and $y(10)\approx -.13$ m.
\end{example}
\ex Interpret the equation $y''+5y'-6y=0$ in terms of the mass-spring system.

\section{Nonhomogeneous Equations: the Method of Undetermined Coefficients}
The method of Undetermined Coefficients is the technique used to guess a solution's form based on the form of the nonhomogeneous function $f(t)$ in a linear equation with 
constant coefficients such as $ay''+by'+cy=f(t)$.

For example the particular solution to $ay''+by'+cy=Ct^m$ is of the form $y_p(t)=A_mt^m+\dots+A_1t+A_0$.

\begin{example}
    Find a particular solution to $y''+3y'+2y=10e^{3t}$.

    Our guess based on the form of $f(t)=10e^{3t}$ is that $y=Ae^{3t}$ is the guess form of the particular solution, so we know that $y'=3A3^{3t}$ and $y''=9Ae^{3t}$.

    Substituting this in gives us $9Ae^{3t}+3(3Ae^{3t})+2(Ae^{3t})=10e^{3t}$. Simplifying this gives us $20Ae^{3t}=10e^{3t}$ so $A=1/2$.

    So our particular solution is $\frac{1}{2}e^{3t}$.
\end{example}

\begin{example}
    Find a particular solution to $y''+3y'+2y=\sin t$.

    Let $y=A\sin t + B\cos t$ as the form of the particualr solution.

    Substituting and solving should result in $A=1/10$ and $B=-3/10$. 
\end{example}

This example suggests an equation of the form $ay''+by'+cy=C\sin\beta t$ (or $C\cos\beta t$) will have a particular solution of the form $y_p(t)=A\cos\beta t+B\sin\beta t$.

\begin{example}
    Find a particular solution to $y''+4y=5t^2e^t$.

    Let $y= At^2e^t+Bte^t+Ce^t = e^t(At^2+Bt+C)$. The result should be $A=1,B=-4/5, C=-2/25$.
\end{example}

To find a particular soultion to the differential equation 
\[ ay''+by'+cy=Ct^me^{rt} \]
where $m$ is a nonnegative integer, use the form 
\[ y_p(t)=t^s(A_mt^m+\dots +A_1t+A_0)e^{rt} \]
with 
\begin{enumerate}
    \item $s=0$ if $r$ is not a root of the associated auxiliary equation;
    \item $s=1$ if $r$ is a simple root of the associated auxiliary equation;
    \item $s=2$ if $r$ is a double root of the associated auxiliary equation.
\end{enumerate}

To find a solution to the differential equation $ay''+by'+cy = Ct^me^{\alpha t}\cos \beta t$ or equal to $Ct^me^{\alpha t}\sin\beta t$ for $\beta \neq 0$, use the form $y_p(t)=t^s(A_mt^m+\dots A_1t+A_0)e^{\alpha t}\cos\beta t+t^s(B_mt^m+\dots +B_1t+B_0)e^{\alpha t}\sin\beta t$, with 
\begin{enumerate}
    \item $s=0$ if $\alpha + i\beta$ is not a root of the associated auxiliary equation; and 
    \item $s=1$ if $\alpha + i\beta$ is a root of the associated auxiliary equation
\end{enumerate}

\section{The Superposition Principle and Undetermined Coefficients Revisited}
\begin{theorem}[Superposition Principle]
    Let $y_1$ be a solution to the differential equation 
    \[ ay''+by'+cy=f_1(t) \]
    and $y_2$ is a solution to 
    \[ ay''+by'+cy=f_2(t) \]
    then for any constants $k_1$ and $k_2$, the function $k_1y_1+k_2y_2$ is a solution to the differential equation 
    \[ ay''+by'+cy = k_1f_1(t)+k_2f_2(t) \]
\end{theorem}

\begin{example}
    Find a particular solution to 
    \[ y''+3y'+2y = 3t+10e^{3t}\]
    The solution for equal to $3t$ is $y=\frac{3t}{2}-\frac{9}{4}$ and for $10e^{3t}$ is $y=\frac{e^{3t}}{2}$ so the solution is $y=\frac{3t}{2}-\frac{9}{4}+\frac{e^{3t}}{2}$.
\end{example}

\ex Find a particualr solution to $y''+3y'+2y=-9t+20e^{3t}$.

General solution for Nonhomogeneous Differential Equations: Let $y_p$ be a particular solution to 
\[ ay''+by'+cy = f(t) \]
and $c_1y_1+c_2y_2$ be the general solution to the homogeneous equation 
\[ ay''+by'+cy =0 \]
Then the general solution to the nonhomogeneous equation is given by 
\[ y(t)=y_p(t)+c_1y_1(t)+c_2y_2(t) \]

\begin{theorem}
    For any real numbers $a(\neq 0), b, c, t_0, Y_0$, and $Y_1$, suppose $y_p(t)$ is a particular solution to above in an interval $I$ containing $t_0$ and 
    that $y_1(t)$ and $y_2(t)$ are linearly independent solutions to the associated homogeneous equation in $I$. Then there exists a unique solution in $I$ to the initial value problem.
    \[ ay''+by'+cy = f(t) \qquad y(t_0)=Y_0 \qquad y'(t_0)= Y_1 \]
\end{theorem}

\begin{example}
    Given that $y_p(t)=t^2$ is a particular solution to 
    \[ y''-y=2-t^2 \]
    Find a general solution and a solution satisfying $y(0)=1, y'(0)=0$.

    Our general solution using the auxiliary equation is $y=c_1e^t+c_2e^{-t}$.

    Our particular solution will be in $At^2+By+C$.

    So $y=t^2+c_1e^t+c_2e^{-t}$. Solving for $c_1$ and $c_2$ by finding the derivative of this and using the initial conditions, the specific solution is $y=t^2+\frac{1}{2}e^t+\frac{1}{2}e^{-t}$.
\end{example}

\begin{example}
    A mass-spring system is driven by a sinusodial external force $5\sin t + 5\cos t$. The mass equals 1, the spring constant equals 2, and the damping coefficient equals 2 (in appropriate units), so the motion is governed by the differential equation 
    \[ y''+2y'+y=5\sin t+5\cos t \]

    If the mass is initially located at $y(0)=1$, with a velocity $y'(0)=2$, find its equation of motion.

    Finding the general solution to this we get that $y=c_1e^{-t}\cos t + c_2e^{-t}\sin t$. 
    
    From $y_p=A\sin t + B\cos t$, we should solve that 
    \[y=3\sin t - \cos t + 2e^{-t}\cos t + e^{-t}\sin t\]
\end{example}

\begin{example}
    Find a particular solution to 
    \[ y''-y=8te^t+2e^t \]

    The general solution for this is $y=c_1e^t+c_2e^{-t}$. $y_p$ is equal to $(At+B)e^t$. Doing some calculations should result in $y_p=(2t^2-t)e^t$. 
\end{example}

Method of Undetermined Coefficients (Revisited)

To find a particular solution to the differential equation 
\[ ay''+by'+cy=P_m(t)e^{rt} \]
where $P_m(t)$ is a polynomial of degree $m$, use the form 
\[ y_p(t)=t^s(A_mt^m+\dots +A_1t+A_0)e^{rt}\]
if $r$ is not a root of the associated auxiliary equation, take $s=0$; ir $r$ is a simple root of the associated 
auxiliary equation, take $s=1$; and if $r$ is a double root of the associated auxiliary equation, take $s=2$.

To find a particular solution to the differential equation 
\[ ay''+by'+cy=P_m(t)e^{\alpha t}\cos \beta t + Q_n(t)e^{\alpha t}\sin \beta t, \beta \neq 0 \]
where $P_m(t)$ is a polynomial of degree $m$ and $Q_n(t)$ is a polynomial of degree $n$, use the form 
$y_p(t)=t^s(A_kt^k+\dots+A_1t+A_0)e^{\alpha t}\cos \beta t + t^s(B_kt^k+\dots +B_1t+B_0)e^{\alpha t}\sin \beta t$, where $k$ is the larger of $m$ and $n$.
If $\alpha + i\beta$ is not a root of the associated auxiliary equation, take $s=0$; if $\alpha + i\beta$ is a root of the associated auxiliary equation, take $s=1$.

\ex Write down the form of a particular solution to the equation $y''+2y'+2y=5e^{-t}\sin t+5t^3e^{-t}\cos t$.

\ex Write down the form of a particular solution to the equation $y'''+2y''+y'=5e^{-t}\sin t+3+7te^{-t}$.

\section{Variation of Parameters}
The Method of Undetermined Coefficients is a procedure for determining a particular solution when the equation has constant coefficients and the nonhomogeneous term is of a special type.

Variation of Paramters is a more general method for finding a particular solution.

Consider a linear second-order equation 
\[ a_2(x)y'' + a_1(x)y'+a_0(x)y=g(x) \]
in the standard form 
\[ y''+P(x)y'+Q(x)y=f(x) \]
Obtain the solution to the associated homogeneous equation 
\[ y=c_1y_1(x)+c_2y_2(x)\]
And replace the constants with functions 
\[ y=u_1y_1(x)+u_2y_2(x)\]

Substituting into the DE yields the system:
\begin{align*}
y_1u'_1+y_2u'_2=0 \\
y'_1u'_1+y'_2u'_2 = f(x)
\end{align*}

By Cramer's Rule, the solution can be expressed in terms of determinants- 
\[ u'_1=\frac{W_1}{W}=\frac{-y_2f(x)}{W} \qquad u'_2=\frac{W_2}{W}=\frac{y_1f(x)}{W} \]

The functions $u_1$ and $u_2$ are found by integrating.

A particular solution is $y_p=u_1y_1+u_2y_2$.

\begin{example}
    Find a general solution on $(-\frac{\pi}{2},\frac{\pi}{2})$ to $\frac{\dd^2 y}{\dd t^2}+y=\tan t$.

    The standard form is $y''+y=\tan t$.

    The solutions to the homogeneous equation is $r=\pm i$, so $\alpha = 0$ and $\beta = 1$, so $y_1=\cos t$ and $y_2=\sin t$.

    Using what was previously introduced, we end up with $y=c_1\cos t + c_2\sin t - \cos t \ln|\sec t+\tan t|$
\end{example}

\ex Find a general solution on $(-\frac{\pi}{2},\frac{\pi}{2})$ to $\frac{\dd^2 y}{\dd t^2}+y=\tan t + 3t - 1$.

\section{Variable-Coefficient Equations}
We now consider equations with variable coefficients of the form 
\[ a_2(t)y''+a_1(t)y'+a_0(t)y=f(t) \]
Typically, the equation is divided by the nonzero coefficient $a_2(t)$ and is expressed in standard form 
\[ y''(t)+p(t)y'+q(t)y(t)=g(t) \]

\begin{theorem}
    Suppose $p(t), q(t)$, and $g(t)$ are continuous on an interval $(a,b)$ that contains the point $t_0$. The, for any choice of the initial values $Y_0$ and $Y_1$, there exists a 
    unique solution $y(t)$ on the same interval $(a,b)$ to the initial value problem 
    \[ y''(t)+p(t)y'(t)+q(t)y(t)=g(t), \qquad y(t_0)=Y_0, \qquad y'(t_0)=Y_1 \]
\end{theorem}

\begin{example}
    Determine the largest interval for which the above theorem ensures the existence and uniqueness of a solution to the initial value problem 
    \[(t-3)y'' + y'+\sqrt{t}y = \ln t \qquad y(1)=3, \qquad y'(1)=-5 \]

    First, let's put this into standard form. $y''+\frac{1}{t-3}y' + \frac{\sqrt{t}}{t-3}y=\frac{\ln t}{t-3}$.

    From this we can see that this is only continuous from $(0,3)$.
\end{example}

\begin{definition}
    A linear second-order equation that can be expressed in the form 
    \[ at^2y''(t)+bty'(t)+cy=f(t) \]
    where $a,b$, and $c$ are constants, is called a Cauchy-Euler, or equidimensional, equation.
\end{definition}

To solve a Cauchy-Euler equation, for $t>0$ look for solutions of the form 
\[ y=t^r \]
and substitute into the homogeneous form 
\[ at^2y''(t)+bty'(t)+cy =0 \]
the resulting equation $ar^2+(b-a)r+c=0$ is called the associated characteristic equation.

\begin{example}
    Find two linearly independent solutions to the equation 
    \[ 3t^2y''+11ty'-3y=0 \qquad t>0 \]

    The solutions are $y=t^r$ and plugging this into the equation results in $3r^2+8r-3$, which factors to $r=\frac{1}{3}$ and $r=-3$.

    The solutions are $y=t^{1/3}$ and $y=t^{-3}$.
\end{example}

If the roots of the associated characteristic equation $r$ are equal, then independent solutions of the Cauchy-Euler equation on $(0, \infty)$ are given by 
\[ y_1 = t^r \qquad \text{and} \qquad y^2 = t^r\ln t\]
IF the roots are complex, $r=\alpha\pm\beta i$, then the independent solutions are given by 
\[ y_1=t^a\cos(\beta \ln t) \qquad \text{and} \qquad y_2=t^{\alpha}\sin(\beta \ln t) \]

\begin{example}
    Find a pair of linearly independent solutions to the Cauchy-Euler equations for $t>0$.

    $t^2y''+5ty'+5y=0$

    Answer: $y_1=t^{-2}\cos (\ln t)$, $y_2=t^{-2}\sin(\ln t)$
\end{example}

\ex Do the same thing for $t^2y''+ty'=0$.

\begin{lemma}
    If $y_1(t)$ and $y_2(t)$ are any two solutions to the homogeneous differential equation 
    \[ y''(t)+p(t)y'(t)+q(t)y(t) =0 \]
    on an interval $I$ where the functions $p(t)$ and $q(t)$ are continuous and if the Wronksian is zero at any point $t$ of $I$, then $y_1$ and $y_2$ are linearly dependent on $I$.
\end{lemma}

\begin{theorem}
    If $y_1(t)$ and $y_2(t)$ are any two solutions to the homogeneous differential equation that are linearly independent on an interval $I$ containing $t_0$, then unique constants $c_1$ and $c_2$ can always be found so that $c_1y_1(t)+c_2y_2(t)$ satisfies the initial
    conditions $y(t_0)=Y_0$, $y'(t_0)=Y_1$ for any $Y_0$ and $Y_1$.
\end{theorem}

$y_h=c_1y_1+c_2y_2$ is called a general solution to the homogeneous differential equation $y''(t)+p(t)y'(t)+q(t)y(t)=0$ if $y_1$ and $y_2$ are linearly independent solutions on $I$.

For the nonhomogeneous equation $y''(t)+p(t)y'(t)+q(t)y(t)=g(t)$ a general solution on $I$ is given by $y=y_p+y_h$ where $y_h=c_1y_1+c_2y_2$ is a general solution to the corresponding homogeneous equation on $I$ and $y_p$ is a particular solution on $I$.

If linearly independent solutions to the homogeneous equation are known, $y_p$ can be determined by the variation of parameters method.

\begin{theorem}
    If $y_1$ and $y_2$ are two linearly independent solutions to the homogeneous equation on an interval $I$ where $p(t)$, $q(t)$, and $g(t)$ are continuous, then a particular solution is given by $y_p=u_1y_1+u_2y_2$, where $u_1$ and $u_2$ are determined up 
    to a constant by the pair of equations, $y_1v'_1+y_2v'_2=0$, $y'_1v'_1+y'_2v'_2=g$, which have the solutions 
    \[ v_1(t)=\int \frac{-g(t)y_2(t)}{W(y_1,y_2)(t)}\dd t \]
    and 
    \[ v_2(t)=\int \frac{g(t)y_1(t)}{W(y_1,y_2)(t)}\dd t \]
    Note that the formulation above presumes that the differential equation has been put into standard form (that is divided by $a_2(t)$).
\end{theorem}

\begin{theorem}
    Let $y_1(t)$ be a solution, not identically zero, to the homogeneous equation in an interval $I$. Then 
    \[ y_2(t)=y_1(t)\int \frac{e^{-\int p(t) \dd t}}{y_1(t)^2} \dd t \]
    is a second, linearly independent solution.
\end{theorem}

\begin{example}
    Given that $y_1(t)=t$ is a solution to 
    \[ y''-\frac{1}{t}y' + \frac{1}{t^2}y = 0 \]
    use the Reduction of Order formula to determine a second linearly independent solution for $t>0$.

    $y_2$ is equal to $t\int \frac{e^{\ln t}}{t^2} \dd t = t\ln t$ from the formula, so the general solution is $y=c_1t+c_2t\ln t$.
\end{example}

\ex The following equation arises in the mathematical modeling of reverse osmosis. \[ (\sin t)y'' - 2(\cos t)y' - (\sin t)y =0, \qquad 0<t<\pi \] Find a general solution.

\end{document}