\documentclass[../diffeq.tex]{subfiles}
\graphicspath{{\subfix{../figures/}}}
\begin{document}
\chapter{Introduction to Differential Equations}
\section{Background}
In a variety of subject areas, mathematical models are developed to aid in understanding. These models often yield an equation that contains derivatives of an unknown function. Such an equation is called a differential equation.

One example is free fall of a body. An object is released from a certain height above the ground and falls under the force of gravity. Newton's second law states that an object's mass times its acceleration equals the total force acting on it.
\[ m\frac{\dd^2 h}{dt^2}=-mg \]

We have $h(t)$ as position, $\frac{\dd h}{\dd t}$ as velocity and $\frac{\dd^2 h}{\dd t}$ as acceleration. The independent variable is $t$ and the dependent variable is $h$.

$m\frac{\dd^2 h}{dt^2}=-mg$ is a differential equation and $h(t)$ is the unknown function that we are trying to find.

From this we have $\frac{\dd^2 h}{\dd t^2} = -g$ and the integral of this is $\frac{\dd h}{\dd t} = -gt + C_1$. To find $h$ we integrate again and we get $h(t) = -\frac{gt^2}{2}+C_1t+C_2$.

Another example is the decay of a radioactive substance. The rate of decay is proportional to the amount of radioactive substance present.
\[ \frac{\dd A}{\dd t} = -kA, \qquad k>0 \]
where $A$ is the unknown amount of radioactive substance present at time $t$ and $k$ Is the proportionality constant.

We are looking for $A(t)$ that satisfies this equation. We can solve this from $\frac{1}{A}\dd A = k\dd t$ and integrating both sides we get that 
$\ln |A| + C_1 = -kt+C_2$. We can rewrite this as $\ln|A| = -kt + C$. So, $e^{-kt+C}=A$.

So $A(t)=e^{-kt}+e^C$, so $A(t)=Ce^{-kt}$. Remember $A$ is the dependent variable and $t$ is the independent variable.

Notice that the solution of a differential equation is a function, not merely a number.

When a mathematical model involves the rate of change of one variable with respect to another, a differential equation is apt to appear.

\subsubsection*{Terminology}
If an equation involves the derivative of one variable with respect to another, then the former is called a dependent variable and the latter an independent variable.

In $\frac{\dd h}{\dd t}$, $h$ is dependent and $t$ is independent.

A differential equation involving only ordinary derivatives with respect to a single independent variable is called an ordinary differential equation. A differential equation involving partial derivatives 
with respect to more than one independent variable is a partial differential equation.

For example we have $z=f(x,y)=4x^2+5xy$, so $\frac{\partial z}{\partial x} = 8x+5y$ and that is partial differentiation.

The order of a differential equation is the order of the highest-order derivatives present in the equation.

For example, $\frac{\dd^2 h}{dt^2}=-g$ has a order of 2.

A linear differential equation is one in which the dependent variable $y$ and its derivatives appear in additive combinations of their first powers. A differential equation is linear if it has the format.
\[ a_n(x)\frac{\dd^n y}{dx^n}+a_{n-1}(x)\frac{\dd^{n-1}y}{dx^{n-1}}+\dots + a_1(x)\frac{\dd y}{\dd x}+a_0(x)y=F(x)\]

$2x+3y=7$ is linear, $2x^2+5xy+7y+8y = 1$ is second-degree. Nothing can have a second degree for this to be linear.

You are just looking at the dependent variable and the derivatives and adding their powers.

If an ordinary differential equation is not linear, we call it nonlinear.

\begin{example}
    For each differential equation, classify as ODE or PDE, linear or nonlinear, and indicate the dependent/independent variables and order.

    (a) $\frac{\dd^2 x}{\dd t^2}+a\frac{\dd x}{\dd t}+kx =0$

    Dependent is $x$, independent is $t$ and the order is 2. This is an ODE and linear.

    (b) $\frac{\partial u}{\partial x} - \frac{\partial u}{\partial y}=x-2y$ 

    The dependent variable is $u$ and the independent variables are $x,y$, so this is a PDE. The order is 1.

    (c) $\frac{\dd^2 y}{\dd x^2}+y^3=0$

    The dependent variable is $y$, the independent variable is $x$, the order is 2 and this is an ODE and this is nonlinear.

    (d) $t^3 \frac{\dd x}{\dd t} = t^3+x$

    Dependent is $x$, independent is $t$, order is 1, this is an ODE. We can rewrite this as $t^3\frac{\dd x}{\dd t}-1x=t^3$, and this matches the form of the linear equation so this is linear.

    (e) $\frac{\dd^2 y}{\dd x^2} - y\frac{\dd y}{\dd x}=\cos x$

    The dependent is $y$, the independent is $x$, the order is 2 and this is an ODE and this is nonlinear because of $y\frac{\dd y}{\dd x}$.
\end{example}

\section{Solutions and Initial Value Problems}
An $n$th-order ordinary differential equation is an equality relating the independent variable to the $n$th derivative (and usually lower-order derivatives as well) of the dependent variable.

\begin{example}
    Identify the order, independent and dependent variable.

    (a) $x^2\frac{\dd^2 y}{\dd x^2} + x\frac{\dd y}{\dd x} + y =x^3$. Independent: $x$, dependent: $y$, order: 2

    (b) $\sqrt{1-\left(\frac{\dd^2 y}{\dd t^2}\right)}-y=0$. Independent: $t$, dependent: $y$, order: 2

    (c) $\frac{\dd^4 x}{\dd t^4}=xt$. Independent: $t$, dependent: $x$, order: 4. (This is also linear.)
\end{example}

A general form for an $n$th-order equation with $x$ independent, $y$ dependent can be expressed as 
\[ F(x,y,\frac{\dd y}{\dd x},\dots,\frac{\dd^n y}{\dd x^n})=0 \]
where $F$ is a function that depends on $x$, $y$, and the derivatives of $y$ up to order $n$. We assume the equations holds for all $x$ in an open interval $I$. In many cases, we can isolate the highest-order term and write the previous equation as 
\[ \frac{\dd^n y}{\dd x^n} = f\left(x,y,\frac{\dd y}{\dd x},\dots,\frac{\dd^{n-1}y}{\dd x^{n-1}}\right)\]
This is called the normal form.

A function $\phi(x)$ that when substituted for $y$ in either the previous two equations satisfies the equation for all $x$ in the interval $I$ is called an explicit solution to the equation on $I$.

\begin{example}
    Show that $\phi(x)=x^2-x^{-1}$ is an explicit solution to the linear equation $\frac{\dd^2 y}{\dd x^2}-frac{2}{x^2}y=0$ but $\psi(x)=x^3$ is not.

    So we have $y=x^2-x^{-1}$. The first derivative of this is $2x+1x^{-2}$. The second derivative is $y'' = 2-2x^{-3}$. If we plug in the values we end up getting from the derivatives, we get that 
    $2-2x^{-3}-2x+2x^{-3} = 0$, so this is satisfied.

    For the second part, the first derivative is $3x^2$ and the second derivative is $6x$. Plugging this in, we get $4x$ which is not $0$, so $\psi(x)$ is not a solution.
\end{example}

\begin{example}
    Show that for any choice of the constants $c_1$ and $c_2$, the function $\phi(x)=c_1e^{-x}+c_2e^{2x}$ is an explicit solution to the linear equation $y''-y'-2y=0$.

    We have that the first derivative is $-c_1e^{-x}+2c_2e^{2x}$ and the second derivative is $c_1e^{-x} + 4c_2e^{2x}$. When we plug this in, we find that this does satisfy the solution for the differential equation.
\end{example}

Methods for solving differential equations do not always yield an explicit solution for the equation. A solution may be defined implicitly.

\begin{example}
    Show that the relation $y^2-x^3+8=0$ implicitly defines a solution to the nonlinear equation $\frac{\dd y}{\dd x}=\frac{3x^2}{2y}$ on the interval $(2,\infty)$.

    We have from the given that $y=\pm\sqrt{x^3-8}$. The derivative (of the positive version) of this is $\frac{3x^2}{2\sqrt{x^3-8}}$. This is the same and defined on the interval.
\end{example}

A relation $G(x,y) =0$ is said to be an implicit solution to the previous equation on the interval $I$ if it defines one or more explicit solutions on $I$.

\begin{example}
    Show that $x+y+e^{xy}=0$ is an implicit solution to the nonlinear equation $(1+xe^{xy})\frac{\dd y}{\dd x}+1+ye^{xy}=0$. 

    Taking the derivative of both sides gets us that $1+\frac{\dd y}{\dd x}+e^{xy}\frac{\dd}{\dd x}(xy) =0$. This does simplify to what was given in the problem.
\end{example}

\begin{example}
    Verify that for every constant $C$ the relation $4x^2-y^2=C$ is an implicit solution to $y\frac{\dd y}{\dd x}-4x=0$. Graph the solution curves for $C=0,\pm 1,\pm 4$.

    The derivative of what is given is $8x-2y\frac{\dd y}{\dd x}=0$. This simplifies to what is given, so it is clearly an implicit solution. 

    For $C=0$, the solution curves for this is $2x=y$ and $-2x=y$.

    For $C=\pm 4$, the solution curves is given by a hyperbola $\frac{x^2}-\frac{y^2}{4}=1$. 
\end{example}

The collection of all solutions in the previous example is called a one-parameter family of solutions.

In general, the methods for solving $n$th-order differential equations evoke $n$ arbitrary constants. We often can evalute these constants 
if we are given $n$ initial values $y(x_0), y'(x_0),\dots, y^{(n-1)}(x_0)$.

\begin{definition}
    By an initial value problem for an $n$th-order differential equation 
    \[ F(x,y,\frac{\dd y}{\dd x},\dots,\frac{\dd^n y}{\dd x^2}) =0 \]

    we mean: Find a solution so the differential equation on an interval $I$ that satisfies at $x_0$ the $n$ initial conditions
    \[ y(x_0)=y_0, \quad \frac{\dd y}{\dd x}(x_0)=y_1 \cdots \frac{\dd^{n-1}y}{\dd x^{n-1}}(x_0)=y_{n-1}\]
    where $x_0\in I$ and $y_0, y_1, \dots, y_{n-1}$ are constants.
\end{definition}

\begin{example}
    Show that $\phi(x)=\sin x-\cos x$ is a solution to the initial value problem 
    \[ \frac{\dd^2 y}{\dd x^2}+y = 0; \qquad y(0)=-1 \qquad \frac{\dd y}{\dd x}(0)=1 \]

    We have $y=\sin x-\cos x$, $y'=\cos x+\sin x$, and $y'' = -\sin x+\cos x$. These satisfy the conditions.
\end{example}

\begin{theorem}[Existence and Uniqueness of Solution]
    Consider the initial value problem 
    \[ \frac{\dd y}{\dd x}=f(x,y), \qquad y(x_0)=y_0 \]

    If $f$ and $\partial f/\partial y$ are continuous functions in some rectangle 
    \[ R = \{(x,y): a<x<b, c<y<d\} \]
    that contains the point $(x_0,y_0)$, then the initial value problem has a unique solution $\phi(x)$ in some interval $x_0-\delta<x<x_0\delta$, where $\delta$ is a positive number.
\end{theorem}

\begin{example}
    Does the theorem above imply the existence for this problem.

    $3\frac{\dd y}{\dd x}=x^2-xy^3, \qquad y(1)=6$

    The derivative exists for all $(x,y)$ and is continuous in all intervals containing $x=1$ and all rectangular regions containing $(1,6)$.

    When we consider the partial derivatives, $\partial f/\partial y = -xy^2$, and this exists and is continuous for all rectangular regions in the $xy$ plane.
\end{example}

\section{Direction Fields}
\section{The Approximation Method of Euler}

\end{document}