\documentclass[../diffeq.tex]{subfiles}
\graphicspath{{\subfix{../figures/}}}
\begin{document}
\chapter{Introduction to Differential Equations}
\section{Background}
In a variety of subject areas, mathematical models are developed to aid in understanding. These models often yield an equation that contains derivatives of an unknown function. Such an equation is called a differential equation.

One example is free fall of a body. An object is released from a certain height above the ground and falls under the force of gravity. Newton's second law states that an object's mass times its acceleration equals the total force acting on it.
\[ m\frac{\dd^2 h}{dt^2}=-mg \]

We have $h(t)$ as position, $\frac{\dd h}{\dd t}$ as velocity and $\frac{\dd^2 h}{\dd t}$ as acceleration. The independent variable is $t$ and the dependent variable is $h$.

$m\frac{\dd^2 h}{dt^2}=-mg$ is a differential equation and $h(t)$ is the unknown function that we are trying to find.

From this we have $\frac{\dd^2 h}{\dd t^2} = -g$ and the integral of this is $\frac{\dd h}{\dd t} = -gt + C_1$. To find $h$ we integrate again and we get $h(t) = -\frac{gt^2}{2}+C_1t+C_2$.

Another example is the decay of a radioactive substance. The rate of decay is proportional to the amount of radioactive substance present.
\[ \frac{\dd A}{\dd t} = -kA, \qquad k>0 \]
where $A$ is the unknown amount of radioactive substance present at time $t$ and $k$ Is the proportionality constant.

We are looking for $A(t)$ that satisfies this equation. We can solve this from $\frac{1}{A}\dd A = k\dd t$ and integrating both sides we get that 
$\ln |A| + C_1 = -kt+C_2$. We can rewrite this as $\ln|A| = -kt + C$. So, $e^{-kt+C}=A$.

So $A(t)=e^{-kt}+e^C$, so $A(t)=Ce^{-kt}$. Remember $A$ is the dependent variable and $t$ is the independent variable.

Notice that the solution of a differential equation is a function, not merely a number.

When a mathematical model involves the rate of change of one variable with respect to another, a differential equation is apt to appear.

\subsubsection*{Terminology}
If an equation involves the derivative of one variable with respect to another, then the former is called a dependent variable and the latter an independent variable.

In $\frac{\dd h}{\dd t}$, $h$ is dependent and $t$ is independent.

A differential equation involving only ordinary derivatives with respect to a single independent variable is called an ordinary differential equation. A differential equation involving partial derivatives 
with respect to more than one independent variable is a partial differential equation.

For example we have $z=f(x,y)=4x^2+5xy$, so $\frac{\partial z}{\partial x} = 8x+5y$ and that is partial differentiation.

The order of a differential equation is the order of the highest-order derivatives present in the equation.

For example, $\frac{\dd^2 h}{dt^2}=-g$ has a order of 2.

A linear differential equation is one in which the dependent variable $y$ and its derivatives appear in additive combinations of their first powers. A differential equation is linear if it has the format.
\[ a_n(x)\frac{\dd^n y}{dx^n}+a_{n-1}(x)\frac{\dd^{n-1}y}{dx^{n-1}}+\dots + a_1(x)\frac{\dd y}{\dd x}+a_0(x)y=F(x)\]

$2x+3y=7$ is linear, $2x^2+5xy+7y+8y = 1$ is second-degree. Nothing can have a second degree for this to be linear.

You are just looking at the dependent variable and the derivatives and adding their powers.

If an ordinary differential equation is not linear, we call it nonlinear.

\begin{example}
    For each differential equation, classify as ODE or PDE, linear or nonlinear, and indicate the dependent/independent variables and order.

    (a) $\frac{\dd^2 x}{\dd t^2}+a\frac{\dd x}{\dd t}+kx =0$

    Dependent is $x$, independent is $t$ and the order is 2. This is an ODE and linear.

    (b) $\frac{\partial u}{\partial x} - \frac{\partial u}{\partial y}=x-2y$ 

    The dependent variable is $u$ and the independent variables are $x,y$, so this is a PDE. The order is 1.

    (c) $\frac{\dd^2 y}{\dd x^2}+y^3=0$

    The dependent variable is $y$, the independent variable is $x$, the order is 2 and this is an ODE and this is nonlinear.

    (d) $t^3 \frac{\dd x}{\dd t} = t^3+x$

    Dependent is $x$, independent is $t$, order is 1, this is an ODE. We can rewrite this as $t^3\frac{\dd x}{\dd t}-1x=t^3$, and this matches the form of the linear equation so this is linear.

    (e) $\frac{\dd^2 y}{\dd x^2} - y\frac{\dd y}{\dd x}=\cos x$

    The dependent is $y$, the independent is $x$, the order is 2 and this is an ODE and this is nonlinear because of $y\frac{\dd y}{\dd x}$.
\end{example}

\section{Solutions and Initial Value Problems}
\section{Direction Fields}
\section{The Approximation Method of Euler}

\end{document}