\documentclass[../diffeq.tex]{subfiles}
\graphicspath{{\subfix{../figures/}}}
\begin{document}
\chapter{Series Solutions of Differential Equations}
\section{Introduction: The Taylor Polynomial Approximation}
The best tool for numerically approximating a function $f(x)$ near a particular point $x_0$ is the Taylor polynomial.

The formula for the Taylor polynomial of degree $n$ centered at $x_0$, approximating a function $f(x)$ possessing $n$ derivatives $x_0$ is given by 
\[ p_n(x)=f(x_0)+f'(x_0)(x-x_0)+\frac{f''(x_0)}{2!}(x-x_0)^2+\frac{f'''(x_0)}{3!}(x-x_0)^3+\dots+\frac{f^{(n)}(x_0)}{n!}(x-x_0)^n = \sum_{j=0}^n \frac{f^{(j)}(x_0)}{j!}(x-x_0)^j \]

\begin{example}
    Find the first four Taylor polynomials for $e^x$, expanded around $x_0=0$.

    $p_n(x)$ is written as $f(0)+f'(0)(x-0)+\frac{f''(0)}{2!}(x-0)^2+\frac{f'''(0)}{3!}(x-0)^3$.

    Since we know the derivatives of $f(x)=e^x$ is just $e(x)$, $f^{(j)}(0)=1$ for all of them.
    
    This simplifies to $1+x+\frac{1}{2}x^2+\frac{1}{6}x^3$.
\end{example}

The Taylor polynomial $p_n$ is just the $(n+1)$st partial sum of the Taylor series 
\[ \sum_{j=0}^{\infty}\frac{f^{(j)}(x_0)}{j!}(x-x_0)^j \]

\begin{example}
    Determine the fourth-degree Taylor polynomials matching the function $\cos x$ at $x_0=2$

    So using what was previously given we have $f(2)+f'(2)(x-2)=\frac{f''(2)}{2!}(x-2)^2+\frac{f'''(2)}{3!}(x-2)^3+\frac{f^{(4)}}{4!}(x-2)^4$.

    Filling in the $f^{(j)}$ values gives us $p_4(x)=\cos 2-\sin 2(x-2)-\frac{\cos 2}{2}(x-2)^2+\frac{\sin 2}{6}(x-2)^3+\frac{\cos 2}{24}(x-2)^4$
\end{example}
\pagebreak
\begin{example}
    Find the first few Taylor polynomials approximating the solution around $x_0=0$ of the initiai value problem 
    \[ y''=3y'+x^{7/2}y \qquad y(0)=10 \quad y'(0)=5 \]

    In general, this is just $y(0)+y'(0)x+\frac{y''(0)}{2!}x^2+\dots + \frac{y^{(n)}(0)}{n!}x^n$.

    Since we are given the problem, we know that $y''(0)=3y'(0)+0 = 15$.

    As we continue taking derivatives with respect to $x$, we get $y'''=3y''+\frac{7}{3}x^{7/3}y+x^{7/3}y'$, and plugging in the numbrs gives us $y'''(0)=45$.

    Calculating the 4th derivative gives us $135$.

    The fifth derivative is no longer defined.
\end{example}

\begin{example}
    Determine the Taylor polynomial of degree $3$ for the solution to the initial value problem 
    \[ y' = \frac{1}{x+y+1} \qquad y(0)=0 \]

    Finding $y'(0)$ gives us $1$, and finding $y''(0)$ gives us $-2$, and $y'''(0)=10$.
\end{example}

We can estimate the accuracy to which a Taylor polynomial $p_n(x)$ approximates its target function $f(x)$ for $x$ near $x_0$. The error $\epsilon_n(x)$ measures the accuracy of the approximation, 
\[ \epsilon_n(x)= f(x)-p_n(x) \]
and can be estimated by $\epsilon_n(x)=\frac{f^{(n+1)}(\aleph)}{(n+1)!}(x-x_0)^{n+1}$, where $\aleph$ is guaranteed to lie between $x_0$ and $x$ if the $(n+1)$st derivative of $f$ exists and is continuous on an interval containing $x_0$ and $x$.


\section{Power Series and Analytic Functions}
A power series about the point $x_0$ is an expression of the form 
\[ \sum_{n=0}^{\infty}a_n(x-x_0)^n = a_0+a_1(x-x_0)+a_2(x-x_0)^2 + \dots \]
where $x$ is a variable and the $a_n$'s are constants.

A power series is convergent at a specified value of $x$ if its sequence of partial sums $\{S_N(x)\}$ converges, that is 
\[ \lim_{N\to \infty} S_N(x)=\lim_{N\to\infty}\sum^N_{n=0}a_n(x-x_0)^n \]
If the limit does not exist at $x$, then the series is said to be divergent.

Every power series has an interval of convergence. The interval of convergence is the set of all real numbers $x$ for which the series converges. The center of the interval of convergence is the center $x_0$ of the series. Within its 
interval of convergence a power series converges absolutely. In other words, if $x$ is in the interval of convergence and is not an endpoint of the interval, then the series of absolute values 
\[ \sum_{n=0}^{\infty} \left|a_n(x-x_0)^n\right| \]
converges.

\begin{theorem}
    For each power series, there is a number $\rho \quad (0\leq \rho < \infty)$, called the radius of convergence of the power series, such that the series converges absolutely for $|x-x_0|<\rho$ and diverges for $|x-x_0|>\rho$.
    If the series converges for all values of $x$, then $\rho=\infty$. WHen the series converges only at $x_0$, then $\rho=0$.
\end{theorem}

\begin{theorem}
    If, for $n$ large, the coefficients $a_n$ are nonzero and satisfy 
    \[ \lim_{n\to \infty} \left|\frac{a_n}{a_{n+1}}\right| = L \qquad (0\leq L\leq \infty) \]
    then the radius of convergence of the power series $\sum_{n=0}^{\infty}a_n(x-x_0)^n$ is $\rho = L$.
\end{theorem}

\begin{example}
    Determine the interval and radius of convergence of 
    \[ \sum_{n=0}^{\infty}\frac{(-2)^n}{n+1}(x-3)^n \]

    From the ratio test, the radius of convergence is $\rho=\frac{1}{2}$.

    The interval of convergence is $|x-3|<\frac{1}{2}$.

    So the interval is $-5/2<x<7/2$.

    For $7/2$, it converges, so $-5/2<x\leq 7/2$.
\end{example}

\begin{theorem}
    If $\sum_{n=0}^{\infty} a_n(x-x_0)^n=0$ for all $x$ in some open interval, then each coefficient $a_n$ equals zero.
\end{theorem}

\begin{theorem}
    If the series $f(x)=\sum_{n=0}^{\infty}a_n(x-x_0)^n$ has a positive radius of convergence $\rho$, then $f$ is differentiable in the interval $|x-x_0|<\rho$ and termwise 
    differentiation gives the power series for the derivative:
    \[ f'(x)=\sum_{n=1}^{\infty}na_n(x-x_0)^{n-1} \qquad \text{for} \qquad |x-x_0|<\rho \]
    Furthermore, termwise integration gives the power series for the integral of $f$:
    \[ \int f(x)\dd x = \sum^{\infty}_{n=0}\frac{a_n}{n+1} (x-x_0)^{n+1} + C \qquad \text{for} \qquad |x-x_0| < \rho \]
\end{theorem}
\pagebreak
\begin{example}
    Starting with the geometric series for $\frac{1}{1-x}=1+x+x^2+x^3+x^4+\dots = \sum_{n=0}^{\infty}x^n \qquad -1<x<1$ find a power series for each of the following functions.

    (a) $\frac{1}{1+x^2}$

    Replace $x$ with $-x^2$ and we get the power series equal to 
    
    $1-x^2+x^4-x^6+x^8+\dots = \sum_{n=0}^{\infty}(-1)^n x^{2n} \qquad -1<x<1$.

    (b) $\frac{1}{(1-x)^2}$

    This becomes $1+2x+3x^2+\dots = \sum_{n=1}^{\infty}nx^{n-1}$

    (c) $\arctan x$
    This becomes $x-\frac{x^3}{3}+\frac{x^5}{5}-\frac{x^7}{7}+\dots = \sum_{n=0}^{\infty}\frac{(-1)^n}{2n+1}x^{2n+1}$
\end{example}

\begin{example}
    Express the series $\sum_{n=2}^{\infty}n(n-1)a_n x^{n-2}$ as a series where the generic term is $x^k$ instead of $x^{n-2}$.

    Let $k=n-2$, so $n=k+2$.

    Plugging this in gives us $\sum_{k=0}^{\infty}(k+2)(k+1)a_{k+2}x^k$.
\end{example}

\begin{example}
    Show that $x^3\sum_{n=0}^{\infty} n^2(n-2)a_nx^n = \sum_{n=3}^{\infty}(n-3)^2(n-5)a_{n-3}x^n$.

    Let $k=n+3$, so $n=k-3$.

    Doing stuff gives you the answer of $\sum_{n=3}^{\infty}(n-3)^2(n-5)a_{n-3}x^n$.
\end{example}

\ex Show that the identity $\sum_{n=1}^{\infty}na_{n-1}x^{n-1}+\sum_{n=2}^{\infty}b_nx^{n+1}=0$ implies that $a_0=a_1=a_2=0$ and $a_n=-\frac{b_{n-1}}{(n+1)}$ for $n\geq 3$.

\begin{definition}
    A function $f$ is said to be analytic at $x_0$ if, in an open interval about $x_0$, this function is the sum of a power series $\sum_{n=0}^{\infty}a_n(x-x_0)^n$ that has a positive radius of convergence.
\end{definition}

A polynomial is analytic at every $x_0$. A rational function $P(x)/Q(x)$ where $P(x)$ and $Q(x)$ are polynomials without a common factor, is analytic except at those $x_0$ for which $Q(x_0)=0$.
The elementary functions $e^x,\sin x,\cos x$ are analytic for all $x$ while $\ln x$ is analytic for $x>0$. Familiar representations are 
\[ e^x = 1+x+\frac{x^2}{2!}+\frac{x^3}{3!}+\dots = \sum_{n=0}^{\infty} \frac{x^n}{n!} \]
\[ \sin x = x-\frac{x^3}{3!}+\frac{x^5}{5!}+\dots = \sum_{n=0}^{\infty} \frac{(-1)^n}{(2n+1)!}x^{2n+1} \]
\[ \cos x = 1-\frac{x^2}{2!}+\frac{x^4}{4!}+\dots = \sum_{n=0}^{\infty} \frac{(-1)^n}{(2n)!}x^{2n} \]
\[ \ln x = (x-1)-\frac{1}{2}(x-1)^2+\frac{1}{3}(x-1)^3 - \dots = \sum_{n=1}^{\infty}\frac{(-1)^{n-1}}{n}(x-1)^n \]
where the first three are valid for all $x$, whereas the last is valid for $x$ in $(0,2]$.



\section{Power Series Solutions to Linear Differential Equations}
\begin{definition}
    A point $x_0$ is called an ordinary point if both $p=a_1/a_2$ and $q=a_0/a_2$ are analytic at $x_0$. If $x_0$ is not an ordinary point, it is called a singular point of the equation.
\end{definition}

\begin{example}
    Determine all the singular points of 
    \[ xy''+x(1-x)^{-1}y'+(\sin x)y=0 \]

    The form of this is $y''+\frac{1}{1-x}y'+\frac{\sin x}{x}y=0$.

    $p(x)=\frac{1}{1-x}$ can be represented as a power series as well as $q(x)=\frac{\sin x}{x}$.

    The only singular point is at $x=1$.
\end{example}

\begin{example}
    Find a power series solution about $x=0$ to 
    \[ y'+2xy=0 \]

    We are substituting around $y=\sum_{n=0}^{\infty}a_nx^n$. The derivative is $y'=\sum_{n=1}^{\infty} n\cdot a_nx^{n-1}$.

    Substituting this in gives $\sum_{n=1}^{\infty} na_nx^{n-1}+2x\sum_{n=0}^{\sum} a_nx^n =0$.

    When we are trying to get $x^1$ in the summations, we get $a_1 + \sum_{n=2}na+nx^{n-1}+\sum_{n=0}2a_nx^{n+1}=0$.

    Simplifying this gives us $a_1+\sum_{k=1}[(k+1)a_{k+1}+2a_{k-1}]x^k=0$.

    We have $a_{k+1}=\frac{-2a_{k-1}}{k+1}$.

    From the expanded form of $y$ we have $a_0x_0+a_1x+a_2x^2+a_3x^3+a_4x^4+\dots$. 

    We already know $a_1=0$.

    We can keep finding the formulas, $a_2=\frac{-2}{2}a_0$, $a_4=\frac{-2}{4}\cdot \frac{-2}{2}a_0$ and $a_6=\frac{-2}{6}\cdot \frac{-2}{4}\cdot \frac{-2}{2}a_0$, and the odd $k$ will result in $0$.

    We have $y=a_0+\frac{-2}{2}a_0x^2+\frac{(-2)^2}{4\cdot 2}a_0x^4+\frac{(-2)^3}{6\cdot 4\cdot 2}a_0x^6+\dots + \frac{(-3)^n}{2\cdot n!}x^{2n}$.

    We can also write this as $y=a_0\sum_{n=0}\frac{(-1)^n}{n!}x^{2n}$, which ends up being $a_0e^{-x^2}$.
\end{example}

\pagebreak
\begin{example}
    Find a general solution to 
    \[ 2y''+xy'+y=0\]
    in the form of a power series about the ordinary point $x=0$.

    We have $y''+\frac{x}{2}y'+\frac{1}{2}y=0$.

    There are no singular points here, so all points are ordinary.

    We will find this with $y=\sum_{n=0}^{\infty}a_nx^n$ and $y'=\sum_{n=1}a_nnx^{n-1}$ and $y''=\sum_{n=2}a_nn(n-1)x^{n-2}$.

    Plugging this in gives $2\sum_{n=2}a_nn(n-1)x^{n-2}+x\sum_{n=1}a_nnx^{n-1}+\sum_{n=0}a_nx^n =0$.

    This will simplify to $4a_2+a_0+\sum_{k=1}[2a_{k+2}(k+2)(k+1)+(k+1)a_k]x^k =0$.

    The recurrence formula ends up being $a_{k+2}=\frac{-a_k}{2(k+2)}$.

    Let's look at $k=1, k=2,k=3,k=4$ until we find a pattern.

    We also know $a_2=-\frac{1}{4}a_0$.

    We have that $a_3=\frac{-a_1}{2\cdot 3}, a_4=-\frac{a_2}{2\cdot 4}, a_5 = -\frac{a_3}{2\cdot 5}, a_6=-\frac{a_4}{2\cdot 6}$.

    We can write $a_4$ in terms of $a_0$ as $-\frac{1}{2\cdot 4}\cdot -\frac{1}{4}a_0$ and $a_6=-\frac{2\cdot 6}\cdot -\frac{1}{2\cdot 4}\cdot -\frac{1}{4}a_0$.

    With these patterns we can write this as $a_{2n+1}=\frac{(-1)^n}{2^n[(2n+1)\cdot \dots 1]}$.

    Ok we know $y=a_0+a_1x+a_2x^2+a_3x^3+a^4x^4+a^5x^5+\dots$.

    So we get this is equal to $a_0+a_1x-\frac{1}{4}a_0x^2-\frac{1}{6}a_1x^3+\frac{1}{32}a_0x^4+\frac{1}{60}a_1x^5$.

    This is a linear combination of $a_0$ and $a_1$.
\end{example}

\begin{example}
    Find the first few terms in a power series expansion about $x=0$ for a general solution to 
    \[ (1+x^2)y''-y'+y=0 \]

    Yea, a lot of stuff happen.

    If you do previous steps of changing the indices and writing out the power series, we get 

    $[2a_2-a_1+a_0]+[6a_3-2a_2+a_1]x+\sum_{k=2}[(k+2)(k+1)a_{k+2}+(k+1)a_{k+1}+(k^2-k+1)a_k]x^k=0$

    And then we can find $a_{k+2}=\frac{-(k+1)a_{k+1}-(k^2-k+1)a_k}{(k+2)(k+1)}$.

    We also know $a_2=\frac{1}{2}(a_1-a_0)$ and $a_3=\frac{1}{6}(2a_2-a_1)=\\frac{-a_0}{6}$.

    Doing many many steps gives you $y=a_0+-\frac{1}{2}a_0x^2-\frac{1}{6}a_0x^3+\frac{1}{12}a_0x^4+\frac{3}{40}a_0x^5-\frac{17}{720}a_0x^6$ for the case of when $a_1=0$.

    When $a_0=0$, then the equation just becomes $a_1[x+\frac{1}{2}x^2-\frac{1}{8}x^4-\frac{1}{40}x^5+\frac{1}{20}x^6+\dots]$.
\end{example}

\section{Equations with Analytic Coefficients}
We start by stating a basic existence theorem for the equation 
\[ y''(x)+p(x)y'(x)+q(x)y(x)=0\]

\begin{theorem}
    Suppose $x_0$ is an ordinary point for the equation. THen this equation has two linearly independent analytic solutions of the form 
    \[ y(x)=\sum_{n=0}^{\infty}a_n(x-x_0)^n \]

    Moreover, the radius of convergence of any power series solution of the form given by the above is at least as large as the distance from $x_0$ to the nearest singular point (real or complex-valued) of the original equation.
\end{theorem}

\begin{example}
    Find a minimum value for the radius of convergence of a power series solution about $x=0$ to 
    \[ 2y''+xy'+y=0 \]

    So we have $y''+\frac{x}{2}y'+\frac{1}{2}y=0$.

    There are no singular points, so the radius of convergence is $\rho = \infty$
\end{example}

\begin{example}
    Find a minimum value for the radius of convergence of a power series solution about $x=0$ to 
    \[ (1+x^2)y''-y'+y=0 \]

    This is $y''-\frac{1}{1+x^2}y'+\frac{1}{1+x^2}y=0$.

    The singular points are $\pm i$.

    The distance from $0$ is $1$, so $\rho = 1$.
\end{example}

\pagebreak
\begin{example}
    Find the first few terms in a power series expansion about $x=1$ for a general solution to 
    \[ 2y''+xy'+y=0 \]
    Also determine the radius of convergence of the series.

    We can let $t=x-1$, and $x=1$ and $t=0$.

    So we can get $y(x)=\sum_{n=0}^{\infty} a_n(x-1)^n$, so $Y(t)=y(x)=y(t+1)$.

    We have $2\frac{\dd^2 Y}{\dd t^2}+(t+1)\frac{\dd Y}{\dd t}+Y =0$.

    Substituting some of this stuff in gives $2\sum_{n=2}^{\infty}n(n-1)a_nt^{n-2}+(t+1)\sum_{n=1}^{\infty}na_nt^{n-1}+\sum_{n=0}^{\infty}a_nt^n=0$.

    We need to break off some stuff, to simplify the sums.

    We get $(4a_2+a_1+a_0)t^0+\sum_{k=1}2(k+2)(k+1)a_{k+2}t^k + \sum_{k=1}ka_kt^k + \sum_{k=1}(k+1)a_{k+1}t^k + \sum_{k=1}a_k t^k$.

    We can get $a_{k+2}=\frac{-a_k-a_{k+1}}{2(k+2)}$.

    We know of course that $Y(t)=a_0+a_1t+a_2t^2+a_3t^3+\dots$.

    We also know it's a linear combination, so $Y(t)=a_0(1-\frac{1}{4}t^2+\frac{1}{24}t^3+\dots)+a_1(t-\frac{1}{4}t^2-\frac{1}{8}t^3+\dots)$

    And just substitute $t=x-1$ into the above to solve it.
\end{example}

\section{Method of Frobenius}
\begin{definition}
    A singular point $x_0$ of 
    \[ y''(x)+p(x)y'(x)+q(x)y(x)=0\]
    is said to be a regular singular point if both $(x-x_0)p(x)$ and $(x-x_0)^2q(x)$ are analytic at $x_0$. Otherwise $x_0$ is called an irregular singular point.
\end{definition}

\begin{example}
    Classify the singular points of the equation 
    \[ (x^2-1)^2 y''(x)+(x+1)y'(x)-y(X)=0 \]

    Rewriting this gives you $y''+\frac{(x+1)}{(x+1)^2(x-1)^2}y'-\frac{1}{(x+1)^2(x-1)^2}y=0$.

    The singular points are $x=1$ and $x=-1$.

    $x=1$ is an irregular singular point because it is not analytic for both $p(x)$ and $q(x)$. $-1$ is a regular singular point.
\end{example}

\pagebreak
\begin{definition}
    If $x_0$ is a regular singular point of $y''+py'+qy=0$, then the indical equation for this point is 
    \[ r(r-1)+p_0r+q_0=0\]

    where 
    \[ p_0 := \lim_{x\to x_0}(x-x_0)p(x), \qquad q_0 := \lim_{x\to x_0}(x-x_0)^2q(x) \]
    The roots of the indicial equation are called the exponents (indices) of the singularity $x_0$.
\end{definition}

\begin{example}
    Find the indical equation and the exponents of the singularity $x=-1$ of 
    \[ (x^2-1)^2y''(x)+(x+1)y'(x)-y(x)=0 \]

    In standard form we have $y''+\frac{(x+1)}{(x+1)^2(x-1)^2}y'-\frac{1}{(x+1)^2(x-1)^2}y=0$.

    We have $(x+1)p(x)=\frac{1}{(x-1)^2}$ and $(x+1)^2q(x)=\frac{-1}{(x-1)^2}$.

    The limits are $1/4$ and $-1/4$ respectively from this.

    So the indical equation becomes $r(r-1)+p_0r+q_0=0$ or $r(r-1)+\frac{1}{4}r-\frac{1}{4}=0$ or $r^2-\frac{3}{4}r-\frac{1}{4}=0$

    Factoring gives $(4r+1)(r-1)$, and the indical roots are $r=-1/4$ and $r=1$.
\end{example}

\begin{example}
    Find a series expansion about the regular singular point $x=0$ for a solution to 
    \[(x+2)x^2y''(x)-xy'(x)+(1+x)y(x)=0, \qquad x>0 \]

    Finding the indical roots gives us $p_0=-1/2$, and $q_0=1/2$.

    The indicial equation is $2r^2-3r+1=0$, so the indicial roots are $r=1/2$ and $r=1$.

    Now expand about $r=1$.

    We get $(x+2)x^2\sum a_n(n+1)nx^{n-1}-x\sum a_n(n+1)x^n+(1+x)\sum a_nx^{n+1}=0$.

    Do some simplification to get 
    
    $\sum{n=0} a_n(n+1)nx^{n+2}+\sum_{n=1}2a_n(n+1)nx^{n+1}-\sum_{n=1}a_n(n+1)x^{n+1}\sum_{n=0} a_n x^{n+2}$.

    Writing them to start all at the same index and combining gives you $\sum_{k=2}[a_{k-2}(k-1)(k-2)+2a_{k-1}k(k-1)-a_{k-1}(k-1)+a_{k-2}]x^k=0$.

    Finding the recurrence formula gives $a_{k-1}=\frac{-(k^2-3k+3)}{(2k-1)(k-1)}a_{k-2}$.

    Putting $k$ values into the formula gives you $y=x-\frac{1}{3}x^2+\frac{1}{10}x^3-\frac{1}{30}x^4+\dots$.
\end{example}

\begin{theorem}
    If $x_0$ is a regular singular point, then there exists at least one series solution, where $r=r_1$ is the larger root of the associated indicial equation. Moreoever, this series converges for all $x$ such that 
    $0<x-x_0<R$, where $R$ is the distance from $x_0$ to the nearest other singular point (real or complex).
\end{theorem}

\begin{example}
    Find a series solution about the regular singular point $x=0$ of 
    \[ x^2y''(x)-xy'(x)+(1-x)y(x)=0, \qquad x>0 \]

    We have $x=0$ is a regular singular point from writing this in general form.

    Writing the indicial equation gives us $r=1$.

    Writing the summations gives you $\sum_{n=0}a_n(n+1)nx^{n+1}-\sum_{n=0}a_n(n+1)x^{n+1}+\sum_{n=0}a_n x^{n+1}-\sum_{n=0}a_n x^{n+2}=0$.

    Simplify this to get $a_{k-1}=\frac{a_{k-2}}{(k=1)^2}$.

    You end up getting $y=x+x^2+\frac{1}{4}x^3+\frac{1}{36}x^4+\dots$.
\end{example}

\end{document}