\documentclass[../diffeq.tex]{subfiles}
\graphicspath{{\subfix{../figures/}}}
\begin{document}
\chapter{Laplace Transforms}
\section{Definition of the Laplace Transform}
\begin{definition}
    Let $f(t)$ be a function on $[0,\infty)$. The Laplace transform of $f$ is the function $F$ defined by the integral 
    \[ F(s)=\int_0^\infty e^{-st}f(t)\dd t \]

    The domain of $F(s)$ is all the values of $s$ for which the integral above exists. The Laplace transform of $f$ is denoted by both $F$ and $\mathcal{L}\{f\}$.
\end{definition}

\begin{example}
    Determine the Laplace transform of the constant function $f(t)=1, t\geq 0$.

    Let $F(s)=\int_0^{\infty}e^{-st}1\dd t = \int_0^{\infty}e^{-st}\dd t$. This is equal to $-\frac{1}{s}e^{-st}$ with bounds $\infty$ and $0$.

    Remember this is an improper integral where we have $\lim_{b\to \infty}-\frac{1}{s}e^{-st}$ from $0$ to $b$.

    This gives $-\frac{1}{s}e^{-sb}-\frac{1}{s}e^0$ on the inside of the limit, so we get $\lim_{b\to \infty}\left[-\frac{1}{s}e^{-sb}+\frac{1}{s}\right]$.

    The above equals $\lim_{b\to\infty}\left[-\frac{1}{s}\cdot\frac{1}{e^{rb}}+\frac{1}{s}\right]$.

    The restriction is $s>0$ because $\frac{1}{e^{sb}}$ has to be greater than 0.

    Our result ends up being $\frac{1}{s}$.

    $\mathcal{L}\{1\}=\frac{1}{s}$.
\end{example}

\begin{example}
    Determine the Laplace transform of $f(t)=t$.

    We have $\mathcal{L}\{t\}=\int_0^{\infty}e^{-st}t\dd t = \lim_{b\to\infty}\left[\int_0^be^{-st}t\dd t\right]$.

    Integrating by parts gives the inside equal to $-\frac{1}{s}\cdot t\cdot \frac{1}{e^{st}}-\frac{1}{s^2}e^{-st}$ with bounds $0$ to $b$.

    Plugging this in gives $\lim_{b\to\infty} -\frac{1}{s}\cdot \frac{b}{e^{sb}}-\frac{1}{s}\cdot\frac{1}{e^{sb}}+\frac{1}{s^2}$.

    We see that $\frac{b}{e^{sb}}$ is indeterminate, so using L'Hopital's Rule, the derivative is $\frac{1}{se^{sb}}$ and the limit as $b$ approaches $\infty$ gives this as $0$.

    We are left with $\frac{1}{s^2}$.

    $\mathcal{L}\{t\}=\frac{1}{s^2}$.
\end{example}

We will see that $\mathcal{L}\{t^n\} = \frac{n!}{s^{n+1}}$.

\begin{example}
    Determine the Laplace transform of $f(t)=e^{at}$, where $a$ is a constant.

    The integral is $\int_0^{\infty}e^{-st}\cdot e^{at}\dd t = \int_0^{\infty}e^{-(s-a)t}\dd t$.

    Integrating this gives $-\frac{1}{s-a}e^{-(s-a)t}$ evaluated from $0$ to $\infty$.

    As $t$ goes to infinity, we get $0$ and then we get $0-\frac{-1}{s-a}e^0 = \frac{1}{s-a}$.

    So $\mathcal{L}\{e^{at}\}=\frac{1}{s-a}$.
\end{example}

If we were to find the Laplace of $e^{5t}$, from the above example it would be $\frac{1}{s-5}$.

\begin{example}
    Find $\mathcal{L}\{\sin bt\}$, where $b$ is a nonzero constant.

    The integral this time is $\int_0^{\infty}e^{-st}\cdot\sin bt \dd t$.

    Integrating gives $-\frac{1}{s}\sin bt e^{-st}+\frac{b}{s}\left[-\frac{1}{s}\cos bt e^{-st}-\int -\frac{1}{s}e^{-st}(-b)\sin bt \dd t \right]$.

    (Do this example later)

    Involves factoring Laplace stuff.

    $\mathcal{L}\{\sin bt\} = \frac{b}{s^2+b^2}$.
\end{example}

\begin{example}
    Determine the Laplace transform of 

    $f(t)=\begin{cases}
        2 \qquad 0<t<5 \\ 
        0 \qquad 5<t<10 \\
        e^{4t} \qquad t>10
    \end{cases}$

    To do this, you just do $\int_0^{\infty}e^{-st}f(t)\dd t = \int_0^5 e^{-st}\cdot 2 \dd t + \int_5^{10}e^{-st}\cdot 0 \dd t + \int_10^{\infty}e^{-st}\cdot e^{4t}\dd t$.

    Evaluating this gives the laplace as $-\frac{2}{s}e^{-5s}+\frac{2}{s}+\frac{1}{s-4}e^{-(s-4)10}$
\end{example}

An important property of the Laplace transform is its linearity. That is, the Laplace transform $\mathcal{L}$ is a linear operator.

\begin{theorem}
    Let $f$, $f_1$, and $f_2$ be functions whose Laplace transforms exist for $s>\alpha$ and let $c$ be a constant. Then, for $s>\alpha$, 
    \[ \mathcal{L}\{f_1+f_2\} = \mathcal{L}\{f_1\} + \mathcal{L}\{f_2\} \]
    \[ \mathcal{L}\{cf\}=c\mathcal{L}\{f\} \]
\end{theorem}

\ex Determine $\mathcal{L}\{11+5e^{4t}-6\sin 2t\}$.

A function $f(t)$ on $[a,b]$ is said to have a jump discontinuity at $t_0\in (a,b)$ if $f(t)$ is discontinuous at $t_0$, but the one-sided limits 
\[ \lim_{t\to t_0^-}f(t) \qquad \text{and} \qquad \lim_{t\to t_0^+}f(t) \]
exist as finite numbers.

\begin{definition}
    A function $f(t)$ is said to be piecewise continuous on a finite interval $[a,b]$ if $f(t)$ is continuous at every point in $[a,b]$, except possibly for a finite number of points at which $f(t)$ has a jump discontinuity.

    A function $f(t)$ is said to be piecewise continuous on $[0,\infty)$ if $f(t)$ is piecewise continuous on $[0,N]$ for all $N>0$.
\end{definition}

In contrast, the function $f(t)=1/t$ is not piecewise continuous on any interval containing the origin, since it has an ``infinite jump'' at the origin.

A function that is piecewise continuous on a finite interval is not necessarily integrable over that interval. However, piecewise continuity on $[0,\infty)$ is not enough to guarantee the existence (as a finite number) of the improper integral over 
$[0,\infty)$; we also need to consider the growth of the integrand for large $t$. The Laplace transform of a piecewise continuous function exists, provided the function does not grow ``faster than an exponential''.

\begin{definition}
    A function $f(t)$ is said to be of exponential order $\alpha$ if there exist positive constants $T$ and $M$ such that 
    \[ |f(T)|\leq Me^{\alpha t} \]
    for all $t\geq T$.
\end{definition}

\begin{theorem}
    If $f(t)$ is piecewise continuous on $[0,\infty)$ and of exponential order $\alpha$, then $\mathcal{L}\{f\}(s)$ exists for $s>a$.
\end{theorem}

Here are common Laplace transforms:
\begin{itemize}
    \item $\mathcal{L}\{1\} = \frac{1}{s}$
    \item $\mathcal{L}\{t\} = \frac{1}{s^2}$
    \item $\mathcal{L}\{t^n\} = \frac{n!}{s^{n+1}}$
    \item $\mathcal{L}\{e^{at}\} = \frac{1}{s-a}$
    \item $\mathcal{L}\{\sin bt\} = \frac{b}{s^2+b^2}$
    \item $\mathcal{L}\{\cos bt\} = \frac{s}{s^2+b^2}$
\end{itemize}

\section{Properties of the Laplace Transform}
\begin{theorem}
    If the Laplace transform $\mathcal{L}\{f\}(s)=F(s)$ exists for $s>\alpha$, then 
    \[ \mathcal{L}\{e^{\alpha t}f(t)\}(s)=F(s-a) \]
    for $s>\alpha+a$
\end{theorem}

\begin{example}
    Determine the Laplace transform of $e^{\alpha t}\sin bt$

    We know the Laplace of $\sin bt$ is equal to $\frac{b}{s^2+b^2}$.

    Multiplying by $e^{\alpha t}$ just shifts it $F(s-\alpha)=\frac{b}{(s-\alpha)^2+b^2}$
\end{example}

\begin{theorem}
    Let $f(t)$ be continuous on $[0,\infty)$ and $f'(t)$ be piecewise continuous on $[0,\infty)$, with both of exponential order $\alpha$. Then for $s>\alpha$,
    \[\mathcal{L}\{f'\}(s)=s\mathcal{L}\{f\}(s)-f(0) \]
\end{theorem}

\begin{theorem}
    Let $f(t),f'(t),\dots,f^{(n-1)}(t)$ be continuous on $[0,\infty)$ and let $f^{(n)}(t)$ be piecewise continuous on $[0,\infty)$, with all these functions of exponential order $\alpha$.
    Then, for $s>\alpha$,
    \[ \mathcal{L}\{f^{(n)}\}(s)=s^n\mathcal{L}\{f\}(s)-s^{n-1}f(0)-s^{n-2}f'(0)-\dots-f^{(n-1)}(0) \]
\end{theorem}

\begin{example}
    Using the above theorems and the fact that $\mathcal{L}\{\sin bt\}(s)=\frac{b}{s^2+b^2}$, determine $\mathcal{L}\{\cos bt\}$

    We know that $f'(t)=b\cos bt$ from this. So $\mathcal{L}\{b\cos bt\}=s\mathcal{L}\{\sin bt\}-f(0)$.

    We know that $b\mathcal{L}\{\cos bt\}=s\mathcal{L}\{\sin bt\}$, since $f(0)=0$.

    So simplifying gives the Laplace transform as $\frac{s}{s^2+b^2}$
\end{example}

\begin{example}
    Prove the following identity for continous functions $f(t)$ (assuming the transforms exist):
    \[ \mathcal{L}\left\{\int_0^t f(\tau)\dd \tau\right\}(s)=\frac{1}{s}\mathcal{L}\{f(t)\}(s) \]

    We know $g(t)=\int_0^t f(\tau)\dd \tau$. From this we know $g'(t)=f(t)$.

    We get that $\mathcal{L}\{g'(t)\}=s\mathcal{L}\{g(t)\}-g(0)$. and that $\mathcal{L}\{f(t)\}=s\mathcal{L}\{\int_0^t f(\tau)\dd \tau\}$.

    We also know $g(0)=0$.

    So the Laplace of the function is equal to $\frac{1}{s}\mathcal{L}\{f(t)\}$.
\end{example}

\begin{theorem}
    Let $F(s)=\mathcal{L}\{f\}(s)$ and assume $f(t)$ is piecewise continuous on $[0,\infty)$ and of exponential order $\alpha$. Then, for $s>\alpha$,
    \[ \mathcal{L}\{t^nf(t)\}(s)=(-1)^n \frac{\dd^n F}{\dd s^n}(s) \]
\end{theorem}

\begin{example}
    Determine $\mathcal{L}\{t\sin bt\}$.

    We know $f(t)=\sin bt$ and that $n=1$.

    This is equal to $(-1)^1\frac{\dd}{\dd s}\mathcal{L}\{\sin bt\}$.

    We end up getting $-\frac{\dd}{\dd s}\left(\frac{b}{s^2+b^2}\right)$.

    We end up getting $\frac{2bs}{(s^2+b^2)^2}$.
\end{example}

Here are some basic properties of Laplace Transforms
\begin{itemize}
    \item $\mathcal{L}\{f+g\}=\mathcal{L}\{f\}+\mathcal{L}\{g\}$.
    \item $\mathcal{L}\{cf\}=c\mathcal{L}\{f\}$ for any constant $c$.
    \item $\mathcal{L}\{e^{at}f(t)\}(s)=\mathcal{L}\{f\}(s-a)$
    \item $\mathcal{L}\{f'\}(s)=s\mathcal{L}\{f\}(s)-f(0)$
    \item $\mathcal{L}\{f''(s)\}=s^2\mathcal{L}\{f\}(s)-sf(0)-f'(0)$
    \item $\mathcal{L}\{f^{(n)}\}(s)=s^n\mathcal{L}\{f\}(s)-s^{n-1}f(0)-s^{n-2}f'(0)-\dots-f^{(n-1)}(0)$
    \item $\mathcal{L}\{t^nf(t)\}(s)=(-1)^n \frac{\dd^n}{\dd s^n}(\mathcal{L}\{f\}(s))$
\end{itemize}

\section{Inverse Laplace Transform}
\begin{example}
    Solve the initial value problem 
    \[ y''-y=-t \qquad y(0)=0, \qquad y'(0)=1 \]

    We can say that $\mathcal{L}\{y''-y\}=\mathcal{L}\{-t\}$.

    Using properties we know that $\mathcal{L}\{y''\}-\mathcal{L}\{y\}=-\mathcal{L}\{t\}$

    This is equal to $s^2\mathcal{L}\{y\}-sy(0)-y'(0)=\mathcal{L}\{y\}=-\frac{1}{s^2}$.

    Now plugging in $\mathcal{L}\{y(t)\}=Y(s)$, we get $s^2Y(s)1-Y(s)=-\frac{1}{s^2}$

    Simplifying gives $Y(s)(s^2-1)=\frac{s^2-1}{s^2}$.

    We see that $Y(s)=\frac{1}{s^2}$. This is the Laplace of $t$, so $y(t)=t$.
\end{example}

\begin{definition}
    Given a function $F(s)$, if there is a function $f(t)$ that is cintinuous on $[0,\infty)$ and satisfies 
    \[ \mathcal{L}\{f\}=F \]
    then we say that $f(t)$ is the inverse Laplace transform of $F(s)$ and employ the notation $f=\mathcal{L}^{-1}\{F\}$.
\end{definition}

\begin{example}
    Determine $\mathcal{L}\{F\}$ for $F(s)=\frac{2}{s^2}$.

    The Inverse Laplace transform of this is $t^2$.

    Determine it for $F(s)=\frac{3}{s^2+9}$.

    This is $\sin 3t$ from the definition.

    Determine it for $\frac{s-1}{s^2-2s+5}$.

    This simplifies to $\frac{s-1}{(s-1)^2+4}=F(s-1)$. This is the same as $\cos 2t$ but shifted by $1$. The Inverse Laplace transform ends up being $e^t \cos 2t$.
\end{example}

\begin{theorem}
    Assume that $\mathcal{L}^{-1}\{F\}, \mathcal{L}^{-1}\{F_1\}$, and $\mathcal{L}^{-1}\{F_2\}$ exist and are continuous on $[0,\infty)$ and let $c$ be any constant. Then 
    \[ \mathcal{L}^{-1}\{F_1+F_2\} = \mathcal{L}^{-1}\{F_1\}+\mathcal{L}^{-1}\{F_2\} \]
    \[ \mathcal{L}^{-1}\{cF\} = c\mathcal{L}^{-1}\{F\} \]
\end{theorem}

\begin{example}
    Determine $\mathcal{L}^{-1}\left\{\frac{5}{s-6}-\frac{6s}{s^2+9}+\frac{3}{2s^2+8s+10}\right\}$.

    The first two terms of this gives $5e^6t-6\cos 3t$.

    For the last term, We see that $\frac{1}{2(s^2+4s+5)}$ lets us put $\frac{3}{2}$ in the front and we can complete the square for this for the denominator to give $\frac{1}{(s+2)^2+1}$.

    The last term ends up being $\frac{3}{2}e^{-2t}\sin t$.
\end{example}

\ex Determine $\mathcal{L}^{-1}\{ \frac{5}{s+2}^4 \}$

\ex Determine $\mathcal{L}^{-1} \{ \frac{3s+2}{s^2+2s+10} \}$.

Method of Partial Fractions - A rational function of the form $\frac{P(s)}{Q(s)}$, where $P(s)$ and $Q(s)$ are polynomials with the degree of $P$ less than the degree of $Q$ has a partial fraction expansion 
whose form is based on the linear and quadratic factors of $Q(s)$. We consider the three cases:
\begin{enumerate}
    \item Nonrepeated linear factors 
    \item Repeated linear factors 
    \item Quadratic factors
\end{enumerate}

Nonrepeated Linear Factors - If $Q(s)$ can be factored into a product of distinct linear factors, $Q(s)=(s-r_1)(s-r_2)\dots(s-r_n)$, where the $r_i$'s are all distinct real numbers, then the partial fraction expansion has the form 
\[ \frac{P(s)}{Q(s)}=\frac{A_1}{s-r_1}+\frac{A_2}{s-r_2}+\dots + \frac{A_n}{s-r_n} \]
where the $A_i$'s are real numbers.

\begin{example}
    Determine $\mathcal{L}^{-1}\{F\}$, where $F(s)=\frac{7s-1}{(s+1)(s+2)(s-3)}$.

    The decomposition is equal to $\frac{A}{s+1}+\frac{B}{s+2}+\frac{C}{s-3}$.

    Solving for $A,B,C$ gives $2,-3,1$ respectively.

    We end up getting $\frac{2}{s+1}+\frac{-3}{s+2}+\frac{1}{s-3}$. This gives us $2e^{-t}-3e^{-2t}+e^{3t}$.
\end{example}

Repeated Linear Factors - Let $s-r$ be a factor of $Q(s)$ and suppose $(s-r)^m$ is the highest power of $s-r$ that divides $Q(s)$. Then the portion of the partial fraction expansion of $P(s)/Q(s)$ that corresponds to the term $(s-r)^m$ is 
\[ \frac{A_1}{s-r}+\frac{A_2}{(s-r^2)}+\dots + \frac{A_m}{(s-r)^m} \]
where the $A_i$'s are real numbers.

\begin{example}
    Determine $\mathcal{L}\left\{\frac{s^2+9s+2}{(s-1)^2(s+3)}\right\}$.

    We end up getting $\frac{A}{s-1}+\frac{B}{(s-1)^2}+\frac{C}{s+3}$.

    Solving for $A,B,C$ gives $2,3,-1$ respectively.

    This gives $2e^t+3t^t t - e^{-3t}$.
\end{example}

Quadratic Factors - Let $(s-\alpha)^2+\beta^2$ be a quadratic factor of $Q(s)$ that cannot be reduced to linear factors with real coefficients. Suppose $m$ is the highest power of 
$(s-\alpha)^2+\beta^2$ that divides $Q(s)$. THen the portion of the partial fraction expansion that corresponds to $(s-\alpha)^2+\beta^2$ is 
\[ \frac{C_1 s+D_1}{(s-\alpha)^2+\beta^2}+\frac{C_2s+D_2}{[(s-\alpha)^2+\beta^2]^2}+\dots + \frac{C_ms+D_m}{[(s-\alpha)^2+\beta^2]^m} \]
When looking up Laplace transforms, the following equivalent form is more convenient 
\[ \frac{A_1(s-\alpha)+\beta B_1}{(s-\alpha)^2+\beta^2}+\frac{A_2(s-\alpha)\beta B_2}{[(s-\alpha)^2+\beta^2]^2}+\dots + \frac{A_m(s-\alpha)+\beta B_m}{[(s-\alpha)^2+\beta^2]^m} \]

\begin{example}
    Determine $\mathcal{L}^{-1}\left\{\frac{2s^2+10s}{(s^2-2s+5)(s+1)}\right\}$.

    The partial fraction is $\frac{As+B}{(s^2-2s+5)}+\frac{C}{s+1}$.

    Solving the system gives $A,B,C=3,5,-1$.

    So we are now finding the Laplace transform of $\frac{3s+5}{(s-1)^2+4}0\frac{1}{s+1}$.

    The first term of this can be rewritten as $\frac{3(s-1)+8}{(s-1)^2+4}$.

    The transform ends up being $3e^t\cos 2t + 4e^t \sin 2t - e^{-t}$.
\end{example}

\section{Solving Initial Value Problems}
Method of Laplace Transforms 

To solve initial value problems:
\begin{itemize}
    \item Take the Laplace transforms of both sides of the equation 
    \item Use the properties of the Laplace transform and the initial conditions to obtain an equation for the Laplace transform of the solution and then solve this equation for the transform 
    \item Determine the inverse Laplace transform of the solution by looking it up in a table or by using a suitable method (such as partial fractions) in combination with the table.
\end{itemize}

\begin{example}
    Solve the initial value problem 
    \[ y''-2y'+5y=-8e^{-t}\qquad y(0)=2, \qquad y'(0)=12 \]

    This is equal to $\mathcal{L}\{y''\}-2\mathcal{L}\{y'\}+5\mathcal{L}\{y\}=-8\mathcal{L}\{e^{-t}\}$.

    This ends up being $s^2\mathcal{L}\{y\}-sy(0)-y'(0)-2[s\mathcal{L}\{y\}-y(0)]+5\mathcal{L}\{y\}=-8\frac{1}{s+1}$.

    We know that $\mathcal{L}\{y\}=Y(s)$.

    So $Y(s)[s^2-2s+5]-2s-12+4=\frac{-8}{s+1}$.

    This is $Y(s)(s^2-2s+5)=2s+8-\frac{8}{s+1}$.

    This ends up being $Y(s)=\frac{2s}{s^2-2s+5}+\frac{8}{s^2-2s+5}-\frac{8}{(s+1)(s^2-2s+5)}$.

    Simplifying ends up getting $\frac{2s^2+10s}{(s+1)(s^2-2s+5)}$.

    Doing partial fraction decomposition gives $\frac{3s+5}{s^2-2s+5}+\frac{-1}{s+1}=\frac{3(s-1)+8}{(s-1)^2+4}+\frac{-1}{s+1}$.

    The Inverse Laplace of this is $3e^t \cos 2t + 4e^t \sin 2t - e^{-t}$.
\end{example}

\ex Solve the initial value problem 
\[ y''+4y'-5y=te^t \qquad y(0)=1 \qquad y'(0)=0 \]

\begin{example}
    Solve the initial value proiblem 
    \[ w''(t)-2w'(t)+5w(t)=-8e^{\pi-t} \qquad w(\pi)=2 \qquad w'(\pi)=12 \]

    Let's introduce a new function $y(t)=w(t+\pi)$.

    Replace $t$ with $t+\pi$ in this equation and we get $w''(t+\pi)-2w'(t+\pi)+5w(t+\pi)=-8e^{\pi-(t+\pi)}$.

    Substituting the derivatives gives $y''(t)-2y'(t)+5y(t)=-8e^{-t}$.

    This basically comes out to $y=3e^t\cos 2t+4e^t \sin 2t - e^{-t}$.

    Replacing everything with $t-\pi$ gives $3e^{t-\pi}\cos 2(t-\pi)+4e^{t-\pi}\sin 2(t-\pi)-e^{-(t-\pi)}=y(t-\pi)$.

    This gives $w(t)=3e^{t-\pi}\cos 2t + 4e^{t-\pi}\sin 2t - e^{-(t-\pi)}$.
\end{example}

\section{Transforms of Discontinuous Functions}
\begin{definition}
    The unit step function $u(t)$ is defined to by 

    \[ u(t) := \begin{cases}
        0, \qquad t<0, \\
        1, \qquad 0<t 
    \end{cases}
    \]
\end{definition}

\begin{example}
    Graph $u(t), u(t-a)$, and $Mu(t-a)$.
    
    The graph of $u(t)$ is just as given above.

    The graph of $u(t-a)$ is just a horizontal shift.

    The graph of $Mu(t-a)$ will just have the one with $1$ multiplied by $M$
\end{example}

\begin{definition}
    The rectangular window function $\prod_{a,b}(t)$ is defined by 
    \[ \prod_{a,b}(t) := u(t-a)-u(t-b) = \begin{cases}
        0, \qquad t<a \\
        1, \qquad a<t<b \\
        0, \qquad b<t 
    \end{cases}
    \]
\end{definition}

\begin{example}
    Write the function 
    \[ f(t) = \begin{cases}
        3 \qquad t<2 \\
        1 \qquad 2<t<5 \\
        t \qquad 5<t<8 \\
        t^2/10 \qquad 8<t 
    \end{cases}\]
    In terms of window and step functions.

    This is $3\prod_{0,2}(t)+1\prod_{2,5}(t)+t\prod_{5,8}(t)+\frac{t^2}{10}u(t-8)$.

    Also this can be written as $3u(t)-2u(t-2)+(t-1)u(t-5)+(\frac{t^2}{10}-t)u(t-8)$.
\end{example}

\[\mathcal{L}\{u(t-a)\}(s)=\frac{e^{-as}}{s} \]

\begin{theorem}
    Let $F(s)=\mathcal{L}\{f\}(s)$ exist for $s>\alpha\geq 0$. If $a$ is a positive constant, then 
    \[ \mathcal{L}\{f(t-a)u(t-a)\}(s)=e^{-\alpha s}F(s)\]
    and, conversely, an inverse Laplace transform of $e^{-as}F(s)$ is given by 
    \[ \mathcal{L}^{-1}\{e^{-asF(s)}\}(t)=f(t-a)u(t-a) \]
\end{theorem}

\[\mathcal{L}\{g(t)u(t-a)\}(s)=e^{-as}\mathcal{L}\{g(t+a)\}(s)\]

\begin{example}
    Determine the Laplace transform of $t^2u(t-1)$.

    $a=$ from here, and $g(t)=t^2$.

    We take $\mathcal{L}\{g(t)u(t-1)\} = e^{-s}\cdot \mathcal{L}\{g(t+1)\}$.

    Replacing $g(t)$ gives that $t^2+2t+1$ for the inside, so the Answer ends up being $e^{-s}\cdot [\frac{2!}{s^3}+\frac{2}{s^2}+\frac{1}{s}]$.
\end{example}

\begin{example}
    Determine $\mathcal{L}\{(\cos t)u(t-\pi)\}$.

    This has $a=\pi$. So we can see that We are doing $e^{-\pi s}\mathcal{L}\{g(t+\pi)\}$.

    $g(t)=\cos t$, so $g(t+\pi)=\cos(t+\pi)=\cos t \cos \pi - \sin t\sin \pi = -\cos t$.

    So the Laplace is $e^{-\pi s}\cdot -1\cdot \frac{s}{s^2+1}$.
\end{example}

\ex Determine $\mathcal{L}^{-1}\left\{\frac{e^{-2s}}{s^2}\right\}$ and sketch its graph.

\begin{example}
    The current $I$ in an $LC$ series circuit is governed by the initial value problem 
    \[ I''+4I(t)=g(t) \qquad I(0)=0 \qquad I'(0)=0 \]
    where 
    \[ g(t)=\begin{cases}
        1 \qquad 0<t<1 \\
        -1 \qquad 1<t<2 \\
        0 \qquad 2<t 
    \end{cases}\]
    Determine the current as a function of time $t$.

    $g(t)=1\prod_{0,1}+-1\prod_{1,2} = 1[u(t-0)-u(t-1)]-1[u(t-1)-u(t-2)]$. This is equal to $g(t)=1u(t-0)-2u(t-1)+u(t-2)$.

    This simplifies to $1-2u(t-1)+u(t-2)$

    The Laplace of the initial value problem is $s^2\mathcal{L}\{I\}-sI(0)-I'(0)+4\mathcal{L}\{I\}=\mathcal{L}\{1-2u(t-1)+u(t-2)\}$

    We end up getting $(s^2+4)\mathcal{L}\{I\}=\frac{1}{s}-\frac{2e^{-s}}{s}+\frac{e^{-2s}}{s}$.

    We get that $\mathcal{L}\{I\} = \frac{1}{s(s^2+4)}-2e^{-s}\left[\frac{1}{s(s^2+4)}\right]+e^{-2s}\left[\frac{1}{s(s^2+4)}\right]$.

    Using partial fraction decomposition of $\frac{1}{s(s^2+4)}$ gives $\frac{1}{4}\cdot \frac{1}{s}+-\frac{1}{4}\cdot \frac{s}{s^2+4}$.

    If we call what we got above to be $F(s)$, we get $F(s)-2e^{-s}F(s)+e^{-2s}F(s)$.

    The inverse of what we have is $I=\mathcal{L}^{-1}\{F(s)\}-2\mathcal{L}^{-1}\{e^{-s}F(s)\}+\mathcal{L}^{-1}\{e^{-2s}F(s)\}$.

    Doing Laplace stuff gives $I=\frac{1}{4}-\frac{1}{4}\cos 2t - 2\left[\frac{1}{4}-\frac{1}{4}\cos 2(t-1)\right]u(t-1)+\left[\frac{1}{4}-\frac{1}{4}\cos 2(t-2)\right]u(t-2)$.
\end{example}

\section{Transforms of Periodic and Power Functions}
\section{Convolution}
\section{Impulses and the Dirac Delta Function}
\section{Solving Linear Systems with Laplace Transforms}

\end{document}