\documentclass[../precalc.tex]{subfiles}
\graphicspath{{\subfix{../figures/}}}
\begin{document}
\chapter{Functions, Rates, and Patterns}
\section{What is a Function?}
\section{Functions and Types of Functions}
\section{A Qualitative Look at Rates}
\section{Sequences and Triangular Differences}
\section{Functions Defined by Patterns}

\section*{Problems}
\begin{enumerate}
    \item Reflect on your informal work in trying to describe what a function is and consider the various group definitions of function presented. Now revise the definition you originally created for describing a function in order to develop a more refined definition. Explain your reasons for refining your definition.
    \item Why is it or is it not important to have a precise definition of the term function? Also, which of the supplied function definitions do you like best, and why?
    \item In general, is it true that $f(g(x))=g(f(x))$? State your conclusion as a property of the composition of functions.
    \item See if you can identify the component functions $f(x)$ and $g(x)$ for $f[g(x)]=\log(2x-5)$.
    \item One property of invertible functions is that if $f$ is invertible, then $f^{-1}(f(x))=x$ and $f(f^{-1}(x))=x$. Can you verify this property?
    \item Report on where, in your mathematical careers so far, you have encountered and work specifically with sequences, and in what capacity.
    \item Try applying triangular differences for the sequence generated by $n^2$, $n$, and $n^2-3n$. Do you notice any patterns? What about $2n^2+4n$ or $5n^2+2n-5$? 
\end{enumerate}
\end{document}