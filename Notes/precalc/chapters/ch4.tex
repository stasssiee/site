\documentclass[../precalc.tex]{subfiles}
\graphicspath{{\subfix{../figures/}}}
\begin{document}
\chapter{Trigonometry}
\section{Working with Identities}
\section{Trigonometric Foundations}
\section{The Trigonometric Functions Off the Unit Circle}
\section{Angular and Linear Speed}
\section{Back to Identities}
\section{The Roller Coaster}

\section*{Problems}
\begin{enumerate}
    \item Prove the identity $\tan(\theta) = \frac{\sin(\theta)}{\cos(\theta)}$.
    \item The domains of the $\csc(x)$, $\sec(x)$, and $\cot(x)$ trigonometric functions can also be restricted such that they are invertible. Do some research to find information about the inverses of these functions and their properties.
    \item Find the formulas for $\sin(\varphi + \theta)$ and $\sin(\varphi - \theta)$.
    \item Find a formula for $\sin(2\varphi)$, then determine two more formulas for $\cos(2\varphi)$.
    \item Use the results from the above two problems to find an algebraic expression for $\cos\left(\frac{8\pi}{3}\right)$. Evaluate this expression by using one of the double-angle formulas derived previously. Verify the value you just calculated with the value of the cosine of the co-terminal angle on the Unit Circle.
    \item Use the Law of Cosines and the distance formula to derive the trigonometric expansion identity for $\cos(\alpha - \beta)$.
\end{enumerate}
\end{document}