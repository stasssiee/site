\documentclass[../precalc.tex]{subfiles}
\graphicspath{{\subfix{../figures/}}}
\begin{document}
\chapter{Algebra and Geometry}
\section{Algebra and Geometry}
\section{Complex Geometry and Roots}
\section{Conic Sections}
\section{Using Matrices to Find Models}
\section{Using Statistical Regression to Fit a Function to Bivariate Data}

\section*{Problems}
\begin{enumerate}
    \item For $f(x)=ax^2+bx+c$, find $d+ei$ that will generate real values.
    \item Using the general formula $ax^2+bx+c=0$, derive the quadratic formula.
    \item Given integers $a$, $b$, and $c$ such that $0<a<b$, and given 
    \[P(x)=x(x-a)(x-b)-13\]
    where $P(x)$ is divisible by $(x-c)$.

    Find $a$, $b$, and $c$. Is your solution unique? Justify your answer.

    \item Can you provide an analytic definition of a circle?
    \item Given: Two fixed points $F$, $G$, and a fixed positive number $k$. The ellipse consists of all points $P$ such that $\overline{FP}+\overline{GP}=k$. 
    
    The fixed points $F$ and $G$ have coordinates $(-c,0)$ and $(c,0)$ respectively. The points $A$ and $B$ are points where the ellipse intersects the positive $x$-axis and positive $y$-axis, respectively.
    
    Verify that $k=2a$ and $c^2=a^2-b^2$, knowing that $A$ and $B$ are points on the ellipse.

    \item Consider $0=Ax^2+Cy^2+Dx+Ey+F$. This is the general form of the equation for any of the conic sections. 
    
    How would one know which conic is represented by a given equation in this form?

    Put each of the conics represented in general form into the standard form of the specific conic.
    \item Find the inverse of matrix $A$ where:
    $A = \begin{bmatrix}
        0 & 1 & 2\\
        1 & 0 & 3\\
        4 & -3 & 8
    \end{bmatrix}$
    Based on this example, can you provide some general informal justification for why this process works?
\end{enumerate}
\end{document}