\documentclass[../precalc.tex]{subfiles}
\graphicspath{{\subfix{../figures/}}}
\begin{document}
\chapter{Limits and Rates of Change of Functions}
\section{First, Some Background - Rational Functions}
\section{Limits}
\section{Approximating Rates of Change}
\section{The Derivative}

\section*{Problems}
\begin{enumerate}
    \item Perform the division of the ration function $f(x)=\frac{x^2-x-2}{x-1}$ and sketch a graph. Make a conjecture about the equations for the asymptotes. Confirm your conjecture by analyzing the results of the function's division.
    \item A formal definition of the limit of a function is:
    \begin{align*}
    L = \lim_{x\to c}f(x) \text{iff for any} \epsilon > 0 \text{there exists a number} \delta > 0\\
    \text{such that if} 0<\mid x-c \mid < \delta \text{then} \mid f(x) - L \mid < \epsilon
    \end{align*}
    On a pair of axes, precisely draw and label a piecture that illustrates the meaning of the given formal definition of limit.
\end{enumerate}
\end{document}