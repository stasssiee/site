\documentclass[../calc3.tex]{subfiles}
\graphicspath{{\subfix{../figures/}}}
\begin{document}
\chapter{Vectors and the Geometry of Space}
\section{Vectors in the Plane}
Vectors are quantities with both magnitude and direction. The vector who's tail is at point 
P and head at point Q is denoted as $\overrightarrow{PQ}$.

Two vectors, \textbf{u} and \textbf{v} are equal if they have equal length and point 
in the same direction. They do not necessarily have to be in the same location.

Scalar multiplication happens when a $c$ is multiplied to vector \textbf{v}. If $c<0$, then 
vector cv and v will point in opposite directions, otherwise they will point in the same direction. 
Two vectors are parallel if they are scalar multiples of each other.

If you place the tail of a vector \textbf{v} at the head of another vector \textbf{u}, the sum 
\textbf{u+v} is the vector that extends from the tail of \textbf{u} to the head of \textbf{v}.

The vector difference \textbf{u-v} is defined as \textbf{u+(-v)}.

In order to do calculations with vectors, we must introduce a cartesian plane. Angle brackets 
$\langle{a,b}\rangle$ show the components of a vector.

The magnitude of a vector is simply its length. Given points $P(x_1,y_1)$ and $Q(x_1,y_1)$, 
the magnitude of the vector $\vec{PQ}=\langle{x_2-x_1,y_2-y_1}\rangle$ is denoted as $|\vec{PQ}|$, is equal to:
    \[|\vec{PQ}|=\sqrt{(x_2-x_1)^2+(y_2-y_1)^2}\]

We can also now do vector addition with components. Given two vectors \textbf{u} and \textbf{v}, the vector sum is:
    \[\textbf{u}+\textbf{v}=\langle{u_1+v_1,u_2+v_2}\rangle\]


For a scalar $c$ and a vector \textbf{u}, the scalar multiple is $c$\textbf{u}. i.e $|c\textbf{u}|=|c||\textbf{u}|$

A unit vector is any vector with length 1. \textbf{i} is a unit vector in the x-direction and \textbf{j} is a 
unit vector in the y-direction. 
\begin{example}
Determine the necessary air speed and heading that a pilot must maintain in order to 
fly her commercial jet north at a speed of 480 mi/hr relative to the ground in a crosswind that is 
blowing 60$\deg$ south of east at 20 mi/hr.

Let $\vec{p}$ be the velocity vector we are trying to find. 

We have ground vector $\langle0,480\rangle$ and a crosswind vector $\langle10,-10\sqrt{3}\rangle$. 
Note we got the x-component and the y-component of the crosswind vector from the formula: 
$\vec{v}=\langle a\cos\theta, b\cos\theta\rangle$

We can find the vector $\vec{p}$ from adding this vector to the crosswind vector resulting in: 
$\vec{p}=\vec{g}-\vec{c} = \langle-10,480+10\sqrt{3}\rangle$. 

The magnitude of this vector is the speed and is equal to 497.2 mi/hr roughly.
\end{example}

\section{Vectors in Three Dimensions}
We can create a z-axis to create a three dimensional system.

The $xyz$-plane is divided into octants and has 3 planes, the $xy$-plane, the $xz$-plane, and the $yz$-plane. 

We can also extend the distance formula to 3 dimensions. It is similar to the distance 
formula in two dimensions, with the z-component added, essentially:
\[|PQ|=\sqrt{|PR|^2+|RQ|^2}=\sqrt{(x_2-x_1)^2+(y_2-y_1)^2+(z_2-z_1)^2}\]

The midpoint formula works the same way:
\[\text{Midpoint}=\left(\frac{x_1+x_2}{2}+\frac{y_1+y_2}{2}+\frac{z_1+z_2}{2}\right)\]

The normal form of the circle equation is: 
\[(x-h)^2+(y-k)^2=r^2\]

For a disk we have 
\[(x-h)^2+(y-k)^2\leq r^2\]

We can generalize this to a sphere. A sphere centered at $(a,b,c)$ with radius $r$ is the set of points satisfying:
\[(x-a)^2+(y-b)^2+(z-c)^2=r^2\]

A ball centered at $(a,b,c)$ with radius $r$ is the set of points satisfying:
\[(x-a)^2+(y-b)^2+(z-c)^2\leq r^2\]

\begin{example}
Find an equation of the sphere passing through $P(-4,2,3)$ and $Q(0,2,7)$ with its center at the midpoint of $PQ$.

We can find the midpoint from the midpoint formula and it is equal to $(-2,2,5)$. 

The radius can be found through the distance formula and is equal to $\sqrt{8}$.

The equation is $(x+2)^2+(y-2)^2+(z-5)^2=8$
\end{example}

All the vector operations from two-dimensions work in three-dimensions. 
\begin{example}
A model airplane is flying horizontally due east at 10 mi/hr when it encounters a horizontal 
crosswind blowing south at 5 mi/hr and an updraft blowing vertically upward at 5 mi/hr. 
\begin{itemize}
    \item Find the position vector that represents the velocity of the plane relative to the ground.
    \item Find the speed of the plane relative to the ground.
\end{itemize}
The velocity vector of the model plane $\vec{p}$ is equal to $\langle 10,0,0 \rangle$

The velocity vector of the horizontal crosswind $\vec{w}$ is equal to $\langle 0,-5,0\rangle$

The velocity vector of the updraft $\vec{u}$ is $\langle 0,0,5\rangle$

Adding the three vectors results in the speed: $\langle 10,-5,5\rangle$

The magnitude of this is roughly 12.25.
\end{example}

\section{Dot Products}
\begin{definition}
    Given two nonzero vectors \textbf{u} and \textbf{v} in two or three dimensions, the dot product is
    \[\textbf{u}\cdot\textbf{v}=|\textbf{u}||\textbf{v}|\cos \theta\]
\end{definition}

The dot product is 0 when $\theta=\frac{\pi}{2}$, negative when $\theta>\frac{\pi}{2}$ and positive when $\theta<\frac{\pi}{2}$.

Two vectors are parallel if and only if \textbf{u}$\cdot$\textbf{v}=$\pm|\textbf{u}||\textbf{v}|$ 

When the dot product is zero, we call $\vec{u}$ and $\vec{v}$ orthogonal.
\begin{theorem}[Dot Product]
    Given two vectors $\textbf{u}=\langle u_1, u_2, u_3 \rangle$ and $\textbf{v}=\langle v_1, v_2, v_3 \rangle$,
    \[\textbf{u}\cdot\textbf{v}=u_1v_1+u_2v_2+u_3v_3\]
\end{theorem}

We now apply the dot product to vector projections.
\begin{definition}
    The orthogonal projection of \textbf{u} on \textbf{v}, denoted proj$_{\textbf{v}}\textbf{u}$ is:
    \[\text{proj}_{\textbf{v}}\textbf{u}=|\textbf{u}|\cos\theta\left(\frac{\textbf{v}}{|\textbf{v}|}\right)\]

    The orthogonal projections can also be computed with the formulas:
    \[\text{proj}_{\textbf{v}}\textbf{u}=\text{scal}_{\textbf{v}}\textbf{u}\left(\frac{\textbf{v}}{|\textbf{v}|}\right)=
    \left(\frac{\textbf{u}\cdot\textbf{v}}{\textbf{v}\cdot\textbf{v}}\right)\textbf{v}\]
    
    where the scalar component of \textbf{u} in the direction of \textbf{v} is 
    \[\text{scal}_{\textbf{v}}\textbf{u}=|\textbf{u}|\cos\theta = \frac{\textbf{u}\cdot\textbf{v}}{|\textbf{v}|}\]
\end{definition}

\section{Cross Products}
\begin{definition}
    Given two vectors $\textbf{u}$ and $\textbf{v}$ in $\mathbb{R}^3$, the cross product $\textbf{u}\times\textbf{v}$ 
    is a vector with magnitude
        \[|\textbf{u}\times\textbf{v}|=|\textbf{u}||\textbf{v}|\sin\theta\]
\end{definition}

Note that $\vec{u}\times\vec{v}=-(\vec{v}\times\vec{u})$

There are some useful properties of the cross product.
\begin{itemize}
    \item The cross product $\textbf{u}\times\textbf{v}$ is orthogonal to both $\vec{u}$ and $\vec{v}$
    \item The cross product is zero when $\sin(\theta)=0$.
    \item Two vectors are parallel if the cross product between them is zero.
\end{itemize}

We define the determinant of a $2\times 2$ array
$\begin{vmatrix}
a & b\\
c & d   
\end{vmatrix}$
as $ad-bc$.

For a matrix $\begin{vmatrix}
a & b & c\\
u_1 & u_2 & u_3\\
v_1 & v_2 & v_3
\end{vmatrix}$

the determinant is $a\begin{vmatrix}
    u_2 & u_3\\
    v_2 & v_3
\end{vmatrix}-b\begin{vmatrix}
    u_1 & u_3\\
    v_1 & v_3
\end{vmatrix}
+c\begin{vmatrix}
    u_1 & u_2\\
    v_1 & v_2
\end{vmatrix}$ 
or:
\[a(u_2v_3-u_3v_2)-b(u_1v_3-u_3v_1)+c(u_1v_2-u_2v_1)\]

\section{Lines and Planes in Space}
Recall that in two dimensions, we needed a point and a slope to write an equation for a line.

We can write an equation in three dimensions as well.

A vector equation of a line passing through the point $P_0(x_0,y_0,z_0)$ in the direction of vector 
$\textbf{v}=\langle a,b,c \rangle$ is $\textbf{r}=\textbf{r}_0+t\textbf{v}$ or:
\[\langle x,y,z \rangle = \langle x_0,y_0,z_0 \rangle + t\langle a,b,c \rangle\]

The corresponding parametric equations of the line also are:
\[x=x_0+at, y=y_0+bt, z=z_0+ct\]

The general equation of a plane in $\mathrm{R}^3$ with the plane passing through 
$P_0(x_0,y_0,z_0)$ with vector $\textbf{v}=\langle a,b,c\rangle$ is described by:
\[a(x-x_0)+b(y-y_0)+c(z-z_0)=0\]
or 
\[ax+by+cz=d\] where $d=ax_0+by_0+cz_0$.

\section{Cylinders and Quadric Surfaces}
A cylinder is a surface that is parallel to a line.

A trace of a surface is the set of points at which the surface intersects a plane that is 
parallel to one of the coordinate planes. The traces in the coordinate planes are called the 
$xy$-trace, the $yz$-trace, and the $xz$-trace.

To sketch quadric surfaces:
\begin{itemize}
    \item Determine the points where the surface intersects the coordinate axes. 
    \item Finding traces of the surface helps visualize the surface
    \item Sketch at least two traces in parallel planes
\end{itemize}

\begin{example}
    For:
    \begin{align*}
        \frac{x^2}{9}+\frac{y^2}{16}+\frac{z^2}{25}=1
    \end{align*}
    We set certain variables to zero to find the $x$-intercept to be $\pm 3$, the $y$-intercept 
    to be $\pm 4$ and the $z$-intercept to be $\pm 5$.

    We can also find the traces:
    \begin{itemize}
        \item $xy$: $\frac{x^2}{9}+\frac{y^2}{16}=1$
        \item $xz$: $\frac{x^2}{9}+\frac{z^2}{25}=1$
        \item $yz$: $\frac{y^2}{16}+\frac{z^2}{25}=1$
    \end{itemize}

    The resulting shape is a ellipsoid.
\end{example}

In general the equation of an ellipsoid is:
\[\frac{x^2}{a^2}+\frac{y^2}{b^2}+\frac{z^2}{c^2}=1\]
In this all traces are ellipses.

For an elliptic cone the equation is:
\[\frac{x^2}{a^2}+\frac{y^2}{b^2}=\frac{z^2}{c^2}\]
Traces with $z = z_0 \neq 0$ are ellipses. Traces with $x=x_0$ or $y=y_0$ are hyperbolas or intersecting lines.

For an elliptic paraboloid the equation is:
\[z=\frac{x^2}{a^2}+\frac{y^2}{b^2}\]
Traces with $z=z_0>0$ are ellipses. Traces with $x=x_0$ or $y=y_0$ are parabolas.

For a hyperbolic paraboloid:
\[z=\frac{x^2}{a^2}-\frac{y^2}{b^2}\]
Traces with $z=z_0\neq 0$ are hyperbolas. Traces with $x=x_0$ or $y=y_0$ are parabolas.

For a hyperboloid of one sheet:
\[\frac{x^2}{a^2}+\frac{y^2}{b^2}-\frac{z^2}{c^2}=1\]
Traces with $z=z_0$ are ellipses for all $z_0$. Traces with $x=x_0$ or $y=y_0$ are hyperbolas.

For a hyperboloid of two sheets:
\[-\frac{x^2}{a^2}-\frac{y^2}{b^2}+\frac{z^2}{c^2}=1\]
Traces with $z=z_0$ with $|z_0|>|c|$ are ellipses. Traces with $x=x_0$ and $y=y_0$ are hyperbolas.
\end{document}