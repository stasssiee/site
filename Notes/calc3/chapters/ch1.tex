\documentclass[../calc3.tex]{subfiles}
\graphicspath{{\subfix{../figures/}}}
\begin{document}
\chapter{Vectors and the Geometry of Space}
\section{Vectors in the Plane}
Vectors are quantities that have both length and direction. The vector whose tail is at the point $P$ and whose head is at the point $Q$ is denoted 
$\vec{PQ}$. We also label vectors with single boldface characters such as $\textbf{u}$ and $\textbf{v}$.

Two vectors are equal, written $\textbf{u}=\textbf{v}$ if they have equal length and point in the same direction. They do not necessarily need 
to have the same location. Any two vectors with the same length and direction are equal.

Not all quantites are represented by vectors. Quantities describes with a magnitude but without a direction, such as mass and temperature are called scalars.

\subsection*{Scalar Multiplication}
A scalar $c$ and a vector $\textbf{v}$ can be combined using scalar-vector multiplication. The resulting vector, denoted $c\textbf{v}$ is called a scalar multiple of $\textbf{v}$.
The length of $c\textbf{v}$ is $|c|$ multiplied by the length of $\textbf{v}$. The vector $c\textbf{v}$ has the same direction as $\textbf{v}$ if $c>0$. If $c<0$, then $c\textbf{v}$ and $\textbf{v}$ point in opposite directions.
If $c=0$, then the product is $\textbf{0}$ (the zero vector).

\begin{definition}[Scalar Multiples and Parallel Vectors]
    Given a scalar $c$ and a vector $\textbf{v}$, the scalar multiple $c\textbf{v}$ is a vector whose length is $|c|$ multiplied by the length of $\textbf{v}$. If $c>0$, then $c\textbf{v}$ has the same direction as $\textbf{v}$. If $c<0$, then $c\textbf{v}$ and $\textbf{v}$ point in opposite directions.
    Two vectors are parallel if they are scalar multiples of each other.    
\end{definition}

Notice that $0\textbf{v}=\textbf{0}$ for all vectors $\textbf{v}$. It follows that the zero vector is parallel to all vectors.

\subsection*{Vector Addition and Subtraction}
We can represent the sum of two vectors $\textbf{u}$ and $\textbf{v}$ geometrically using the parallelogram rule. The tails of $\textbf{u}$ and $\textbf{v}$ are connected to form adjacent sides of a parallelogram; 
then the remaining two sides of the parallelogram are sketched. The sum $\textbf{u}+\textbf{v}$ is the vector that coincides with the diagonal of the parallelogram, beginning at the tails of $\textbf{u}$ and $\textbf{v}$.

The difference $\textbf{u}-\textbf{v}$ is defined to be the sum $\textbf{u} + (-\textbf{v})$. 

\subsection*{Vector Components}
To do calculations with vectors, it is necessary to introduce a coordinate system. We will start by considering a vector $\textbf{v}$ whose tail is at the origin in the Cartesian plane and whose head is at the point $(v_1,v_2)$.

\begin{definition}[Position Vectors and Vector Components]
    A vector $\textbf{v}$ with its tail at the origin and head at the point $(v_1,v_2)$ is called a position vector and is written $\langle v_1,v_2\rangle$.
    The real numbers $v_1$ and $v_2$ are the $x$ and $y$ components of $\textbf{v}$, respectively. The Position vectors $\textbf{u}=\langle u_1,u_2\rangle$ and $\textbf{v}=\langle v_1,v_2\rangle$ are equal if and only if $u_1=v_1$ and $u_2=v_2$.
\end{definition}

There are infinitely many vectors equal to the position vector $\textbf{v}$, all with the same length and direction. It is important to abide by the convention that $\textbf{v}=\langle v_1,v_2\rangle$ refers to the position vector $\textbf{v}$ or to any other vector equal to $\textbf{v}$.

Now consider the vector $\vec{PQ}$ equal to $\textbf{v}=\langle v_1,v_2\rangle$, but not in standard position, with its tail at the point $P(x_1,y_1)$ and its head at the point $Q(x_2,y_2)$. The $x$-component of $\vec{PQ}$ is the difference in the $x$-coordinates of $Q$ and $P$, or $x_2-x_1$. The $y$-component 
of $\vec{PQ}$ is the difference in the $y$-coordinates, $y_2,y_1$. Therefore, $\vec{PQ}=\langle x_2-x_1,y_2-y_1\rangle = \langle v_1,v_2\rangle = \textbf{v}$.

\subsection*{Magnitude}
The magnitude of a vector is simpliy its length.

\begin{definition}[Magnitude of a Vector]
    Given the points $P(x_1,y_1)$ and $Q(x_2,y_2)$, the magnitude, or length, of $\vec{PQ}=\langle x_2,x_1,y_2-y_1\rangle$, denoted $|\vec{PQ}|$ is the distance between $P$ and $Q$:
    \[ |\vec{PQ}| = \sqrt{(x_2-x_1)^2+(y_2-y_1)^2} \]
    The magnitude of the position vector $\textbf{v}=\langle v_1,v_2\rangle$ is $|\textbf{v}|=\sqrt{v_1^2+v_2^2}$.
\end{definition}

\begin{example}
    Given the points $O(0,0), P(-3,4)$, and $Q(6,5)$, find the components and magnitude of the following vectors:
    \begin{itemize}
        \item a. $\vec{OP}$
        \item b. $\vec{PQ}$
    \end{itemize}

    $|\vec{OP}|=\sqrt{(-3)^2+4^2}=5$

    $|\vec{PQ}|=\sqrt{9^2+1^2}=\sqrt{82}$
\end{example}

\subsection*{Vector Operations in Terms of Components}
We now show how vector addition, vector subtraction, and scalar multiplication are performed using components. 
Suppose $\textbf{u}=\langle u_1,u_2\rangle$ and $\textbf{v}=\langle v_1,v_2\rangle$. The vector sum $\textbf{u}$ and $\textbf{v}$ is 
$\textbf{u}+\textbf{v}=\langle u_1+v_1,u_2+v_2\rangle$. The definition of a vector sum is conistent with the Parallelogram Rule given earlier.

For a scalar $c$ and a vector $\textbf{u}$, the scalar multiple $c\textbf{u}$ is $c\textbf{u}=\langle cu_1,cu_2\rangle$; that is, the scalar $c$ multiplies each component of $\textbf{u}$. If $c>0$, $\textbf{u}$ and $c\textbf{u}$ have the same direction.
If $c<0$, $\textbf{u}$ and $c\textbf{u}$ have opposite directions. In either case, $|c\textbf{u}|=|c||\textbf{u}|$.

Notice that $\textbf{u}-\textbf{v}=\textbf{u}+(-\textbf{v})$, where $-\textbf{v}=\langle -v_1,-v_2\rangle$. Therefore the vector difference of 
$\textbf{u}$ and $\textbf{v}$ is $\textbf{u}-\textbf{v}=\langle u_1-v_1,u_2,v_2\rangle$.

\begin{definition}[Vector Operations in $\mathbb{R}^2$]
    Suppose $c$ is a scalar, $\textbf{u}=\langle u_1,u_2\rangle$ and $\textbf{v}=\langle v_1,v_2\rangle$.
    \[ \textbf{u}+\textbf{v}=\langle u_1+v_1,u_2+v_2\rangle \] 
    \[ \textbf{u}-\textbf{v}=\langle u_1-v_1,u_2-v_2\rangle \]
    \[ c\textbf{u}=\langle cu_1, cu_2\rangle \]
\end{definition}
\pagebreak
\begin{example}
    Let $\textbf{u}=\langle -1,2\rangle$ and $\textbf{v}=\langle 2,3\rangle$.
    \begin{itemize}
        \item a. Evaluate $|\textbf{u}+\textbf{v}|$
        \item b. Simplify $2\textbf{u}-3\textbf{v}$
        \item c. Find two vectors half as long as $\textbf{u}$ and parallel to $\textbf{u}$
    \end{itemize}

    a. Because $\textbf{u}+\textbf{v}=\langle 1,5\rangle$, the magnitude is $\sqrt{26}$.
    
    b. $2\textbf{u}-3\textbf{v}=\langle -2,4\rangle - \langle 6,9\rangle = \langle -8,-5\rangle$.

    c. The vectors $\frac{1}{2}\textbf{u}=\langle -\frac{1}{2}, 1\rangle$ and $-\frac{1}{2}\textbf{u}=\langle \frac{1}{2},-1\rangle$ have half the length of $\textbf{u}$ and are parallel to $\textbf{u}$.
\end{example}

\subsection*{Unit Vectors}
A unit vector is any vector with length 1. Two useful unit vectors are the coordiante unit vectors $\textbf{i}=\langle 1,0\rangle$ and $\textbf{j}=\langle 0,1\rangle$.
These vectors are directed along the coordinate axes and enable us to express all vector in an alternative form.

For example $\langle 3,4\rangle = 3\textbf{i}+4\textbf{j}$.

In general, the vector $\textbf{v}=\langle v_1,v_2\rangle$ is also written as $v_1\textbf{i}+v_2\textbf{j}$.

Given a nonzero vector $\textbf{v}$, we sometimes need to construct a new vector parallel to $\textbf{v}$ of a specified length. Dividing $\textbf{v}$ by its length, we obtain the vector $\textbf{u}=\frac{\textbf{v}}{|\textbf{v}|}$. Because $\textbf{u}$ is a 
positive scalar multiple of $\textbf{v}$, it follows that $\textbf{u}$ has the same direction as $\textbf{v}$. Furthermore, $\textbf{u}$ is a unit vector because $|\textbf{u}|=\frac{|\textbf{v}|}{|\textbf{v}|}=1$. The vector $-\textbf{u}$ is also a unit vector with a direction opposite that of $\textbf{v}$.
Therefore $\pm \frac{\textbf{v}}{|\textbf{v}|}$ are unit vectors parallel to $\textbf{v}$ that point in opposite directions.

To construct a vector that points in the direction of $\textbf{v}$ and has a specified length $c>0$, we form the vector $\frac{c\textbf{v}}{|\textbf{v}|}$. It is a positive scalar multiple of $\textbf{v}$, so it points in the direction of $\textbf{v}$, and its length is 
$\left| \frac{c\textbf{v}}{|\textbf{v}|}\right|=|c|\frac{|\textbf{v}|}{|\textbf{v}|}=c$. The vector $-\frac{c\textbf{v}}{|\textbf{v}|}$ points in the opposite direction and also has length $c$. With this construction, we can also write $\textbf{v}$ as the prodcut of its magnitude and a unit vector in the direction of $\textbf{v}$:
\[ \textbf{v}=|\textbf{v}|\cdot \frac{\textbf{v}}{|\textbf{v}|} \]

\begin{example}
    Consider the points $P(1,-2)$ and $Q(6,10)$.
    \begin{itemize}
        \item a. Find $\vec{PQ}$ and two unit vectors parallel to $\vec{PQ}$.
        \item b. Find two vectors of length 2 parallel to $\vec{PQ}$.
        \item c. Express $\vec{PQ}$ as the product of its magnitude and a unit vector.
    \end{itemize}

    a. $\vec{PQ} = \langle 5,12\rangle$. The magnitude of this vector is $13$, so a unit vector parallel to $\vec{PQ}$ is $\frac{\vec{PQ}}{|\vec{PQ}|}=\frac{5}{13}\textbf{i}+\frac{5}{13}\textbf{j}$.
    The unit vector parallel in the opposite direction is $\langle -\frac{5}{13},-\frac{12}{13}\rangle$.

    b. To obtain two vectors of length $2$ that are parallel to $\vec{PQ}$, we multiply the unit vector $\frac{5}{13}\textbf{i}+\frac{5}{13}\textbf{j}$ by $\pm 2$.
    This obtains $\frac{10}{13}\textbf{i}+\frac{24}{13}\textbf{j}$ and $-\frac{10}{13}\textbf{i}-\frac{24}{13}\textbf{j}$.

    c. The unit vector $\langle \frac{5}{13},\frac{12}{13}\rangle$ points in the direction of $\vec{PQ}$, so we have $\vec{PQ}=|\vec{PQ}|\frac{\vec{PQ}}{|\vec{PQ}|}=13\langle \frac{5}{13},\frac{12}{13}\rangle$.
\end{example}
\subsection*{Properties of Vector Operations}
When we stand back and look at vector operations, ten general properties emerge. 

The following are the ten properties of vector operations:
\begin{enumerate}
    \item $\textbf{u}+\textbf{v}=\textbf{v}+\textbf{u}$
    \item $(\textbf{u}+\textbf{v})+\textbf{w}=\textbf{u}+(\textbf{v}+\textbf{w})$
    \item $\textbf{v}+\textbf{0}=\textbf{v}$
    \item $\textbf{v}+(-\textbf{v})=\textbf{0}$
    \item $c(\textbf{u}+\textbf{v})=c\textbf{u}+c\textbf{v}$
    \item $(a+c)\textbf{v}=a\textbf{v}+c\textbf{v}$
    \item $0\textbf{v}=\textbf{0}$
    \item $c\textbf{0}=\textbf{0}$
    \item $1\textbf{v}=\textbf{v}$
    \item $a(c\textbf{v})=(ac)\textbf{v}$
\end{enumerate}

These properties allow us to solve vector equations.

\subsection*{Applications of Vectors}
Vectors have countless practical applications, particularly in the physical sciences and engineering. For now, we will present two common uses of vectors: to describe velocities and forces.

Consider a motorboat crossing a river whose current is everywhere represented by the constant vector $\textbf{w}$; this means that $|\textbf{w}|$ is the speed of the moving water and 
$\textbf{w}$ points in the direction of the moving water. Assume the vector $\textbf{v}_w$ is the sum $\textbf{v}_g=\textbf{v}_w+\textbf{w}$, which is the velocity of the boat that would be observed by someone on the shore.

\begin{example}
    Suppose the water in a river moves southwest ($45\degree$ west of south) at 4 mi/hr and a motorboat travels due east at 15 mi/hr relative to the shore. Determine the speed of the boat and its heading relative to the moving water.

    Because the boat moves east at 15 mi/hr, the velocity relative to the shore is $\textbf{v}_g = \langle 15,0\rangle$. 

    To obtain the components of $\textbf{w}$, observe that $|\textbf{w}|=4$ and since it is a 45-45-90 triangle, we can find that $|w_x|=|w_y|=2\sqrt{2}$.

    Given the orientation of $\textbf{w}$, $\textbf{w}=\langle -2\sqrt{2},-2\sqrt{2}\rangle$. Because $\textbf{v}_g=\textbf{v}_w+\textbf{w}$, $\textbf{v}_w=\langle 15+2\sqrt{2}, 2\sqrt{2}\rangle$.

    The magnitude of this is approximately 18, so the speed of the boat relative to the water is approximately 18 mi/hr.

    The heading of the boat is given by the angle $\theta$ between $\textbf{v}_w$ and the positive $x$-axis. Therefore, $\theta = \tan^{-1}\left(\frac{2\sqrt{2}}{15+2\sqrt{2}}\right)\approx 9\degree$.

    Therefore the heading of the boat is approximately $9\degree$ north of east, and its speed relative to the water is approximately 18 mi/hr.
\end{example}

Suppose a child pulls on the handle of a wagon at an angle of $\theta$ with the horizontal. The vector $\textbf{F}$ represents the force exerted on the wagon; it has a magnitude 
$|\textbf{F}|$ and a direction given by $\theta$. We denote the horizontal and vertical components of $\textbf{F}$ as $F_x$ and $F_y$ respectively. 
We see that $F_x=|\textbf{F}|\cos\theta, F_y=|\textbf{F}|\sin\theta$, and the force vector is $\textbf{F}=\langle |\textbf{F}|\cos\theta, |\textbf{F}|\sin\theta\rangle$.

\begin{example}
    A child pulls a wagon with a force of $|\textbf{F}| = 20$ lb at an angle of $\theta = 30\degree$ to the horizontal. Find the force vector $\textbf{F}$.

    $\textbf{F}=\langle 20\cos 30\degree, 20\sin 30\degree\rangle = \langle 10\sqrt{3}, 10\rangle$.
\end{example}

\ex A 400-lb engine is suspended from two chains that form $60\degree$ angles with a horizontal ceiling. How much weight does each chain support?

\ex If $\textbf{u}=\langle u_1,u_2\rangle$ and $\textbf{v}=\langle v_1,v_2\rangle$, how do you find $\textbf{u}+\textbf{v}$?

\ex Find the vector $\textbf{v}$ that has a magnitude of 10 and a direction opposite that of the unit vector $\langle 3/5, -4/5\rangle$.

\ex Find the position vector $\vec{OP}$ if $O$ is the origin and $P$ has polar coordinates $(8, 5\pi/6)$.

\ex Determine the necessary air speed and heading that a pilot must maintain in order to fly her commercial jet north at a speed of 480 mi/hr relative to the ground in a crosswind that is blowing $60\degree$ south of east at 20 mi/hr.

\ex For the points $A(3,4), B(6,10), C(a+2, b+5)$, and $D(b+4,a-2)$, find the values of $a$ and $b$ such that $\vec{AB}=\vec{CD}$.

\ex An ant walks due east at a constant speed of 2 mi/hr on a sheet of paper that rests on a table. Suddenly, the sheet of paper starts moving southeast at $\sqrt{2}$ mi/hr. Describe the motion of the ant relative to the table.

\ex Prove that $|c\textbf{v}|=|c||\textbf{v}|$, where $c$ is a scalar and $\textbf{v}$ is a vector.

\section{Vectors in Three Dimensions}
We can create a z-axis to create a three dimensional system.

The $xyz$-plane is divided into octants and has 3 planes, the $xy$-plane, the $xz$-plane, and the $yz$-plane. 

We can also extend the distance formula to 3 dimensions. It is similar to the distance 
formula in two dimensions, with the z-component added, essentially:
\[|PQ|=\sqrt{|PR|^2+|RQ|^2}=\sqrt{(x_2-x_1)^2+(y_2-y_1)^2+(z_2-z_1)^2}\]

The midpoint formula works the same way:
\[\text{Midpoint}=\left(\frac{x_1+x_2}{2}+\frac{y_1+y_2}{2}+\frac{z_1+z_2}{2}\right)\]

The normal form of the circle equation is: 
\[(x-h)^2+(y-k)^2=r^2\]

For a disk we have 
\[(x-h)^2+(y-k)^2\leq r^2\]

We can generalize this to a sphere. A sphere centered at $(a,b,c)$ with radius $r$ is the set of points satisfying:
\[(x-a)^2+(y-b)^2+(z-c)^2=r^2\]

A ball centered at $(a,b,c)$ with radius $r$ is the set of points satisfying:
\[(x-a)^2+(y-b)^2+(z-c)^2\leq r^2\]

\begin{example}
Find an equation of the sphere passing through $P(-4,2,3)$ and $Q(0,2,7)$ with its center at the midpoint of $PQ$.

We can find the midpoint from the midpoint formula and it is equal to $(-2,2,5)$. 

The radius can be found through the distance formula and is equal to $\sqrt{8}$.

The equation is $(x+2)^2+(y-2)^2+(z-5)^2=8$
\end{example}

All the vector operations from two-dimensions work in three-dimensions. 
\begin{example}
A model airplane is flying horizontally due east at 10 mi/hr when it encounters a horizontal 
crosswind blowing south at 5 mi/hr and an updraft blowing vertically upward at 5 mi/hr. 
\begin{itemize}
    \item Find the position vector that represents the velocity of the plane relative to the ground.
    \item Find the speed of the plane relative to the ground.
\end{itemize}
The velocity vector of the model plane $\vec{p}$ is equal to $\langle 10,0,0 \rangle$

The velocity vector of the horizontal crosswind $\vec{w}$ is equal to $\langle 0,-5,0\rangle$

The velocity vector of the updraft $\vec{u}$ is $\langle 0,0,5\rangle$

Adding the three vectors results in the speed: $\langle 10,-5,5\rangle$

The magnitude of this is roughly 12.25.
\end{example}

\ex A plane is flying horizontally due north in calm air at 300 mi/hr when it encounters a horizontal crosswind blowing southeast at 40 mi/hr and a downdraft blowing vertically downward at 30 mi/hr. 
What are the resulting speed and direction of the plane relative to the ground?

\ex Describe the set of poinst that satisfy the equation $x^2+y^2+z^2-2x+6y-8z=-1$.

\section{Dot Products}
The dot product is used to determine the angle between two vectors. It is also a tool for calculating projections - the measure of how much of a given vector lies in the direction of another vector.

\begin{definition}
    Given two nonzero vectors \textbf{u} and \textbf{v} in two or three dimensions, the dot product is
    \[\textbf{u}\cdot\textbf{v}=|\textbf{u}||\textbf{v}|\cos \theta\]
\end{definition}

The dot product is 0 when $\theta=\frac{\pi}{2}$, negative when $\theta>\frac{\pi}{2}$ and positive when $\theta<\frac{\pi}{2}$.

Two vectors are parallel if and only if \textbf{u}$\cdot$\textbf{v}=$\pm|\textbf{u}||\textbf{v}|$ 

When the dot product is zero, we call $\vec{u}$ and $\vec{v}$ orthogonal.
\begin{theorem}[Dot Product]
    Given two vectors $\textbf{u}=\langle u_1, u_2, u_3 \rangle$ and $\textbf{v}=\langle v_1, v_2, v_3 \rangle$,
    \[\textbf{u}\cdot\textbf{v}=u_1v_1+u_2v_2+u_3v_3\]
\end{theorem}

We now apply the dot product to vector projections.
\begin{definition}
    The orthogonal projection of \textbf{u} on \textbf{v}, denoted proj$_{\textbf{v}}\textbf{u}$ is:
    \[\text{proj}_{\textbf{v}}\textbf{u}=|\textbf{u}|\cos\theta\left(\frac{\textbf{v}}{|\textbf{v}|}\right)\]

    The orthogonal projections can also be computed with the formulas:
    \[\text{proj}_{\textbf{v}}\textbf{u}=\text{scal}_{\textbf{v}}\textbf{u}\left(\frac{\textbf{v}}{|\textbf{v}|}\right)=
    \left(\frac{\textbf{u}\cdot\textbf{v}}{\textbf{v}\cdot\textbf{v}}\right)\textbf{v}\]
    
    where the scalar component of \textbf{u} in the direction of \textbf{v} is 
    \[\text{scal}_{\textbf{v}}\textbf{u}=|\textbf{u}|\cos\theta = \frac{\textbf{u}\cdot\textbf{v}}{|\textbf{v}|}\]
\end{definition}



\section{Cross Products}
\begin{definition}
    Given two vectors $\textbf{u}$ and $\textbf{v}$ in $\mathbb{R}^3$, the cross product $\textbf{u}\times\textbf{v}$ 
    is a vector with magnitude
        \[|\textbf{u}\times\textbf{v}|=|\textbf{u}||\textbf{v}|\sin\theta\]
\end{definition}

Note that $\vec{u}\times\vec{v}=-(\vec{v}\times\vec{u})$

There are some useful properties of the cross product.
\begin{itemize}
    \item The cross product $\textbf{u}\times\textbf{v}$ is orthogonal to both $\vec{u}$ and $\vec{v}$
    \item The cross product is zero when $\sin(\theta)=0$.
    \item Two vectors are parallel if the cross product between them is zero.
\end{itemize}

We define the determinant of a $2\times 2$ array
$\begin{vmatrix}
a & b\\
c & d   
\end{vmatrix}$
as $ad-bc$.

For a matrix $\begin{vmatrix}
a & b & c\\
u_1 & u_2 & u_3\\
v_1 & v_2 & v_3
\end{vmatrix}$

the determinant is $a\begin{vmatrix}
    u_2 & u_3\\
    v_2 & v_3
\end{vmatrix}-b\begin{vmatrix}
    u_1 & u_3\\
    v_1 & v_3
\end{vmatrix}
+c\begin{vmatrix}
    u_1 & u_2\\
    v_1 & v_2
\end{vmatrix}$ 
or:
\[a(u_2v_3-u_3v_2)-b(u_1v_3-u_3v_1)+c(u_1v_2-u_2v_1)\]

\section{Lines and Planes in Space}
Recall that in two dimensions, we needed a point and a slope to write an equation for a line.

We can write an equation in three dimensions as well.

A vector equation of a line passing through the point $P_0(x_0,y_0,z_0)$ in the direction of vector 
$\textbf{v}=\langle a,b,c \rangle$ is $\textbf{r}=\textbf{r}_0+t\textbf{v}$ or:
\[\langle x,y,z \rangle = \langle x_0,y_0,z_0 \rangle + t\langle a,b,c \rangle\]

The corresponding parametric equations of the line also are:
\[x=x_0+at, y=y_0+bt, z=z_0+ct\]

The general equation of a plane in $\mathrm{R}^3$ with the plane passing through 
$P_0(x_0,y_0,z_0)$ with vector $\textbf{v}=\langle a,b,c\rangle$ is described by:
\[a(x-x_0)+b(y-y_0)+c(z-z_0)=0\]
or 
\[ax+by+cz=d\] where $d=ax_0+by_0+cz_0$.

\section{Cylinders and Quadric Surfaces}
A cylinder is a surface that is parallel to a line.

A trace of a surface is the set of points at which the surface intersects a plane that is 
parallel to one of the coordinate planes. The traces in the coordinate planes are called the 
$xy$-trace, the $yz$-trace, and the $xz$-trace.

To sketch quadric surfaces:
\begin{itemize}
    \item Determine the points where the surface intersects the coordinate axes. 
    \item Finding traces of the surface helps visualize the surface
    \item Sketch at least two traces in parallel planes
\end{itemize}

\begin{example}
    For:
    \begin{align*}
        \frac{x^2}{9}+\frac{y^2}{16}+\frac{z^2}{25}=1
    \end{align*}
    We set certain variables to zero to find the $x$-intercept to be $\pm 3$, the $y$-intercept 
    to be $\pm 4$ and the $z$-intercept to be $\pm 5$.

    We can also find the traces:
    \begin{itemize}
        \item $xy$: $\frac{x^2}{9}+\frac{y^2}{16}=1$
        \item $xz$: $\frac{x^2}{9}+\frac{z^2}{25}=1$
        \item $yz$: $\frac{y^2}{16}+\frac{z^2}{25}=1$
    \end{itemize}

    The resulting shape is a ellipsoid.
\end{example}

In general the equation of an ellipsoid is:
\[\frac{x^2}{a^2}+\frac{y^2}{b^2}+\frac{z^2}{c^2}=1\]
In this all traces are ellipses.

For an elliptic cone the equation is:
\[\frac{x^2}{a^2}+\frac{y^2}{b^2}=\frac{z^2}{c^2}\]
Traces with $z = z_0 \neq 0$ are ellipses. Traces with $x=x_0$ or $y=y_0$ are hyperbolas or intersecting lines.

For an elliptic paraboloid the equation is:
\[z=\frac{x^2}{a^2}+\frac{y^2}{b^2}\]
Traces with $z=z_0>0$ are ellipses. Traces with $x=x_0$ or $y=y_0$ are parabolas.

For a hyperbolic paraboloid:
\[z=\frac{x^2}{a^2}-\frac{y^2}{b^2}\]
Traces with $z=z_0\neq 0$ are hyperbolas. Traces with $x=x_0$ or $y=y_0$ are parabolas.

For a hyperboloid of one sheet:
\[\frac{x^2}{a^2}+\frac{y^2}{b^2}-\frac{z^2}{c^2}=1\]
Traces with $z=z_0$ are ellipses for all $z_0$. Traces with $x=x_0$ or $y=y_0$ are hyperbolas.

For a hyperboloid of two sheets:
\[-\frac{x^2}{a^2}-\frac{y^2}{b^2}+\frac{z^2}{c^2}=1\]
Traces with $z=z_0$ with $|z_0|>|c|$ are ellipses. Traces with $x=x_0$ and $y=y_0$ are hyperbolas.
\end{document}