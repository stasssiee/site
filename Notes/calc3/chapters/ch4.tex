\documentclass[../calc3.tex]{subfiles}
\graphicspath{{\subfix{../figures/}}}
\begin{document}
\chapter{Multiple Integrals}
\section{Double Integrals over Rectangles}
Recall that $\int_a^b f(x)dx$ gives the area under the curve from $x=a$ to $x=b$.

Also, $\int_a^b f(x)dx=\lim_{x\to\infty}\sum_{k=1}^n f(x_k^i)\Delta x_k$ (Riemann sums)

Now we have volume.

Volume Problem: Given a function of $2$ variables that is continuous and non-negative on a region $R$ in the $xy$-plane, find the volume of the solid enclosed between the surface $z=f(x,y)$ and the $xy$-plane.

Plan:
\begin{enumerate}
    \item Partition $R$ in rectangles.
    \item Choose a point $(x_k^*, y_k^*)$ in each rectangle.
    \item Map onto $z$.
    \item Form parallelpiped.
\end{enumerate}

We can then approximate the volume using rectangular parallelpipeds.

Volume $\approx$ area of rectangle $\times$ height $\rightarrow$ Volume $\approx \Delta A_k f(x_k^*, y_k^*)$.

Volume $\approx \Delta A_k f(x_k^*, y_k^*)$.

Volume $=\lim_{n\to\infty}\sum_{k=1}^n f(x_k^*, y_k^*)\Delta A_k$ (this is the formal definition for the volume problem using Riemann sums)

If $f$ has both positive and negative values, then the volume is the difference in volumes between $R$ and the surface above and below the $xy$-plane.

\begin{definition}
    If $f(x,y)\geq 0$, then the volume of the solid that lies above the rectangle $R$ and below the surface $z=f(x,y)$ is:
    \[ V = \lim_{n\to\infty}\sum_{k=1}^{n}f(x_k^*, y_k^*)\Delta A_k = \lim_{m,n\to \infty}\sum_{i=1}^m\sum_{j=1}^nf(x_k^*,y_k^*)\Delta A_k \]
    \[ = \iint_R f(x,y)dA \]
\end{definition}

\begin{example}
    Estimate the volume of the solid that lies above the square $R=[0,2]\times [0,2]$ and below $z=16-x^2-2y^2$. Divide $R$ into $4$ equal squares and choose the sample point to be the upper right corner of each square.

    So $V=\iint_R (16-x^2-2y^2)dA$.

    This is $\lim_{n\to\infty}\sum_{k=1}^n f(x_k^*,y_k^*)\Delta A_k$. We will use approximately $4$ parallelpipeds.

    We get $V\approx 1\cdot f(1,1)+1\cdot f(2,1)+1\cdot f(1,2)+1\cdot f(2,2)$.

    This is equal to $V\approx 34$ u$^3$. This is an estimate.
\end{example}

\begin{example}
    If $R=\{ (x,y)|-1\leq x\leq 1, -2\leq y\leq 2\}$, evaluate $\iint_R \sqrt{1-x^2}dA$.

    The graph will look like half of a cylinder.

    The volume of a cylinder is $\pi r^2 h$.

    $V=\frac{1}{2}\pi r^2 h=\frac{1}{2}\pi (1)^2(4)$, so $\iint_R dA=2\pi$.
\end{example}

Midpoint Rule - This tells us to evaluate $\iint_R f(x,y)dA$ using Riemann sums and midpoints.

Evaluating Double Integrals $\rightarrow$ uses iterated integration.
\[ \int_c^d \int_a^b f(x,y)dx dy = \int_c^d \left[ \int_a^b f(x,y)dx\right]dy \]
This integral can be $dxdy$ or $dydx$.

\begin{example}
    Integrate $\int_0^3 \int_1^2 x^2y dy dx$.

    First integrate $\int_1^2 x^2ydy$.

    This is $x^2\cdot \frac{1}{2}y^2$ from $y=1$ to $y=2$.

    And then we have $\int_0^3 \left(\frac{1}{2}x^2\cdot 2^2-\frac{1}{2}x^2\cdot 1^2\right)dx$.

    This is $\int_0^3 \frac{3}{2}x^2 dx = \frac{1}{2}x^3$ from $x=0$ to $x=3$, and the answer is $\frac{27}{2}$.
\end{example}

\begin{example}
    Evaluate $\int_1^2 \int_0^3 x^2 ydxdy$.

    First evaluate $\int_0^3 x^2y dx$ to get $y\cdot \frac{1}{3}x^3$ from $x=0$ to $x=3$.

    Then we have $\int_1^2 9y dy$ and this is $\frac{9}{2}y^2$ from $y=1$ to $y=2$, so the result of the integral is $\frac{27}{2}$.
\end{example}

$\frac{27}{2}$ in both cases is the area underneath $f(x,y)=x^2y$ and above $R:[0,3]\times [1,2]$. Note that $dx$ and $dy$ are not always interchangable.

\begin{theorem}[Fubini's Theorem]
    Let $R$ be a rectangular region defined by $a\leq x\leq b$, $c\leq y\leq d$. If $f(x,y)$ is continuous on this rectangle, then:
    \[ \iint_R f(x,y)dA = \int_c^d \int_a^b f(x,y)dxdy = \int_a^b \int_c^d f(x,y)dydx \]
\end{theorem}

\begin{example}
    Evaluate $\iint_R (x-3y^2)dA$ with $R=[0,2]\times [1,2]$.

    The integral is $\int_0^2 \int_1^2 (x-3y^2)dydx$.

    We start with the inner integral and get $xy-y^3$ from $y=1$ to $y=2$.

    Then we evaluate $\int_0^2 (2x-8)-(1x-1)dx$ to get $\int_0^2 (x-7)dx$.

    The result of this integral is $-12$.
\end{example}

A few properties:
\begin{enumerate}
    \item $\iint_R cf(x,y)dA = c\iint_R f(x,y)dA$
    \item $\iint_R [f(x,y)\pm g(x,y)]dA = \iint_R f(x,y)dA \pm \iint_R g(x,y)dA$
    \item $\iint_R f(x,y)dA = \iint_{R_1}f(x,y)dA + \iint_{R_2}f(x,y)dA$
    \item $\iint_R g(x)h(y)dA = \int_a^b g(x)\int_c^d h(y)dy$ where $R=[a,b]\times [c,d]$.
\end{enumerate}

\begin{example}
    Evaluate $\iint_R \sin x\cos y dA$ where $R=\left[0,\frac{\pi}{2}\right]\times [0,\pi]$.

    We can use the fourth property above to get $\int_0^{\pi/2}\sin xdx\cdot \int_0^{\pi}\cos y dy$.

    We get $\cos x$ from $0$ to $\pi/2$ and subtract $\sin y$ from $0$ to $\pi$ from this.

    The answer is $0$.
\end{example}

\begin{example}
    Find the volume of the solid that is bounded above by $f(x,y)=y\sin(xy)$ and below by $R=[1,2]\times [0,\pi]$.

    $V=\int_0^{\pi}\int_1^2 y\sin(xy)dxdy=\int_1^2 \int_0^{\pi}y\sin(y)dydx$.

    The first option is better.

    So we start with $y\cdot -\frac{1}{y}\cos(xy)$ from $x=1$ to $x=2$ 

    Then, $\int_0^{\pi}(-\cos 2y+\cos y)dy = 0$.
\end{example}

Average Value: $f_{avg}=\frac{1}{A(R)}\iint_R f(x,y)dA$

\begin{example}
    Find the average value of $f(x,y)=x^2y$ over $R$ with vertices $(-1,0)$, $(-1,5)$, $(1,5)$, and $(1,0)$.

    The area of the rectangle is $A_R=10$.

    Set up the integral to get $f_{avg}=\frac{1}{10}\int_{-1}^1 \int_0^5 x^2y dydx$/
\end{example}

\section{Double Integrals over General Regions}
There are $2$ Types of regions:

The biggest question is finding the limits of integration.

So if we have $y$ in terms of $x$, then we use $\iint_D f(x,y)dA = \iint_{g_1(x)}^{g_2(x)}f(x,y)dydx$.

If we have $x$ in terms of $y$, then $\iint_D f(x,y)dA=\iint_{h_1(y)}^{h_2(y)} f(x,y)dxdy$

\begin{example}
    Evaluate $\iint_D (x+2y)dA$ where $D$ is the region bounded by $y=2x^2$ and $y=1+x^2$.

    We have both equations being $y=$ something, so we are integrating with respect to $y$ first in this case. Though the reason for this is mostly because we are integrating vertically.

    So the integration is $\int_{-1}^1 \int_{2x^2}^{1+x^2}(x+2y)dydx$.

    We then get $xy+y^2$ with limits of integration $y=2x^2$ to $y=1+x^2$.

    So this results in $\int_{-1}^1 [x(1+x^2)+(1+x^2)^2]-[x(2x^2)+(2x^2)^2]dx = \int_{-1}^1 (-3x^4-x^3+2x^2+x+1)dx = \frac{32}{15}$.
\end{example}

\begin{example}
    Set up only! Evaluate $\iint_R xydA$ where $R$ is the region bounded by $y=-x+1, y=x+1$, and $y=3$.

    Draw to see the region. We can see from the graph that integrating by $x$ first is probably the better idea, because $y$ would require $2$ double integrals.

    So setting up the integral gives $\int_1^3 \int_{1-y}^{y-1}xy dxdy$. (Setting the equations in terms of $x=$)
\end{example}

\begin{example}
    Find the volume of the solid that lies under $z=xy$ and above $D$ where $D$ is the region bounded by $y=x-1$ and $y^2=2x+6$.

    By drawing this, we see we will go by $dx$ first.

    Setting up the integral is $\int_{-2}^4 \int_{\frac{1}{2}y^2-3}^{y+1}xydx dy$.

    Starting the integration, we get $\frac{1}{2}x^2y$ with the limits of the first integral.

    This gives $\frac{1}{2}\int_{-2}^4 (y+1)^2\cdot y-\left(\frac{1}{2}y^2-3\right)^2\cdot y dy$.

    Simplifying some more gives $\frac{1}{2}\int_{-2}^4 -\frac{1}{4}y^5+4y^3+2y^2-8y dy$ and this gives $36$ as an answer.
\end{example}

\begin{example}
    $\int_0^1 \int_x^1 \sin(y^2)dy dx$

    If we draw the region, we have $y=1$ and $y=x$.

    We go from $x=0$ to $x=y$ and for the $y$ limits, we go from $0$ to $1$.

    Therefore the integral is $\int_0^1 \int_0^y \sin(y^2)dxdy$.

    Solving this integral gives $-\frac{1}{2}\cos(1)+\frac{1}{2}$.
\end{example}

\begin{example}
    Use a double integral to find the area of the region enclosed between $y=x^3$ and $y=2x$ in the first quadrant.

    Set up the integral $A=\int_0^{\sqrt{2}}\int_{x^3}^2x$ to get the area.
\end{example}

\ex Evaluate by reversing the order of integration: $\int_0^2 \int_{y/2}^1 e^{x^2}dxdy$.

\section{Double Integrals in Polar Coordinates}
Recall that polar coordinates are in form $(r,\theta)$ and rectangular coordinates are in $(x,y)$.

In polar, for the unit circle, we can write $0\leq r\leq 1$ or $0\leq \theta \leq 2\pi$.

We have $3$ equations for converting: 
\begin{itemize}
    \item $r^2=x^2+y^2$
    \item $x=r\cos\theta$
    \item $y=r\sin\theta$
\end{itemize}

To find the volume, the process is similar to the Riemann sum process for double integrals.
\[ V = \lim_{n\to\infty}\sum_{k=1}^n f(r_k^*, \theta_k^*)\Delta A_k = \iint f(r,\theta)dA \]

In the above, $\Delta A_k$ represents the area of the polar rectangle. 

Now we need to find the area of a polar rectangle.

First we know that the area of a sector is $\frac{1}{2}r^2\theta$ and that $\Delta A_k$ is the large sector minus the little sector.

Therefore we have $\frac{1}{2}\left(r_k^*+\frac{1}{2}\Delta r_k\right)^2\Delta \theta_k - \frac{1}{2}\left(r_k^* - \frac{1}{2}\Delta r_k\right)^2\Delta \theta k$.

And this simplifies to $r_k^* \Delta r_k \Delta \theta_k$, so $\Delta A_k = r_k^*\Delta r_k\Delta \theta_k$.

So, $V=\iint_R f(r,\theta)dA=\lim_{n\to\infty}\sum_{k=1}^n f(r_k^*, \theta_k^*)r_k^* \Delta r_k\Delta \theta_k$.

So we have 
\[ V = \iint_R f(r,\theta)rdrd\theta\]

\begin{example}
    Find $\iint_R \sin\theta dA$ where $R$ is the region outside the circle $r=2$ and inside $r=2+2\cos\theta$ in the $1$st quadrant.

    We see that the circle will be hit first, then the other polar curve.

    So the integral is $\int_0^{\pi/2}\int_2^{2+2\cos\theta}\sin\theta\cdot rdrd\theta$.

    Note the outer integral goes to $\frac{\pi}{2}$ because that is the first quadrant.

    So the inner integral becomes $\sin\theta \cdot \frac{1}{2}r^2$ from the bounds $r=2$ to $r=2+2\cos\theta$.

    We then integrate $\frac{1}{2}\int_0^{\pi/2}\sin\theta [(2+2\cos\theta)^2-2^2]d\theta$.

    Solving this gives $8/3$.
\end{example}

\begin{example}
    Find the volume of the solid bounded by $z=0$ and $z=1-x^2-y^2$.

    The graph of $z=1-x^2-y^2$ will be a paraboloid.

    So $R$ is a circle with radius $1$ when we draw this paraboloid.

    We could integrate as $V=\int_{-1}^1 \int_{-\sqrt{1-x^2}}^{\sqrt{1-x^2}}(1-x^2-y^2)dydx$, or we could convert to polar.

    In polar, we know $R$ is a circle and then we can convert to polar to get $\int_0^{2\pi}\int_0^1 (1-r^2)\cdot r drd\theta$, which is equal to $V=\frac{\pi}{2}$.
\end{example}

Area is the same as before, recall this was $A=\iint_R 1\cdot dA$, and now it is $\iint_R r dr d\theta$.

\begin{example}
    Use a double integral to find the area enclosed by one loop of the four-leaf rose of $r=\cos2\theta$.

    If you know how to draw this, then we can find the area $A=\int_{-\pi/4}^{\pi/4}\int_0^{\cos 2\theta}rdrd\theta$, and this integral is simple to solve, the answer is $\pi/8$.
\end{example}

\begin{example}
    Find the volume of the solid that lies under $z=x^2+y^2$, above the $xy$-plane, and inside $x^2+y^2=2x$.

    The graph $x^2+y^2=2x$ can be rearranged to complete the square. We have then $(x-1)^2+y^2=1$.

    So if we were to convert $x^2+y^2$ to polar, we get $r^2$.

    We have $r^2=2r\cos\theta$ and $r=2\cos\theta$.

    The integral then we get $\int_{-\pi/2}^{\pi/2} \int_0^{2\cos\theta} (r^2)\cdot rdrd\theta$.

    The integral is $V=\frac{3\pi}{2}$
\end{example}

\section{Surface Area}
The formula for surface area is 
\[ A(S)=\lim_{n\to 0}\sum_{k=1}^n \sqrt{(z_x)^2+(z_y)^2+1}\Delta A\]
which becomes 
\[ A(S)=\iint_R \sqrt{(z_x)^2+(z_y)^2+1}dA \]

\begin{example}
    Find the surface area of the part of the surface $z=x^2+2y$ that lies above the triangular region with vertices $(0,0), (1,0)$, and $(1,1)$.

    $z=x^2+2y$, $z_x=2x$ and $z_y=2$.

    So $A(S)=\int_0^1 \int_0^x \sqrt{(2x)^2+(2)^2+1} dydx$.

    This integral gievs $\frac{1}{12}(27-5\sqrt{5})$.
\end{example}

\begin{example}
    Find the surface area of the portion of $z=x^2+y^2$ below the plane $z=9$.

    Paraboloid!

    The integral is $A(S)=\iint_R \sqrt{(2x)^2+(2y)^2+1}dA$.

    We might be able to see that simplifying this gives $4x^2+4y^2+1$ inside the square root, and we have an $x^2+y^2=9$ in the paraboloid.

    We should use polar.

    So we now have $\int_0^{2\pi}\int_0^3 \sqrt{4r^2+1}\cdot r dr d\theta$.

    This gives you $\frac{\pi}{6}(37^{3/2}-1)$.
\end{example}


\begin{example}
    Find the surface area of $z=\sqrt{4-x^2}$ above $R: [0,1]\times [0,4]$.

    Setting up the integral gives $\iint_R \sqrt{\left(-\frac{x}{\sqrt{4-x^2}}\right)^2+0^2+1}$.

    It doesn't really matter the way we integrate, so we get $\int_0^1 \int_0^4 \sqrt{\frac{x^2}{4-x^2}+\frac{4-x^2}{4-x^2}}dydx$.

    This simplifies to $\int_0^1 \frac{8}{\sqrt{4-x^2}}dx$.

    We notice that this becomes $8\sin^{-1}\left(\frac{x}{2}\right)$ with bounds $0$ to $1$, and this gives $\frac{4\pi}{3}$.
\end{example}

\section{Triple Integrals}

\section{Triple Integrals in Cylindrical Coordinates}

\section{Triple Integrals in Spherical Coordinates}

\section{Change of Variables in Multiple Integrals}

\end{document}