\documentclass[../calc3.tex]{subfiles}
\graphicspath{{\subfix{../figures/}}}
\begin{document}
\chapter{Multiple Integrals}
\section{Double Integrals over Rectangles}
Recall that $\int_a^b f(x)dx$ gives the area under the curve from $x=a$ to $x=b$.

Also, $\int_a^b f(x)dx=\lim_{x\to\infty}\sum_{k=1}^n f(x_k^i)\Delta x_k$ (Riemann sums)

Now we have volume.

Volume Problem: Given a function of $2$ variables that is continuous and non-negative on a region $R$ in the $xy$-plane, find the volume of the solid enclosed between the surface $z=f(x,y)$ and the $xy$-plane.

Plan:
\begin{enumerate}
    \item Partition $R$ in rectangles.
    \item Choose a point $(x_k^*, y_k^*)$ in each rectangle.
    \item Map onto $z$.
    \item Form parallelpiped.
\end{enumerate}

We can then approximate the volume using rectangular parallelpipeds.

Volume $\approx$ area of rectangle $\times$ height $\rightarrow$ Volume $\approx \Delta A_k f(x_k^*, y_k^*)$.

Volume $\approx \Delta A_k f(x_k^*, y_k^*)$.

Volume $=\lim_{n\to\infty}\sum_{k=1}^n f(x_k^*, y_k^*)\Delta A_k$ (this is the formal definition for the volume problem using Riemann sums)

If $f$ has both positive and negative values, then the volume is the difference in volumes between $R$ and the surface above and below the $xy$-plane.

\begin{definition}
    If $f(x,y)\geq 0$, then the volume of the solid that lies above the rectangle $R$ and below the surface $z=f(x,y)$ is:
    \[ V = \lim_{n\to\infty}\sum_{k=1}^{n}f(x_k^*, y_k^*)\Delta A_k = \lim_{m,n\to \infty}\sum_{i=1}^m\sum_{j=1}^nf(x_k^*,y_k^*)\Delta A_k \]
    \[ = \iint\limits_{R} f(x,y)dA \]
\end{definition}

\begin{example}
    Estimate the volume of the solid that lies above the square $R=[0,2]\times [0,2]$ and below $z=16-x^2-2y^2$. Divide $R$ into $4$ equal squares and choose the sample point to be the upper right corner of each square.

    So $V=\iint\limits_{R} (16-x^2-2y^2)dA$.

    This is $\lim_{n\to\infty}\sum_{k=1}^n f(x_k^*,y_k^*)\Delta A_k$. We will use approximately $4$ parallelpipeds.

    We get $V\approx 1\cdot f(1,1)+1\cdot f(2,1)+1\cdot f(1,2)+1\cdot f(2,2)$.

    This is equal to $V\approx 34$ u$^3$. This is an estimate.
\end{example}

\begin{example}
    If $R=\{ (x,y)|-1\leq x\leq 1, -2\leq y\leq 2\}$, evaluate $\iint\limits_{R} \sqrt{1-x^2}dA$.

    The graph will look like half of a cylinder.

    The volume of a cylinder is $\pi r^2 h$.

    $V=\frac{1}{2}\pi r^2 h=\frac{1}{2}\pi (1)^2(4)$, so $\iint\limits_{R} dA=2\pi$.
\end{example}

Midpoint Rule - This tells us to evaluate $\iint\limits_{R} f(x,y)dA$ using Riemann sums and midpoints.

Evaluating Double Integrals $\rightarrow$ uses iterated integration.
\[ \int_c^d \int_a^b f(x,y)dx dy = \int_c^d \left[ \int_a^b f(x,y)dx\right]dy \]
This integral can be $dxdy$ or $dydx$.

\begin{example}
    Integrate $\int_0^3 \int_1^2 x^2y dy dx$.

    First integrate $\int_1^2 x^2ydy$.

    This is $x^2\cdot \frac{1}{2}y^2$ from $y=1$ to $y=2$.

    And then we have $\int_0^3 \left(\frac{1}{2}x^2\cdot 2^2-\frac{1}{2}x^2\cdot 1^2\right)dx$.

    This is $\int_0^3 \frac{3}{2}x^2 dx = \frac{1}{2}x^3$ from $x=0$ to $x=3$, and the answer is $\frac{27}{2}$.
\end{example}

\begin{example}
    Evaluate $\int_1^2 \int_0^3 x^2 ydxdy$.

    First evaluate $\int_0^3 x^2y dx$ to get $y\cdot \frac{1}{3}x^3$ from $x=0$ to $x=3$.

    Then we have $\int_1^2 9y dy$ and this is $\frac{9}{2}y^2$ from $y=1$ to $y=2$, so the result of the integral is $\frac{27}{2}$.
\end{example}

$\frac{27}{2}$ in both cases is the area underneath $f(x,y)=x^2y$ and above $R:[0,3]\times [1,2]$. Note that $dx$ and $dy$ are not always interchangable.

\pagebreak
\begin{theorem}[Fubini's Theorem]
    Let $R$ be a rectangular region defined by $a\leq x\leq b$, $c\leq y\leq d$. If $f(x,y)$ is continuous on this rectangle, then:
    \[ \iint\limits_{R} f(x,y)dA = \int_c^d \int_a^b f(x,y)dxdy = \int_a^b \int_c^d f(x,y)dydx \]
\end{theorem}

\begin{example}
    Evaluate $\iint\limits_{R} (x-3y^2)dA$ with $R=[0,2]\times [1,2]$.

    The integral is $\int_0^2 \int_1^2 (x-3y^2)dydx$.

    We start with the inner integral and get $xy-y^3$ from $y=1$ to $y=2$.

    Then we evaluate $\int_0^2 (2x-8)-(1x-1)dx$ to get $\int_0^2 (x-7)dx$.

    The result of this integral is $-12$.
\end{example}

A few properties:
\begin{enumerate}
    \item $\iint\limits_{R} cf(x,y)dA = c\iint\limits_{R} f(x,y)dA$
    \item $\iint\limits_{R} [f(x,y)\pm g(x,y)]dA = \iint\limits_{R} f(x,y)dA \pm \iint\limits_{R} g(x,y)dA$
    \item $\iint\limits_{R} f(x,y)dA = \iint_{R_1}f(x,y)dA + \iint_{R_2}f(x,y)dA$
    \item $\iint\limits_{R} g(x)h(y)dA = \int_a^b g(x)\int_c^d h(y)dy$ where $R=[a,b]\times [c,d]$.
\end{enumerate}

\begin{example}
    Evaluate $\iint\limits_{R} \sin x\cos y dA$ where $R=\left[0,\frac{\pi}{2}\right]\times [0,\pi]$.

    We can use the fourth property above to get $\int_0^{\pi/2}\sin xdx\cdot \int_0^{\pi}\cos y dy$.

    We get $\cos x$ from $0$ to $\pi/2$ and subtract $\sin y$ from $0$ to $\pi$ from this.

    The answer is $0$.
\end{example}

\begin{example}
    Find the volume of the solid that is bounded above by $f(x,y)=y\sin(xy)$ and below by $R=[1,2]\times [0,\pi]$.

    $V=\int_0^{\pi}\int_1^2 y\sin(xy)dxdy=\int_1^2 \int_0^{\pi}y\sin(y)dydx$.

    The first option is better.

    So we start with $y\cdot -\frac{1}{y}\cos(xy)$ from $x=1$ to $x=2$ 

    Then, $\int_0^{\pi}(-\cos 2y+\cos y)dy = 0$.
\end{example}

Average Value: $f_{avg}=\frac{1}{A(R)}\iint\limits_{R} f(x,y)dA$

\pagebreak
\begin{example}
    Find the average value of $f(x,y)=x^2y$ over $R$ with vertices $(-1,0)$, $(-1,5)$, $(1,5)$, and $(1,0)$.

    The area of the rectangle is $A_R=10$.

    Set up the integral to get $f_{avg}=\frac{1}{10}\int_{-1}^1 \int_0^5 x^2y dydx$/
\end{example}

\section{Double Integrals over General Regions}
There are $2$ Types of regions:

The biggest question is finding the limits of integration.

So if we have $y$ in terms of $x$, then we use $\iint\limits_{D} f(x,y)dA = \iint_{g_1(x)}^{g_2(x)}f(x,y)dydx$.

If we have $x$ in terms of $y$, then $\iint\limits_{D} f(x,y)dA=\iint_{h_1(y)}^{h_2(y)} f(x,y)dxdy$

\begin{example}
    Evaluate $\iint\limits_{D} (x+2y)dA$ where $D$ is the region bounded by $y=2x^2$ and $y=1+x^2$.

    We have both equations being $y=$ something, so we are integrating with respect to $y$ first in this case. Though the reason for this is mostly because we are integrating vertically.

    So the integration is $\int_{-1}^1 \int_{2x^2}^{1+x^2}(x+2y)dydx$.

    We then get $xy+y^2$ with limits of integration $y=2x^2$ to $y=1+x^2$.

    So this results in $\int_{-1}^1 [x(1+x^2)+(1+x^2)^2]-[x(2x^2)+(2x^2)^2]dx = \int_{-1}^1 (-3x^4-x^3+2x^2+x+1)dx = \frac{32}{15}$.
\end{example}

\begin{example}
    Set up only! Evaluate $\iint\limits_{R} xydA$ where $R$ is the region bounded by $y=-x+1, y=x+1$, and $y=3$.

    Draw to see the region. We can see from the graph that integrating by $x$ first is probably the better idea, because $y$ would require $2$ double integrals.

    So setting up the integral gives $\int_1^3 \int_{1-y}^{y-1}xy dxdy$. (Setting the equations in terms of $x=$)
\end{example}

\begin{example}
    Find the volume of the solid that lies under $z=xy$ and above $D$ where $D$ is the region bounded by $y=x-1$ and $y^2=2x+6$.

    By drawing this, we see we will go by $dx$ first.

    Setting up the integral is $\int_{-2}^4 \int_{\frac{1}{2}y^2-3}^{y+1}xydx dy$.

    Starting the integration, we get $\frac{1}{2}x^2y$ with the limits of the first integral.

    This gives $\frac{1}{2}\int_{-2}^4 (y+1)^2\cdot y-\left(\frac{1}{2}y^2-3\right)^2\cdot y dy$.

    Simplifying some more gives $\frac{1}{2}\int_{-2}^4 -\frac{1}{4}y^5+4y^3+2y^2-8y dy$ and this gives $36$ as an answer.
\end{example}

\pagebreak
\begin{example}
    $\int_0^1 \int_x^1 \sin(y^2)dy dx$

    If we draw the region, we have $y=1$ and $y=x$.

    We go from $x=0$ to $x=y$ and for the $y$ limits, we go from $0$ to $1$.

    Therefore the integral is $\int_0^1 \int_0^y \sin(y^2)dxdy$.

    Solving this integral gives $-\frac{1}{2}\cos(1)+\frac{1}{2}$.
\end{example}

\begin{example}
    Use a double integral to find the area of the region enclosed between $y=x^3$ and $y=2x$ in the first quadrant.

    Set up the integral $A=\int_0^{\sqrt{2}}\int_{x^3}^2x$ to get the area.
\end{example}

\ex Evaluate by reversing the order of integration: $\int_0^2 \int_{y/2}^1 e^{x^2}dxdy$.

\section{Double Integrals in Polar Coordinates}
Recall that polar coordinates are in form $(r,\theta)$ and rectangular coordinates are in $(x,y)$.

In polar, for the unit circle, we can write $0\leq r\leq 1$ or $0\leq \theta \leq 2\pi$.

We have $3$ equations for converting: 
\begin{itemize}
    \item $r^2=x^2+y^2$
    \item $x=r\cos\theta$
    \item $y=r\sin\theta$
\end{itemize}

To find the volume, the process is similar to the Riemann sum process for double integrals.
\[ V = \lim_{n\to\infty}\sum_{k=1}^n f(r_k^*, \theta_k^*)\Delta A_k = \iint f(r,\theta)dA \]

In the above, $\Delta A_k$ represents the area of the polar rectangle. 

Now we need to find the area of a polar rectangle.

First we know that the area of a sector is $\frac{1}{2}r^2\theta$ and that $\Delta A_k$ is the large sector minus the little sector.

Therefore we have $\frac{1}{2}\left(r_k^*+\frac{1}{2}\Delta r_k\right)^2\Delta \theta_k - \frac{1}{2}\left(r_k^* - \frac{1}{2}\Delta r_k\right)^2\Delta \theta k$.

And this simplifies to $r_k^* \Delta r_k \Delta \theta_k$, so $\Delta A_k = r_k^*\Delta r_k\Delta \theta_k$.

So, $V=\iint\limits_{R} f(r,\theta)dA=\lim_{n\to\infty}\sum_{k=1}^n f(r_k^*, \theta_k^*)r_k^* \Delta r_k\Delta \theta_k$.

So we have 
\[ V = \iint\limits_{R} f(r,\theta)rdrd\theta\]

\pagebreak
\begin{example}
    Find $\iint\limits_{R} \sin\theta dA$ where $R$ is the region outside the circle $r=2$ and inside $r=2+2\cos\theta$ in the $1$st quadrant.

    We see that the circle will be hit first, then the other polar curve.

    So the integral is $\int_0^{\pi/2}\int_2^{2+2\cos\theta}\sin\theta\cdot rdrd\theta$.

    Note the outer integral goes to $\frac{\pi}{2}$ because that is the first quadrant.

    So the inner integral becomes $\sin\theta \cdot \frac{1}{2}r^2$ from the bounds $r=2$ to $r=2+2\cos\theta$.

    We then integrate $\frac{1}{2}\int_0^{\pi/2}\sin\theta [(2+2\cos\theta)^2-2^2]d\theta$.

    Solving this gives $8/3$.
\end{example}

\begin{example}
    Find the volume of the solid bounded by $z=0$ and $z=1-x^2-y^2$.

    The graph of $z=1-x^2-y^2$ will be a paraboloid.

    So $R$ is a circle with radius $1$ when we draw this paraboloid.

    We could integrate as $V=\int_{-1}^1 \int_{-\sqrt{1-x^2}}^{\sqrt{1-x^2}}(1-x^2-y^2)dydx$, or we could convert to polar.

    In polar, we know $R$ is a circle and then we can convert to polar to get $\int_0^{2\pi}\int_0^1 (1-r^2)\cdot r drd\theta$, which is equal to $V=\frac{\pi}{2}$.
\end{example}

Area is the same as before, recall this was $A=\iint\limits_{R} 1\cdot dA$, and now it is $\iint\limits_{R} r dr d\theta$.

\begin{example}
    Use a double integral to find the area enclosed by one loop of the four-leaf rose of $r=\cos2\theta$.

    If you know how to draw this, then we can find the area $A=\int_{-\pi/4}^{\pi/4}\int_0^{\cos 2\theta}rdrd\theta$, and this integral is simple to solve, the answer is $\pi/8$.
\end{example}

\begin{example}
    Find the volume of the solid that lies under $z=x^2+y^2$, above the $xy$-plane, and inside $x^2+y^2=2x$.

    The graph $x^2+y^2=2x$ can be rearranged to complete the square. We have then $(x-1)^2+y^2=1$.

    So if we were to convert $x^2+y^2$ to polar, we get $r^2$.

    We have $r^2=2r\cos\theta$ and $r=2\cos\theta$.

    The integral then we get $\int_{-\pi/2}^{\pi/2} \int_0^{2\cos\theta} (r^2)\cdot rdrd\theta$.

    The integral is $V=\frac{3\pi}{2}$
\end{example}

\section{Surface Area}
The formula for surface area is 
\[ A(S)=\lim_{n\to 0}\sum_{k=1}^n \sqrt{(z_x)^2+(z_y)^2+1}\Delta A\]
which becomes 
\[ A(S)=\iint\limits_{R} \sqrt{(z_x)^2+(z_y)^2+1}dA \]

\begin{example}
    Find the surface area of the part of the surface $z=x^2+2y$ that lies above the triangular region with vertices $(0,0), (1,0)$, and $(1,1)$.

    $z=x^2+2y$, $z_x=2x$ and $z_y=2$.

    So $A(S)=\int_0^1 \int_0^x \sqrt{(2x)^2+(2)^2+1} dydx$.

    This integral gives $\frac{1}{12}(27-5\sqrt{5})$.
\end{example}

\begin{example}
    Find the surface area of the portion of $z=x^2+y^2$ below the plane $z=9$.

    Paraboloid!

    The integral is $A(S)=\iint\limits_{R} \sqrt{(2x)^2+(2y)^2+1}dA$.

    We might be able to see that simplifying this gives $4x^2+4y^2+1$ inside the square root, and we have an $x^2+y^2=9$ in the paraboloid.

    We should use polar.

    So we now have $\int_0^{2\pi}\int_0^3 \sqrt{4r^2+1}\cdot r dr d\theta$.

    This gives you $\frac{\pi}{6}(37^{3/2}-1)$.
\end{example}


\begin{example}
    Find the surface area of $z=\sqrt{4-x^2}$ above $R: [0,1]\times [0,4]$.

    Setting up the integral gives $\iint\limits_{R} \sqrt{\left(-\frac{x}{\sqrt{4-x^2}}\right)^2+0^2+1}$.

    It doesn't really matter the way we integrate, so we get $\int_0^1 \int_0^4 \sqrt{\frac{x^2}{4-x^2}+\frac{4-x^2}{4-x^2}}dydx$.

    This simplifies to $\int_0^1 \frac{8}{\sqrt{4-x^2}}dx$.

    We notice that this becomes $8\sin^{-1}\left(\frac{x}{2}\right)$ with bounds $0$ to $1$, and this gives $\frac{4\pi}{3}$.
\end{example}

\section{Triple Integrals}
So far we have 
\begin{itemize}
    \item $D$ is closed (can be contained in a rectangle)
    \item Taking limit as $n\rightarrow \infty$ gave us the volume under $z=f(x,y)$.
\end{itemize}

Now, triple integrals.
\begin{itemize}
    \item Closed solid $B$ (can be contained in a box).
    \item Divide $B$ into $n$ sub-boxes.
    \item Volume of each box is $\Delta V$ and a point in the box is $(x_{ijk}^*, y_{ijk}^*, z_{ijk}^*)$.
\end{itemize}

Then $\lim_{n\to\infty}\sum_{k=1}^n f\left(x_{ijk}^*, y_{ijk}^*, z_{ijk}^*\right)\Delta V = \iiint\limits_{B} f(x,y,z)dV$.

This gives us hypervolume.
\begin{itemize}
    \item Same properties and evaluation as before (double integrals)
    \item If $B$ is a box defined by $a\leq x\leq b$, $c\leq y\leq d$, $e\leq z\leq f$, then $\iiint\limits_{B} f(x,y,z)dV = \int_e^f \int_c^d \int_a^b f(x,y,z)dxdydz$.
\end{itemize}

\begin{example}
    Evaluate $\iiint\limits_{G} xyz^2 dV$ where $G: \{ (x,y,z)|0\leq x\leq 1,-1\leq y\leq 2, 0\leq z\leq 3\}$.

    Convention is to do $\int_0^1 \int_{-1}^2 \int_0^3 xyz^2 dz dy dx$.

    Let us start with the $z$ part to get $\int_0^1 \int_{-1}^2 \frac{1}{3}xyz^3$ from $z=0$ to $z=3$.

    Then we get $\int_0^1 \frac{9}{2}xy^2$ from $y=-1$ to $y=2$.

    And then we get $\frac{9}{2}\int_0^1 3x dx = \frac{27}{4}$.
\end{example}

If the region $B$ is rectangular and the function is a product (such as $f(x,y,z)=g(x)\cdot h(y)\cdot j(z)$), we can split up the integral.

The above integral will become: $\int_0^1 xdx \cdot \int_{-1}^2 y dy \cdot \int_0^3 z^2 dz$.

Type I Solid: $E$ is a solid with upper surface $z=u_2(x,y)$ and lower surface $z=u_1(x,y)$. $D$ is the projection of $E$ onto the $xy$-plane.

Then $\iiint\limits_{E} f(x,y,z)dV = \iint\limits_{D} \int_{u_1(x,y)}^{u_2(x,y)}f(x,y,z)dz dA$.

To find limits of integration:
\begin{enumerate}
    \item Find upper and lower surfaces bounding $E$. These are limits for $z$.
    \item Make a $2$-d sketch of the projection $D$ on $xy$-plane.
    \item Treat like usual.
\end{enumerate}

\begin{example}
    Evaluate $\iiint\limits_{T} zdV$ where $T$ is the tetrahedron bounded by $x=0$, $y=0$, $z=0$ and $x+y+z=1$.

    This is a tetrahedral in the first octant and $x+y+z=1$ is a plane.

    We have $z=1-x-y$, so $\int \int \int_0^{1-x-y}z\cdot dz dy dz$.

    And from the other bounds we get $\int_0^1 \int_0^{1-x} \int_0^{1-x-y}z\cdot dz dy dx$.

    The outer integral note should always have constants.

    The answer of this integral becomes $\frac{1}{24}$.
\end{example}

Sometimes we have lateral surfaces bounding, not the top or bottom.

Type II Solid:
\[ \iiint\limits_{E} f(x,y,z)dV = \iint\limits_{D} \left[ \int_{u_1(y,z)}^{u_2(y,z)}f(x,y,z)dx\right]dA \]

Type III Solid:
\[ \iiint\limits_{E} f(x,y,z)dV = \iint\limits_{D} \left[ \int_{u_1(x,z)}^{u_2(x,z)}f(x,y,z)dy\right] dA \]

\begin{example}
    Evaluate $\iiint\limits_{R} \sqrt{x^2+z^2}dV$ where $E$ is bounded by $y=x^2+z^2$ and $y=4$.

    $y=x^2+z^2$ is a paraboloid and $y=4$ is a plane.

    We want to start integrating from $y$ since the paraboloid goes towards the plane always.

    We project the figure now on the $xz$-plane and now get a circle with equation $x^2+z^2=4$.

    So we can write the integral now as $\int_{-2}^2 \int_{-\sqrt{4-x^2}}^{\sqrt{4-x^2}} \int_{x^2+z^2}^4 \sqrt{x^2+z^2}dy dz dx$.

    First we have the first part of the integral be equal to $\int_{-2}^2 \int_{-\sqrt{4-x^2}}^{\sqrt{4-x^2}} (4-x^2-z^2)\sqrt{x^2+z^2}dzdx$.

    In this we see that $r^2=x^2+z^2$.

    When we switch to polar we end up getting $\int_{0}^{2\pi} \int_{0}^{2} (4-r^2)r\cdot r dr d\theta$.

    This gives you $\frac{128\pi}{15}$.
\end{example}

Volume as a triple integral: $V=\iiint\limits_{E} 1\cdot dV$.

\begin{example}
    Setup an integral to find the volume of the wedge in the 1st octant that is cut from the solid cylinder $y^2+z^2\leq 1$ by the planes $y=x$ and $x=0$.

    We start with $z$ then we can see $y=x$ from the projection.

    So $V=\int_0^1 \int_0^x \int_0^{\sqrt{1-y^2}}1\cdot dz dx dy$.
\end{example}

Note it is not always the case that the order of integration can be changed without changing the bounds.

\begin{example}
    Find the volume of the solid enclosed between the paraboloids $z=5x^2+5y^2$ and $z=6-7x^2-y^2$.

    To find projection $D$ we have $5x^2+5y^2=6-7x^2-y^2$. Solving gives us $y=\pm \sqrt{1-2x^2}$ (or a circle).

    The integral becomes $V=\int_{-\frac{1}{\sqrt{2}}}^{\frac{1}{\sqrt{2}}}\int_{-\sqrt{1-2x^2}}^{\sqrt{1-2x^2}}\int_{5x^2+5y^2}^{6-7x^2-y^2}1\cdot dz dy dx$.
\end{example}

\section{Triple Integrals in Cylindrical Coordinates}
Cylindrical coordinates are like polar, but in $3$-D.

To convert cylindrical to rectangular, it is the same as polar. $x=r\cos\theta$, $y=r\sin\theta$, and the additional part is $z=z$.

We also get $r^2=x^2+y^2$, $\tan\theta = \frac{y}{x}$ and $z=z$ once again to convert rectangular to cylindrical.

\ex Plot $\left(2,\frac{2\pi}{3}, 1\right)$ and find rectangular coordinates.

\pagebreak
\begin{example}
    Find cylindrical coordinates for $(3,-3,-7)$.

    $r^2=3^2+(-3)^2$ gives $r=3\sqrt{2}$.

    $\tan\theta = -1$, so $\theta = \frac{3\pi}{4}$ or $\frac{7\pi}{4}$.

    We pick $\frac{7\pi}{4}$ for this, and then we get coordinates $(3\sqrt{2}, \frac{7\pi}{4}, -7)$.
\end{example}

\begin{example}
    Describe the surface $z=r$.

    We can rewrite this was $z=\sqrt{x^2+y^2}$ and therefore $z^2=x^2+y^2$, which describes a cone.
\end{example}

Recall from earlier: $\iiint\limits_{E} f(x,y,z)dV = \lim_{n\to\infty}\sum_{k=1}^n f\left(x_{ijk}^*, y_{ijk}^*, z_{ijk}^*\right)\Delta V_k$.

$\Delta V_k$ is the area of base times the height. From earlier, $\Delta V_k = r_k^* \Delta r_k \Delta \theta_k\cdot \Delta z_k$.

So, $\iiint\limits_{E} f(x,y,z)dV = \iiint f(r,\theta,z)\cdot r dr d\theta dz$.

To find limits of integration:
\begin{enumerate}
    \item Identify upper surface $z=g_2(r,\theta)$ and lower surface $z=g_1(r,\theta)$.
    \item Make a $2$-D sketch of projection onto $xy$-plane to determine bounds for $r$ and $\theta$.
\end{enumerate}


\begin{example}
    Evaluate $\iiint\limits_{G} dV$ where $G$ lies within $x^2+y^2=1$, below $z=4$, and above $z=1-x^2-y^2$.

    We are finding a volume.

    The graph gives a cylinder, with a hemisphere at the bottom removing a part of the cylinder (bad explanation but it's ok).

    So the integral is $V=\int_{-1}^1 \int_{-\sqrt{1-x^2}}^{\sqrt{1-x^2}} \int_{1-x^2-y^2}^4 1\cdot dz dy dx$.

    As we can see, the projection on the $xy$-plane is a circle, so cylindrical coordinates are best.

    Now we have $\int_0^{2\pi} \int_0^1 \int_{1-r^2}^4 1\cdot r\cdot dz dr d\theta$

    Integrating this gives you $V=\frac{7}{2}\pi$.
\end{example}

\begin{example}
    Convert from rectangular to cylindrical: $\int_{-2}^2 \int_{-\sqrt{4-x^2}}^{\sqrt{4-x^2}}\int_{\sqrt{x^2+y^2}}^2 (x^2+y^2)dzdydx$.

    $R$ is a circle with radius $r=2$, so the integral becomes $\int_0^{2\pi}\int_0^2 \int_r^2 r^2\cdot r dz dr d\theta$.

    The integral evaluates to $\frac{16\pi}{5}$.
\end{example}

\pagebreak
\begin{example}
    Let $E$ be the region inside the sphere of radius $2$ centered at the origin and above the plane $z=1$. Find the volume of $E$.

    THe equation of the sphere is $x^2+y^2+z^2=4$, so we have $z^2=4-r^2$.

    The integral can be $V=\int_0^{2\pi} \int_0^{\sqrt{3}} \int_1^{\sqrt{4-r^2}} 1\cdot r dz dr d\theta$.

    We have figure $D: x^2+y^2+z^2=4$ and $z=1$, so $x^2+y^2=3$.
\end{example}


\section{Triple Integrals in Spherical Coordinates}
Spherical Coordinates: $(\rho, \theta, \phi)$, where $\rho$ is the distance from the origin to the point, $\theta$ represents the same as before (the angle on the $xy$-plane from the $x$-axis), and $\phi$ represents the angle between the positive $z$-axis and point.

Bounds for $\rho,\theta$, and $\phi$:
\begin{itemize}
    \item $\rho\geq 0$
    \item $0\leq \theta\leq 2\pi$
    \item $0\leq \phi\leq \pi$
\end{itemize}

A few common graphs:
\begin{itemize}
    \item $\rho = c$ gives a sphere 
    \item $\theta = c$ gives a ``half'' plane 
    \item $\phi = c$ gives a cone ($0<c<\pi/2$ will give the top part)
\end{itemize}

Converting:
\begin{itemize}
    \item $x=\rho \sin\phi \cos\theta$
    \item $y=\rho \sin\phi \sin\theta$
    \item $z=\rho \cos\phi$
\end{itemize}

Also, $\rho^2 = x^2+y^2+z^2$.

\begin{example}
    Convert $\left(2,\frac{\pi}{4},\frac{\pi}{3}\right)$ to rectangular.

    We know $x=\rho \sin\phi \cos\theta$, so $x=2\sin\frac{\pi}{3}\cos\frac{\pi}{4}=\sqrt{\frac{3}{2}}$.

    $y=\rho\sin\phi\sin\theta$, so pluggin in gives $\sqrt{\frac{3}{2}}$.

    $z=\rho\cos\phi = 2\cos\frac{\pi}{3}=1$.

    The coordinates are $\left(\sqrt{\frac{3}{2}},\sqrt{\frac{3}{2}}, 1\right)$.
\end{example}

\pagebreak
\begin{example}
    Convert $(0,2\sqrt{3},-2)$ to spherical.

    $\rho^2 = x^2+y^2+z^2$, so $\rho=4$.

    $z=\rho\cos\phi$, and we can see that $\cos\phi = \frac{z}{\rho}$, so $\phi = \frac{2\pi}{3}$ or $\frac{4\pi}{3}$, but knowing the bounds for $\phi$ gives $\phi = \frac{2\pi}{3}$.

    We know that $\cos\theta = \frac{x}{\rho\sin\phi}$, so $\cos\theta = 0$, and $\theta = \frac{\pi}{2}$ is the only $\theta$ that works in the $xy$-plane for this.

    The point is $\left(4,\frac{\pi}{2},\frac{2\pi}{3}\right)$
\end{example}

Recall: $\iiint\limits_{E} f(x,y,z)dV = \lim_{n\to\infty}\sum_{k=1}^n f\left(x_{ijk}^*,y_{ijk}^*.z_{ijk}^*\right)\Delta V_k = \int_{\theta_1}^{\theta_2}\int_{\phi_1}^{\phi^2}\int_{\rho_1}^{\rho_2} f(\rho,\theta,\phi)\rho^2 \sin\phi d\rho d\phi d\theta$.

\begin{example}
    Evaluate $\iiint\limits_{B} e^{(x^2+y^2+z^2)^{\frac{3}{2}}}dz dy dx$ were $B$ is the unit sphere.

    If try rectangular, we would find the limits of integration are messy.

    So we use spherical.

    The integral is $\int_0^{2\pi}\int_0^{\pi}\int_0^1 e^{(\rho^2)^{3/2}}\rho^2 \sin\phi d\rho d\phi d\theta$.

    This is equal to $\int_0^{2\pi}\int_0^{\pi}\int_0^1 \rho^2 e^{\rho^3}\sin\phi d\rho d\phi d\theta$.

    Integrating this fully gives $\frac{4\pi(e-1)}{3}$.
\end{example}

\begin{example}
    Convert to spherical: $\int_{-2}^2 \int_{-\sqrt{4-x^2}}^{\sqrt{4-x^2}}\int_{0}^{\sqrt{4-x^2-y^2}}z^2 \sqrt{x^2+y^2+z^2}dzdydx$.

    We can see that the whole inside part gives $(\rho\cos\phi)^2\cdot \rho \cdot \rho^2 \sin\phi d\rho d\phi d\theta$.

    The most inner integrand is a hemisphere so bounds go from $0$ to $2$.

    The second integrand is a circle so bounds go from $0$ to $\pi/2$.

    The integral is $\int_0^{2\pi}\int_0^{\frac{\pi}{2}},\int_0^2 \rho^5 \cos^2\phi\sin\phi d\rho d\phi d\theta$.

    The answer of this integral is $\frac{64}{9}\pi$.
\end{example}


\begin{example}
    Find the volume of the ice cream cone bounded by $x^2+y^2+z^2=z$ and $z=\sqrt{x^2+y^2}$.

    Remember $V=\iiint\limits_{E} 1\cdot dV$.

    So the sphere equation can be rewritten as $x^2+y^2+\left(z-\frac{1}{2}\right)^2=\frac{1}{4}$.

    So the center of the sphere is $\left(0,0,\frac{1}{2}\right)$ with $r=\frac{1}{2}$.

    $\rho: x^2+y^2+z^2=z$, $\rho^2 = \rho\cos\phi$, $\rho=\cos\phi$.

    $\phi: z=\sqrt{x^2+y^2}$, $\rho\cos\phi = \sqrt{\rho^2\sin^2\phi\cos^2\theta+\rho^2\sin^2\phi\sin^2\theta}$, so $\cos\phi = \sin\phi$, gives $\phi = \frac{\pi}{4}$.

    So the integral becomes $V=\int_0^{2\pi}\int_0^{\pi/4}\int_0^{\cos\phi}1\cdot \rho^2 \sin\phi d\rho d\phi d\theta$.

    Solving this integral gives $V=\frac{\pi}{8}$.
\end{example}

We can split a triple integral into 3 separate integrals when all the bounds are numbers, and functions are as products of functions.


\section{Change of Variables in Multiple Integrals}
\begin{example}
    Let $T$ be the transformation from $uv$-plane to $xy$-plane defined by $x=\frac{1}{4}(u+v)$ and $y=\frac{1}{2}(u-v)$.

    (a) Find $T(1,3)$.

    $x=\frac{1}{4}(1+3)=1$ and doing the same gives $y=-1$, so $(1,-1)$.

    (b) Sketch the image under $T$ bounded by $-2\leq u\leq 2$ and $-2\leq v\leq 2$.

    The figure for $uv$-plane is a square centered at the origin with side lengths of $4$.

    The plan to draw the image on the $xy$-plane is to sketch several $u$, $v$ curves and then you need to know $u$, $v$ in terms of $x$ and $y$.

    $4x=u+v$ and $2y=u-v$, and this gives $4x+2y=2u$ and $u=2x+y$ as a result.

    We can see $u=-2$ gives $y=-2x-2$, then $y=-2x-1$ when $u=-1$, and when $u=2$, $y=-2x+2$.
    
    Similarly, $v=2x-y$, and we see a pattern here as well.
\end{example}

In a way, it is easier to integrate over a different region like mapped above.

\begin{definition}[Jacobian]
    If $x=g(u,v)$ and $y=h(u,v)$, then the Jacobian of $x$ and $y$ with respect to $u$ and $v$ is: 
    \[ \frac{\partial (x,y)}{\partial (u,v)}\begin{vmatrix}
        \frac{\partial x}{\partial u}& \frac{\partial x}{\partial v} \\
        \frac{\partial y}{\partial u}& \frac{\partial y}{\partial v}
    \end{vmatrix} \]
\end{definition}

We use this to give us the extra factor when converting integrals.

It is similar to converting in polar (or cylindrical), where you added $r$, or simliar to spherical when you added $\rho^2 \sin\phi$.

So 
\[ \iint\limits_{R} f(x,y)dA = \iint_S f(x(u,v),y(u,v))\left| \frac{\partial (x,y)}{\partial (u,v)}\right| du dv \]

\begin{example}
    Consider $x=r\cos\theta$ and $y=r\sin\theta$. Find the Jacobian.

    The determinant of this will be $\begin{vmatrix}
        \frac{\partial x}{\partial r} & \frac{\partial x}{\partial \theta}\\
        \frac{\partial y}{\partial r} & \frac{\partial y}{\partial \theta}
    \end{vmatrix} = \begin{vmatrix}
        \cos\theta & -r\sin\theta \\ \sin\theta & r\cos\theta
    \end{vmatrix} = r^2\cos^2\theta + r\sin^2\theta = r$
\end{example}

\pagebreak
\begin{example}
    Evaluate $\iint\limits_{R} 3xydA$ where $R$ is bounded by $x-2y=0$, $x-2y=-4$, $x+y=4$, and $x+y=1$.

    We let $u=x-2y$, $v=x+y$, so $u=0,-4$ and $v=4,1$. The region of this is rectangular, much easier to solve than if you were to do it based on $y$.

    For change of variables, first you need to find the Jacobian.

    We need $x(u,v)$ and $y(u,v)$.

    So we have $y=\frac{1}{3}(v-u)$ and $x=\frac{1}{3}(u+2v)$ from the values of $u$ and $v$.

    The jacobian is then $\begin{vmatrix}
        \frac{1}{3} & \frac{2}{3} \\ -\frac{1}{3} & \frac{1}{3}
    \end{vmatrix} = \frac{1}{3}$.

    Then we have to do change of variables in the integral.

    So the integral is $\int_1^4 \int_{-4}^0 3\left(\frac{1}{3}(u+2v)\right)\left(\frac{1}{3}(v-u)\right)\cdot \left|\frac{1}{3}\right|du dv$.

    This integral results in $\frac{164}{9}$.
\end{example}

Now for $3$ variables.

\begin{example}
    Find the Jacobian of $x=\rho\sin\phi \cos\theta$, $y=x=\rho\sin\phi\sin\theta$, and $z=\rho\cos\phi$.

    The Jacobian becomes $\begin{vmatrix}
        \sin\phi \cos\theta & \rho\cos\phi\cos\theta & -\rho\sin\phi\sin\theta \\
        \sin\phi\sin\theta & \rho\cos\phi\sin\theta & \rho\sin\phi\cos\theta \\
        \cos\phi & -\rho\sin\phi & 0
    \end{vmatrix}$

    From this, we get and doing some simplifications gives you the determinant of this, which is $\rho^2\sin\phi$.
\end{example}

\end{document}