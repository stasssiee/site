\documentclass[../calc3.tex]{subfiles}
\graphicspath{{\subfix{../figures/}}}
\begin{document}
\chapter{Vector Calculus}
\section{Vector Fields}
\begin{definition}
    Let $f$ and $g$ be defined on a region $R$ of $\mathbb{R}^2$.
    A vector field in $\mathbb{R}^2$ is a function $\textbf{F}$ that assigns to each 
    point $(x,y)$ in $R$ a vector $\textbf{F}(x,y)$ where 
    \[\textbf{F}(x,y) = f(x,y)\textbf{i}+g(x,y)\textbf{j}\]
    \begin{center}
    or
    \end{center}
    \[F(x,y)=\langle f(x,y), g(x,y) \rangle \]
    The vector field $\textbf{R}$ is continuous or differentiable on $R$ is $f$ 
    and $g$ are continuous or differentiable on $R$.

    Note: A vector field is both a vector valued function and a function of several variables. 
\end{definition}

\begin{definition}
    Let $f$, $g$, and $h$ be defined on a region $D$ of $\mathbb{R}^3$. A vector field in $\mathbb{R}^3$ is a function 
    \textbf{F} that assigns to each point $(x,y,z)$ in $D$ a vector \textbf{F}$(x,y,z)$ where 
    \[\textbf{F}(x,y,z)=f(x,y,z)\textbf{i}+g(x,y,z)\textbf{j}+h(x,y,z)\textbf{k}\]
    \begin{center}
        or
    \end{center}
    \[\textbf{F}(x,y,z)=\langle f(x,y,z), g(x,y,z), h(x,y,z)\rangle \]

    The vector field \textbf{F} is continuous or differentiable on $D$ is $f$, $g$, and $h$ are continuous or differentiable on $D$.
\end{definition}

\begin{definition}[Radial Vector Field in $\mathbb{R}^2$]
    Let \textbf{r} = $\langle x,y\rangle$ and $p$ is any real number, then 
    \[\textbf{F}(x,y)=\frac{\textbf{r}}{|\textbf{r}|^p}=\frac{\langle x,y \rangle}{|\textbf{r}|^p}\]
    is a radial vector field.
\end{definition}
\pagebreak 
\begin{definition}
    Let $\varphi$ be a differentiable region of $\mathbb{R}^2$ or $\mathbb{R}^3$. The vector field \textbf{F}$=\nabla \varphi$ is a gradient field 
    and the function $\varphi$ is a potential function for \textbf{F}.

    Recall $\nabla \varphi = \langle \varphi_x, \varphi_y \rangle$ or $\langle \varphi_x, \varphi_y, \varphi_z\rangle$ 
    and that the vector field \textbf{F} $=\nabla_{\varphi}$ is orthogonal to the level curves of $\varphi$ at $(x,y)$.

    In $\mathbb{R}^3$ the gradient field will be orthogonal to level surfaces of $\varphi$.    
\end{definition}

\begin{definition}
    Let $\varphi$ be a potential function for a vector field in \textbf{F} in $\mathbb{R}^2$. That is, \textbf{F}$=\nabla_{\varphi}$.

    The level curves of a potential function are called equipotential curves.

    Also, the vector field may be visualized by drawing continuous flow curves or streamlines 
    that are everywhere orthogonal to the equipotential curves.

    These ideas can be extended to $\mathbb{R}^3$ in which case we will have equipotential surfaces.
\end{definition}

\section{Line Integrals}
\begin{definition}
    Suppose the scalar-valued function $f$ is defined on the region containing the smooth curve 
    $C$ given by \textbf{r}$(t)=\langle x(t),y(t) \rangle$, for $a\leq t\leq b$.

    The line integral of $f$ over $C$ is 
    \[\int_C f(x(t),y(t))\mathrm{d} = \lim_{\Delta \to 0} \sum^n_{k=1}f(x(t^*_k),y(t_k^*))\Delta s_k\]
    provided this limit exists over all partitions of $[a,b]$.

    If $f>0$ then the line integral computes the area of the ``curtain'' under $f$ and over $C$.
\end{definition}
Scalar line integrals are independent of the orientation and parameterization of the curve $C$.

Evaluting Scalar Line integrals in $\mathbb{R}^2$.
\[\int_C f\mathrm{d}s\]
What is $\mathrm{d}s$?

Let $C$ be given by \textbf{r}$(t)=\langle x(t),y(t) \rangle$ for $a\leq t\leq b$.

Recall: The length of the curve $C$ over $[a,t]$ is given by $s(t)=\int_a^t |\textbf{r}'(u)|\mathrm{d}u$.

By differentiating both sides, $s'(t)=|\textbf{r}'(t)|$.\\
Thus, $\mathrm{d}s = s'(t)\mathrm{d}t = |\textbf{r}'(t)|$.
\pagebreak
\begin{theorem}
    Let $f$ be continuous on a region containing a smooth curve $C$: $\textbf{r}(t)=\langle x(t),y(t)\rangle$, for $a\leq t\leq b$. Then 
    \[\int_C f\mathrm{d}s = \int_a^b f(x(t),y(t))|\textbf{r}'(t)|\mathrm{d}t = \int_a^b f(x(t),y(t))\sqrt{((x'(t))^2+(y'(t))^2)}\mathrm{d}t\]
\end{theorem}

\begin{theorem}
    Let $f$ be continuous on a region containing a smooth curve $C$: $\textbf{r}(t)=\langle x(t),y(t),z(t)\rangle$, for $a\leq t\leq b$. Then 
    \begin{align*}
        \int_C f\mathrm{d}s = \int_a^b f(x(t),y(t),z(t))|\textbf{r}'(t)|\mathrm{d}t \\
        =\int_a^b f(x(t),y(t),z(t))\sqrt{(x'(t))^2+(y'(t))^2+(z'(t))^2}\mathrm{d}t
    \end{align*}
\end{theorem}

\begin{definition}[Line Integral of a Vector Field]
    Let \textbf{F} be a vector field that is continuous on a region containing a smooth oriented curve 
    $C$ parametrized by arc length. Let \textbf{T} be the unit tangent vector at each point of $C$ 
    consistent with the orientation. The line integral of \textbf{F} over $C$ is 
    \[\int_C \textbf{F}\cdot\textbf{T}\mathrm{d}s\]
    Observations
    \begin{itemize}
        \item \textbf{F}$\cdot$\textbf{T}=$|$\textbf{f}$|$$|$\textbf{T}$|$$\cos\theta = |\textbf{F}|\cos\theta$ 
        \item The line integral adds up these components 
        \item The orientation of the curve matters! $\int_{-C}\textbf{F}\cdot\textbf{T}\mathrm{d}s = -\int_C \textbf{f}\cdot\textbf{T}\mathrm{d}s$
    \end{itemize}
\end{definition}

Evaluating The Line Integral of a Vector Field 
\[\int_C \textbf{F}\cdot\textbf{T}\mathrm{d}s = \int_a^b \textbf{F}\cdot \frac{\textbf{r}'(t)}{|\textbf{r}'(t)|}|\textbf{r}'(t)|\mathrm{d}t = \int_a^b \textbf{F}\cdot\textbf{r}'(t)\mathrm{d}t\]

Different Forms of Line Integrals of Vector Fields: Given \textbf{F} = $\langle f,g,h \rangle$ and $C$ with parameterization $\textbf{r}(t) = \langle x(t), y(t), z(t)\rangle$ for $a \leq t \leq b$:
\begin{align*}
    \int_C \textbf{F}\cdot \textbf{T}\mathrm{d}s = \int_a^b \textbf{F}\cdot\textbf{r}'(t)\mathrm{d}t = \int_a^b (f(t)x'(t)+g(t)y'(t)+h(t)z'(t))\mathrm{d}t\\
    =\int_C (f\mathrm{d}x+g\mathrm{d}y+h\mathrm{d}z)\\
    =\int_C \textbf{F}\cdot\mathrm{d}\textbf{r}
\end{align*}
This works similarly for vector fields in $\mathbb{R}^2$.

\begin{definition}
    Let $\textbf{F}$ be a continuous force field in a region $D$ of $\mathbb{R}^3$. Let 
    \[C:\textbf{r}(t)=\langle x(t),y(t),z(t)\rangle \text{for} a\leq t \leq b\]
    be a smooth curve in $D$ with a unit tangent vector $\textbf{T}$ consistent with the orientation.

    The work done (by the force field) in moving an object along $C$ in the positive direction is 
    \[W = \int_C \textbf{F}\cdot \textbf{T}\mathrm{d}s = \int_a^b \textbf{F}\cdot \textbf{r}'(t)\mathrm{d}t\]
    \begin{itemize}
        \item $\textbf{F}\cdot\textbf{T} = |\textbf{F}|\cos\theta$ is the tangential component of \textbf{F} along $C$ (in direction of the motion).
        \item The vector line integral sums the work done at each point along $C$.
    \end{itemize}
\end{definition}

\begin{definition}
    Definition: Circulation

    Let $\textbf{F}$ be a continuous vector field on a region $R$ of $\mathbb{R}^2$, and let $C$ be a closed 
    smooth oriented curve in $R$. The circulation of $\textbf{F}$ on $C$ is $\int_C \textbf{F}\cdot\textbf{T}\mathrm{d}s$, where 
    $\textbf{T}$ is the unit vector tangent to $C$ consistent with the orientation.

    A curve $C$ in $\mathbb{R}^2$ is closed if its initial and terminal points are the same.

    Circulation is a measure of how much of the vector field points in the direction of $C$.
\end{definition}

\begin{definition}
    Definition: Flux 

    Let $\textbf{F}$ be a continuous vector field on a region $R$ of $\mathbb{R}^2$, and let $C$ be a closed smooth oriented curve in $R$.
    The flux of the vector field $\textbf{F}$ across $C$ is $\int_C \textbf{F}\cdot\textbf{n}\mathrm{d}s$, where $\textbf{n}=\textbf{T}\times\textbf{k}$ is the 
    unit normal vector and $\textbf{T}$ is the vector tangent to $C$ consistent with the orientation.

    Flux is a measure of how much the vector field points orthogonally to $C$.

    In practice, use $\textbf{n}=\textbf{T}\times\textbf{k}$ and $\mathrm{d}s=|\textbf{r}'(t)|\mathrm{d}t$.

    Given $\textbf{F}=\langle f,g\rangle$ and $C:\textbf{r}(t)=\langle x(t), y(t)\rangle$ for $a\leq t\leq b$, then 
    \begin{align*}
        \int_C \textbf{F}\cdot\textbf{n}\mathrm{d}s=\int_a^b (f(t)y'(t)-g(t)x'(t))\mathrm{d}t\\
        =\int_C f\mathrm{d}y-g\mathrm{d}x
    \end{align*}
\end{definition}

\section{Conservative Vector Fields}
\begin{definition}[Simple and Closed Curves]
    Suppose a curve $C$ (in $\mathbb{R}^2$ or $\mathbb{R}^3$) is described parametrically by $\textbf{r}(t)$, where $a\leq t\leq b$.
    \begin{itemize}
        \item Then $C$ is a simple curve if $\textbf{r}(t_1)\neq \textbf{r}(t_2)$ for all $t_1$ and $t_2$, with $a<t_1<t_2<b$; that is, $C$ never intersects itself between its endpoints.
        \item The curve $C$ is closed if $\textbf{r}(a)=\textbf{r}(b)$; that is, the initial and terminal points of $C$ are the same.
    \end{itemize}
\end{definition}

\begin{definition}[Connected and Simply Connected Regions]
    \begin{itemize}
        \item An open region $R$ in $\mathbb{R}^2$ (or $D$ in $\mathbb{R}^3$) is connected if it is possible to connect any two points in $R$ by a continuous curve lying in $R$. (Think: $R$ is in one piece)
        \item An open region $R$ is simply connected if every closed simple curve in $R$ can be deformed and contracted to a point in $R$. (Think: $R$ has no holes.)
    \end{itemize}
    Recall that all points of an open set are interior points. An open set does not contain any of its boundary points.
\end{definition}

\begin{definition}[Conservative Vector Fields]
    A vector field $\textbf{F}$ is said to be conservative on a region (in $\mathbb{R}^2$ or $\mathbb{R}^3$) if there exists a 
    scalar function $\varphi$ such that $\textbf{F} = \nabla\varphi$ on that region.

    Recall that when $\textbf{F}=\nabla\varphi$, the function $\varphi$ is a potential function for $\textbf{F}$.

    Note: any function of the form $\varphi(x,y)=xy+C$ would be a potential function for $\textbf{F}$.
\end{definition}

\begin{definition}[Test for Conservative Vector Fields]
    Suppose $\textbf{F}=\langle f,g\rangle$ has continuous first partial derivatives on a connected and simply connected region $D$ in $\mathbb{R}^2$.

    If $\textbf{F}$ is conservative $\implies$ there is a function $\varphi$ such that $\textbf{F}=\nabla \varphi$.
    \begin{center}
        $\implies$(1)$f=\varphi_x$ \qquad and \qquad (2) $g=\varphi_y$.
    \end{center}
    Now by taking partial derivatives,
    \begin{center}
        $f_y=\varphi_{xy}$ \qquad and \qquad $g_x = \varphi_{yx}$.
    \end{center}
    By equality of mixed partial derivatives, 
    \begin{center}
        $\varphi_{xy}=\varphi_yx$.
    \end{center}
    Thus we can conclude, $f_y=g_x$.

    The other direction is also true. That is, if $f_y=g_x$ then $\textbf{F}$ is conservative.

    This provides us with a test for a conservative vector field in two dimensions, which can be 
    extended to the following test for vector fields in three dimensions.
\end{definition}

\begin{theorem}[Test for Conservative Vector Fields]
    Let $\textbf{F}=\langle f,g,h\rangle$ be a vector field defined on a connected and simply connected region $D$ in 
    $\mathbb{R}^3$, where $f$, $g$, and $h$ have continuous first partial derivatives on $D$.

    Then $\textbf{F}$ is a conservative vector field on $D$ if and only if 
    \begin{center}
        $\frac{\partial f}{\partial y}=\frac{\partial g}{\partial x}, \qquad \frac{\partial f}{\partial z}=\frac{\partial h}{\partial x}$ \qquad and \qquad $\frac{\partial g}{\partial z}=\frac{\partial h}{\partial y}$. 
    \end{center}
    For vector fields in $\mathbb{R}^2$, we have the single condition $\frac{\partial f}{\partial y}=\frac{\partial g}{\partial x}$.
\end{theorem}

\begin{theorem}[Fundamental Theorem for Line Integrals]
    Let $R$ be a region in $\mathbb{R}^2$ or $\mathbb{R}^3$ and let $\varphi$ be a differentiable potential function defined on $R$.
    If $\textbf{F}=\nabla\varphi$ (which means that $\textbf{F}$ is conservative), then 
    \[\int_C \textbf{F}\cdot\textbf{T}\mathrm{d}s=\int_C \textbf{F}\cdot \mathrm{d}\textbf{r}=\varphi(B)-\varphi(A)\]
    for all points $A$ and $B$ in $R$ and all piecewise-smooth oriented curves $C$ in $R$ from $A$ to $B$.
\end{theorem}

Why? Let $\textbf{r}(t)$ be any parameterization of $C$ for $a\leq t\leq b$ and one can use the chain rule to show that $\frac{\mathrm{d}\varphi}{\mathrm{d}t}=\textbf{F}\cdot\textbf{r}'(t)$. This, 
\[\int_C \textbf{F}\cdot\mathrm{d}\textbf{r}=\int_a^b \textbf{F}\cdot\textbf{r}'(t)\mathrm{d}t = \int_a^b \frac{\mathrm{d}\varphi}{\mathrm{d}t}\mathrm{d}t = \varphi(B)-\varphi(A)\]

Interpretation: If $\textbf{F}$ is a conservative vector field, then the value of a line integral of $\textbf{F}$ depends only on the endpoints of the path!

\begin{definition}[Path Independence]
    Let $\textbf{F}$ be a continuous vector field with domain $R$. If 
    \[\int_{C_1}\textbf{F}\cdot\mathrm{d}\textbf{r}=\int_{C_2}\textbf{F}\mathrm{d}\textbf{r}\]
    for all piecewise-smooth curves $C_1$ and $C_2$ in $R$ with the same initial and terminal points, then the line integral is 
    independent of path.
\end{definition}

\begin{theorem}
    Let $\textbf{F}$ be a continuous vector field on an open connected region $R$ in $\mathbb{R}^2$. If 
    \[\int_C \textbf{F}\cdot\mathrm{d}\textbf{r}\]
    is independent of path, then $\textbf{F}$ is conservative; that is, there exists a potential function $\varphi$ such that $\textbf{F}=\nabla\varphi$ on $R$.
\end{theorem}

\textbf{Line Integrals on Closed Curves}
Notation: We will ues $\oint_C \textbf{F}\cdot\mathrm{d}\textbf{r}$ to denote a line integral over a closed curve $C$.

\begin{theorem}
    Let $R$ be an open connected region in $\mathbb{R}^2$ or $\mathbb{R}^3$. Then $\textbf{F}$ is a conservative vector field on $R$ 
    if an only if $\oint_C \textbf{F}\cdot\mathrm{d}\textbf{r}=0$ on all simple closed piecewise-smooth oriented 
    curves $C$ in $R$. 
\end{theorem}

Why?\\
\textbf{F} is a conservative $\implies \oint_C \textbf{F}\cdot\mathrm{d}\textbf{r}=\varphi(B)-\varphi(A)=\varphi(A)-\varphi(A)=0$\\
$\oint_C \textbf{F}\cdot\mathrm{d}\textbf{r}=0 \implies 0 = \int_{C_1}\textbf{F}\cdot\mathrm{d}\textbf{r}+\int_{C_2}\textbf{F}\cdot\mathrm{d}\textbf{r}$\\
$\implies \int_{C_1}\textbf{F}\cdot\mathrm{d}\textbf{r}=-\int_{C_2}\textbf{F}\cdot\mathrm{d}\textbf{r}=\int_{-C_2}\textbf{F}\cdot\mathrm{d}\textbf{r}$

\section{Green's Theorem}
\begin{theorem}[Green's Theorem - Circulation Form]
    Let $C$ be a simple closed piecewise-smooth curve, oriented counterclockwise, that encloses a connected and simply connected region 
    $R$ in the plane. Assume $\textbf{F}=\langle f,g\rangle$ where $f$ and $g$ have continuous first partial derivatives in $R$. Then 
    \[\oint_C \textbf{F}\cdot\mathrm{d}\textbf{r}=\oint_C f\mathrm{d}x+g\mathrm{d}y=\iint_R \left(\frac{\partial g}{\partial x}-\frac{\partial f}{\partial y}\right)\mathrm{d}A\]

    Green's Theorem relates the circulation on $C$ to a double integral over the region $R$.

    If needed: $\oint_{-C}\textbf{F}\cdot\mathrm{d}\textbf{r}=-\oint_C \textbf{F}\cdot\mathrm{d}\textbf{r}$
\end{theorem}

\begin{definition}[Two-Dimensional Curl]
    The two-dimensional curl of the vector field $\textbf{F}=\langle f,g \rangle$ is 
    \[\frac{\partial g}{\partial x}-\frac{\partial f}{\partial y}\]
    If the curl is zero throughout a regio, the vector field is irrotational throughout that region.
\end{definition}

Recall: If $\textbf{F}=\langle f,g\rangle$ is conservative then $f_y = g_x$

Thus, $g_x-f_y=0$ and the curl of $\textbf{F}$ is zero.

Under the conditions of Green's Theorem: $\oint_C \textbf{F}\cdot\mathrm{d}\textbf{r}=\iint_R \left(\frac{\partial g}{\partial x}-\frac{\partial f}{\partial y}\right)\mathrm{d}A=0$.

Circulation integrals of conservative vector fields are always zero!

\begin{theorem}[Area of a Plane Region by Line Integrals]
    Let $C$ be a simple closed piecewise-smooth curve, oriented counterclockwise, that encloses a connected and simply connected region $R$ in the plane.

    Then the area of $R$ is given by:
    \[\oint_C x\mathrm{d}y = -\oint_C y\mathrm{d}x = \frac{1}{2}\oint_C (x\mathrm{d}y-y\mathrm{d}x)\]
\end{theorem}

\begin{theorem}[Green's Theorem - Flux Form]
    Let $C$ be a simple closed piecewise-smooth curve, oriented counterclockwise, that encloses a connected and simply connected region $R$ in the plane.
    Assume $\textbf{F}=\langle f,g\rangle$ where $f$ and $g$ have continuous first partial derivatives in $R$. Then 
    \[\oint_C \textbf{F}\cdot\textbf{n}\mathrm{d}s=\oint_C f\mathrm{d}y-g\mathrm{d}x=\iint_R \left(\frac{\partial f}{\partial x}+\frac{\partial g}{\partial y}\right)\mathrm{d}A\]
\end{theorem}
Interpretation: 
\begin{itemize}
    \item Green's theorem says that the net divergence throughout the region $R$ equals the flux across the boundary of $R$.
\end{itemize}

\begin{definition}[Two-Dimensional Divergence]
    The two-dimensional divergence of the vector field $\textbf{F}=\langle f,g\rangle$ is 
    \[\frac{\partial f}{\partial x}+\frac{\partial g}{\partial y}\]
    If the divergence is zero throughout a region, the vector field is source free throughout that region.
\end{definition}

The outward flux of a source free vector field is always zero!



\section{Divergence and Curl}
\begin{definition}[Divergence]
    The divergence of a vector field $\textbf{F}=\langle f,g,h\rangle$ that is differentiable on a region of $\mathbb{R}^3$ is 
    \[\text{div} \textbf{F} =\frac{\partial f}{\partial x}+\frac{\partial g}{\partial y}+\frac{\partial h}{\partial z}\]

    If div $\textbf{F}=0$, the vector field is source free. 

    Divergence measures the expansion or contraction of the vector field at each point.

    Del operator: $\nabla = \left\langle\frac{\partial}{\partial x},\frac{\partial}{\partial y},\frac{\partial}{\partial z} \right\rangle$

    Alternation notation: $\nabla \cdot \textbf{F} = \left\langle\frac{\partial}{\partial x},\frac{\partial}{\partial y},\frac{\partial}{\partial z} \right\rangle \cdot \langle f,g,h\rangle = \frac{\partial f}{\partial x}+\frac{\partial g}{\partial y}+\frac{\partial h}{\partial z} = \text{div} \textbf{F}$
\end{definition}

\begin{theorem}[Divergence of Radial Vector Fields]
    For a real number, $p$, the divergence of the radial vector field 
    \[\textbf{F}=\frac{\textbf{r}}{|\textbf{r}|^p}=\frac{\langle x,y,z\rangle}{(x^2+y^2+z^2)^{p/2}} \qquad \text{ is } \qquad \nabla \cdot \textbf{F} = \frac{3-p}{|\textbf{r}|^p}\]
\end{theorem}

\begin{definition}[Curl]
    The curl of a vector field $\textbf{F} = \langle f,g,h \rangle$ that is differentiable in a region of $\mathbb{R}^3$ is 
    \[\text{curl}\textbf{F} = \left(\frac{\partial h}{\partial y}-\frac{\partial g}{\partial z}\right)\textbf{i} + \left(\frac{\partial f}{\partial z}-\frac{\partial h}{\partial x}\right)\textbf{j}+\left(\frac{\partial g}{\partial x}-\frac{\partial f}{\partial y}\right)\textbf{k}\]

    Curl is a measure of rotation within a vector field at each point. 

    We can express the curl as a cross product:

    curl $\textbf{F} = \nabla \times \textbf{F} = \begin{vmatrix}
        \textbf{i} & \textbf{j} & \textbf{k} \\
        \frac{\partial}{\partial x} & \frac{\partial}{\partial y} & \frac{\partial}{\partial z}\\
        f & g & h
    \end{vmatrix}$

    If curl $\textbf{F = 0}$, the vector field is irrotational.
\end{definition}

The $\textbf{k}$-component of the curl (or the two-dimensional curl) gives the rotation of the vector field in the $xy$-plane at a point. 

The other components of the curl give similar information about the rotation of the vector field.

\begin{theorem}[Curl of a Conservative Vector Field]
    Suppose $\textbf{F}$ is a conservative vector field on an open region $D$ of $\mathbb{R}^3$. Let $\textbf{F}=\nabla\varphi$, where $\varphi$ 
    is a potential function with continuous second partial derivatives on $D$.

    Then curl $\textbf{F=0}$ and $F$ is irrotational.
\end{theorem}

\begin{theorem}[Divergence of the Curl]
    Suppose $\textbf{F}=\langle f,g,h\rangle$, where $f$, $g$, and $h$ have continuous second partial derivatives, then 
    \begin{center}
        div curl $\textbf{F} = 0$.
    \end{center}

    That is, the divergence of the curl is zero.

    Note: If $\textbf{F}$ is a vector field in $\mathbb{R}^3$ then div $\textbf{F}$ is a scalar valued function and not a vector field. Hence, curl div $\textbf{F}$ is not defined.
\end{theorem}

General Rotation Field:

$\textbf{F}=\textbf{a}\times\textbf{r}=\langle a_1,a_2,a_3 \rangle \times \langle x,y,z \rangle$.
\begin{itemize}
    \item The vector $\textbf{a}$ is the axis of rotation for the vector field $\textbf{F}$.
    \item The length of the curl of $\textbf{F}$, $|\nabla \times \textbf{F}| = 2|\textbf{a}|$.
    \item The divergence of the vector field $\textbf{F}$ is zero, or $\textbf{F}$ is source free.
\end{itemize}



\section{Surface Integrals}
Recall a curve in $\mathbb{R}^2$ is defined parametrically by $\textbf{r}(t)=\langle x(t), y(t)\rangle$, for $a\leq t\leq b$.

For a surface in $\mathbb{R}^3$ we'll need two parameters and three dependent variables:
\[\textbf{r}(u,v)=\langle x(u,v), y(u,v), z(u,v)\rangle\]

A cylinder with radius $a>0$ and height $h>0$ can be described parametrically as $\textbf{r}(u,v)=\langle a\cos u, a\sin u, v\rangle$ where $0\leq u\leq 2\pi$ and $0\leq v\leq h$.

A cone with radius $a>0$ and height $h>0$ can be described parametrically as $\textbf{r}(u,v)\left\langle \frac{av}{h}\cos u, \frac{av}{h} \sin u , v \right\rangle$ where $0\leq u\leq 2\pi$ and $0\leq v\leq h$.

A sphere with radius $a>0$ can be described parametrically as $\textbf{r}(u,v)=\langle a\sin u\cos v, a\sin u \sin v, a\cos u\rangle$, where $0\leq u\leq \pi$ and $0\leq v\leq 2\pi$.

For an explicitly defined surface:
\[z=g(x,y) \text{ on } R={(x,y): a\leq x\leq b, c\leq y\leq d}\]
can be parametrically described:
\[\textbf{u,v}=\langle u,v,g(u,v)\rangle\]
where $a\leq u\leq b$ and $c\leq v\leq d$.

Now we will develop the surface integral of a scalar-valued function $f$ defined on a smooth parametrized surface $S$.
\[\iint_S f(x,y,z)\dd S\]
Applications:
\begin{itemize}
    \item Compute the surface area of $S$.
    \item Compute the mass of a thin sheet described by the surface $S$ with mass density function $f$.
    \item Compute the average value of $f$ over the surface $S$.
\end{itemize}

\begin{definition}[Surface Integrals of Scalar-Valued Functions]
    Let $f$ be a continuous scalar-valued function on a smooth surface $S$ given parametrically by 
    $\textbf{r}(u,v)=\langle x(u,v), y(u,v), z(u,v)\rangle$, where $u$ and $v$ vary over the rectangle 
    $R={(u,v):a\leq u\leq b, c\leq v\leq d}$. Assume also that the tangent vectors
    \[ \textbf{t}_u = \frac{\partial \textbf{r}}{\partial u} = \left\langle \frac{\partial x}{\partial u}, \frac{\partial y}{\partial u}, \frac{\partial z}{\partial u}\right\rangle \qquad \text{ and } \qquad \textbf{t}_v = \frac{\partial\textbf{r}}{\partial v}=\left\langle\frac{\partial x}{\partial v},\frac{\partial y}{\partial v},\frac{\partial z}{\partial v} \right\rangle\]
    are continuous on $R$ and the normal vector $\textbf{t}_u \times \textbf{t}_v$ is nonzero on $R$.

    The surface integral of $f$ over $S$ is 
    \[\iint_S f(x,y,z)\dd S = \iint_R f(x(u,v), y(u,v), z(u,v))|\textbf{t}_u\times \textbf{t}_v|\dd A\]
    If $f(x,y,z)=1$, this integral equals the surface area of $S$.
\end{definition}

\begin{theorem}[Evaluation of Surface Integrals of Scalar-Valued Functions on Explicitly Defined Surfaces]
    Let $f$ be a continuous scalar-valued function on a smooth surface $S$ given parametrically by 
    $z=g(x,y)$, for $(x,y)$ in a region $R$. The surface integral of $f$ over $S$ is 
    \[\iint_S f(x,y,z)\dd S = \iint_R f(x,y,g(x,y))\sqrt{z_x^2 + z_y^2 + 1}\dd A\]
    
    If $f(x,y,z)=1$, the surface integral equals the area of the surface.
\end{theorem}

Orientable Surfaces: To be orientable, a surface must have a choice of normal vectors that varies continuously over the surface. (The surface is two-sided.)

If a surface encloses a region then we will choose normal vectors to point in the outward direction. For other surfaces, we must specify the direction of the normal vector.

\begin{definition}
    Flux Integrals: Consider a continuous vector field $\textbf{F}=\langle f,g,h\rangle$. Let $S$ be a smooth oriented surface with unit normal vector $\textbf{n}$.

    The flux integral 
    \[ \iint_S \textbf{F}\cdot \textbf{n}\dd S\]
    computes the net flux of the vector field across the surface.

    $\textbf{F}\cdot\textbf{n}=|\textbf{F}||\textbf{n}|\cos\theta = |\textbf{F}|\cos\theta$.

    The flux integral adds up the components of the vector field $\textbf{F}$ normal to the surface.
\end{definition}

\begin{definition}[Surface Integral of a Vector Field]
    Suppose $\textbf{F} = \langle f,g,h\rangle$ is a continuous vector field on a region of $\mathbb{R}^3$ containing a smooth 
    oriented surface $S$. If $S$ is defined parametrically as $\textbf{u,v}=\langle x(u,v), y(u,v), z(u,v)\rangle$, for $(u,v)$ in a region $R$, then 
    
    \[\iint_S \textbf{F}\cdot\textbf{n}\dd S = \iint_R \textbf{F}\cdot(\textbf{t}_u\times\textbf{t}_v)\dd A\]

    where $\textbf{t}_u = \frac{\partial \textbf{r}}{\partial u} = \left\langle \frac{\partial x}{\partial u}, \frac{\partial y}{\partial u}, \frac{\partial z}{\partial u}\right\rangle \qquad \text{ and } \qquad \textbf{t}_v = \frac{\partial\textbf{r}}{\partial v}=\left\langle\frac{\partial x}{\partial v},\frac{\partial y}{\partial v},\frac{\partial z}{\partial v} \right\rangle$
    are continuous on $R$, and the normal vector $\textbf{t}_u\times\textbf{t}_v$ is nonzero on $R$, and the direction of the normal vector is consistent with the orientation of $S$.
\end{definition}

\begin{definition}[Surface Integral of a Vector Field]
    If $S$ is defined in the form $z=w(x,y)$ for $(x,y)$ in a region $R$, then 
    \[\iint_S \textbf{F}\cdot \textbf{n}\dd S = \iint_R \textbf{F}\cdot(\textbf{t}_u\times \textbf{t}_v)\dd A = \iint_R (-fz_x-gz_y+h)\dd A\]

\end{definition}
\section{Stokes' Theorem}
\section{Divergence Theorem}

\end{document}