\documentclass[../calc3.tex]{subfiles}
\graphicspath{{\subfix{../figures/}}}
\begin{document}
\chapter{Vector-Valued Functions}
\section{Vector Functions and Space Curves}
Review: Parametric Curves 
\begin{itemize}
    \item $x=f(t)$
    \item $y=g(t)$
    \item $z=h(t)$
\end{itemize}
These represent a curve in 3-space (for 2-space, it is just $x$ and $y$.)

The above represents a path in space that is traced in a specific direction as $t$ increases (orientation). The domain is ($-\infty$, $\infty$), unless specified otherwise.

\begin{definition}
    \[ \vec{r}=\vec{r}(t) = \langle f(t),g(t),h(t)\rangle \]
    At any given $t$ value, $\vec{r}$ represents a vector whose initial point is at the origin and terminal point is $(f(t),g(t),h(t))$.

    The domain is $(-\infty,\infty)$ and the range is the set of vectors.
\end{definition}

Graphs of vector-valued functions: curve that is traced by connecting tips of ``radius vectors''.

\begin{example}
    Graph $\vec{r}(t)=2\cos t\vec{i}-3\sin t\vec{j}$ for $0\leq t\leq 2\pi$.

    We could write this as $x=2\cos t$ and $y=-3\sin t$ (parametric).

    We could instead write a table.
    \[ \begin{tabular}{c|c|c}
        t & x & y\\ \hline 0 & 2 & 0\\ \hline $\pi/2$ & 0 & -3 \\\hline $\pi$ & -2 & 0 \\\hline $3\pi/2$ & 0 & 3 \\\hline $2\pi$ & 2 & 0
    \end{tabular} \]

    As you draw this, you can see that this will be an ellipse.
\end{example}

\begin{example}
    \[ \vec{r}(t) = \langle 4\cos t, 4\sin t, t\rangle \]

    We should know that since there are trig things in here, that we go from $0$ to $2\pi$, and if we put this on a table, we can see that $x$ and $y$ will give you a circle from the table. The $z$ is moving up though, so basically the function will just be circling around a cylinder of radius $2$.
\end{example}
\pagebreak
\begin{example}
    Find a vector and parametric equations for the line segment that joins $A(1,-3,4)$ to $B(-5,1,7)$.

    We have $\vec{r}=\vec{AB}=\langle -6,4,3\rangle$. So $\vec{r}(t)=\langle 1-6t,-3+4t,4+3t\rangle$, and we want to put the bound $0\leq t\leq 1$

    The parametrics are $x(t)=1-6t, y(t)=-3+4t$, and $z=4+3t$, with $0\leq t\leq 1$.
\end{example}

\begin{example}
    Find a vector function that represents the curve of intersection of $x^2+y^2=1$ and $y+z=2$.

    $x^2+y^2=1$ is a cylinder and $y+z=2$ is a plane.

    We can represent $x^2+y^2=1$ as $x=\cos t$ and $y=\sin t$, with bounds $0\leq t\leq 2\pi$.

    $y+z=2$ can be represented as $z=2-y$ or $z=2-\sin t$ with $0\leq t\leq 2\pi$.

    So $\vec{r}(t)=(\cos t)\vec{i}+(\sin t)\vec{j}+(2-\sin t)\vec{k} = \langle \cos t, \sin t, 2-\sin t\rangle$ with $0\leq t\leq 2\pi$.
\end{example}

\begin{example}
    Find the domain of $\vec{r}(t)=\langle \ln|t-1|, e^t, \sqrt{t}\rangle$.

    The domain is all values of $t$ for which $\vec{r}(t)$ is defined.

    So we have $x=\ln |t-1|$, $y=e^t$ and $z=\sqrt{t}$.

    For $x$, we have the domain as $(-\infty, 1)\cup (1,\infty)$, for $y$ we have the domain as $t\in \mathbb{R}$, and for $z$, we have $t\geq 0$, so combining them gives domain $[0,1)\cup (1,\infty)$.  
\end{example}

\begin{definition}
    If $\vec{r}(t)=\langle f(t),g(t),h(t)\rangle$, then $\lim_{t\to a}\vec{r}(t)=\langle \lim_{t\to a}f(t),\lim_{t\to a}g(t),\lim_{t\to a}h(t)\rangle$ (as long as all 3 limits exist).
\end{definition}

\begin{example}
    Let $\vec{r}(t)=t^2\vec{i}+e^t\vec{j}-(2\cos \pi t)\vec{k}$. Find $\lim_{t\to 0}\vec{r}(t)$.

    The limit of the $\vec{i}$ term is 0 as it goes to 0.

    The limit of the $\vec{j}$ term is 1 as it approaches 0.

    The limit of the $\vec{k}$ term is -2 as it approaches 0.

    So the limit is $\lim_{t\to 0} \vec{r}(t) = \vec{j}-2\vec{k}$
\end{example}

\begin{example}
    Let $\vec{r}(t) = \left(\frac{4t^3+5}{3t^3+1}\right)\vec{i}+\left(\frac{1-\cos t}{t}\right)\vec{j}+\left(\frac{\ln(t+1)}{t}\right)\vec{k}$. Find $\lim_{t\to 0}\vec{r}(t)$.

    For the first term, we get 5 as the limit.

    For the other two, we will use L'Hopital's Rule.

    Doing this and finding the limits should give that $\lim_{t\to 0}\vec{r}(t) = \langle 5,0,1\rangle$.
\end{example}

Continuity: A vector function $\vec{r}(t)$ is continuous at $a$ if: $\lim_{t\to a} \vec{r}(t)=\vec{r}(a)$. (This is just AP Calculus BC)

\section{Derivatives and Integrals of Vector Functions}
\begin{definition}
    If $\vec{r}(t)$ is a vector function, the derivative of $\vec{r}(t)$ with respect to $t$ is 
    \[ \vec{r'} = \vec{r}(t)'=\frac{d\vec{r}}{dt}=\frac{d}{dt}(\vec{r}(t)) = \lim_{h\to 0}\frac{\vec{r}(t+h)-\vec{r}(t)}{h} \]
\end{definition}

Geometrically, this would have $\vec{r}(t)$ as a vector tangent to the curve at the tip of $\vec{r}(t)$. It points in the direction of increasing parameter.

\begin{theorem}
    If $\vec{r}(t)=\langle f(t),g(t),h(t)\rangle$, where $f$, $g$, and $h$ are differentiable functions, then 
    \[ \vec{r'}(t)=\langle f'(t),g'(t),h'(t)\rangle \]
\end{theorem}
\begin{proof}
    Let $\vec{r}(t)=\langle x(t),y(t)\rangle$

    By definition, $\vec{r'}(t)=\lim_{h\to 0}\frac{\vec{r}(t+h)-\vec{r}(t)}{h}$.

    This is equal to $\lim_{h\to 0}\frac{[x(t+h)\vec{i}+y(t+h)\vec{j}]-[x(t)\vec{i}+y(t)\vec{j}]}{h}$.

    Which is equal to 
    \[ \left(\lim_{h\to 0}\frac{x(t+h)\vec{i}-x(t)\vec{i}}{h}\right) + \left(\lim_{h\to 0}\frac{y(t+h)\vec{j}-y(t)}{h}\right) \]

    Taking out the $\vec{i}$ and $\vec{j}$, allows us to see that this equals to $x'(t)\vec{i}+y'(t)\vec{j}$. \qed
\end{proof}

\begin{example}
    $\vec{r}(t) = \frac{1}{t}\vec{i}+e^{2t}\vec{j}-2\cos\pi t\vec{k}$. Find $\vec{r}(t)$.

    The derivative of this is simply $\langle \frac{-1}{t^2},2e^{2t}, 2\pi\sin\pi t\rangle$.
\end{example}

$\vec{r'}(t)$ refers to the tangent vector. The tangent line is the line through $P$ that is parallel to $\vec{r'}(t)$. 

Unit Tangent Vector: $\vec{T}(t)=\frac{\vec{r'}(t)}{|\vec{r'}(t)|}$.

\begin{example}
    From the previous example, find the unit tangent vector at $t=1$.

    We know that $\vec{r'}(t)=\langle \frac{-1}{t^2},2e^{2t}, 2\pi\sin\pi t\rangle$. 

    From this, $\vec{r'}(1)=\langle -1,2e^2,0\rangle$, and the magnitude of this is $\sqrt{1+4e^4}$.

    Therefore, $\vec{T}(1)=\langle \frac{-1}{\sqrt{1+4e^4}}, \frac{2e^2}{\sqrt{1+4e^4}}, 0\rangle$.
\end{example}

\ex For the curve $\vec{r}(t)=\sqrt{t}\vec{i}+(2-t)\vec{j}$, find $\vec{r'}(t)$. Sketch $\vec{r}(1)$ and $\vec{r'}(1)$.

\pagebreak
\begin{example}
    Find parametric equations for the tangent line to the helix with equations $x=2\cos t$, $y=\sin t$, and $z=t$ at the point $(0,1,\pi/2)$.

    We have $\vec{r}(t)=\langle 2\cos t,\sin t, t\rangle$, so $\vec{r'}(t)=\langle -2\sin t,\cos t,1\rangle$.

    We get $0=2\cos t$, $1=\sin t$, and $\frac{\pi}{2}=t$, so we know that $t$ is.
    
    Plugging this in gives $\vec{r'}\left(\frac{\pi}{2}\right) = \langle -2,0,1\rangle$. This is the tangent vector.

    So $\vec{r}(t)=\langle 0,1,\frac{\pi}{2}\rangle + t\langle -2,0,1\rangle$. 

    Parametrically: $x=-2t$, $y=1$, $z=\frac{\pi}{2}+t$.
\end{example}

Differentiation Rules:
\begin{enumerate}
    \item $\frac{d}{dt}[\vec{u}(t)+\vec{v}(t)]=\vec{u}'(t)+\vec{v}'(t)$
    \item $\frac{d}{dt}[c\vec{u}(t)]=c\vec{u}'(t)$
    \item $\frac{d}{dt}[f(t)\vec{u}(t)]=f'(t)\vec{u}(t)+f(t)\vec{u}'(t)$
    \item $\frac{d}{dt}[\vec{u}(t)\cdot \vec{v}(t)]=\vec{u}'(t)\cdot\vec{v}(t)+\vec{u}(t)\cdot \vec{v}'(t)$
    \item $\frac{d}{dt}[\vec{u}(t)\times \vec{v}(t)]=\vec{u}'(t)\times \vec{v}(t)+\vec{u}(t)\times \vec{v}'(t)$ (Order matters here)
    \item $\frac{d}{dt}[\vec{u}(f(t))]=f'(t)\vec{u}'(f(t))$
\end{enumerate}

\begin{theorem}
    If $\vec{r}(t)$ is differentiable and $\|\vec{r}(t)\|$ is constant for all $t$, then $\vec{r}(t)\cdot \vec{r'}(t)=0$.

    This means they are orthogonal for all $t$.
\end{theorem}

\begin{example}
    The graphs of $\textbf{r}_1(t)$ and $\textbf{r}_2(t)$ intersect at the origin. Find the degree measure of the acute angle between the tangent lines to the graphs of $\textbf{r}_1(t)$ and $\textbf{r}_2(t)$ at the origin.

    We have $\vec{r}_1(t)=\langle \tan^{-1}t, \sin t, t^2\rangle$ and $\vec{r}_2(t)=\langle t^2-t, 2t-2,\ln t\rangle$.

    $\vec{r}_1(t)=\langle 0,0,0\rangle$ at $t=0$.

    $\vec{r}_2(t)=\langle 0,0,0\rangle$ at $t=1$.

    We need the derivatives of the functions.

    $\vec{r}_1'(t)=\langle \frac{1}{1+t^2},\cos t, 2t\rangle$
    
    $\vec{r}_2'(t)=\langle 2t-1, 2,\frac{1}{t}\rangle$

    $\vec{r}_1'(0)=\langle 1,1,0\rangle$ and $\vec{r}_2'(1)=\langle 1,2,1\rangle$.

    If we want to find the angles between then we have to use the dot product.

    We get $\cos\theta = \frac{1+2+0}{\sqrt{2}\cdot \sqrt{6}} = \frac{\sqrt{3}}{2}$.

    So $\theta = \frac{\pi}{6}$.
\end{example}

\pagebreak
\begin{example}
    Calculate $\frac{d}{dt}\left[ \vec{r}_1(t)\cdot \vec{r}_2(t)\right]$ and $\frac{d}{dt}\left[\vec{r}_1(t)\times \vec{r}_2(t)\right]$ by differentiating the product directly and using the formulas.
    \[ \vec{r}_1(t)=2t\vec{i}+3t^2\vec{j}+t^3\vec{k} \]
    \[ \vec{r}_2(t)=t^4 \vec{k}\]

    \textbf{Directly:}

    The dot product $\vec{r}_1\cdot \vec{r}_2 = t^7$. The derivative of this is $7t^6$.


    \textbf{Formula:}
    The formula is $\vec{r}_1'\cdot \vec{r}_2+\vec{r}_1\cdot \vec{r}_2'$.

    Using this formula gives you $3t^4t^6 = 7t^6$.

    
    Now for the cross product.

    \textbf{Directly:}
    The cross product gives $\langle 3t^6-0, -(2t^5-0),0\rangle = \langle 3t^6, -2t^5,0\rangle$.

    The derivative of this is $\langle 18t^5, -10t^4,0\rangle$.

    \textbf{Formula:}
    The formula is $\vec{r}_1'\times \vec{r}_2 + \vec{r}_1' \times \vec{r}_2'$.

    You should get the same answer.
\end{example}

\[ \int_a^b \vec{r}(t)=\lim_{n\to \infty}\sum_{i=1}^{n}\vec{r}(t_i^*)\Delta t \]

Or, more helpfully 
\[\int_a^b \vec{r}(t)dt = \left(\int_a^b f(t)dt\right)\vec{i}+\left(\int_a^b g(t)dt\right)\vec{j}+\left(\int_a^b h(t)dt\right)\vec{k}(t)\]

\begin{example}
    Let $\vec{r}(t)=t^2\vec{i}+e^t\vec{j}-2\cos\pi t\vec{k}$. Find $\int_0^1 \vec{r}(t)dt$.

    Integrating each component and plugging in the limits of integration results in $\int_0^1 \vec{r}(t)dt = \frac{1}{3}\vec{i}+(e^t)\vec{j}$.
\end{example}

\begin{example}
    Find $\int(2t\vec{i}+3t^2\vec{j})dt$.

    Remember in an indefinite integral to add a constant at the end.

    The result is $t^2\vec{i}+t^3\vec{j}+\vec{c}$.
\end{example}

\begin{example}
    Find $\vec{r}(t)$ given that $\vec{r'}(t)=\langle 3,2t\rangle$ and $\vec{r}(1)=\langle 2,5\rangle$.

    If we start by integrating, then $\vec{r}(t)=\langle 3t,t^2\rangle + \vec{c}$.

    We have $\langle 2,5\rangle = \langle 3,1\rangle+ \langle c_1,c_2\rangle$.

    We get $\vec{c}=\langle -1,4\rangle$ from this.

    So $\vec{r}(t)=\langle 3t-1, t^2+4\rangle$.
\end{example}

\section{Arc Length and Curvature}
Consider a curve given by parametric equations $x=x(t)$ and $y=y(t), a\leq t\leq b$.

Then arc length 
\[ L = \int_a^b \sqrt{\left(\frac{dx}{dt}\right)^2+\left(\frac{dy}{dt}\right)^2}dt \]

The Arc Length of a Vector Valued Function is the exact same idea 
\[ L = \int_a^b \sqrt{\left(\frac{dx}{dt}\right)^2 + \left(\frac{dy}{dt}\right)^2 + \left(\frac{dz}{dt}\right)^2}dt = \int_a^b \|\vec{r'}(t)\|dt \]

\begin{example}
    Find the arc length of the portion of the curve $x=3\cos t$, $y=3\sin t$, $z=4t$ from $(3,0,0)$ to $(-3,0,4\pi)$.

    If we use $z=4t$ we get $t=0$ and $t=\pi$ from both points.

    The integral is $L=\int_0^{\pi}\sqrt{(-3\sin t)^2+(3\cos t)^2+4^2}dt$.

    This is equal to $\int_0^{\pi}\sqrt{25}dt = 5\pi$.
\end{example}

A curve can be represented by more than one function.

\begin{example}
    Given $\vec{r}_1(t)=\langle t,t^2,t^3\rangle, 1\leq t\leq 2$.

    If we use $t=e^u$ then $\vec{r}_1(u)=\langle e^u, e^{2u}, e^{3u}\rangle, 0\leq u\leq \ln 2$.

    Both represent the same curve. These are called parametrizations of the curve. Both can be use dto find arc length (because arc length does not depend on the parameter).
\end{example}

\begin{example}
    Find the length of the curve above using both parametrizations.

    $\vec{r}_1(t)=\langle t,t^2,t^3\rangle$.

    $\vec{r}_1'(t)=\langle 1,2t,3t^2\rangle$.

    Then we integrate $L=\int_1^2 \sqrt{1+4t^2+9t^4}dt \approx 7.075$.

    For $\vec{r}_2(u)=\langle e^4,e^{2u},e^{3u}\rangle$.

    The derivative of this is $\vec{r}_2'(u)=\langle e^u, 2e^{2u}, 3e^{3u}\rangle$.

    The integral is $\int_0^{\ln 2}\sqrt{e^{2u}4e^{4u}+9e^{6u}}du \approx 7.075$.

    As you can see, they are the same.
\end{example}

We want to parametrize a curve in terms of arc length, $s$, rather than an arbitrary value in a particular coordinate system.

We first must recognize that $s(t)=\int_a^t |\vec{r'}(u)|du$.

This of course is equal to 
\[ \int_a^t \sqrt{\left(\frac{dx}{dt}\right)^2+\left( \frac{dy}{du}\right)^2 + \left(\frac{dz}{du}\right)^2}du \]

We can also see that $\frac{ds}{dt}=|\vec{r'}(t)|$.

\begin{example}
    Find the arc length parametrization of $\vec{r}(t)=\cos t\vec{i}+\sin t\vec{j}+t\vec{k}$ with reference point $(1,0,0)$ and the same orientation as the helix.

    We know that $\frac{ds}{dt}=|\vec{r'}(t)| = \sqrt{(-\sin t)^2+(\cos t)^2+1^2}=\sqrt{2}$.

    $s=s(t)=\int_0^t \sqrt{2}du = \sqrt{2}t$.

    We get that $t=\frac{s}{\sqrt{2}}$ as a result.

    Therefore $\vec{r}(s)=\cos \left(\frac{s}{\sqrt{2}}\right)\vec{i}+\sin\left(\frac{s}{\sqrt{2}}\right)\vec{j}+\left(\frac{s}{\sqrt{2}}\right)\vec{k}$.

    Arc length formula guarantees same orientation.

    This is useful because let's say we need to move along the curve for a certain amount of units, well we can just plug in that value and find the point at which we are.

    For example, $\vec{r}(5)\approx (-0.923, -0.384, 3.5636)$.
\end{example}


\begin{example}
    Find the arc length parametrization of the curve below measured from $(0,0)$ in the direction of increasing $t$.
    \[ \vec{r}(t)=\langle 1/3t^2, 1/2t^2\rangle, t\geq 0 \]

    $\vec{r'}(t)=\langle t^2,t\rangle$ and the magnitude of this is $t\sqrt{t^2+1}$.

    We are now integrating $s=\int_0^t u\sqrt{u^2+1}du$.

    This gives you $\frac{1}{3}(u^2+1)^{3/2}$ from $0$ to $t$.

    Integrating this and solving for $t$ gives you $t=\sqrt{(3s+1)^{2/3}-1}$.

    Therefore the parametrization of this is $\vec{r}(s)=\langle \frac{1}{3}[(3s+1)^{2/3}-1]^{3/2}, \frac{1}{2}[(3s+1)^{2/3}-1]\rangle$.
\end{example}

\begin{example}
    Let $\vec{r}(t)=\langle \ln t, 2t, t^2\rangle$. Find 

    (a) $\| \vec{r'}(t) \|$

    $\vec{r'}(t)=\langle \frac{1}{t},2,2t\rangle$, so the magnitude of this is $\sqrt{\frac{1}{t^2}+4+4t^2}=2t+\frac{1}{t}$.

    (b) $\frac{ds}{dt}$

    This is the exact same thing as $\| \vec{r'}(t)\| = 2t+\frac{1}{t}$

    (c) $\int_1^3 \| \vec{r'}(t)\| dt$

    We are integrating $\int_1^3 (2t+\frac{1}{t})dt = 9+\ln 3-1-0 = 8+\ln 3$.
\end{example}

A parametrization is called smoth on $I$ if $\vec{r'}(t)$ is continuous and $\vec{r'}(t)\neq 0$ on $I$ (a smooth curve has smooth parametrization). Smooth means no sharp corners or cusps.

\pagebreak
\begin{example}
    $\vec{r}(t)=\langle \cos t, \sin t, t\rangle$.
    
    Is $\vec{r}(t)$ smooth?

    The derivative of the vector is $\langle -\sin t,\cos t, 1\rangle$. This is continuous on $(-\infty,\infty)$ and this is not equal to $\vec{0}$, so $\vec{r}(t)$ is smooth.
\end{example}

Recall: $\vec{T}(t)=\frac{\vec{r'}(t)}{|\vec{r'}(t)|}$ (called unit tangent vector) indicated the direction of curve.

Curvature is as followed.
\[ \kappa = \left| \frac{d\vec{T}}{ds}\right| \]
$\vec{T}$ has a constant length so $\kappa$ is only affected by a change in direction.

\begin{example}
    Show that the curvature of a circle with radius $a$ is $1/a$.

    $\vec{r}(t)=\langle a\cos t, a\sin t\rangle$.

    The derivative $\vec{r'}(t)=\langle -a\sin t,a\cos t\rangle$.

    $s(t)=\int_0^t \sqrt{a^2\sin^2 u+a^2\cos^2 u}du =\int_0^t a du$.

    We get $s(t)=s=at$ so $t=\frac{a}{s}$.

    The circle in terms of $s$ is $\vec{r}(S)=\langle a\cos \frac{a}{s},a\sin\frac{a}{s}\rangle$.

    The derivative of this is $\langle -\sin\frac{a}{s},\cos\frac{a}{s}\rangle$.

    The magnitude of this is 1.

    The unit tangent vector $\vec{T}(s)=\langle -\sin \frac{a}{s},\cos\frac{a}{s}\rangle$.

    The derivative of this vector is $\langle -\frac{1}{a}\cos \frac{a}{s}, -\frac{1}{a}\sin\frac{a}{s}\rangle$.

    The magnitude of this vector is $\kappa = \frac{1}{a}$. A big radius means a small curvature.
\end{example}

The curvature of a straight line is $\kappa =0$.

A circle has constant curvature.

Other formulas for $\kappa$ are the following
\[ \kappa = \left| \frac{d\vec{T}}{dS}\right| = \left| \frac{\frac{dT}{dt}}{\frac{dS}{dt}} \right|\]
\[ \kappa = \frac{|\vec{T'}(t)|}{|\vec{r'}(t)|} \]
\[ \kappa = \frac{|\vec{r'}(t)\times \vec{r''}(t)}{\vec{r'}(t)|^3} \]
\[ \kappa(t) = \frac{|x'y''-y'x''|}{[(x')^2+(y')^2]^{3/2}}\]

\ex Use another formula to calculate $\kappa$ for $\vec{r}(t)=\langle a\cos t,a\sin t\rangle$.

\pagebreak
\begin{example}
    Find $\kappa$ for $\vec{r'}(t)=\langle 2t,t^2,-\frac{1}{3}t^3\rangle$.

    The derivative $\vec{r'}(t)=\langle 2,2t,-t^2\rangle$.

    $|\vec{r'}(t)|=\sqrt{4+4t^2+t^4}=t^2+2$

    $\vec{T}(t)=\frac{\langle 2,2t,-t^2\rangle}{t^2}+2 = \langle \frac{2}{t^2+2}, \frac{2t}{t^2+2}, \frac{-t^2}{t^2+2}\rangle$.

    $\vec{T'}(t)=\langle \frac{4t}{(t^2+2)^2}, \frac{-2t^2+4}{(t^2+2)^2}, \frac{-4t}{(t^2+2)^2}\rangle$.

    $\| \vec{T'}(t)\| = \sqrt{\frac{16t^2+4t^4-16t^2+16+16t^2}{(t^2+2)^4}} = \frac{2}{t^2+2}$

    $\kappa (t)=\frac{2/t^2+2}{t^2+2}=\frac{2}{(t^2+2)^2}$



    We can also use the other formula using the cross product.

    $\vec{r'}(t)=\langle 2,2t,-t^2\rangle$ and $\vec{r''}(t)=\langle 0,2,-2t\rangle$.

    The cross product of these two vectors will result in $\langle -4t^2--2t^2, -(-4t-0), 4-0\rangle = \langle -2t^2,4t,4\rangle$.

    The magnitude of this is $2(t^2+2)$, so $\kappa (t)=\frac{2(t^2+2)}{(t^2+2)^3} = \frac{2}{(t^2+2)^2}$.

    Both ways give an equivalent answer.
\end{example}

There is one more curvature formula in terms of $x$ rather than $t$.
\[ \kappa(x)=\frac{|f''(x)|}{[1+(f'(x))^2]^{3/2}} \]

\begin{example}
    Find the curvature of the parabola $y=x^2$ at the points $(0,0)$, $(1,1)$, and $(2,4)$.

    So $f(x)=x^2$, $f'(x)=2x$, and $f''(x)=2$.

    $\kappa (x)=\frac{|2|}{(1+(2x)^2)^{3/2}} = \frac{2}{(1+4x^2)^{7/2}}$

    $\kappa(0)=2$, $\kappa(1)\approx 0.18$, $\kappa(2) \approx 0.03$.

    As $\kappa \rightarrow \infty$, $\kappa (x)\rightarrow 0$.
\end{example}

Radius of curvature: $\rho = \frac{1}{\kappa}$

We have also shown $\kappa = \frac{1}{a}$

\begin{example}
    From the previous example, calculate the curvature at $(0,0)$. Then draw a circle of curvature.

    $\kappa(0)=2$ and $\rho(0,0)=\frac{1}{2}$.

    At the point $(0,0)$, $\kappa$ is same as circle with radius $\frac{1}{2}$.
\end{example}

Recall the unit tangent vector, $\vec{T}(t)=\frac{\vec{r'}(t)}{|\vec{r'}(t)|}$ which points in the direction of increasing parameter.

The unit tangent vector is orthogonal to its derivative.

Unit normal vector $\vec{N}(t)=\frac{\vec{T'}(t)}{|\vec{T'}(t)|}$. This points inward towards the concave part of curve $c$.

Binormal vector $\vec{B}(t) = \vec{T}(t)\times \vec{N}(t)$. 

$\| \vec{T}\times \vec{N}\| = \| \vec{T}\| \| \vec{N}\| \sin 90$. This is a also a unit vector.

\begin{example}
    Find the unit tangent, unit normal, and binormal vectors for $\vec{r}(t)=\langle 3\sin t, 3\cos t, 4t\rangle$.

    $\vec{r'}(t)=\langle 3\cos t, -3\sin t,4\rangle$.

    $\| \vec{r'}(t)\| = 5$

    $\vec{T}(t)=\langle \frac{3}{5}\cos t,-\frac{3}{5}\sin t, \frac{4}{5}\rangle$.

    $\vec{T'}(t)=\langle -\frac{3}{5}\sin t, -\frac{3}{5}\cos t,0\rangle$

    $\| \vec{T'}(t)\| = \frac{3}{5}$

    $\vec{N}(t)=\langle -\sin t,-\cos t,0 \rangle$.

    $\vec{B}(t)=\vec{T}\times \vec{N} =\langle \frac{4}{5}\cos t, -\frac{4}{5}\sin t,-\frac{3}{5}\rangle$
\end{example}

Another wayt to find $\vec{B}(t)$ is the following 
\[ \vec{B}(t)=\frac{\vec{r'}(t)\times \vec{r''}(t)}{\| \vec{r'}(t)\times \vec{r''}(t)\|} \]

\begin{example}
    Consider $\vec{r}(t)=\langle t,\frac{\sqrt{2}}{2}t^2, \frac{1}{3}t^3\rangle$. Find $\vec{T}, \vec{N}$ at $t=2$.

    $\vec{r'}(t)=\langle 1,\sqrt{2}t,t^2\rangle$

    $\| \vec{r'}(t)\|=\sqrt{1+2t^2+t^4} = t^2+1$

    $\vec{T}(t)=\langle \frac{1}{1+t^2}, \frac{\sqrt{2}t}{1+t^2}, \frac{t^2}{1+t^2}\rangle$

    $\vec{T}(2)=\langle \frac{1}{5}, \frac{2\sqrt{2}}{5}, \frac{4}{5}\rangle$

    Now to find $\vec{N}(2)$.

    $\vec{T'}(t)=\langle \frac{-2t}{(1+t^2)2}, \frac{(1+t^2)\sqrt{2}-2t(\sqrt{2}t)}{(1+t^2)^2}, \frac{2t(1+t^2)-t^2(2t)}{(1+t^2)^2}\rangle = \langle \frac{-2t}{(1+t^2)^2}, \frac{-2t^2+2}{(1+t^2)^2}, \frac{2t}{(1+t^2)^2}\rangle$

    $\vec{N}(t)=\frac{\vec{T'}(t)}{|\vec{T'}(t)|}$

    We should instead of finding the magnitude, find $\vec{T'}(2)=\langle \frac{-4}{25}, \frac{-8+\sqrt{2}}{25}, \frac{4}{25}\rangle$

    The magnitude of this is $\| \vec{T'}(2)\| = \sqrt{\frac{16}{625}+\frac{64-16\sqrt{2}+2}{625}+\frac{16}{625}} = \frac{\sqrt{98-16\sqrt{2}}}{25}$

    So $\vec{N}(2)= \frac{\langle \frac{-4}{25}, \frac{-8+\sqrt{2}}{25}, \frac{4}{25}\rangle}{\sqrt{98-16\sqrt{2}}{25}}$

    This is equal to $\langle \frac{-4}{\sqrt{98-16\sqrt{2}}}, \frac{-8+\sqrt{2}}{\sqrt{98-16\sqrt{2}}}, \frac{4}{\sqrt{98-16\sqrt{2}}}\rangle$.
\end{example}

A normal plane contains $\vec{N}$ and $\vec{B}$. It contains all lines perpendicular to $\vec{T}$.

The osculating plane contains $\vec{T}$ and $\vec{N}$. It is related to the circle of curvature or osculating circle.

The rectifying plane contains $\vec{T}$ and $\vec{B}$.

To find the equation of a plane you need a point and a perpendicular vector.

\pagebreak
\begin{example}
    Find the equations of the normal and osculating plantes at $(3,0,2\pi)$ for the following:
    \[ \vec{T}(t)=\langle \frac{3}{5}\cos t, -\frac{3}{5}\sin t, \frac{4}{5}\rangle \]
    \[ \vec{N}(t) = \langle -\sin t,-\cos t,0\rangle \]
    \[ \vec{B}(t) = \langle \frac{4}{5}\cos t, -\frac{4}{5}\sin t, -\frac{3}{5}\rangle \]

    The normal plane has point $(3,0,2\pi)$ and normalvector at $\frac{\pi}{2}$ is $\vec{T}\left(\frac{\pi}{2}\right) = \langle 0,-\frac{3}{5},\frac{4}{5}\rangle$.

    We have $0(x-3)+\frac{-3}{5}(y-0)+\frac{4}{5}(z-2\pi)=0$ and this gives $\frac{-3}{5}y+\frac{4}{5}z=\frac{8}{5}\pi$.

    The osculating plane we need the binormal vector. $\vec{B}\left(\frac{\pi}{2}\right) = \langle 0,-\frac{4}{5},\frac{3}{5}\rangle$.

    $0(x-3)+-\frac{4}{5}(y-0)+\frac{3}{5}(z-2\pi)=0$ so we get $-\frac{4}{5}y-\frac{3}{5}z=-\frac{6}{5}\pi$
\end{example}

\begin{example}
    Consider the ellipse given by 
    \[ \vec{r}(t) = 2\cos t\vec{i}+3\sin t\vec{j}, 0\leq t\leq 2\pi \]

    Note: $\kappa(t)=\frac{6}{[4\sin^2 t+9\cos^2 t]^{3/2}}$

    Find and draw the osculating circles at $(2,0)$ and $(0,-3)$.

    So we have $t=0$ and $t=\frac{3\pi}{2}$.

    For $(2,0)\rightarrow \kappa(0)=\frac{2}{9}$. so circle with radius $\frac{9}{2}$ and diamater 9.
    
    For $(0,-3)$, $\kappa\left(\frac{3\pi}{2}\right) = \frac{3}{4}$ so radius $r=\frac{3}{4}$ and diameter $\frac{8}{3}$.

    For the point $(2,0)$, we also have the point $(-7,0)$, so the center is $\left(-\frac{5}{2},0\right)$.

    So the equation for that is $\left(x+\frac{5}{2}\right)^2 + y^2 = \frac{81}{4}$.
\end{example}

\section{Motion in Space - Velocity and Acceleration}
\begin{enumerate}
    \item Direction of motion time $t$ is in the direction of $\vec{T}$.
    \item speed = $\frac{ds}{dt}$ (instantaneous rate of change of the arc length traveled). This is a scalar
    \item velocity vector $\vec{v}(t)=\frac{ds}{dt}\vec{T}(t)$
\end{enumerate}
$\frac{ds}{dt}$ is the magnitude of $\vec{v}(t)$.

$\vec{T}(t)$ denotes direction.

$\vec{v}(t)$ points in direction of motion and has magnitude = speed

If $\vec{r}(t)$ is a position function, then $\vec{v}(t)=\frac{d\vec{r}}{dt}(t)$ and $\vec{a}(t)=\frac{d\vec{v}}{dt} = \frac{d^2\vec{r}}{dt}$.

Speed is $\| \vec{v}(t)\| = \frac{ds}{dt}$

\pagebreak
\begin{example}
    A particle moves along $C$: $\vec{r}(t)=\langle 2\sin\left(\frac{t}{2}\right), 2\cos\left(\frac{t}{2}\right)\rangle$.

    (a) Find its velocity, acceleration, and speed at time $t$.

    $\vec{v}(t)=\vec{r'}(t)=\langle \cos \frac{t}{2}, -\sin\frac{t}{2}\rangle = \vec{v}(t)$.

    $\vec{a}(t)=\vec{v'}(t)=\langle -\frac{1}{2}\sin \frac{t}{2}, -\frac{1}{2}\cos \frac{t}{2}\rangle = \vec{a}(t)$

    speed = $\| \vec{v}(t)\| = 1$

    (b) Show that $\vec{a}(t)$ is orthogonal to $\vec{v}(t)$ for this path only.

    $\vec{a}(t)\cdot \vec{v}(t)=-\frac{1}{2}\cos \frac{t}{2}\sin \frac{t}{2}+\frac{1}{2}\sin \frac{t}{2}\cos \frac{t}{2}=0$.

    This implies that $\vec{a}(t)$ is orthogonal to $\vec{v}(t)$.
\end{example}

\begin{example}
    An object moves in 3-space so that $\vec{v}(t)=\langle 1,t,t^2\rangle$. Find the coordinates of the particle at time $t=1$ given that at $t=0$, the particle is at $(-1,2,4)$.

    $\vec{r}(t)=\int \vec{v}(t)dt = \langle t,\frac{1}{2}t^2,\frac{1}{3}t^3\rangle + \vec{c}$

    We know that $\vec{r}(0)=\langle -1,2,4\rangle$. This means that $\vec{c}=\langle -1,2,4\rangle$.

    So, $\vec{r}(t)=\langle t-1, \frac{1}{2}t^2, \frac{1}{3}t^3+4\rangle$.

    $\vec{r}(1)=\langle 0, \frac{5}{2}, \frac{13}{3}\rangle$.

    So this becomes the point $\left( 0,\frac{5}{2}, \frac{13}{3}\right)$ at $t=1$.
\end{example}

\begin{example}
    An object with mass $m$ that moves in a circular pattern with constant angular speed $\omega$ has position vector $\vec{r}(t)=a\cos \omega t \vec{i}+a\sin\omega t \vec{j}$. Find the force acting on the object and show that it is directed toward the origin.

    We have a circle toward the origin with radius $a$ and we have points on the circle $P$ at an angle $\theta$.

    Newton's 2nd law states that $\vec{F}(t)=m\vec{a}(t)$.

    We have the position vector.

    $\vec{v}(t)=\langle -a\omega \sin\omega t, a\omega \cos\omega t\rangle$

    $\vec{a}(t)=\langle -a\omega^2\cos\omega t, -a\omega^2\sin\omega t\rangle$

    $\vec{F}(t)=m\vec{a}(t)=m\langle -a\omega^2\cos\omega t, -a\omega^2\sin\omega t\rangle$.

    This can be simplified to $-m\omega^2\langle a\cos\omega t, a\sin\omega t\rangle$. As you can see the vector is just $\vec{r}(t)$.

    So $\vec{F}(t)=-m\omega^2 \vec{r}(t)$.

    The force acts in direction opposite to radius vector $\vec{r}(t)$. It points towards the origin.
\end{example}

Newton's Second Law is $\vec{F}=m\vec{a}$ as we talked about earlier.

Assumptions:
\begin{itemize}
    \item Mass is constant 
    \item Only force acting on the object after launch is Earth's gravity
    \item Assume the force of gravity is constant because the object is sufficiently close to the earth 
\end{itemize}

$\vec{F}=m\vec{a}$. $m$ is mass, $g$ is the acceleration due to gravity.

We can find $\vec{a}$ by letting $\vec{F}=-mg\vec{j}$, and we can rewrite as $m\vec{a}=-mg\vec{j}$.

This gives $\vec{a}=-g\vec{j}$.

$\vec{v}(t)=\int \vec{a}(t)=\int -g\vec{j}dt = -gt\vec{j}+c$ at $t=0, v(0)=v_0$.

This leads us to $\vec{v}(t)=-gt\vec{j}+\vec{v_0}$.

To find position, we need to integrate once more.

$\vec{r}(t)= -\frac{1}{2}gt^2\vec{j}+\vec{v_0}t+\vec{c_2}$, we have initial conditions $\vec{r}(0)=s_0$ and $\vec{c_2}=s_0 \vec{j}$ (up)

We can find that $\vec{r}(t)=-\frac{1}{2}gt^2\vec{j}+\vec{v_0}t+s_0\vec{j}$ or written as $\left( -\frac{1}{2}gt^2+s_0\right)\vec{j}+t\vec{v_0}$

We can express $\vec{v_0}$ in two components, with the $x$ component being $\vec{v_0}\cos\alpha$ and the $y$ component being $\vec{v_0}\sin\alpha$.

So $\vec{v_0} = v_0\cos\alpha \vec{i}+v_0\sin\alpha \vec{j}$.

So $\vec{r}(t)=\left(-\frac{1}{2}gt^2+s_0\right)\vec{j}+t(v_0\cos\alpha \vec{i}+v_0\sin\alpha\vec{j})$.

This simplifies to $\vec{r}(t)=(v_0\cos\alpha t)\vec{i}+(s_0+v_0\sin\alpha t-\frac{1}{2}gt^2)\vec{j}$

So $x(t)=v_0\cos \alpha \cdot t$ and $y(t)=s_0+v_0\sin\alpha \cdot t-\frac{1}{2}gt^2$.

Velocity in each direction is $v_x = v_0\cos\alpha$ and $v_y=v_0\sin\alpha-gt$

\begin{example}
    A basketball is hit with an initial speed of 80 ft/sec at an angle of $30^{\circ}$ and an initial height of 3 feet. 

    (a) Find parametric equations for the trajectory of the ball.

    $x=80\cos(30)t$ so $x(t)=40\sqrt{3}t$

    $y=3+80\sin 30 t-\frac{1}{2}(32)t^2$ so $y(t)=3+40t-16t^2$

    (b) How high does the ball get?

    We need to find the maximum of $y$ so $\frac{dy}{dt}=40-32t$. $0=40-32t$ and that gives $t=\frac{5}{4}$ seconds.

    Substituting that back in gives $y\left(\frac{5}{4}\right) = 28$ ft.

    Before $t=\frac{5}{4}$ the value is positive and after this time it is negative, so it is a maximum by the first derivative test.

    (c) How far does it travel horizontally?

    $0=3+40t-16t^2$ and $t\approx 2.57$ sec. $x(2.57)\approx 178.25$ ft.

    (d) What is the speed of the ball when it lands?

    It lands at $t\approx 2.57$ sec and speed is $\| \vec{v}(t)\|$.

    $\vec{v}(t)=\vec{r'}(t)=\langle 40\sqrt{3},40-32t\rangle$.

    Speed is $\sqrt{(40\sqrt{3})^2+(40-32(2.57))^2} \approx 81.19$ ft/sec
\end{example}

It if often useful to break acceleration into 2 components - one that is in the direction of the tangent vector and one in the direction of the normal vector. 

We will define $\| \vec{v}(t)\| = v$. Then $\vec{T}(t)=\frac{\vec{r'}(t)}{| \vec{r'}(t)|}=\frac{\vec{v}(t)}{|\vec{v}(t)|} = \frac{\vec{v}(t)}{v}$

So, $\vec{v}(t)=v\cdot \vec{T}(t)=v\cdot \vec{T}$.

Differentiating this gives $\vec{v'}=v'\vec{T}+v\vec{T'}$.

To get $\vec{T'}$, use $\kappa$ (curvature).

$\kappa = \frac{|\vec{T'}(t)|}{|\vec{r'}(t)|}=\frac{|\vec{T'}(t)|}{v}\implies |\vec{T'}(t)|=\kappa\cdot v$

Also $\vec{N}=\frac{\vec{T'}(t)}{|\vec{T'}(t)|}\implies \vec{T'}(t)=|\vec{T'}(t)|\vec{N}$

Substituting in gives $\vec{T'}(t)=\kappa v\cdot \vec{N}$.

$\vec{v'}=\vec{a}(t)=v'\vec{T}+v(\kappa v \vec{N}) = v'\vec{T}+\kappa v^2 \vec{N}$

We can write $\vec{a}(t)= a_T \vec{T}+a_N \vec{N}$.

This tells us that the object always moves according to the direction of motion ($\vec{T}$) and direction the curve is turning ($\vec{N}$)

We can dot $\vec{a}(t)$ with $v$ to get $\vec{v}\cdot \vec{a}=(v\vec{T})\cdot(v'\vec{T}+\kappa v^2\vec{N})$

This gives us $\vec{v}\cdot \vec{a}=vv'\vec{T}\cdot \vec{T}+\kappa v^3\vec{T}\cdot \vec{N}$. Hence, $\vec{v}\cdot \vec{a}=vv'$.

We know that $v'=a_T$, so $a_T = \frac{\vec{v}\cdot \vec{a}}{v} = \frac{\vec{r'}(t)\cdot \vec{r''}(t)}{| \vec{r'}(t)|}$.

We also know that $a_N = \kappa v^2 = \frac{|\vec{r'}(t)|\times \vec{r''}(t)|}{|\vec{r'}(t)|^3}|\vec{r'(t)}^2$.

This gives $a_N = \frac{|\vec{r'}(t)\times \vec{r''}(t)|}{|\vec{r'}(t)|}$

In summary:

Scalar Tangential component of acceleration 
\[ a_T = \frac{\vec{r'}(t)\cdot \vec{r''}(t)}{\| \vec{r'}(t)\|} \]

Scalar Normal component of acceleration 
\[ a_N = \frac{\| \vec{r'}(t)\times \vec{r''}(t)\| }{\| \vec{r'}(t)\|} \]

\pagebreak
\begin{example}
    Suppose a particle moves along $C: \vec{r}(t)=\langle t,t^2,t^3\rangle$.

    (a) Find the scalar tangential and normal components of $\vec{a}$.

    The first derivative is $\vec{r'}(t)=\langle 1,2t,3t^2\rangle$ and $\vec{r''}(t)=\langle 0,2,6t\rangle$.

    $\vec{r'}(t)\cdot \vec{r''}(t)=4t+18t^3$.

    $|\vec{r'}(t)| = \sqrt{9t^4+4t^2+1}$.

    So, $a_T = \frac{18t^3+4t}{\sqrt{9t^4+4t^2+1}}$.

    The cross product of $\vec{r'}(t)$ and $\vec{r''}(t)= \langle 6t^2,-6t,2\rangle$.

    The magnitude of this vector is $\sqrt{36t^4+36t^2+4}$.

    The scalar normal component $a_N = \sqrt{\frac{36t^4+36t^2+4}{9t^4+4t^2+1}}$.

    (b) Find the scalar tangential and normal components of $\vec{a}$ at $(1,1,1)$

    Plug in to get $a_T = \frac{22}{\sqrt{14}}$ and $a_N = \sqrt{\frac{28}{7}}$.

    (c) Find the vector tangential and normal components at $t=1$.

    $\vec{a}=a_T\vec{T}+a_n\vec{N}$.

    $\vec{T}(t)=\frac{\vec{r'}(t)}{|\vec{r'}(t)|}$. So $\vec{T}(1)=\frac{\langle 1,2,3\rangle}{\sqrt{14}}$.

    So $a_T \vec{T} = \frac{22}{\sqrt{14}}\frac{\langle 1,2,3\rangle}{\sqrt{14}}=\langle \frac{11}{7}, \frac{22}{7}, \frac{33}{7}\rangle$.

    Now to find the normal one, we can either find $\vec{N}$ or we can use that $\vec{a}=a_T\vec{T}+a_N\vec{N}$.

    We know that $\vec{a}(1)=\langle 0,2,6\rangle$ and we can substitute this to find $a_N \vec{N}$.

    $\langle 0,2,6\rangle -\langle \frac{11}{7}, \frac{22}{7},\frac{33}{7}\rangle = a_N\vec{N}=\langle -\frac{11}{7}, -\frac{8}{7}, \frac{9}{7}\rangle$.

    (d) Find the curvature of the path at the point $(1,1,1)$.

    Remember $\kappa(t)=\frac{|\vec{r'}\times \vec{r''}|}{|\vec{r'}(t)|^3}$

    Using what we previously found, $\vec{r'}(1)=\langle 1,2,3\rangle$ and $\vec{r''}(1)=\langle 0,2,6\rangle$.

    The cross product of these gives $\langle 6,-6,2\rangle$.

    $\kappa(1)=\frac{\sqrt{76}}{\sqrt{14}^3}=\frac{1}{14}\sqrt{\frac{38}{7}}$
\end{example}

\begin{example}
    The position particle of a function is given by $\vec{r}(t)=\langle -5t^2, -t,t^2+t\rangle$. At what time is the speed at a minimum?

    speed is $\| \vec{v}(t)\|$.

    $\vec{v}(t)=\langle -10t, -1, 2t+1\rangle$

    speed = $\sqrt{100t^2+1+4t^2+4t+1}=\sqrt{104t^2+4t+2}$

    $\frac{d\text{speed}}{dt}=\frac{1}{2}(104t^2+4t+2)^{-1/2}(208t+4)$

    $0=\frac{1}{2}(104t^2+4t+2)^{-1/2}(208t+4)$

    The first factor is never $0$, the second factor is $0$ when $t=-\frac{1}{52}$ sec 

    Now using the first derivative test, we see values before $-\frac{1}{52}$ are decreasing and after this point are positive, so $t=-\frac{1}{52}$ is a minimum.
\end{example}

\end{document}