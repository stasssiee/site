\documentclass[../calc3.tex]{subfiles}
\graphicspath{{\subfix{../figures/}}}
\begin{document}
\chapter{Vector-Valued Functions}
\section{Vector Functions and Space Curves}
Review: Parametric Curves 
\begin{itemize}
    \item $x=f(t)$
    \item $y=g(t)$
    \item $z=h(t)$
\end{itemize}
These represent a curve in 3-space (for 2-space, it is just $x$ and $y$.)

The above represents a path in space that is traced in a specific direction as $t$ increases (orientation). The domain is ($-\infty$, $\infty$), unless specified otherwise.

\begin{definition}
    \[ \vec{r}=\vec{r}(t) = \langle f(t),g(t),h(t)\rangle \]
    At any given $t$ value, $\vec{r}$ represents a vector whose initial point is at the origin and terminal point is $(f(t),g(t),h(t))$.

    The domain is $(-\infty,\infty)$ and the range is the set of vectors.
\end{definition}

Graphs of vector-valued functions: curve that is traced by connecting tips of ``radius vectors''.

\begin{example}
    Graph $\vec{r}(t)=2\cos t\vec{i}-3\sin t\vec{j}$ for $0\leq t\leq 2\pi$.

    We could write this as $x=2\cos t$ and $y=-3\sin t$ (parametric).

    We could instead write a table.
    \[ \begin{tabular}{c|c|c}
        t & x & y\\ \hline 0 & 2 & 0\\ \hline $\pi/2$ & 0 & -3 \\\hline $\pi$ & -2 & 0 \\\hline $3\pi/2$ & 0 & 3 \\\hline $2\pi$ & 2 & 0
    \end{tabular} \]

    As you draw this, you can see that this will be an ellipse.
\end{example}

\begin{example}
    \[ \vec{r}(t) = \langle 4\cos t, 4\sin t, t\rangle \]

    We should know that since there are trig things in here, that we go from $0$ to $2\pi$, and if we put this on a table, we can see that $x$ and $y$ will give you a circle from the table. The $z$ is moving up though, so basically the function will just be circling around a cylinder of radius $2$.
\end{example}
\medbreak
\begin{example}
    Find a vector and parametric equations for the line segment that joins $A(1,-3,4)$ to $B(-5,1,7)$.

    We have $\vec{r}=\vec{AB}=\langle -6,4,3\rangle$. So $\vec{r}(t)=\langle 1-6t,-3+4t,4+3t\rangle$, and we want to put the bound $0\leq t\leq 1$

    The parametrics are $x(t)=1-6t, y(t)=-3+4t$, and $z=4+3t$, with $0\leq t\leq 1$.
\end{example}

\begin{example}
    Find a vector function that represents the curve of intersection of $x^2+y^2=1$ and $y+z=2$.

    $x^2+y^2=1$ is a cylinder and $y+z=2$ is a plane.

    We can represent $x^2+y^2=1$ as $x=\cos t$ and $y=\sin t$, with bounds $0\leq t\leq 2\pi$.

    $y+z=2$ can be represneted as $z=2-y$ or $z=2-\sin t$ with $0\leq t\leq 2\pi$.

    So $\vec{r}(t)=(\cos t)\vec{i}+(\sin t)\vec{j}+(2-\sin t)\vec{k} = \langle \cos t, \sin t, 2-\sin t\rangle$ with $0\leq t\leq 2\pi$.
\end{example}

\begin{example}
    Find the domain of $\vec{r}(t)=\langle \ln|t-1|, e^t, \sqrt{t}\rangle$.

    The domain is all values of $t$ for which $\vec{r}(t)$ is defined.

    So we have $x=\ln |t-1|$, $y=e^t$ and $z=\sqrt{t}$.

    For $x$, we have the domain as $(-\infty, 1)\cup (1,\infty)$, for $y$ we have the domain as $t\in \mathbb{R}$, and for $z$, we have $t\geq 0$, so combining them gives domain $[0,1)\cup (1,\infty)$.  
\end{example}

\begin{definition}
    If $\vec{r}(t)=\langle f(t),g(t),h(t)\rangle$, then $\lim_{t\to a}\vec{r}(t)=\langle \lim_{t\to a}f(t),\lim_{t\to a}g(t),\lim_{t\to a}h(t)\rangle$ (as long as all 3 limits exist).
\end{definition}

\begin{example}
    Let $\vec{r}(t)=t^2\vec{i}+e^t\vec{j}-(2\cos \pi t)\vec{k}$. Find $\lim_{t\to 0}\vec{r}(t)$.

    The limit of the $\vec{i}$ term is 0 as it goes to 0.

    The limit of the $\vec{j}$ term is 1 as it approaches 0.

    The limit of the $\vec{k}$ term is -2 as it approaches 0.

    So the limit is $\lim_{t\to 0} \vec{r}(t) = \vec{j}-2\vec{k}$
\end{example}

\begin{example}
    Let $\vec{r}(t) = \left(\frac{4t^3+5}{3t^3+1}\right)\vec{i}+\left(\frac{1-\cos t}{t}\right)\vec{j}+\left(\frac{\ln(t+1)}{t}\right)\vec{k}$. Find $\lim_{t\to 0}\vec{r}(t)$.

    For the first term, we get 5 as the limit.

    For the other two, we will use L'Hopital's Rule.

    Doing this and finding the limits should give that $\lim_{t\to 0}\vec{r}(t) = \langle 5,0,1\rangle$.
\end{example}

Continuity: A vector function $\vec{r}(t)$ is continuous at $a$ if: $\lim_{t\to a} \vec{r}(t)=\vec{r}(a)$. (This is just AP Calculus BC)

\section{Derivatives and Integrals of Vector Functions}
\begin{definition}
    If $\vec{r}(t)$ is a vector function, the derivative of $\vec{r}(t)$ with respect to $t$ is 
    \[ \vec{r}' = \vec{r}(t)'=\frac{d\vec{r}}{dt}=\frac{d}{dt}(\vec{r}(t)) = \lim_{h\to 0}\frac{\vec{r}(t+h)-\vec{r}(t)}{h} \]
\end{definition}

Geometrically, this would have $\vec{r}(t)$ as a vector tangent to the curve at the tip of $\vec{r}(t)$. It points in the direction of increasing parameter.

\begin{theorem}
    If $\vec{r}(t)=\langle f(t),g(t),h(t)\rangle$, where $f$, $g$, and $h$ are differentiable functions, then 
    \[ \vec{r}'(t)=\langle f'(t),g'(t),h'(t)\rangle \]
\end{theorem}
\begin{proof}
    Let $\vec{r}(t)=\langle x(t),y(t)\rangle$

    By definition, $\vec{r}'(t)=\lim_{h\to 0}\frac{\vec{r}(t+h)-\vec{r}(t)}{h}$.

    This is equal to $\lim_{h\to 0}\frac{[x(t+h)\vec{i}+y(t+h)\vec{j}]-[x(t)\vec{i}+y(t)\vec{j}]}{h}$.

    Which is equal to 
    \[ \left(\lim_{h\to 0}\frac{x(t+h)\vec{i}-x(t)\vec{i}}{h}\right) + \left(\lim_{h\to 0}\frac{y(t+h)\vec{j}-y(t)}{h}\right) \]

    Taking out the $\vec{i}$ and $\vec{j}$, allows us to see that this equals to $x'(t)\vec{i}+y'(t)\vec{j}$. \qed
\end{proof}

\begin{example}
    $\vec{r}(t) = \frac{1}{t}\vec{i}+e^{2t}\vec{j}-2\cos\pi t\vec{k}$. Find $\vec{r}(t)$.

    The derivative of this is simply $\langle \frac{-1}{t^2},2e^{2t}, 2\pi\sin\pi t\rangle$.
\end{example}

$\vec{r}'(t)$ refers to the tangent vector. The tangent line is the line through $P$ that is parallel to $\vec{r}'(t)$. 

Unit Tangent Vector: $\vec{T}(t)=\frac{\vec{r}'(t)}{|\vec{r}'(t)|}$.

\begin{example}
    From the previous example, find the unit tangent vector at $t=1$.

    We know that $\vec{r}'(t)=\langle \frac{-1}{t^2},2e^{2t}, 2\pi\sin\pi t\rangle$. 

    From this, $\vec{r}'(1)=\langle -1,2e^2,0\rangle$, and the magnitude of this is $\sqrt{1+4e^4}$.

    Therefore, $\vec{T}(1)=\langle \frac{-1}{\sqrt{1+4e^4}}, \frac{2e^2}{\sqrt{1+4e^4}}, 0\rangle$.
\end{example}

\ex For the curve $\vec{r}(t)=\sqrt{t}\vec{i}+(2-t)\vec{j}$, find $\vec{r}'(t)$. Sketch $\vec{r}(1)$ and $\vec{r}'(1)$.

\begin{example}
    Find parametric equations for the tangent line to the helix with equations $x=2\cos t$, $y=\sin t$, and $z=t$ at the point $(0,1,\pi/2)$.

    We have $\vec{r}(t)=\langle 2\cos t,\sin t, t\rangle$, so $\vec{r}'(t)=\langle -2\sin t,\cos t,1\rangle$.

    We get $0=2\cos t$, $1=\sin t$, and $\frac{\pi}{2}=t$, so we know that $t$ is.
    
    Plugging this in gives $\vec{r}'\left(\frac{\pi}{2}\right) = \langle -2,0,1\rangle$. This is the tangent vector.

    So $\vec{r}(t)=\langle 0,1,\frac{\pi}{2}\rangle + t\langle -2,0,1\rangle$. 

    Parametrically: $x=-2t$, $y=1$, $z=\frac{\pi}{2}+t$.
\end{example}

Differentiation Rules:
\begin{enumerate}
    \item $\frac{d}{dt}[\vec{u}(t)+\vec{v}(t)]=\vec{u}'(t)+\vec{v}'(t)$
    \item $\frac{d}{dt}[c\vec{u}(t)]=c\vec{u}'(t)$
    \item $\frac{d}{dt}[f(t)\vec{u}(t)]=f'(t)\vec{u}(t)+f(t)\vec{u}'(t)$
    \item $\frac{d}{dt}[\vec{u}(t)\cdot \vec{v}(t)]=\vec{u}'(t)\cdot\vec{v}(t)+\vec{u}(t)\cdot \vec{v}'(t)$
    \item $\frac{d}{dt}[\vec{u}(t)\times \vec{v}(t)]=\vec{u}'(t)\times \vec{v}(t)+\vec{u}(t)\times \vec{v}'(t)$ (Order matters here)
    \item $\frac{d}{dt}[\vec{u}(f(t))]=f'(t)\vec{u}'(f(t))$
\end{enumerate}

\begin{theorem}
    If $\vec{r}(t)$ is differentiable and $||\vec{r}(t)||$ is constant for all $t$, then $\vec{r}(t)\cdot \vec{r}'(t)=0$.

    This means they are orthogonal for all $t$.
\end{theorem}



\section{Arc Length and Curvature}

\section{Motion in Space - Velocity and Acceleration}

\end{document}