\documentclass[../em.tex]{subfiles}
\graphicspath{{\subfix{../figures/}}}
\begin{document}
\chapter{Conductors and Capacitors}
\section{Electrostatics with Conductors}
An ideal conductor is a material in which electrons are able to move freely.

When a conductor is in electrostatic equilibrium, mutual repulsion of excess 
charge carriers results in those charge carriers residing entirely on the surface of the conductor.
\begin{itemize}
    \item In a conductor with a negative net charge, excess electrons reside on the surface of the conductor.
    \item In a conductor with a positive net charge, the surface becomes deficient in electrons and can be modeled as if positive charge carriers reside on the surface of the conductor.
\end{itemize}

Excess charges will move to the surface of a conductor to create a state of electrostatic equilibrium within the conductor.
\begin{itemize}
    \item At electrostatic equilibrium, the electroc potential of the surface is the same everywhere and the conductor becomes an equipotential surface.
\end{itemize}

Recall there is no electric field inside of a conductor.

The charge density on the surface is greater where there are points or edges compared to planar areas.

Electrostatic shielding is the process of an area with a closed conducting shell to 
create a region inside the conductor that is far from external electric fields.

\begin{example}
    A solid, uncharged conducting sphere of radius $3a$ contains a hollowed spherical region of radius $a$.
    A point charge $+Q$ is placed at the common center of the spheres. Taking $V = 0$ as $r$ approaches infinity,
    the potential at position $r = 3a$ from the center of the sphere is:

    We start with (treat as a point charge)
    \[V=\frac{kq}{r}\]
    and end up getting 
    \[V=\frac{k[Q]}{3a}\]
    which is equal to 
    \[V=\frac{kQ}{3a}\]  
\end{example}

\ex A spherical conductor of radius $r$ is given a charge $+Q$. What is the electric potential inside the spherical conductor at half of its radius and why?

\section{Redistribution of Charge between Conductors}
When conductors are in electrical contact, charges will be redistributed such that the surfaces of each conductor are at the same electric potential.

Ground is an idealized reference point that has zero electric potential and can absorb an 
infinite amount of charge without changing its electric potential. Charge can be induced by a conductor by grounding the conductor is the presence of an external electric field.

\pagebreak
\begin{example}
    Charge is placed on two conducting spheres that are very far apart and connected by a long thin wire. 
    The radius of the smaller sphere is 5 cm and that of the larger sphere is 12 cm. The electric field at the 
    surface of the larger sphere is 358 kV/m. Find the surface charge density on each sphere.

    We start with electric field:
    \[E = \frac{kQ}{R^2}\] 
    \[Q = \frac{ER^2}{k}=\frac{(358\times 10^3 \text{V/m})}{(0.12\text{m})^2}{9\times 10^9}= 5.78\times 10^{-8}\text{C}\]

    We know that $V_1 = V_2$, so 
    \[\frac{kQ_1}{R_1}=\frac{kQ_2}{R_2}\] 
    Rearranging for $Q_2$:
    \[Q_2=\frac{R_2}{R_1}Q_1 = \frac{5}{12}(5.78\times10^{-7})=2.4\times10^{-7}\text{C}\]

    Now we can find the surface charge on each sphere. 
    \[\sigma_1=\frac{Q_1}{A_1} \quad \sigma_2 = \frac{Q_2}{A_2}\]

    So \[\sigma_1 = 3.2\times 10^{-6} C/m^2\]
    and \[\sigma_2 = 7.6\times 10^{-6} C/m^2\] 
\end{example}

\ex A conducting sphere with net charge $Q_0$ and radius $R$ is connected by a wire to an initially uncharged conducting sphere of 
radius $2R$. After electrostatic equilibrium has been reached, describe the electric potential at the surface of the smaller sphere.

\ex Two conducting spheres are initially isolated. Sphere 1 has radius $r_1$ and is initially uncharged. Sphere 2 has radius $r_2$, such that $r_1>r_2$ and has a charge of $+Q$. The two spheres are then connected 
by a conducting wire and allowed to reach equilibrium, resulting in a charge of Sphere 1. 
Represent the charge $q$ on Sphere 1 after the two spheres have reached equilibrium.



\section{Capacitors}
A parallel-plate capacitor consists of two separated parallel conducting surfaces 
that can hold equal amounts of charge with opposite signs.

Capacitance relates the magnitude of the charge stored on each plate to the electric 
potential difference created by the separation of those charges.
\[C = Q/\Delta V\] 
Unit = Farads
\[C = \frac{\kappa \epsilon_0 A}{d}\]

The electric field between the two charged plates with uniformly distributed 
electric charge is constant in both magnitude and direction, except nera the edges of the plates.
\[E = \frac{Q}{\epsilon_0 A} = \sigma/\epsilon_0\]

The electric potential stored in a capacitor is equal to the work done by an external force 
to separate that amount of charge on the capacitor.
\[U_c \frac{1}{2}Q\Delta V = \frac{1}{2}C(\Delta V)^2 = \frac{1}{2}\frac{Q^2}{C}\]

\begin{example}
    A capacitor with circular parallel plates of radius $R$ that are separated by a distance $d$ 
    has a capacitance of $C$. What would the capacitance be if the plates has radius $2R$ and 
    were separated by a distance $d/2$?

    \begin{align*}
        C_0 = \frac{\epsilon_0 A}{d}\\ 
        C_0 = \frac{\epsilon_0\pi r^2}{d}=\frac{\epsilon_0 \pi R^2}{d}\\
    \end{align*}
    \begin{align*}
        C = \frac{\epsilon_0 A}{d}\\
        C = \frac{\epsilon_0 \pi r^2}{d}\\
        C = \frac{\epsilon_0 \pi (2R)^2}{d/2} = \frac{8\epsilon_0\pi R^2}{d}
    \end{align*}
\end{example}

\ex A parallel-plate capacitor with plates of area $A$ and plate separation $d$ is attached to a bettery and given a charge $Q$. The potential energy stored in the capacitor is $U_1$. The capacitor is then detached from the battery 
and then the plates are pulled apart to a distance of $2d$. The potential energy stored in the capacitor is now $U_2$. 
Describe the ratio of the potential energies $\frac{U_1}{U_2}$.

\ex An isolated, charged parallel-plate capacitor has charge $Q$ and the absolute value of the potential difference across the plates is $|\Delta V|$. 
A slab of conductive material is inserted between the plates such that it fills half of the distance between the plates. 
Describe the change, if any, in $Q$ and $|\Delta v|$ as the conductive material is inserted between the plates.

\section{Dielectrics}
In a dielectric material, electric charges are not as free to move as they are in a conductor.

The material becomes polarized in the presence of an external electric field.

The dielectric constant of a material relates the electric permittivity of that material to the permittivity of free space.
\[\kappa = \epsilon/\epsilon_0\]

For a dielectric, 
\[\int \vec{E}\cdot \mathrm{d}\vec{A} = \frac{Q}{\kappa \epsilon_0}=EA \]

The electric field created by a polarized dielectric is opposite in direction to the external field.

The electric field between the plates of an isolated parallel-plate capacitor decreases when a 
dielectric is placed between the plates.
\[\kappa = E_0/E\]

The insertion of a dielectric into the capacitor may change the capacitance of the capacitor.
\[C = \kappa C_0 = \frac{\kappa \epsilon_0 A}{d}\]

\pagebreak
\begin{example}
    A capacitor consists of two conducting, coaxial, cylindrical shells of radius $a$ and $b$, respectively,
    and length $L$ $>>$ $b$. The space between the cylinder is filled with oil that has a dielectric constant $\kappa$.
    Initially both cylinders are uncharged, but then a battery is used to charge the capacitor, leaving charge $+Q$ 
    on the inner cylinder and $-Q$ on the outer cylinder. Let $r$ be the radial distance from the axis of the capacitor.
    Determine:
    \begin{itemize}
        \item The electric field from $a$ to $b$.
        \item The electric potential from $a$ to $b$.
        \item The capacitance of the capacitor.
    \end{itemize}

    a.
    \begin{align*}
        \int \vec{E}\mathrm{d}\vec{A}=\frac{Q}{\epsilon}\\ 
        EA = \frac{Q}{\kappa\epsilon_0}\\
        E = \frac{Q}{2\kappa \pi\epsilon_0 L r}
    \end{align*} 

    b.
    \begin{align*}
        V = -\int_a^b \vec{E}\cdot\mathrm{d}\vec{r}\\
        V = \int^a_b \left(\frac{Q}{2\kappa\epsilon_0 \pi L}\frac{1}{r}\right)\cdot \mathrm{d}r\\
        V = \frac{Q}{2\kappa\epsilon_0\pi L}\int_b^a \frac{1}{r}\mathrm{d}r\\
        V = \frac{Q}{2\kappa\epsilon_0\pi L}[\ln r]^a_b \\
        V = \frac{Q}{2\kappa\epsilon_0\pi L}ln\left[\frac{b}{a}\right]
    \end{align*}

    c.
    \begin{align*}
        C = \frac{Q}{V}\\
        C = \frac{2\kappa \epsilon_0 \pi L}{\ln b/a}
    \end{align*}

\end{example}
\ex An air-filled parallel plate capacitor has a capacitance of $C$ = 12 pF. The space between the plates is filled with a dielectric, and the new capacitance of the capacitor is $C=48$ pF. What is the dielectric constant for the dielectric?

\ex A capacitor initially has a capacitance of $C_1$. If a dielectric with a dielectric constant of 3 is added between the plates of the capacitor the new 
capacitance is $C_2$. Expresse the ratio of the new capacitance $C_2$ to $C_1$.

\end{document}