\documentclass[../em.tex]{subfiles}
\graphicspath{{\subfix{../figures/}}}
\begin{document}
\chapter{Electromagnetic Induction}
\section{Magnetic Flux}
For a magnetic field that is constant across an area, the magnetic flux through the area is defined as 
\[ \varphi_B = \vec{B}\cdot\vec{A}=BA\cos\theta \]
The area vector is defined as perpendicular to the plane of the surface area and outward from a closed surface.

The sign of the flux is given by the dot product of the magnetic field vector and the area vector.

The total magnetic flux passing through a surface is defined by the surface integral of the magnetic field over the surface area.
\[\varphi_B = \oint \vec{B}\cdot\dd\vec{A} \]
which is usually equal to $BA$.
\section{Electromagnetic Induction}
Faraday's law describes the relationship between changing magnetic flux and the resulting induced emf in a system.
\[\epsilon = \frac{-\dd\varphi_B}{\dd t} = \frac{-\dd(\vec{B}-\vec{A})}{\dd t} = \frac{-\dd BA\cos\theta}{\dd t}\]

Lenz's law is used to determine the direction of an induced emf resulting from a changing magnetic flux.
\begin{itemize}
    \item An induced emf generates a current that creates a magnetic field that opposes the change in magnetic flux.
\end{itemize}

Faraday's law of induction is the third of Maxwell's equations.
\[ \epsilon = \oint \vec{E}\cdot\dd \vec{l} = \frac{\dd \varphi_B}{\dd t}\]

\section{Induced Currents and Magnetic Forces}
When an induced current is created in a conductive loop, the already present magnetic field will exert a magnetic force on the moving charge carriers in the loop.
\[ F_B = \int I(\dd \vec{l}\times \vec{B}) = ILB\cos\theta \]

When current is induced in a conducting loop, magnetic forces are only exerted on the segments of the loop that are within the external magnetic field.

The force on a conducting is proportional to the induced current in the loop, which depends on the rate of change of magnetic flux, the resistance 
of the loop and the velocity of the loop.

Newton's second law can be applied to a conducting loop moving in a magnetic field as it experiences an induced emf.
\section{Inductance}
Inductance is the tendency of a conductor to oppose a change in electrical current.
\begin{itemize}
    \item Depends on the physical properties of the conductor.
    \item An inductor, such as solenoid, is a circuit element that has significant inductance.
    \item The inductance of a solenoid is 
    \[ L_{sol} = \frac{\mu N^2 A}{l} \]
\end{itemize}

Inductors store energy in the magnetic field that is generated by current in the inductor.
\[ U_L = \frac{1}{2}LI^2 \]
The energy stored in the magnetic field generated by an inductor in which current is flowing can be dissipated through a resistor or used to charge a capacitor.

The transfer of energy generated in an inductor to other forms of energy obeys conservation laws.

By applying Faraday's law to an inductor and using the definition of inductance, induced emf can be related to inductance and the rate of change of current.
\[ \epsilon_i = -L \frac{\dd I}{\dd t} \]

\section{LR Circuits}
A resistor will dissipate energy that was stored in an inductor as the current charges.

Kirchhoff's loop rule can be applied to a series LR circuit with a bettery emf $\epsilon$, resulting in a differential equation that describes the current in the loop.
\[ \epsilon = IR + L\frac{\dd I}{\dd t}\]
The time constant is a significant feature of the behavior of an LR circuit.

The time constant of a circuit is a measure of how quickly an LR circuit will reach a steady state and is described with the equation 
\[ \tau = \frac{L}{R}\]

The electric properties of inductors change during the time interval in which the current in the inductor changes, but will 
exhibit steady state behavior after a long time interval.

\section{LC Circuits}
In circuits containing only a charged capacitor and an inductor (LC Circuits), the maximum current in the inductor 
can be determined using conservation of energy within the circuit.

In LC circuits, the time dependence of the charge stroed in the capacitor can be modeled as simple harmonic motion:
\[ \frac{\dd^2 q}{\dd t^2} = -\frac{I}{LC}q \]

The angular frequency of an oscillating LC circuit can be derived from the differnetial equation that describes an LC circuit.
\[ \omega = \frac{1}{\sqrt{LC}} \implies \omega = \frac{2\pi}{T} \implies T = \frac{2\pi}{\omega} = 2\pi\sqrt{LC}\]



\end{document}