\documentclass[../em.tex]{subfiles}
\graphicspath{{\subfix{../figures/}}}
\begin{document}
\chapter{Electric Circuits}
\section{Electric Current}
Current is the rate at which charge passes through a cross-sectional area of a wire.
\[I=\mathrm{d}q/\mathrm{d}t \implies q = \int I \mathrm{d}t\]
Current within a conductor consists charge carriers traveling through the conductor with an average drift velocity.
\[I=nqv_D A\]
Electric charge moves in a circuit in response to an electron potential difference, sometimes referred to as electromotive force, or emf ($\epsilon$)

Current density if the flow of charge per unit area.
\[I=\int{\vec{J}\cdot\mathrm{d}\vec{A}} \implies I = JA\]
Current density is related to the motion of the charge carriers within a conductor and is a vector quantity.
\[J=nqv_D\]
A potential difference across a conductor creates an electric field within the conductor that is proportional to the resistivity and the current density.
\[\vec{E}=\rho \vec{J}\]
If a function of current density is given, the total current can be determined by integrating the density over the area.
\[I=\int \vec{J(r)}\cdot\mathrm{d}\vec{A}\]gg6
Although current is a scalr quantity, it does have direction.
\begin{itemize}
    \item The direction of conventional current is chosen to be the direction in which positive charge would move.
    \item In common circuits, the current is actually due to the movement of electrons.
\end{itemize}

\begin{example}
    A long conducting cylinder has radius $R$, and carries a current to the left. The current density varies with distance 
    $r$ from the cylinder's central axis according to the equation $J=kr^2$, where $r\leq R$ and $k$ is a positive constant. Derive an expression
    for the total current in the cylinder.

    \begin{align*}
        I=\int\vec{J}(r)\cdot\mathrm{d}\vec{A} \\
        = \int (kr^2)(2\pi r\mathrm{d}r)
        = \frac{\pi k}{2}R^4
    \end{align*}
\end{example}

\section{Simple Circuits}
\section{Ohm's Law and Electrical Power}
\section{Compound Direct Current Circuits}
\section{Kirchoff's Rules}
\section{Resistor-Capacitor (RC) Circuits}
\end{document}