\documentclass[../em.tex]{subfiles}
\graphicspath{{\subfix{../figures/}}}
\begin{document}
\chapter{Magnetic Fields and Electromagnetism}
\section{Magnetic Fields}
A magnetic field is a vector field that can be used to determine the magnetic force exerted 
on moving electric charges, electric currents or magnetic materials.
\begin{itemize}
    \item Produced by magnetic dipoles or combinations of dipoles, but never by monopoles.
    \item Magnetic dipoles have north and south polarity.
\end{itemize}

A magnetic field can be represented by using vector field maps.

Magnetic field lines must form closed loops, as described by Gauss's law of magnetism.
\begin{itemize}
    \item Gauss's law of magnetism is Maxwell's second equation.
    \item $\oint \vec{B} \cdot \dd\vec{A} = 0$
\end{itemize}

Magnetic dipoles results from the circular of rotational motion of electric charges.

A magnetic dipole when placed in a magnetic field will align with the magnetic field.

A material's composition influences its magnetic behavior in the presence of an external magnetic field.

Magnetic permeability is a measurement of magnetization in a material in response to an external magnetic field.

Free space has a constant value of magnetic permeability, known as the vacuum permeability, $\mu_0$, that appears in equations representing physical relationships.
\[ \mu_0 = 4\pi \times 10^{-7} \text{T$\cdot$ m/A} \]
\section{Magnetism and Moving Charges}
A single moving charged object produces a magnetic field.
\begin{itemize}
    \item It is dependent on the object's velocity and the distance between the point and the object.
    \item The direction of the magnetic field is perpendicular to both the velocity and the position vector from the object. 
    \begin{itemize}
        \item Determined by using a right-hand rule.
    \end{itemize}
\end{itemize}

A magnetic field will exert a force on a charged object within that field, with magnitude and direction that depend on the cross-product of the charge's velocity and the magnetic field.
\[ F_B = q(\vec{V}\times \vec{B})=qvB\sin\theta \]

In a region containing both a magnetic field and an electric field, a moving charged particle will experience independent forces from each field.

The Hall effect describes the potential difference created in a conductor by an external magnetic field that has a component perpendicular to the direction of charges moving in the conductor.
\section{Biot-Savart Law}
The Biot-Savart law defines the magnitude and direction of a magnetic field created by an electric current.
\[ \dd \vec{B} = \frac{\mu_0}{4\pi} \frac{I(\dd l\times \hat{r})}{r^2} \]

The magnetic field vectors around a small segment of a current-carrying wire are tangent to concentric circles centered on that wire.

The Biot-Savart Law can be used to derive the magnitudes and directions of magnetic fields around segments of current-carrying wires.
\[ B_{loop} = \frac{\mu_0 I}{2R} \]

A magnetic field will exert a force on a current-carrying wire.
\[ F_B = \int I(\dd \vec{l}\times \vec{B}) = IlB\sin\theta \]

\section{Ampere's Law}
Ampere's law relates the magnitude of the magnetic field to the current enclosed by an imaginary path called an Amperian Loop.
\[\oint \vec{B}\cdot \dd \vec{l} = \mu_0 I_{enc} \]

Ampere's law can be used to determine the magnetic field near a long, straight current carrying wire.

All solenoids are assumed to very long, with uniform magnetic fields inside the solenoids and negligible magnetic fields outside the solenoids.

Ampere's law can be used to determine the magnetic field inside of a long solenoid.
\[ B_{sol} = \mu_0 nI\]

$n = \frac{\text{Turns}}{\text{Lenght}} = N/L$

An Amperian loop is a closed path arround a current-carrying conductor.

Ampere's law is the third of Maxwell's equations.
\[ \oint \vec{B} \cdot \dd \vec{l} = \mu_0 I + \mu_0 \epsilon_0 \frac{\dd\varphi_E}{\dd t}\]

\end{document}