\documentclass[../mech.tex]{subfiles}
\graphicspath{{\subfix{../figures/}}}
\begin{document}
\chapter{Work, Energy and Power}
\section{Translational Kinetic Energy}
An object's translational kinetic energy is given by the equation 
\[ K=\frac{1}{2}mv^2 \]
Translational kinetic energy is a scalar quantity.

Different observers may measure different values of the translational kinetic energy of an object, depending on the observer's frame of reference.
\section{Work}
Work is the amount of energy transferred into or out of a system by a force exerted on that system over a distance.
\begin{itemize}
    \item The work done by a conservative force is path-independent and only depends on the initial and final configurations of that system.
    \item The work done by a nonconservative force is path-dependent.
\end{itemize}

Work is scalar quantity that may be positive, negative or zero.

The work done on an object by a variable force is calculated as 
\[ W=\int_a^b \vec{F}(r)\cdot \dd \vec{r} \qquad [W=Fd\cos\theta] \]
The work-energy theorem states that the change in an object's kinetic energy is equal to the sum of the work being done by all forces exerted on the object.
\[ \delta K = W \]
Work is equal to the area under the curve of a graph of $F$ as a function of displacement.

\section{Potential Energy}
A system composed of two or more objects has potential energy if the objects within that system only interact with each other through conservative forces.

Potential energy is a scalar quantity associated with the position of objects within a system.

The definition of zero potential energy for a given system is a decision made by the observer considering the situation to simplify or otherwise assist in analysis.

The relationship between conservative forces exerted on a system and the system's potential energy is 
\[ \delta U = -\int \vec{F}(r)\cdot \dd \vec{r} \]
The conservative forces exerted on a single dimension can be determined using the slope of a system's potential energy with respect to position in that dimension, these forces point in the direction of decreasing potential energy.
\[ F_x = -\dd u(x)/\dd x\]
The potential energy of common physical systems can be described using the physical properties of that system.

\section{Conservation of Energy}
A system that contains objects that interact via conservative forces or that can change its shape reversibly may have both kinetic and potential energies.

Mechanical energy is the sum of a system's kinetic and potential energy.

A system may be selected so that the total energy of the system is constant.

If the total energy of a system changes, that change will be equivalent to the energy transferred into or out of the system.

Energy is conserved in all interactions.

If the work done on a selected system is zero and there are no nonconservative interactions within the system, the total mechanical energy of the system is constant.

If the work done on a selected system is nonzero, the energy is transferred between the system and the environment.

\section{Power}
Power is the rate at which energy changes with respect to time, either by transfer into or out of a system or by converstion from one type to another within the system.

Average power is the amount of energy being transferred or converted, divided by the time it took for that transfer to happen.

The instantaneous power delivered to an object by a force is given by the equation:
\[ p_{ins}=\frac{\dd E}{\dd t} \]

The instantaneous power delivered to an object by the component of a constant force parallel to the object's velocity can be described with the derived equation:
\[ p_{ins}=Fv\cos\theta \]


\end{document}