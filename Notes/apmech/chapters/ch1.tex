\documentclass[../mech.tex]{subfiles}
\graphicspath{{\subfix{../figures/}}}
\begin{document}
\chapter{Kinematics}
\section{Scalars and Vectors}
Scalars are quantities described by magnitude only, vectors are quantities described by both magnitude and direction.

Vectors can be visually modeled as arrows with appropriate direction and lengths proportional to their magnitudes.

Vectors can be expressed in unit vector notation or as a magnitude and a direction.
\begin{itemize}
    \item Unit vector notation can be used to represent vectors as the sum of their constituent components in the x, y, and z directions, denoted by $\hat{i}, \hat{j}$ and $\hat{k}$.
    \[ \vec{r} = A\hat{i}+B\hat{j}+C\hat{k}\]
    \item The position vector of a point is given by $\vec{r}$ and the unit vector in the direction of the position vector is denoted $\hat{r}$.
    \item A resultant vector is the vector sum of the addend vectors' components. 
    \[ \vec{C}=\vec{A}+\vec{B}=(A_x+B_y)\hat{i}+(A_y+B_y)\hat{j} \]
\end{itemize}

In a given one-dimensional coordinate system, opposite directions are denoted by opposite signs.
\section{Displacement, Velocity, and Acceleration}
When using the object model, the size, shape and internal configuration are ignored.
\begin{itemize}
    \item The object may be treated as a single point with extensive properties such as mass and charge.
\end{itemize}
Displacement is the change in an object's position: $\Delta x = x-x_0$

Averages of velocity and acceleration are calculated considering the initial and final states of an object over an interval of time.

Average velocity is the displacement of an object divided by the interval of time in which that displacement occurs:
\[ \vec{v}_{avg}=\frac{\Delta \vec{x}}{\Delta t}\]
Average acceleration is the change in velocity divided by the interval of time in which that change in velocity occurs.
\[ \vec{a}_{avg}=\frac{\Delta \vec{v}}{\Delta t}\]
As the time interval used to calculate the average value of a quantity approaches zero, the average value of that quantity approaches the value of the quantity that is instant, called the instantaneous value.
\begin{itemize}
    \item $\vec{v} = \frac{\dd x}{\dd t}$
    \item $\vec{a} = \frac{\dd v}{\dd t}$
\end{itemize}

Time dependent functions and instantaneous values of position, velocity and acceleration can be determined using differentiation and integration.

\section{Representing Motion}
Motion can be represented by motion diagrams, figures, graphs, equations and narrative descriptions.

For constant acceleration, three kinematics equations can used to describe the instantaneous linear motion in one dimension:
\begin{itemize}
    \item $v=v_0 + at$
    \item $x=x_0+v_0 t + \frac{1}{2}at^2$
    \item $v^2=v_0^2+2a\Delta x$
\end{itemize}

Near the surface of the Earth, the vertical acceleration caused by the force of gravity is downward, constant and has a measured value of $g=9.8$ m/s$^2$ or $g=10$ m/s$^2$.

Graphs of position, velocity and acceleration as functions of time can be used to find the relationships between those quantities.

\section{Reference Frames and Relative Motion}
The choice of reference frame will determine the direction and magnitude of quantities measured by an observer in that reference frame.

Measurements from a given reference frame may be converted to measurements from another reference frame.

The observed velocity of an object results from the combination of the object's velocity and the velocity of the observer's reference frame.
\begin{itemize}
    \item Combining the motion of an object and the motion of an observer in a given reference frame involves the addition or subtraction of vectors.
    \item The acceleration of any object is the same as measured from all inertial reference frames.
\end{itemize}

\section{Motion in Two or Three Dimensions}
Motion in two or three dimensions can be analyzed using one-dimensional kinematic relationships if the motion is separated into componenets.

Velocity and acceleration may be different in each dimension and be nonuniform.

Motion in one dimension may be changed without causing a change in the perpendicular dimension.

Projectile motion is a special case of two-dimensional motion that has zero acceleration in one dimension and constant, nonzero acceleration in the second dimension.
\end{document}