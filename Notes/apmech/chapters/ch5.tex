\documentclass[../mech.tex]{subfiles}
\graphicspath{{\subfix{../figures/}}}
\begin{document}
\chapter{Torque and Rotational Dynamics}
\section{Rotational Kinematics}
Angular displacement is the measurement of the angle, in radians, through which a point on a rigid system rotates about a specific axis.
\begin{itemize}
    \item In general, the counterclockwise motion is positive, and the clockwise motion is negative.
\end{itemize}

Angular velocity is the rate at which angular displacement position changes with respect to time.
\[ \omega = \frac{\dd \theta}{\dd t} \]

Angular acceleration is the rate at which angular velocity changes with respect to time.
\[ \alpha = \frac{\dd \omega}{\dd t} \]

Angular displacement, angular velocity, and angular acceleration around one axis are analogous to linear displacement, velocity, and acceleration in one dimension and demonstrate the same mathematical relationships.

Graphs of angular displacement, angular velocity, and angular acceleration as functions of time can be used to find the relationships between the above quantitites.

\begin{example}
    A windmill is spinning because of the nonuniform force of the wind. The windmill is originally spinning at a speed of $\omega_0$, and a crosswind slows it with an angular acceleration 
    of $-A\omega^2$. What will be the angular speed of the windmill at time $t=T$?

    From $\alpha = -A\omega^2$, we get that $\frac{1}{\omega^2}=-A$.

    Integrating this from $\omega_0$ to $\omega$ gives us $\omega = \frac{\omega_0}{\omega_0 AT + 1}$.
\end{example}

\ex Two disks, Disk 1 and Disk 2, are initially at rest and begin to rotate with a constant angular acceleration. Disk 1 has an angular acceleration $\alpha_1$ and rotates through an angle 
$\theta_1$ in a time $\Delta t$. Disk 2 has an angular acceleration $\alpha_2 = 2a_1$ and rotates through an angle $\theta_2$ in the same amount of time $\Delta t$. What is $\theta_2$ in terms of $\theta_1$?

\ex A wheel spins with an initial angular velocity of 18 rad/s in the clockwise direction and a constant angular acceleration. After 3 seconds the wheel is spinning at 6 rad/s in the counterclockwise direction. What is the magnitude and direction of the angular acceleration?

\section{Connecting Linear and Rotational Motion}
\section{Torque}
\section{Rotational Inertia}
\section{Rotational Equilibrium and Newton's First Law in Rotational Form}
\section{Newton's Second Law in Rotational Form}

\end{document}