\documentclass[../discrete.tex]{subfiles}
\graphicspath{{\subfix{../figures/}}}
\begin{document}
\chapter{Discrete Probability}
An experiment is a procedure that yields one of a given set of possible outcomes. 
The sample space of the experiment is the set of possible outcomes. An event is a subset of the sample space.
Laplace's definition of the probability of an event with finitely many possible outcomes will now be stated.

\begin{definition}
    If $S$ is a finite nonempty sample space of equally likely outcomes, and $E$ is an event, 
    that is, a subset of $S$, then the probability of $E$ is $p(E)=\frac{\mid E \mid}{\mid S \mid}$.
\end{definition}
The probability of an event can never be negative or more than one!

We can use counting techniques to find the probability of events derived from other events.
\begin{theorem}
    Let $E$ be an event in a sample space of $S$. The probability of the event $\overline{E}=S-E$, the complementary event of $E$, is given by 
    \[p(\overline{E})=1-p(E)\]
\end{theorem}

We can also find the probability of the union of two events.
\begin{theorem}
    Let $E_1$ and $E_2$ be events in the sample space $S$. Then 
    \[p(E_1\cup E_2)=p(E_1)+p(E_2)-p(E_1\cap E_2)\]
\end{theorem}


\end{document}