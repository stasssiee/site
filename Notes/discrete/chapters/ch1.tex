\documentclass[../discrete.tex]{subfiles}
\graphicspath{{\subfix{../figures/}}}
\begin{document}
\chapter{Logic and Proofs}
\section{Propositional Logic}
A proposition is a declarative statement that is either true or false.

For example, we can say there is a proposition $p$ and $q$, such as ``The sky is blue'' or ``The moon is made of cheese''. Something that is not a propositional statement is a statement that is neither true or false, such as ``Sit down'', which is just a statement. $x+1=2$ is not a propositional statement because it is either true or false, unless you assign a value for $x$.

A compound proposition is comprised of propositions and one or more of the following connectives:
\begin{itemize}
    \item negation: $\neg$ ``NOT'' ($\neg p$)
    \item conjunction: $\land$ ``AND'' ($p\land q$)
    \item disjunction: $\lor$ ``OR'' ($p\lor q$)
    \item implication: $\rightarrow$ ``IF, THEN'' ($p\rightarrow q$)
    \item biconditional $\leftrightarrow$ ``IF AND ONLY IF'' ($p\leftrightarrow q$)
\end{itemize}
Each proposition is represented by a propositional variable.

The negation of the proposition $p$ is $\neg p$ (not $p$). For example, If $p$ denotes ``The grass is green'', then $\neg p$ denotes ``it is not the case that the grass is green'', or ``the grass it not green''.
\begin{example}
    If $p$ is the statement ``my dog is the cutest dog'', then $\neg p$ is the statement ``my dog is not the cutest dog''.
\end{example}
\ex Negate the statement ``the door is not open.''
\ex Can you negate the statement ``Are we there yet?''

Truth Tables: Each row of a truth table gives us one possibility for the truth values of our proposition(s). Since each proposition has two possible truth values, true or false, we will have 2 rows for each proposition (or $2^n$ rows where $n$ is the number of propositions).

For $\neg p$, a truth table would look like 
\begin{tabular}{c|c}
    $p$ & $\neg p$\\ \hline
    T & F\\
    F & T
\end{tabular}
The left side is the combinations, and the right side is the connectives. For example, if we were looking at a truth table of $p\lor q$, the combinations would be $p$ and $q$, and the connectives would be $p\lor q$.

Conjunction: The conjunction of propositions $p$ and $q$ is denoted $p\land q$ and read $p$ ``and'' $q$. For a conjunction to be true, both propositions must be true.
The truth table looks like the following. 
\[\begin{tabular}{c|c|c}
    $p$ & $q$ & $p\land q$ \\ \hline 
    T & T & T \\\hline
    T & F & F\\\hline
    F&T&F\\\hline 
    F&F&F
\end{tabular}\]

Disjunction: The disjunction of propositions $p$ and $q$ is denoted $p\lor q$ and read $p$ ``or'' $q$. For a disjunction to be true, either proposition must be true.
\[ \begin{tabular}{c|c|c}
    $p$ & $q$ & $p\lor q$ \\ \hline
    T & T & T \\\hline
    T&F&T\\\hline
    F&T&T\\\hline
    F&F&F
\end{tabular}\]

The connective ``OR'' in English ``XOR''. The inclusive or is a disjunction $p\lor q$. The exclusive or is $p\oplus q$. Basically this is true only if one of the inputs is true, but not both or neither.
\[ \begin{tabular}{c|c|c|c}
    $p$ & $q$ & $p\lor q$ & $p\oplus q$\\ \hline
    T & T & T & F\\\hline
    T&F&T & T\\\hline
    F&T&T&T \\\hline
    F&F&F&F
\end{tabular}\]

Implication (Conditional Statement): The implication of propositions $p$ and $q$ is denoted $p\rightarrow q$ and read ``if $p$ then $q$'' or ``$p$ implies $q$''. 
When the hypothesis is true, the conclusion must be true for the implication to be true. WHen the hypothesis is false, the conclusion is true.
\[ \begin{tabular}{c|c|c}
    $p$ & $q$ & $p\rightarrow q$\\\hline 
    T & T & T\\\hline
    T & F & F\\\hline
    F&T&T\\\hline
    F&F&T 
\end{tabular}\]

From the implication $p\rightarrow q$, we can construct 3 news conditional statements.
\begin{itemize}
    \item Converse: $q\rightarrow p$ (Switch Order)
    \item Inverse: $\neg p\rightarrow \neg q$ (Negate),
    \item Contrapositive: $\neg q\rightarrow \neg p$ (Switch and Negative)
\end{itemize}
Contrapositives have the same truth value as $p\rightarrow q$.

\begin{example}
    Give the converse, inverse, and contrapositive of the implication: If your homework is complete, then your teacher is happy.

    The converse is, if the teacher is happy, then you completed your homework.

    The inverse is if you didn't complete your homework, then your teacher is not happy.

    The contrapositive is if your teacher is not happy, then you did not complete your homework.
\end{example}

Biconditional: The biconditional of propositions $p$ and $q$ is denoted $p\leftrightarrow q$ and read ``$p$ if and only if $q$''. 
For a biconditional to be true, both propositions must share the same truth value.
\[ \begin{tabular}{c|c|c}
    $p$ & $q$ & $p\leftrightarrow q$\\\hline
    T & T & T\\\hline
    T&F&F\\\hline
    F&F&T
\end{tabular}\]

The biconditional $p\leftrightarrow q$ can also be written as a compound proposition.
\[ (p\leftrightarrow q)\equiv (p\rightarrow q)\land (q\rightarrow p) \]
Can you show this?

The steps for constructing a truth table for compound propositions are as follows:

You need rows and columns. You need a row for every possible combination of values for the compound proposition. You need a column for each propositional variable, you need a column for the truth value of each expression that occurs in the compound proposition as it is built up, and you need a column for the compound proposition. The order of operations is as follows: $\neg, \land,\lor,\rightarrow,\leftrightarrow$.

\begin{example}
    Create a truth table for $(p\lor \neg q)\rightarrow (p\land q)$.

    \[ \begin{tabular}{c|c|c|c|c|c}
        $p$ & $q$ & $\neg q$ & $p\lor \neg q$ & $p\land q$ & $(p\lor\neg q)\rightarrow (p\land q)$\\\hline
        T & T & F & T & T & T\\\hline
        T&F&T&T&F&F\\\hline
        F&T&F&F&F&T\\\hline
        F&F&T&T&F&F
    \end{tabular}\]
\end{example}


\section{Propositional Equivalences}
Translating English Sentences:
\begin{enumerate}
    \item Identify atomic propositions.
    \item Determine appropriate logical connectives.
\end{enumerate}

For example consider the statement ``If I go to the store or the movies, I won't do my homework.''.

In that statement $p$ is going to the store, $q$ is going to the movies, and $r$ is doing homework, so $(p\lor q)\rightarrow \neg r$.

\ex Translate the propositional statement ``You can get a free sandwich on thursday if you buy a sandwich or a cup of soup.''

\ex Translate the propositional statement ``You can get a free sandwich on thursday only if you buy a sandwich or a cup of soup.''

\ex Translate the propositional statement ``The automated reply can't be sent when the system is full.'

Translating Propositions:
\begin{example}
    Let $q$ be ``You can ride the rollercoaster'', $r$ be ``You are under 4 feet tall'', and $s$ be ``you are older than 16 years old.''

    Translating $(r\lor \neg s)\rightarrow \neg q$ to english gives us, If $r$ or not $s$, then not $q$, or ``If you are under 4 feet tall or younger than 16 years old then you cannot ride the rollercoaster.''
\end{example}


\begin{example}
    An island has two kinds of inhabitants, knights, who always tell the truth, and knaves, who always lie. You go to the island and meet A and B. A says ``B is a knight''. B says, ``The two of us are opposite types.'' What are A and B? Let $p$ represent that A is a knight and $q$ represent that B is a knight.

    Using the propositional statements, $p\land q$, $p\land \neg q$, $\neg p \land q$, and $\neg p \land \neg q$, we can see that $\neg p \land \neg q$ satisfies the conditions that A and B are both knaves. (Note that this can also be done with a truth table.)
\end{example}

\ex When planning a party, you want to know whom to invite. Among the poeple you would like to invite are three touchy friends. You know that if Jasmine attends, she will become unhappy if Samir is there. Samir will attend only if Kanti will be there, and Kanti will not attend unless Jasmine also does. Which combinations of these three friends can you invite so as not to make someone unhappy?

Logic circuit - A circuit designed to perform complex tasks designed in terms of elementary logic functinos. Three `gates' are used used implementing a boolean function on one or more binary inputs producing one output.

The ``NOT GATE'' will have one input $p$, and output $\neg p$. The ``OR GATE'' will have two inputs, and output the disjunction of those two inputs. The ``AND GATE'' will have two inputs, and output the conjunction of the propositions. (Search up images yourself)

\section{Predicates and Quantifiers}
A tautology is a proposition which is always true. Ex. $p\lor \neg p$

A contradiction is a proposition which is always false. Ex. $p\land \neg p$

A contingency is a proposition which is neither a tautology nor a contradiction. Ex. $p$

Two compound propositions $p$ and $q$ are logically equivalent if $p\leftrightarrow q$ is a tautology, meaning they have the same truth value in all possible cases. This is written as $p\equiv q$, where $p$ and $q$ are compound propositions.

For example, the truth table below shows $\neg p\lor q\equiv p\rightarrow q$.
\[ \begin{tabular}{c|c|c|c|c}
    $p$ & $q$ & $\neg p$ & $\neg p\lor q$ & $p\rightarrow q$\\\hline
    T&T&F&T&T\\\hline
    T&F&F&F&F\\\hline
    F&T&T&T&T\\\hline
    F&F&T&T&T
\end{tabular}\]

\ex Determine if $\neg p\land q\equiv \neg p\lor \neg q$.

\ex Determine if $p\lor (q\land r)\equiv (p\lor q)\land (p\lor r)$.

Identity Laws:
\begin{itemize}
    \item $p\land T\equiv p$
    \item $p\lor F\equiv p$
\end{itemize}

Domination Laws:
\begin{itemize}
    \item $p\lor T\equiv T$
    \item $p\land F\equiv F$
\end{itemize}

Indempotent Laws:
\begin{itemize}
    \item $p\lor p\equiv p$
    \item $p\land p\equiv p$
\end{itemize}

Double Negation Law: $\neg(\neg p)\equiv p$

Absorbtion Laws:
\begin{itemize}
    \item $p\lor (p\land q)\equiv p$
    \item $p\land (p\lor q)\equiv p$
\end{itemize}

Negation Laws:
\begin{itemize}
    \item $p\lor \neg p\equiv T$
    \item $p\land \neg p\equiv F$
\end{itemize}

Commutative Laws:
\begin{itemize}
    \item $p\lor q\equiv q\lor p$
    \item $p\land q\equiv q\land p$
\end{itemize}

Associative Laws:
\begin{itemize}
    \item $(p\lor q)\lor r\equiv p\lor(q\lor r)$
    \item $(p\land q)\land r\equiv p\land(q\land r)$
\end{itemize}

Distributive Laws:
\begin{itemize}
    \item $p\lor (q\land r)\equiv(p\lor q)\land (p\lor r)$
    \item $p\land (q\lor r)\equiv (p\land q)\lor (p\land r)$
\end{itemize}

De Morgan's Laws:
\begin{itemize}
    \item $\neg(p\land q)\equiv \neg p\lor \neg q$
    \item $\neg(p\lor q)\equiv \neg p\land \neg q$
\end{itemize}

More equivalencies:
\begin{itemize}
    \item $p\rightarrow q\equiv \neg p \lor q$
    \item $p\rightarrow q\equiv \neg q\rightarrow \neg p$
    \item $p\lor q\equiv \neg p\rightarrow q$
    \item $p\land q\equiv \neg(p\rightarrow \neg q)$
    \item $(p\rightarrow q)\land (p\rightarrow r)\equiv p\rightarrow (q\land r)$
    \item $(p\rightarrow r)\land (q\rightarrow r)\equiv (p\land q)\rightarrow r$
    \item $(p\rightarrow q)\lor (p\rightarrow r)\equiv p\rightarrow (q\lor r)$
    \item $(p\rightarrow r)\lor (q\rightarrow r)\equiv (p\land q)\rightarrow r$
    \item $p\leftrightarrow q\equiv (p\rightarrow q)\land (q\rightarrow p)$
    \item $p\leftrightarrow q\equiv \neg p\leftrightarrow \neg q$
    \item $p\leftrightarrow q\equiv (p\land q)\lor (\neg p\land \neg q)$
    \item $\neg(p\leftrightarrow q)\equiv p\leftrightarrow \neg q$
\end{itemize}

We can construct new logical equivalences.
\begin{example}
    Show $\neg(p\lor (\neg p\land q))\equiv \neg p\land \neg q$ by developing a series of logical equivalences 
    
    The original statement is equivalent to $\neg p\land \neg(\neg p\land q)$ from the 2nd demorgan law.
    
    By the 1st De Morgan law, this is $\neg p\land \neg\neg p\lor \neg q$.

    Double negation: $\neg p\land (p\lor \neg q)$

    2nd Distributive Law: $(\neg p\land p)\lor (\neg p\land \neg q)$

    2nd Negation Law: $F\lor (\neg p\land \neg q)$

    Commutative Law for Disjunction: $(\neg p\land \neg q)\lor F$

    Identity Law: $\neg p\land \neg q$.
\end{example}

\ex Show $(p\land q)\rightarrow (p\lor q)$ is a tautology by developing a series of logical equivalences.

\ex Use logical equivalences to show $\neg(\neg p\lor q)\equiv \neg q\land p$


\section{Nested Quantifiers}
If I say,
\begin{itemize}
    \item All candy made with chocolate is delicious 
    \item M \& M's are made with chocolate 
\end{itemize}
Does it follow that M\&M's are delicious?

We can't model this relationship with propositions.

This is where we need predicate logic which include:
\begin{itemize}
    \item Variables: $x$, $y$, $z$, these are the subjects of the statement(s)
    \item Predicates: a property the variable can have 
    \item Quantifiers: covered later 
\end{itemize}

Statements involving variables, such as ``$x<2$'' and ``$x+y=z$'' are often found in mathematical assertions, in computer programs, and in system specifications. The statements are neither true nor false when the values of the variables aren't specified.

The statement ``x is less than 2'' has two parts. First, the variable $x$ is the subject of the statement. The second part, the predicate, ``is less than 2'' refers to the property that the subject of our statement can have. The predicate, ``is less than 2'' can be denoted by $P(x)$, where $P$ denotes the predicate and $x$ the variable.

Propositional functions become propositions (and have truth values) when their variables are each replaced by a value from the domain (or bound by a quantifier, as we will see later).

The statement $P(x)$ is said to be the value of the propositional function $P$ at $x$.

For example, let $P(x)$ denote ``$x>0$'' and the domain be the integers. Then, $P(-3)$ is false, $P(0)$ is false, and $P(3)$ is true.

Often the domain is denoted by $U$. So in this example, $U$ is the integers.

\begin{example}
    Let ``$x+y=z$'' be denoted by $R(x,y,z)$ and $U$ (for all three variables) be the integers. Find these truth values:

    (a) $R(2,-1,5)$

    FALSE, because $2+-1=5$ is false 

    (b) $R(3,4,7)$

    TRUE, because $3+4=7$

    (c) $R(x,3,z)$

    This is not a proposition 
\end{example}

Connectives from propositional logic carry over the predicate logic.

If $P(x)$ denotes ``$x>0$'', find these truth values:
\begin{itemize}
    \item $P(3)\lor P(-1)$: True 
    \item $P(3)\land P(-1)$: False 
    \item $P(3)\rightarrow P(-1)$: False 
    \item $P(3)\rightarrow \neg P(-1)$: True 
\end{itemize}

Expressions with variables are not propositions and therefore do not have truth values. For example, 
\begin{center}
    $P(3)\land P(y)$ \\
    $P(x)\rightarrow P(y)$
\end{center}

When used with quantifiers (see below), these expressions (propositional functions) become propositions.

As we know, a propositional function $P(x)$ is not a proposition until it has a truth value. Up to this point, we could only do this by assigning a value to our variable.

For example, if $P(x)$ represents ``$x>0$'', the truth value for $P(4)$ returns true.

Now we will turn a propositional function into a proposition using a quantifier.

Let's focus on the two most widely used quantifiers:
\begin{itemize}
    \item ``forall'': $\forall$ (Universal Quantifier)
    \item ``there exists'': $\exists$ (Existential Quantifier)
\end{itemize}

The statement, $\forall xP(x)$ tells us that the proposition $P(x)$ must be true for all values of $x$ in the domain of discourse/universe.
\begin{example}
    Let $P(x)$ represent ``$x>0$''. Find each truth value for $\forall xP(x)$

    (a) $U$ is $\mathbb{Z}$

    False, because negative numbers exist.

    (b) $U$ is $\mathbb{Z}^+$

    True 
\end{example}

The statement, $\exists xP(x)$, tells us that the proposition $P(x)$ is true for some value(s) of $x$ in the domain of discourse/universe.

\begin{example}
    Same example as above, but the truth value is for $\exists xP(x)$.

    (a) $U$ is $\mathbb{Z}$

    True 

    (b) $U$ is $\mathbb{Z}^-$

    False 
\end{example}

\section{Rules of Inference}

\section{Introduction to Proofs}

\section{Proof Methods and Strategy}

\end{document}