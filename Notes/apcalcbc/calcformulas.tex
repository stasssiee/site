\documentclass[10pt,a4paper,oneside]{book}

\title{AP Calculus BC Formulas}
\author{anastasia}

\usepackage[utf8]{inputenc}
\usepackage[margin=1.0in]{geometry}
\usepackage{amsmath}
\usepackage{amsfonts}
\usepackage{amssymb}
\usepackage{enumitem}                       % custom enum labels
\usepackage{parskip}                        % add vertical paragraph space
\usepackage{tocloft}						% modify toc position
\usepackage{xr}								% cross-references
\usepackage{mathtools}						% Aboxed
\usepackage{empheq}							% box multiple lines
\usepackage{upgreek}
\usepackage{gensymb}
\usepackage{chemformula}
\usepackage{esint}							% oiint
\usepackage{cancel}
\usepackage{tikz}
\usetikzlibrary{calc}
\usepackage{asymptote}
\usepackage[framemethod=TikZ]{mdframed}     % graphics and framed envs
\usepackage[hang,flushmargin]{footmisc}		% remove footnote indentation
\usepackage[hyperfootnotes=false,
					hidelinks]{hyperref}	% create clickable table of contents
\usepackage{cancel}

\newcommand{\tarc}{\mbox{\large$\frown$}}
\newcommand{\arc}[1]{\stackrel{\tarc}{#1}}
%\newcommand{\degree}{^{\circ}}
\newcommand{\blank}{\_\_\_\_\_\_}

\DeclareMathOperator\cis{cis}
\DeclareMathOperator\Arg{Arg}


\renewcommand{\familydefault}{\sfdefault}  	% sans serifs text
\setlength{\parindent}{0pt}                	% no paragraph indentation

% region TITLES
%\setbox0=\hbox{\Huge{\textbf{\textsf{\courseid: }}}}
\setlength{\cftbeforetoctitleskip}{0em}
\setlength{\cftaftertoctitleskip}{1em}
%\renewcommand{\contentsname}{\hangindent=\wd0 \strut \courseid: \coursetitle \\ \medskip {\professor, \campus, \semester}}

% region COMMANDS
\newcommand{\ds}{\displaystyle}
\newcommand{\pfn}[1]{\textrm{#1}}	% enables new functions
\newcommand{\mbf}[1]{\mathbf{#1}}	% mathbf
\newcommand{\C}{\mathbb{C}}			% fancy C
\newcommand{\R}{\mathbb{R}}			% fancy R
\newcommand{\Q}{\mathbb{Q}}			% fancy Q
\newcommand{\Z}{\mathbb{Z}}			% fancy Z
\newcommand{\N}{\mathbb{N}}			% fancy N
\newcommand{\dd}{\mathrm{d}}
\newcommand{\from}{\leftarrow}
\newcommand{\qed}{$\square$}
\newcommand{\ex}{\textit{Exercise }}
% endregion

\let\footnoterule\relax
\newcommand\blfootnote[1]{%
  \begingroup
  \renewcommand\thefootnote{}\footnote{#1}%
  \addtocounter{footnote}{-1}%
  \endgroup
}

\usepackage{titlesec}
\titleformat{\chapter}{\Huge\normalfont\bfseries}{\thechapter}{1em}{}
\titlespacing*{\chapter}{0em}{0em}{2em}
% endregion

% region ENVIRONMENTS
\newcounter{theo}[chapter]\setcounter{theo}{0}
\newcommand{\numTheo}{\arabic{chapter}.\arabic{theo}}
\newcommand{\mdftheo}[3]{
	\mdfsetup{
		frametitle={
			\tikz[baseline=(current bounding box.east),outer sep=0pt]
			\node[anchor=east,rectangle,fill=#3]
			{\ifstrempty{#2}{\strut #1~\numTheo}{\strut #1~\numTheo:~#2}};
		},
		innertopmargin=4pt,linecolor=#3,linewidth=2pt,
		frametitleaboveskip=\dimexpr-\ht\strutbox\relax,
		skipabove=11pt,skipbelow=0pt
	}
}
\newcommand{\mdfnontheo}[3]{
	\mdfsetup{
		frametitle={
			\tikz[baseline=(current bounding box.east),outer sep=0pt]
			\node[anchor=east,rectangle,fill=#3]
			{\ifstrempty{#2}{\strut #1}{\strut #1:~#2}};
		},
		innertopmargin=4pt,linecolor=#3,linewidth=2pt,
		frametitleaboveskip=\dimexpr-\ht\strutbox\relax,
		skipabove=11pt,skipbelow=0pt
	}
}
\newcommand{\mdfproof}[1]{
	\mdfsetup{
		skipabove=11pt,skipbelow=0pt,
		innertopmargin=4pt,innerbottommargin=4pt,
		topline=false,rightline=false,
		linecolor=#1,linewidth=2pt
	}
}


\newenvironment{theorem}[1][]{
	\refstepcounter{theo}
	\mdftheo{Theorem}{#1}{red!25}
	\begin{mdframed}[]\relax
}{\end{mdframed}}

\newenvironment{lemma}[1][]{
	\refstepcounter{theo}
	\mdftheo{Lemma}{#1}{red!15}
	\begin{mdframed}[]\relax
}{\end{mdframed}}

\newenvironment{corollary}[1][]{
	\refstepcounter{theo}
	\mdftheo{Corollary}{#1}{red!15}
	\begin{mdframed}[]\relax
}{\end{mdframed}}

\newenvironment{definition}[1][]{
	\mdfnontheo{Definition}{#1}{blue!20}
	\begin{mdframed}[]\relax
}{\end{mdframed}}

\newenvironment{exercise}[1][]{
	\mdfproof{black!15}
	\textit{Exercise. }
}

\newenvironment{proof}[1][]{
	\mdfproof{black!15}
	\begin{mdframed}[]\relax
\textit{Proof. }}{\qed \end{mdframed}}

\newenvironment{claim}[1][]{
	\mdfproof{black!15}
	\begin{mdframed}[]\relax
\textit{Claim. }}{\end{mdframed}}

\newenvironment{example}[1][]{
	\mdfnontheo{Example}{#1}{yellow!40}
	\begin{mdframed}[]\relax
}{\end{mdframed}}

\newenvironment{summary}[1][]{
	\mdfnontheo{Summary}{#1}{green!70!black!30}
	\begin{mdframed}[]\relax
}{\end{mdframed}}
% endregion
\graphicspath{{figures/}}

\begin{document}
\maketitle

\textbf{Average Rate of Changing over $[a,b]$:} $\frac{f(b)-f(a)}{b-a}$

\textbf{Limits at a point:}

If $L$, $M$, $c$, and $k$ are real numbers and $\lim_{x\to c} f(x) = L$ and $\lim_{x\to c}g(x) = M$, then the following properties are true:
\begin{itemize}
    \item Limit of a constant: $\lim_{x\to c}k=k$
    \item Limit of $x$: $\lim_{x\to c}x=c$
    \item Sum rule: $\lim_{x\to c}(f(x)+g(x))=L+M$
    \item Difference rule: $\lim_{x\to c}(f(x)-g(x))=L-M$
    \item Product rule: $\lim_{x\to c}(f(x)\cdot g(x))=L\cdot M$
    \item Constant multiple rule: $\lim_{x\to c}(k(f(x)))=k\cdot L$
    \item Quotient rule: $\lim_{x\to c}\frac{f(x)}{g(x)}=\frac{L}{M}, M\neq 0$
    \item Power rule: $\lim_{x\to c}(f(x))^{r/s}=L^{r/s}$, if $r$ and $s$ are integers, and $s\neq 0$
    \item Limit of a composite function: $\lim_{x\to c}f(g(x))=f(\lim_{x\to c}g(x))$, if $f$ is a continuous function
\end{itemize}

\textbf{Properties of limits as $x\rightarrow \pm \infty$}

If $L$, $M$, $c$, and $k$ are real numbers and $\lim_{x\to \pm\infty}f(x)=L$ and $\lim_{x\to \pm\infty}g(x)=M$, then the following properties are true:
\begin{itemize}
    \item Constant rule: $\lim_{x\to\pm\infty}c=c$
    \item Sum rule: $\lim_{x\to\pm\infty}(f(x)+g(x))=L+M$
    \item Difference rule: $\lim_{x\to\pm\infty}(f(x)-g(x))=L-M$
    \item Product rule: $\lim_{x\to\pm\infty}(f(x)\cdot g(x))=L\cdot M$
    \item Constant multiple rule: $\lim_{x\to\pm\infty}(k(f(x)))= k\cdot L$
    \item Quotient rule: $\lim_{x\to\pm\infty}\frac{f(x)}{g(x)}=\frac{L}{M}$, $M\neq 0$
    \item Power rule: $\lim_{x\to\pm\infty}(f(x))^{r/s}=L^{r/s}$, if $r$ and $s$ are integers and $s\neq 0$
    \item Limit of $\frac{c}{x^r}$: $\lim_{x\to\pm\infty}\frac{c}{x^r}=0$
\end{itemize}

\textbf{Squeeze Theorem}

Conditions: 
\begin{itemize}
    \item $g(x)\leq f(x)\leq h(x)$ for $x\neq c$
    \item $\lim_{x\to c}g(x)=L$ and $\lim_{x\to c}h(x)=L$
\end{itemize}

Conclusion: $\lim_{x\to c}f(x)=L$

\textbf{Definition of Continuity}

A function $f(x)$ is continuous at $x=c$ if all of the following conditions are met:
\begin{itemize}
    \item $f(c)$ is defined
    \item $\lim_{x\to c}f(x)$ exists 
    \item $\lim_{x\to c}f(x)=f(c)$
\end{itemize}
The graph of a continuous function has no ``gaps''.

\textbf{Intermediate Value Theorem}

If a function $f$ is continuous on the interval $[a,b]$ and $k$ is a number between $f(a)$ and $f(b)$, then there is at least one $x$-value $c$ between $a$ and $b$ 
such that $f(c)=k$.

Any continuous function connecting $(a,f(a))$ and $(b,f(b))$ must pass through every $y$-value between $f(a)$ and $f(b)$ at least once.

\textbf{Limit Definitions of the Derivative}
Derivative of $f$ at $x=a$: $f'(a)=\lim_{x\to a}\frac{f(x)-f(a)}{x-a}$

Defintion of derivative of $f$ at $x=a$: $f'(a)=\lim_{h\to 0}\frac{f(a+h)-f(a)}{h}$

\textbf{Definition of Differentiability}

$f$ is differentiable at $x=c$: $\lim_{x\to c}\frac{f(x)-f(c)}{x-c}$ exists and is equal to $f'(c)$. $\frac{f(x)-f(c)}{x-c}$ is the difference quotient.

\textbf{Derivative Rules}
Basic:
\begin{itemize}
    \item Constant: $\frac{\dd}{\dd x}[c]=0$
    \item Power: $\frac{\dd}{\dd x}[x^n]=nx^{n-1}$
    \item Natural exponential: $\frac{\dd}{\dd x}[e^x]=e^x$
    \item Exponential: $\frac{\dd}{\dd x}[a^x]=(\ln a)a^x$
    \item Natural log: $\frac{\dd}{\dd x}[\ln(x)]=\frac{1}{x}$
    \item Constant multiple: $\frac{\dd}{\dd x}[cf(x)]=cf'(x)$
    \item Sum and difference: $\frac{\dd}{\dd x}[f(x)\pm g(x)]=f'(x)\pm g'(x)$
\end{itemize}

Trig:
\begin{itemize}
    \item $\frac{\dd}{\dd x}[\sin x]=\cos x$
    \item $\frac{\dd}{\dd x}[\cos x]=-\sin x$
    \item $\frac{\dd}{\dd x}[\tan x]=\sec^2 x$
    \item $\frac{\dd}{\dd x}[\cot x]=-\csc^2 x$
    \item $\frac{\dd}{\dd x}[\csc x]=-\csc(x)\cot(x)$
    \item $\frac{\dd}{\dd x}[\sec x]=\sec(x)\tan(x)$
\end{itemize}

\textbf{Product Rule}: $\frac{\dd}{\dd x}[uv]=uv'+vu'$

\textbf{Quotient Rule}: $\frac{\dd}{\dd x}[\frac{u}{v}]=\frac{vu'-uv'}{v^2}$

\textbf{Chain Rule}: $\frac{\dd}{\dd x}[f(g(x))] = f'(g(x))\cdot g'(x)$

\textbf{Derivatives of Inverse Functions}: $(f^{-1})'(a)=\frac{1}{f'(b)}$

\textbf{PVA}
Derivatives:
\begin{itemize}
    \item Position: $x(t)$
    \item Velocity: $v(t)=x'(t)$
    \item Acceleration: $a(t)=x''(t)$
\end{itemize}

Integrals:
\begin{itemize}
    \item Integrate $a(t)$ to get $v(t)$
    \item Integrate $v(t)$ to get $s(t)$
\end{itemize}

\textbf{L'Hospital's Rule}

Use L'Hospital's Rule to find the limit of the ratio of two differentiable functions $\frac{f(x)}{g(x)}$ as $x$ approaches $c$.
If direct substitution produces one of the indeterminate forms $\frac{0}{0}$ or $\frac{\infty}{\infty}$, then differentiate the numerator $f$ and the denominator $g$ independently.
\[ \lim_{x\to c}\frac{f(x)}{g(x)}=\lim_{x\to c}\frac{f'(x)}{g'(x)} \]
L'Hospital's Rule also applies to limits such as $x\rightarrow \infty$ or $x\rightarrow -\infty$

\textbf{Mean Value Theorem}

Conditions:
\begin{itemize}
    \item $f$ is continuous on $[a,b]$
    \item $f$ is differentiable on $(a,b)$
\end{itemize}

Conclusion: For some $c$ in $(a,b)$: $f'(c)=\frac{f(b)-f(a)}{b-a}$. $f'(c)$ is the instantaneous rate of change at $x=c$ and $\frac{f(b)-f(a)}{b-a}$ is the average rate of change on $[a,b]$.

\textbf{Rolle's Theorem}

If a function $f$ satisfies each of the following conditions:
\begin{itemize}
    \item continuous on the closed interval $[a,b]$
    \item differentiable on the open interval $(a,b)$
    \item $f(a)=f(b)$
\end{itemize}
then there is at least one number $c$ in $(a,b)$ such that $f'(c)=0$

Graphically, the slope of the secant line on $[a,b]$ and the slope of the tangent line at $x=c$ both equal zero for at least one value of $c$ in $(a,b)$.

Rolle's Theorem is a special case of the Mean Value Theorem in which the average rate of change is 0:
\[ f'(c)=\frac{f(b)-f(a)}{b-a}=0 \]

\textbf{Extreme Value Theorem}

If a function $f$ is continuous on the closed interval $[a,b]$, then $f$ is guaranteed to attain an absolute minimum and absolute maximum value on $[a,b]$.

\textbf{First Derivative Test}

If $f'(c)=0$ or undefined, there is a local maximum if $f'(x)$ changes from positive to negative and a local minimum when $f'(x)$ changes from negative to positive.

\textbf{Second Derivative Test}

If $f''(c)<0$, $f(c)$ is a relative maximum. If $f''(c)=0$ the test is inconclusive. If $f''(c)>0$ then $f'(c)$ is a relative minimum.

\textbf{Riemann Sums}
A left Riemann sum approximates the value of a definite integral $\int_a^b f(x)\dd x$. The interval $[a,b]$ is divided 
into subintervals, and the area boudned by the graph of $f$ and the $x$-axis on each subinterval is estimated with a rectangle.

The base length $b_n$ of each rectangle is the distance between the endpoints of the subinterval, and the height $h_n$ is the function value at the left endpoint.
\[ \int_a^b f(x)\dd x \approx b_1h_1+b_2h_2+\dots \]

A midpoint Riemann sum approximates the value of a definite integral $\int_a^b f(x)\dd x$. The interval $[a,b]$ is divided into subintervals, and the area bounded by the graph of $f$ and the $x$-axis on each subinterval is estimated with a rectangle.

The base length $b_n$ of each rectangle is the distance between the endpoints of the subinterval, and the height $h_n$ is the function value at the midpoint of the subinterval $m_n$.
\[ \int_a^b f(x)\dd x \approx b_1h_1+b_2h_2+\dots \]

A right Riemann sum approximates the value of a definite integral $\int_a^b f(x)\dd x$. The interval $[a,b]$ is divided 
into subintervals, and the area boudned by the graph of $f$ and the $x$-axis on each subinterval is estimated with a rectangle.

The base length $b_n$ of each rectangle is the distance between the endpoints of the subinterval, and the height $h_n$ is the function value at the right endpoint.
\[ \int_a^b f(x)\dd x \approx b_1h_1+b_2h_2+\dots \]

A trapezoidal sum approximates the value of a definite integral $\int_a^b f(x)\dd x$. The interval $[a,b]$ is divided into subintervals, and the area bounded by the graph of $f$ and the $x$-axis on each subinterval is estimated with a trapezoid.

The height $h_n$ of each trapezoid is the distance between the endpoints of the subinterval, and the bases $b_n$ and $b_{n+1}$ are the function values at the endpoints.
\[ \int_a^b f(x)\dd x\approx \frac{1}{2}h_1(b_1+b_2)+\frac{1}{2}h_2(b_2+b_3) \]

\textbf{Limit of a Right Riemann Sum}
\[ \lim_{n\to \infty}\sum^n_{i=1}f(a+\Delta xi)\Delta x = \int_a^b f(x)\dd x\]

\textbf{Fundamental Theorem of Calculus}
$\int_a^b f(x) \dd x = F(b)-F(a)$

$\int_a^b f'(x) \dd x = f(b)-f(a)$

$f(b)=f(a)+\int_a^b f'(t)\dd t$, where $f(b)$ is the final quantity, $f(a)$ is the initial quantity and $\int_a^b f'(t)\dd t$ is the net change.

\textbf{Second FTC}
$\frac{\dd}{\dd x}[\int_a^x f(t)\dd t] = f(x)$

\textbf{Basic Integration Rules}
\begin{itemize}
    \item Constant: $\int c\dd x = cx+C$
    \item Power: $\int x^n \dd x = \frac{x^{n+1}}{n+1}+C$
    \item Constant multiple: $\int cf(x)\dd x = c\int f(x)\dd x$
    \item Sum and difference: $\int f(x)\pm g(x)\dd x = \int f(x)\dd x\pm \int g(x)\dd x$
    \item Natural exponential: $\int e^x \dd x = e^x + C$
    \item Natural log: $\int \frac{1}{x}\dd x = \ln|x|+C$
\end{itemize}

\textbf{Trig Integrals}
\begin{itemize}
    \item $\int \sin u \dd u = -\cos u +C$
    \item $\int \cos u \dd u = \sin u +C$
    \item $\int \sec^2 u \dd u = \tan u +C$
    \item $\int \csc^2 u \dd u = -\cot u +C$
    \item $\int (\sec u \tan u)\dd u = \sec u + C$
    \item $\int (\csc u \cot u)\dd u = -\csc u +C$
    \item $\int \tan u \dd u = -\ln|\cos u|+C$
    \item $\int \cot u \dd u = \ln|\sin u|+C$
    \item $\int \sec u \dd u = \ln|\sec u + \tan u|+C$
    \item $\int \csc u \dd u = -\ln|\csc u +\cot u|+C$
\end{itemize}

\textbf{Properties of Definite Integrals}
The following are properties of definite integrals, where functions $f$ and $g$ are continuous on the closed interval $[a,b]$ and $a$, $b$, and $k$ are constants.
\begin{itemize}
    \item $\int_a^a f(x)\dd x =0$
    \item $\int_a^b f(x)\dd x = -\int_b^a f(x)\dd x$
    \item $\int_a^c f(x)\dd x = \int_a^b f(x)\dd x+\int_b^c f(x)\dd x$
    \item $\int_a^b kf(x)\dd x = k\int_a^b f(x)\dd x$
    \item $\int_a^b [f(x)\pm g(x)]\dd x = \int_a^b f(x)\dd x\pm \int_a^b g(x)\dd x$
\end{itemize}

\textbf{Improper Integral}
\[ \int_a^{\infty}f(x)\dd x = \lim_{t\to\infty}\int_a^t f(x)\dd x \]

\textbf{Integration by Parts}
$\int u\dd v = uv-\int v\dd u$

\textbf{Euler's Method}
$y_{n+1}=y_n+f'(x_n)(\Delta x)$, where $y_{n+1}$ is the next $y$-value, $f'(x_n)$ is the derivative at current $x_n$-value and $\Delta x$ is the step size.

\textbf{Exponential Growth and Decay}
Differential Equation: $\frac{\dd y}{\dd t}=k\cdot y$, where $\frac{\dd y}{\dd t}$ is the rate of change of $y$ and $k$ is the constant of proportionality.

General Solution: $y=C\cdot e^{k\cdot t}$, where $C$ is the initial value of $y$ (when $t=0$), $k$ is the constant of proportionality, and $t$ is time.

\textbf{Logistic Growth/Decay}
$\frac{\dd P}{\dd t}=kP\left(1-\frac{P}{a}\right)$

$\frac{\dd P}{\dd t}=kP(a-P)$

\textbf{Average Value}
$\frac{1}{b-a}\int_a^b f(x)\dd x$

\textbf{Total Distance Traveled}
$\int_{t_1}^{t_2}|v(t)|\dd t$

\textbf{Area Between Curves}
In terms of $x$: $A=\int_{x_1}^{x_2}(\text{top}-\text{bottom})\dd x$ is the area bounded by two functions on $[x_1,x_2]$.

In terms of $y$: $A=\int_{y_1}^{y_2}(\text{right}-\text{left})\dd y$ is the area bounded by two functions on $[y_1,y_2]$.

\textbf{Disk Method}
Use the disk method to determine the volume of a solid of revolution formed by rotating a region about a horizontal line $y=c$ (axis of revolution) over the interval 
$a<x<b$ when $y=c$ is a boundary of the region - there is no space between the region and $y=c$.
\[ \pi \int_a^b r^2 \dd x\]

When a region is revolved about an axis of revolution, a perpendicular cross section of the solid is a disk where 
\begin{itemize}
    \item $r$ is the distance from the axis of revolution to the closest function $f(x)$
    \item $\dd x$ is the thickness of the disk 
\end{itemize}

Use the disk method to determine the volume of a solid of revolution formed by rotating a region about a vertical line $x=k$ (axis of revolution) over the interval 
$c<y<d$ when $x=k$ is a boundary of the region - there is no space between the region and the line $x=k$.
\[ \int_c^d (r(y))^2 \dd y \]
When a region is revolved about an axis of revolution, a perpendicular cross section of the solid is a disk where:
\begin{itemize}
    \item $r$ is the distance from the axis of revolution to the closest function $f(y)$
    \item $\dd y$ is the thickness of the disk 
\end{itemize}

\textbf{Washer Method}
Use the washer method to determine the volume of a solid of revolution formed by rotating a region bounded by $f(x)$ and $g(x)$ about a horizontal line $y=c$ (axis of revolution) over the interval 
$a<x<b$ when $y=c$ is not a boundary of the region - there is space between the region and $y=c$.
\[ \pi \int_a^b ((R(x))^2 - (r(x))^2)\dd x \]
When a region is revolved about an axis of revolution, a perpendicular cross section of the resulting solid is a disk with a hole (washer) where:
\begin{itemize}
    \item $R$ is the distance from the axis of revolution to the farthest function $f(x)$ 
    \item $r$ is the distance from the axis of revolution to the closest function $g(x)$
    \item $\dd x$ is the thickness of the washer 
\end{itemize}

Use the washer method to determine the volume of a solid of revolution formed by rotating a region bounded by $f(y)$ and $g(y)$ about a horizontal line $x=k$ (axis of revolution) over the interval 
$c<y<d$ when $x=k$ is not a boundary of the region - there is space between the region and $x=k$.
\[ \pi \int_c^d ((R(y))^2 - (r(y))^2)\dd y \]
When a region is revolved about an axis of revolution, a perpendicular cross section of the resulting solid is a disk with a hole (washer) where:
\begin{itemize}
    \item $R$ is the distance from the axis of revolution to the farthest function $f(y)$ 
    \item $r$ is the distance from the axis of revolution to the closest function $g(y)$
    \item $\dd y$ is the thickness of the washer 
\end{itemize}

\textbf{Arc Length}
\[ \int_a^b \sqrt{1+(f'(x))^2}\dd x\]

\textbf{Parametrics}
Parametric Slope: $\frac{\dd y}{\dd x}=\frac{y'(t)}{x'(t)}$

Parametric Speed: $s(t)=\sqrt{(x'(t))^2+(y'(t))^2}$

Parametric Arc Length: $\int_{t_1}^{t_2}\sqrt{(x'(t))^2+(y'(t))^2}\dd t$

\textbf{Derivatives of Vector Valued Functions}
$f(t)=\langle x(t), y(t)\rangle$

$f'(t) = \langle x'(t),y'(t)\rangle$

$f''(t)=\langle x''(t),y''(t)\rangle$

\textbf{Total Distance of Vectors}
$\int_{t_1}^{t_2}\sqrt{(x'(t))^2+(y'(t))^2}\dd t$

\textbf{Polar to Rectangular Coordinates}
$x=r\cos\theta$

$y=r\sin\theta$

\textbf{Slope of Polar Curve}
$\frac{\dd y}{\dd x}=\frac{\frac{\dd}{\dd \theta}[y]}{\frac{\dd}{\dd\theta}[x]}$

\textbf{Sum of Geometric Series}
$S=\frac{a_1}{1-r}$

\textbf{Convergence Tests}
The harmonic series is an infinite series given by 
\[ \sum_{n=1}^\infty \frac{1}{n}=1+\frac{1}{2}+\frac{1}{3}+\frac{1}{4}+\dots \]
The harmonic series diverges by the $p$-series test. 

$p$-series test.
\begin{itemize}
    \item $p$-series of the form $\frac{1}{n^p}$ converges for $p>1$
\end{itemize}
\[ \sum_{n=1}^\infty \frac{1}{n}\implies \sum_{n=1}^\infty \frac{1}{n^1}, p=1\]

$n$th Term Test 
$\sum_{n=1}^\infty a_n$ diverges if $\lim_{n\to \infty}a_n \neq 0$ and is inconclusive when $\lim_{n\to\infty}a_n=0$

A series of the form $\sum_{n=1}^\infty \frac{1}{n^p}=\frac{1}{1^p}+\frac{1}{2^p}+\frac{1}{3^p}+\frac{1}{4^p}+\dots$ is called a $p$-series.
\begin{itemize}
    \item $p$-series converges if $p>1$
    \item $p$-series diverges if $0<p\leq 1$
\end{itemize}
If $p=1$, the resulting series $\sum_{n=1}^\infty \frac{1}{n}=1+\frac{1}{2}+\frac{1}{3}+\frac{1}{4}+\dots$ is called a harmonic series, which diverges.

Geometric Series 
\[ \sum_{n=0}^\infty ar^n \]
When $|r|<1$ the series converges to $S=\frac{a_1}{1-r}$, where $a_1$ is the first term of the series. If $|r|\geq 1$ the series diverges.

Integral Test

If $f$ is continuous, positive, and eventually decreases as $x\rightarrow \infty$, and $\int_c^\infty f(x)$:
\begin{itemize}
    \item converges then $\sum_{n=c}^\infty f(n)$ converges and $\sum_{n=c}^\infty f(n)>\int_c^\infty f(x)\dd x$
    \item diverges: then, $\sum_{n=c}^\infty f(n)$ diverges 
\end{itemize}

Direct Comparison Test 
\[ 0<a_n<b_n \]
If the larger series $\sum_{n=1}^\infty b_n$ converges, the smaller series $\sum_{n=1}^\infty a_n$ converges 

If the smaller series $\sum_{n=1}^\infty a_n$ diverges, the larger series $\sum_{n=1}^\infty b_n$ diverges.

Limit Comparison Test 

If $\lim_{n\to \infty} \frac{a_n}{b_n}=L$, where $L$ is finite and positive and $a_n>0$, $b_n>0$, then: 

$\sum_{n=1}^\infty b_n$ and $\sum_{n=1}^\infty a_n$ converge or $\sum_{n=1}^\infty b_n$ and $\sum_{n=1}^\infty a_n$ diverge.

Ratio Test 

If $\lim_{n\to\infty}\left|\frac{a_{n+1}}{a_n}\right| = k$, then $\sum_{n=1}^\infty a_n$ converges absolutely if $k<1$ or diverges if $k>1$.

Alternating Series Test 
\[ \sum_{n=1}^\infty (-1)^na_n \]
converges if:
\begin{itemize}
    \item $\lim_{n\to\infty}a_n = 0$ and 
    \item $a_n$ is a positive, decreasing sequence 
\end{itemize}

\textbf{Taylor/Maclaurin Polynomials}
$n$th-degree Taylor polynomial of $f$ about $x=c$
\[ P_n(x)=f(c)+f'(c)(x-c)+\frac{f''(c)}{2!}(x-c)^2+\dots + \frac{f^(n)(c)}{n!}(x-c)^n \]

Maclaurin polynomial 
\[ P_n(x)=f(0)+f'(0)x+\frac{f''(0)}{2!}x^2+\frac{f'''(0)}{3!}x^3+\dots + \frac{f^{(n)}(0)}{n!}x^n \]

\textbf{Known Power Series}
\begin{itemize}
    \item $\frac{1}{1-x}=1+x+x^2+x^3+\dots+x^n+\dots$
    \item $e^x = 1+x+\frac{x^2}{2!}+\frac{x^3}{3!}+\dots+\frac{x^n}{n!}+\dots$
    \item $\sin x = x-\frac{x^3}{3!}+\frac{x^5}{5!}-\frac{x^7}{7!}+\dots+(-1)^n\frac{x^{2n+1}}{(2n+1)!}+\dots$
    \item $\cos x = 1-\frac{x^2}{2!}+\frac{x^4}{4!}-\frac{x^6}{6!}+\dots+(-1)^n\frac{x^{2n}}{(2n)!}+\dots$
\end{itemize}

\end{document}