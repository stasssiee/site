\documentclass[../chem.tex]{subfiles}
\graphicspath{{\subfix{../figures/}}}
\begin{document}
\chapter{Infinite Sequences and Series}
\section{Defining Convergent and Divergent Infinite Series}
Writing terms of a sequence.
\[ a_n = \{1+(-2)^n\} \]
ends up becoming $-1,5,-7,17,-31$/

A sequence is a collection of numbers that are in one-to-one correspondence with positive integers.

A monotonic sequence never decreases or never increases.
\begin{itemize}
    \item $a_1\leq a_2\leq a_3\leq \dots \leq a_n$
    \item $a_1\geq a_2 \geq a_3\geq \dots \geq a_n$
\end{itemize}

Bounded Sequences 
\begin{itemize}
    \item $a_n\leq M$ (upper bound/above)
    \item $a_n\geq N$ (lower bound/below)
    \item $\{a_n\}$ bounded if both are true 
\end{itemize}

An infinite series is 
\[ \sum_{n=1}^{\infty} a_n = a_1+a_2+a_3+\dots + a_n \]

Partial sums are 
\[ S_n=a_1+a_2+a_3+\dots + a_n \]

$a_n$ is an expression that gives the $n$th term in a sequence.

$s_n$ is an expression that gives the sum of the first $n$ terms.

\begin{example}
    Use the following sequence $2,4,6,8,10$ to find $a_4$ and $S_4$.

    The fourth term is $a_4=8$.

    $S_4=2+4+6+8=20$.
\end{example}

That leads us to $\sum_{n=1}^{\infty}a_n = \lim_{n\to \infty}S_n$ 

\begin{definition}
    For the infinite series $\sum_{n=1}^{\infty} a_n$, the $n$th partial sum is $S_n=a_1+a_2+a_3+\dots + a_n$.

    If the sequence of the partial sum $\{S_n\}$ converges to $S$, then the series $\sum_{n=1}^{\infty}a_n$ converges. The limit $S$ is called the sum of the series.

    Likewise, if $\{S_n\}$ diverges, then the series diverges.
\end{definition}

\begin{example}
    Does the series converge or diverge? 
    \[ \sum_{n=1}^{\infty} \frac{1}{2^n} \]

    The first term is $S_1=\frac{1}{2}$. Then $S_2 = \frac{1}{2}+\frac{1}{4} = \frac{3}{4}$. $S_3=\frac{7}{8}$ and $S_4 = \frac{15}{16}$.

    This keeps on getting closer to $1$, so the guess is that this converges.

    A strategy is to find $S_n$. This ends up being $\frac{1}{2}+\frac{1}{2^2}+\frac{1}{2^3}+\dots + \frac{1}{2^n}$.
    
    We get $\frac{1}{2}S_n=\frac{1}{2^2}+\frac{1}{2^3}+\frac{1}{2^4}+\dots + \frac{1}{2^{n+1}}$.

    We say that $S_n-\frac{1}{2}S_n$ by subtracting the two.

    If we do this, we see that everything cancels all the way until you get $\frac{1}{2}+\frac{1}{2^{n+1}}$.

    Simplifying for $S_n$ gives $1-\frac{1}{2^n}$.

    Letting this appraoch infinity gives $1$.
\end{example}

\begin{example}
    Use a calculator to find the partial sum $S_n$ of the series $\sum_{n=1}^{\infty} \frac{10}{n(n+2)}$ for $n=200,1000$.

    Hint: Use a calculator
\end{example}

\begin{example}
    Does the series converge or diverge? 
    \[ \sum_{n=1}^{\infty} n \]

    If you see this, it keeps getting bigger, so it diverges.
\end{example}

\ex Given the infinite series $\sum_{n=1}^{\infty} (-1)^n$, find the sequence of partial sums $S_1,S_2,S_3,S_4$, and $S_5$. (Answer: $-1,0,-1,0,-1$)

\ex Find the sequence of partial sums $S_1,S_2,S_3,S_4$, and $S_5$ for the infinite series $1+\frac{1}{2}+\frac{1}{4}+\frac{1}{6}+\frac{1}{8}+\frac{1}{10}+\dots$. (Answer: $1,\frac{3}{2},\frac{7}{4},\frac{23}{12},\frac{49}{24}$)

\ex If the infinite series $\sum_{n=1}^{\infty} a_n$ has $n$th partial sum $S_n=(-1)^{n+1}$ for $n\geq 1$, what is the sum of the series. (Answer: $1,-1,1,-1$ diverges)

\ex The infinite series $\sum_{n=1}^{\infty}a_n$ has $n$th partial sum $S_n=\frac{n}{4n+1}$ for $n\geq 1$. What is the sum of the series? (The limit of the $S_n$ to infinity gives $\frac{1}{4}$)

\ex Use a calculator to find the partial sum $S_n$ of the series $\sum_{n=1}^{\infty}\frac{6}{n(n+3)}$ for $n=100,500,1000$. (Answer: Trivial)

\ex Show that the sequence with the given $n$th term $a_n=1+2n$ is monotonic. (Answer: $3,5,7,9$ is monotic because it is strictly increasing)

\ex What is the $n$th partial sum of the infinite series $\sum_{n=1}^{\infty} \frac{1}{2^{n+1}}$. (Answer: Do what was above you get $\frac{1}{2}-\frac{1}{2^{n+1}}$)

$\textbf{Special}$

\ex Which of the following could be the $n$th partial sum for the infinite series $\sum_{n=1}^{\infty}\frac{1}{4^n}$? (Answer: DO same process as above, get $S_n=\frac{1}{3}(1-\frac{1}{4^n})$)
$\textbf{(A) } S_n = \frac{1}{3}(1+\frac{1}{4^n}) \qquad \textbf{(B) } S_n=\frac{1}{3}(1-\frac{1}{4^{n+1}}) \qquad \textbf{(C) } S_n=\frac{1}{3}(1-\frac{1}{4^n}) \qquad \textbf{(D) } S_n=\frac{1}{4}(1-\frac{1}{3^n})$

\ex If the infinite series $\sum_{n=1}^{\infty}a_n$ is convergent and has a sum of $\frac{7}{8}$, what could be the $n$th partial sum? (The limit as $n$ goes to infinity for $S_n$ is $\frac{7}{8}$)
$\textbf{(A) } S_n = \frac{7n+1}{8n^2+1} \qquad \textbf{(B) } S_n = \frac{7n^2+1}{8n+1} \qquad \textbf{(C) } S_n = 2(\frac{7}{8}-\frac{1}{n+2}-\frac{1}{n+3}) \qquad \textbf{(D) } S_n = (\frac{7}{8}-\frac{1}{n+2}-\frac{1}{n+3})$

\ex Which of the following sequences with the given $n$th term is bounded and monotonic?
$\textbf{(A) } a_n=2+(-1)^n \qquad \textbf{(B) } a_n=\frac{n^2}{n+1} \qquad \textbf{(C) } a_n = \frac{3n}{n+2} \qquad \textbf{(D) } a_n = \frac{\cos n}{n}$




\end{document}