\documentclass[../bccalc.tex]{subfiles}
\graphicspath{{\subfix{../figures/}}}
\begin{document}
\chapter{Improper Integrals, Sequences and Series, Taylor Polynomials, and Taylor Series}
\section{Improper Integrals}
\begin{example}
    \[ \int_1^{\infty} \frac{1}{x^2}dx \]

    This can be written as 
    \[ \lim_{b\to \infty}\int_1^b \frac{1}{x^2}dx=\lim_{b\to \infty}\left[-\frac{1}{x}\right]^b_1 = \lim_{b\to\infty}\left[-\frac{1}{b}-\frac{-1}{1}\right] \]
    Which converges to 1.
\end{example}

\ex $\int_{-1}^0 \frac{1}{x^2}dx$

\ex $\int_{-1}^2 \frac{1}{x^3}dx$
\section{nth Term Test}
A sequence $\{a_n\} = a_1, a_2, a_3, a_4,\dots, a_n,a_{n2}$.

$\{a_n\}$ converges if $\lim_{n\to \infty}a_n=L$ and diverges if $L\rightarrow \infty$ or $L$ does not exist.

A series $\sum_{n=1}^{\infty}a_n = a_1+a_2+a_3+\dots a_n+\dots$.

If $\sum_{n=1}^{\infty}a_n$ becomes $\infty$ we say the series diverges.

If $\sum_{n=1}^{\infty}a_n$ stays finite, we say it converges.

The 2 big questions are 
\begin{enumerate}
    \item Does $\sum_{n=1}^{\infty}a_n$ converge?
    \item If so, to what value?
\end{enumerate}

Let's say we have a sum $\sum_{n=1}^{\infty}2n = 2+4+6+8+\dots$.

$S_1$ would be $2$, $S_2$ would be $2+4=6$, $S_3$ would be $2+4+6=12$. We can see that the $\lim_{n\to\infty}=\infty$ so it diverges.
\pagebreak
\begin{example}
    Does this diverge or converge?
    \[ \sum_{n=1}^{\infty} \frac{n}{2n-1}\]

    Expanding we see that This ends up being $\frac{1}{1}+\frac{2}{3}+\frac{3}{5}+\frac{4}{7}+\dots$

    The limit $\lim_{n\to\infty}\frac{n}{2n-1}=\frac{1}{2}$.

    $S_n\rightarrow \infty$ so diverges.
\end{example}

To ``stand a chance'' to converge 
\[ \lim_{n\to\infty}a_n=0\]

nth term for divergence:

If $\lim_{n\to\infty}a_n\neq 0$ then $\sum_{n=1}^{\infty}a_n$ diverges.

\ex Use the nth term test to determine whether the series diverges for $\sum_{n=1}^{\infty}\frac{1}{3n}$.

\ex Use the nth term test to determine whether the series diverges for $\sum_{n=1}^{\infty}\frac{n!}{3n!+2}$.

Never use the nth term test to argue for convergence.
\section{Geometric Series and Telescopic}
\ex Does $\sum_{n=1}^{\infty}\tan^{-1}n$ diverge? Use the nth term test.

\begin{example}
    \[ \sum_{n=1}^{\infty}\frac{e^n}{3^n}\]

    The summation can be written as $\left(\frac{e}{3}\right)^n$. Using the nth term test, the limit goes to 0.

    Whenever you have $\sum_{n=0}^{\infty}a_1(r)^n = \frac{a_1}{1-r}$ if $-1<r<1$ is what it converges to.
    
    This converges if $-1<r<1$.

    Converges to $\frac{a_1}{1-r}$.

    In the above case, $a_1=\frac{e}{3}$ and $r=\frac{e}{3}$ so it converges to $\frac{\frac{e}{3}}{1-\frac{e}{3}}$.
\end{example}

\ex Can you use geometric series or nth term test on $\sum_{n=1}^{\infty}\frac{1}{10n+7}$?

\begin{example}
    \[ \sum_{n=1}^{\infty}\frac{1}{n(n+1)} \]

    We can separate this into $\sum_{n=1}^{\infty}\frac{1}{n}-\frac{1}{n+1}=1$ becauase you can see that every term will cancel out except $\frac{1}{1}$.

    Converges to 1.
\end{example}

\ex Does $\sum_{n=1}^{\infty} \frac{3^{n-2}}{2^n}$ diverge or converge?

\ex Does $\sum_{n=1}^{\infty}\frac{3^n+1}{3^{n+1}}$ diverge or converge?

\section{Integral Test and p-Series}
Integral Test:

If $f$ is positive, continuous, and decreasing for $x\geq 1$ and  $a_n=f(n)$, then $\sum_{n=1}^{\infty}a_n$ and $\int_1^{\infty}f(x)dx$ either both converge or both diverge.

\begin{example}
    Determine whether the following series converges or diverges.
    \[ \sum_{n=1}^{\infty}\frac{n}{n^2+1} \]

    $\lim_{n\to\infty}a_n=0$, so we use a different test.

    $\int_1^{\infty} \frac{x}{x^2+1}$ diverges, so the series will diverge as a result.
\end{example}

\ex Determine if the series converges or diverges. $\sum_{n=1}^{\infty}\frac{1}{n^2+1}$

If $f$ is positive, continuous, and decreasing for $x\geq 1$ and $a_n=f(n)$ and if $\sum_{n=1}^{\infty} a_n$ and $\int_1^{\infty}f(x)dx$ both converge, then the series converges to $S$, and the remainder, $R_N=S-S_N$ is bounded by $0\leq R_N\leq \int_N^{\infty}f(x)dx$.

\begin{example}
    Approximate the sum of the convergent series $\sum_{n=1}^{\infty}\frac{1}{n^4}$ by using six terms. Include an estimate of the maximum error for your approximation.

    Using the first six terms gives you $1.07209$.

    The error is error $\leq \int_6^{\infty}\frac{1}{x^4}dx=0.0015$.

    $1.07289<\sum < 1.07289+0.0015$
\end{example}

p-Series 

A series of the form $\sum_{n=1}^{\infty} \frac{1}{n^p}=\frac{1}{1^p}+\frac{1}{2^p}+\frac{1}{3^p}+\dots+\frac{1}{n_p}+\dots$ is called a $p$-series, where $p$ is a positive constant. For $p=1$, the series $\sum_{n=1}^{\infty}\frac{1}{n}=1+\frac{1}{2}+\frac{1}{3}+\dots+\frac{1}{n}+\dots$ is called the harmonic series.

The p-series diverges if $0<p\leq 1$ and converges if $p>1$. The harmonic series diverges.

\begin{example}
    \[ \sum_{n=1}^{\infty}\frac{1}{n\sqrt{n}}\]

    $p=3/2$ for this, so the series converges.
\end{example}

\section{Comparison of Series}
Direct Comparison Test 

If $a_n\geq 0$ and $b_n\geq 0$,
\begin{enumerate}
    \item If $\sum_{n=1}^{\infty}b_n$ converges and $0\leq a_n\leq b_n$, then $\sum_{n=1}^{\infty}a_n$ converges.
    \item If $\sum_{n=1}^{\infty}a_n$ diverges and $0\leq a_n\leq b_n$, then $\sum_{n=1}^{\infty}b_n$ diverges.
\end{enumerate}

\begin{example}
    Determine if this converges or diverges.
    \[ \sum_{n=1}^{\infty}\frac{1}{n^3+1} \]

    Compare this to $\sum_{1}^{\infty}\frac{1}{n^3}$. This is a p-series that converges, so this series converges.
\end{example}

\ex Determine whether the following series converges or diverges. $\sum_{n=1}^{\infty}\frac{1}{3^n+2}$

\ex Determine whether the following series converges or diverges. $\sum_{n=4}^{\infty}\frac{1}{\sqrt{n}-1}$

Limit Comparison Test 

Suppose $a_n>0, b_n>0$ and $\lim_{n\to\infty}\frac{a_n}{b_n}=L$, where $L$ is both finite and positive. Then the two series $\sum_{n=1}^{\infty}a_n$ and $\sum_{n=1}^{\infty}b_n$ either both converge or both diverge.

\begin{example}
    Determine whether the following converge or diverge.
    \[ \sum_{n=1}^{\infty}\frac{n^4+10}{4n^5-n^3+7} \]

    Compare this to $\frac{1}{n}$ and this diverges. The limit of what is in the summation as $n\rightarrow \infty$ is $\frac{1}{4}$ so this diverges so they diverge.
\end{example}

\ex Determine whether the series converges or diverges. $\sum_{n=2}^{\infty}\frac{1}{n^3-2}$.

\section{Alternating Series}
An alternating series is a series whose terms are alternately positive and negative.

Examples are $\sum_{n=1}^{\infty}(-1)^{n+1}\frac{1}{n}$ or $\sum_{n=1}^{\infty}(-1)^n \frac{1}{n!}$.

In general, just knowing that $\lim_{n\to\infty}a_n=0$ tells us very little about the convergence of the series $\sum_{n=1}^{\infty}a_n$ however it turns out that an alternating series must converge if the terms have a limit of 0 and the terms decrease in magnitude.

Alternating Series Test 

Let $a_n>0$. The alternating series $\sum_{n=1}^{\infty}(-1)^n a_n$ and $\sum_{n=1}^{\infty}(-1)^{n+1}a_n$ converge if the following two conditions are met 
\begin{enumerate}
    \item $\lim_{n\to\infty}a_n=0$
    \item $a_{n+1}<a_n$ for all $n$
\end{enumerate}
In other words, a series converges if its terms 
\begin{enumerate}
    \item alternate in sign 
    \item decrease in magnitude 
    \item have a limit of 0
\end{enumerate}

Note: This does not say that if $\lim_{n\to\infty}a_n\neq 0$, the series diverges by the Alternating Series Test. The Alternating Series Test can only be used to prove convergence. If $\lim_{n\to\infty}a_n\neq 0$ then the series diverges by the nth Term Test for Divergence not by the Alternating Series Test.
\pagebreak
\begin{example}
    Determine if the series converges or diverges.
    \[ \sum_{n=1}^{\infty}\frac{(-1)^{n+1}n}{2n-1} \]

    The limit $\lim_{n\to\infty}\frac{n}{2n-1}=\frac{1}{2}$ so this diverges by the nth term test.
\end{example}

\ex Determine if the series converges or diverges. $\sum_{n=1}^{\infty}\frac{(-1)^n n}{\ln(2n)}$

\ex Determine if the series converges or diverges. $\sum_{n=1}^{\infty}\frac{(-1)^n}{n}$

The above series is called the alternating harmonic series. If an alternating series converges to a sum $S$, then its partial sums jump around $S$ from side to side with decreasing distances from $S$. 

\begin{definition}
    $\sum_{n=1}^{\infty}a_n$ is absolutely convergent if $\sum_{n=1}^{\infty}|a_n|$ converges.

    $\sum_{n=1}^{\infty}a_n$ is conditionally convergent if $\sum_{n=1}^{\infty}a_n$ converges but $\sum_{n=1}^{\infty}|a_n|$ diverges.
\end{definition}

\begin{example}
    Determine whether the given alternating series converges or diverges. If it converges, determine whether it is absolutely convergent or conditionally convergent.

    \[ \sum_{n=1}^{\infty}\frac{(-1)^n}{\sqrt{n}} \]

    The limit of $\frac{1}{\sqrt{n}}=0$ as $n$ goes to infinity.

    It is an alternating series because terms are decreasing in magnitude.

    So we see that $\left| \frac{(-1)^n}{\sqrt{n}}\right|$ converges, but $\frac{1}{\sqrt{n}}$ will diverge.

    This is conditionally convergent.
\end{example}

Alternating Series Remainder 

If a series has terms that are alternating, decreasing in magnitude, and having a limit of 0, then the series converges so that it has a sum $S$. If the sum $S$ is approximated by the nth partial sum, $S_n$, then the error in the approximation, $|R_n|$, which equals $|S-S_n|$, will be less than the absolute value of the first omitted or trucnated term.

In other words, if the three conditions are met, you can approximate the sum of the series by using the nth partial sum, $S_n$, and your error will be bounded by the absolute value of the first truncated term.
\pagebreak
\begin{example}
    Given the series $\sum_{n=1}^{\infty} \frac{(-1)^{n+1}}{n!}$

    (a) Approximate hte sum $S$ of the series by using its first four terms.

    First four terms comes out to $0.625$

    (b) Explain why the estimate found in (a) differs from the actual value by less than $\frac{1}{100}$.

    error$<\left|\frac{(-1)^{5+1}}{5!}\right| = \frac{1}{120}<\frac{1}{100}$

    (c) Use your results to explain why $S\neq 0.7$

    $0.625-\frac{1}{120} < $ sum$<0.625+\frac{1}{120}$.

    $0.7$ is not in this interval.
\end{example}

\ex How many terms are needed to approximate the sum of the series $\sum_{n=1}^{\infty}\frac{(-1)^{n+1}}{n^4}$ so that the estimate differs from the actual sum by less than $\frac{1}{1000}$? Justify your answer.

\section{Ratio Test}
Ratio Test 

Let $\sum_{n=1}^{\infty}a_n$ be a series of nonzero terms.
\begin{enumerate}
    \item $\sum_{n=1}^{\infty}a_n$ converges if $\lim_{n\to\infty}\left|\frac{a_{n+1}}{a_n}\right|<1$.
    \item $\sum_{n=1}^{\infty}a_n$ diverges if $\lim_{n\to\infty}\left|\frac{a_{n+1}}{a_n}\right|>1$.
    \item If $\lim_{n\to\infty}\left|\frac{a_{n+1}}{a_n}\right|=1$ the Ratio Test is inconclusive so another test would need to be used.
\end{enumerate}

\begin{example}
    Determine whether the following converge or diverge.
    \[ \sum_{n=1}^{\infty}\frac{2^n}{n!} \]

    We do $\lim_{n\to\infty}\left| \frac{2^{n+1}}{(n+1)!}\cdot \frac{n!}{2^n} \right|$.

    This simplies to $\lim_{n\to\infty}\left|\frac{2}{n+1}\right|=0<1$ converges.
\end{example}

\ex Determine whether the following converges or diverges. $\sum_{n=1}^{\infty}\frac{n^2 3^{n+1}}{2^n}$

\ex Determine whether the following converges or diverges. $\sum_{n=1}^{\infty}\frac{(n+1)!}{3^n}$

\section{Taylor Polynomials}

\section{Radius and Interval of Convergence}

\section{Taylor Series}

\section{Lagrange Error Bound}

\section{More on Error}
\end{document}