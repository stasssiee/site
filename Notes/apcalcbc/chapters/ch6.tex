\documentclass[../bccalc.tex]{subfiles}
\graphicspath{{\subfix{../figures/}}}
\begin{document}
\chapter{Differential Equations}
\section{Differential Equations}
The process is the following:
\begin{enumerate}
    \item Separate Variables (multiply or divide to get the $x$ and $y$'s on opposite sides. $dx$ and $dy$ must always be on top).
    \item Integrate both sides 
    \item Add $+C$ with the $x$ side.
\end{enumerate}

\begin{example}
    Use integration to find the general solution to the differential equation 
    \[ \frac{dy}{dx}=2x(x-4) \]

    We are finding $dy=2x^2-8x dx$

    Integrate both sides to get $y=\frac{2}{3}x^3-\frac{8x^2}{2}+C$.

    $y=\frac{2}{3}x^3-4x^2+C$ is the general solution.
\end{example}

\ex Find the particular solution of $f'(x)=7x-6$ knowing thta $f(1)=\frac{3}{2}$.

\ex Use integration to find the general solution to the differential equation $\frac{dy}{dx}=\frac{x-1}{y-6}$

\section{Euler's Method}
This is basically baby steps with tangent lines.

The general procedure is 
\begin{center}
    New $y$ = Old $y$ + Slope(step size)
\end{center}

\begin{example}
    Consider a function whose slope is given by $\frac{dy}{dx}=2xy$ with initial condition $f(1)=1$.

    Use Euler's method with a step size of $0.1$ to approximate $f(1.3)$.

    $f(1.1)\approx 1+2(1)(1)(.1)=1.2$

    $f(1.2)\approx 1.2+2(1.1)(1.2)(.1)$

    Keep doing this to find $f(1.3)$.
\end{example}

\pagebreak
\section{Exponential Growth and Decay}
Solving problems where the rate of growth is proportional to the amount present.

\[ \frac{dy}{dt}=kt \implies \ln|y|=kt +C\]
\[ \implies y=Ce^{kt}\]

\begin{example}
    Write an equation for the amount $Q$ of a radioactive substance with a half-life of 30 days, if 10 grams are present when $t=0$.

    $Q(t)=Ce^{kt}$.

    $Q(t)=10e^{kt}$, so we need to find $k$.

    $5=10e^{30k}$, so $k= \frac{\ln \frac{1}{2}}{30}$
\end{example}

\ex The balance in an account triples in 30 years. Assuming that interest is compounded continuously, what is the annual percentage rate?

\ex In 1990 the population of a village was 21,000 and in 2000 it was 20,000. Assuming the population decreases continuously at a constant rate proportional to the existing population, estimate the population in the year 2020.

\ex A certain type of bacteria increases continuously at a rate proportional to the number present. If there are 500 present at a given time and 1,000 present 2 hours later, how many will there be 5 hours from the initial time given?
\end{document}