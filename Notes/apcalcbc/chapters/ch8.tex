\documentclass[../bccalc.tex]{subfiles}
\graphicspath{{\subfix{../figures/}}}
\begin{document}
\chapter{Techniques of Integration}
\section{Integration by Parts}
Integration by parts is used to integrate a product, such as the product of an algebraic and a transcendental function:

For example, $\int xe^x dx, \int x\sin x dx, \int x\ln x dx$.

Recall the product rule is $\frac{d}{dx}[uv]=u\frac{dv}{dx}+v\frac{du}{dx}$.

Integrating both sides, we get $uv=\int u dv + vdu$.

Rearranging we get the formula for integration by parts:
\[ \int u dv = uv-\int v du \]

\begin{example}
    \[ \int x\sin x dx \]

    Let $u=x$, $dv= \sin x dx$, then $du=dx$ and $v=-\cos x$.

    We get $x\cos x+\int \cos dx$.

    Simplifying, we get $-x\cos x + \sin x+C$
\end{example}

\ex $\int x^2e^x dx$

A tabular approach is helpful with these ``repeated'' integration by parts problems 
\begin{example}
    \[ x^2e^x dx \]
\begin{center}
\begin{tabular}{c|c}
    $u$ & $v$ \\ \hline
    $x^2$ & $e^x$ \\
    $2x$  & $e^x$ \\
    $2$   & $e^x$ \\
    $0$   & $e^x$
\end{tabular}
\end{center}
    

    Criss crossing gives you $x^2e^x - 2xe^x+2e^x$.
\end{example}

If you have limits of integration, first integrate without them.
\begin{example}
    \[ \int_0^{\pi/2}x\sin x dx \]

    Using integration by parts you get $[-x\cos x+\sin x]$.

    Using the limits of integration, you get $1$.
\end{example}

\ex $\int e^x \cos x dx$
\section{Integration by Partial Fractions}
Fractions which have a denominator that can be factored can be decomposed into a sum or difference of fractions.

Fractions which have a denominator that can be factored into distinct linear factors 
\[ \frac{4x+1}{x^2-5x+6}=\frac{4x+1}{(x-3)(x-2)}=\frac{A}{x-3}+\frac{B}{x-2} \]
Solving for $A$ and $B$ results in $A=13$ and $B=-9$, so that the above equals 
\[ \frac{13}{x-3}-\frac{9}{x-2} \]

\begin{example}
    \[ \int \frac{4x+41}{x^2+3x-10}dx \]

    This is $\frac{4x+41}{(x-2)(x+5)}=\frac{A}{x-2}+\frac{B}{x+5}$ inside the integral.

    $4x+41=A(x+5)+B(x-2)$. If we let $x=-5$, $B=-3$. Let $x=2$, then $A=7$.

    We are now integrating $\int \frac{7}{x-2}-\frac{3}{x+5}dx$.

    This is $7\ln |x-2|-3\ln |x+5|+C$.
\end{example} 

\section{Logistic Growth}
In exponential growth (or decay), we assume that the rate of increase (or decrease) of a population at any time $t$ is directly proportional to the population $P$. In otehr words, $\frac{dP}{dt}=kP$. However, in many situations population growth levels off and approaches a limiting number $L$ (the carrying capacity) because of limited resources.
In this situation the rate of increase (or decrease) is directly proportional to both $P$ and $L-P$. This type of growth is called logistic growth. It is modeleted by the differential equation $\frac{dP}{dt}=kP{L-P}$.

If we find $\frac{d^2P}{dt^2}$ we can find out an important fact about the time when $P$ is growing the fastest. We will do this in the example below.

\begin{example}
    The population $P(t)$ of fish in a lake satisfies the logistic differnetial equation $\frac{dP}{dt}=3P-\frac{P^2}{6000}$ where $t$ is measured in years and $P(0)=4000$.

    (a) $\lim_{t\to \infty}P(t)=$?

    18000

    (b) What is the range of the solution curve?

    $4000\leq P(t)<18000$

    (c) For what values of $P$ is the solution curve increasing? Decreasing? Justify your answer.

    $P(t)$ is increasing because $\frac{dP}{dt}>0$.

    (d) For what values of $P$ is the solution curve concave up? Concave down? Justify your answer.

    Concave up from $(4000,9000)$ and concave down $(9000,18000)$.

    (e) Does the solution curve have an inflection point? Justify your answer.

    Yes because the second derivative changed signs.
\end{example}

\ex The population $P(t)$ of fish in a lake satisfies the logistic differential equation $\frac{dP}{dt}=3P-\frac{P^2}{6000}$ where $t$ is measured in years and $P(0)=10000$.

(a) $\lim_{t\to\infty}P(t)$

(b) What is the range of the solution cuve?

(c) For what values of $P$ is the solution curve increasing? Decreasing? Justify your answer.

(d) For what values of $P$ is the solution curve concave up? Concave down? Justify your answer.

(e) Does the solution curve have an inflection point? Justify your answer.

\ex The population $P(t)$ of fish in a lake satisfies the logistic differential equation $\frac{dP}{dt}=3P-\frac{P^2}{6000}$ where $t$ is measured in years and $P(0)=20000$.

(a) $\lim_{t\to\infty}P(t)$

(b) What is the range of the solution cuve?

(c) For what values of $P$ is the solution curve increasing? Decreasing? Justify your answer.

(d) For what values of $P$ is the solution curve concave up? Concave down? Justify your answer.

(e) Does the solution curve have an inflection point? Justify your answer.

\end{document}