\documentclass[../chem.tex]{subfiles}
\graphicspath{{\subfix{../figures/}}}
\begin{document}
\chapter{Properties of Substances and Mixtures}
\section{Solids, Liquids, and Gases}
AP Topic: 3.3

All matter has two distinct characteristics. It has mass and it occupies space. 

Solids have a definite shape and definite volume. The particles in a solid are packed tightly together and only vibrate gently around fixed positions.

Liquids have no shape of their own but take the shape of their container. A liquid has a definite volume. The particles in a liquid are free to move.

Gases have neither a definite shape nor volume. The particles in a gas spread apart filling all the space of the container avaliable to them.

Solids fall broadly into two categories; crystalline, where a regular, ordered, repeatable 3-D structure of particles if found, or 
amorphous, where the arrangement of particles is not regular or ordered. However, in both cases, the particles in the solid have very 
little energy and move very little in relation to one another.

Liquids have some properties 
\begin{itemize}
    \item Fluidity: liquids have the ability to flow. Gases are also fluid.
    \item Viscosity: the measure of the resistance to flow. Molecules with large intermolecular forces have greater viscosity but viscosity also increases with molecular complexity. Viscosity decreases with increasing temperature.
    \item Buoyancy: the upward force a liquid exerts on an object
\end{itemize}

Modeling a liquid is difficult. Like solids, liquids have particles that are very close together, but unlike solids the particles in liquids 
are constantly moving and colliding with one another, have significantly greater energy, and move a lot in relation to one another. Liquids have both 
strong intermolecular forces and quite a bit of motion.

Since solids and liquids tend to have their particles very close together, their volumes are often very similar.

Since, in a gas, the particles possess enough energy to overcome any intermolecular forces and hence move around completely freely and with large 
spaces between them, a liquid represents particles in an intermediate state between the extremely ordered and low energy state of a solid, and the extremely 
disordered and high energy state of a gas. Solids have very strong IMFs and next to no motion.

Gases will be described later in this chapter.

If you consider the solid, liquid, and gas state of one particular substance, in most cases teh solid is more dense than liquid which is more dense than gas.

An exception to this is water. Ice floats meaning that ice is less dense than liquid water. 

All of the properties of solids and liquids such as viscosity, surface tension, hardness, etc., are dependent upon how the particles 
that make up the solid or liquid are arranged, and the extent of the attractions between those particles. 

Since the different states of matter have particles with differing energies, converting between them requires a change in energy. The 
changes in energy associated with phase changes can be quantified in heating and cooling curves. You should know that matter can change from 
one phase to another by adding or removing energy and there are six phase changes.

Phase changes that require energy (also known as endothermic)
\begin{itemize}
    \item Melting: solid to liquid 
    \item Vaporization: liquid to gas, occurs when molecules have enough energy to escape the pull of the other molecules 
    \item Sublimation: solid to gas 
\end{itemize}

Phase changes that release energy (exothermic)
\begin{itemize}
    \item Condensation: gas to liquid 
    \item Freezing: liquid to solid 
    \item Deposition: gas to solid 
\end{itemize}

A phase diagram represents phases of matter as a function of temperature and pressure. 

There is some vocab related to phase diagrams 
\begin{itemize}
    \item Triple Point - the point on a phase diagram that shows the temperature and pressure combinations at which all the phases exist at equilibrium 
    \item Critical Temperature - temperature above which the vapor cannot be liquefied 
    \item Critical Pressure - pressure required to liquefy at the critical temperature 
    \item Critical Point - critical temperature and pressure coodinates
\end{itemize}

Each phase boundary represents an equilibrium set of pressure and temperature conditions.
\section{Intermolecular Forces}
AP Topic: 3.1

Now it is time to consider the forces that condense matter. The forces that hold one molecule to another molecule are referred to as intermolecular forces (IMFs).
These forces arise from unequal distribution of the electrons in the molecule and the electrostatic attraction between oppositely charged 
portions of molecules. IMFs are the forces between molecules.

Physical properties such as melting points, boiling points, vapor pressures, etc. can be attributed to the strength of the intermolecular attractions present between molecules. 
The lower the boiling point or vapor pressure of melting point, the weaker the intermolecular attractions. When the intermolecular forces of two different substances are similar, 
the substances tend to be miscible.

Inter bonding is based on the Coulombic attractions between opposite charges, but since the individual charges are usually relatively small, 
inter bonding is a relatively weak attraction when compared to intra bonding.

The presence of dipoles, and the intermolecular Coulombic attraction between them, determines the type of intermolecular forces present 
in covalently bonded compounds. In turn, the type of intermolecular forces present can greatly influence the properties. 

Londin Dispersion Forces (LDF's) are small electrostatic forces that are caused by the movement of electrons within the covalent bonds of molecules 
that would otherwise have no permanent dipole. As one molecule approaches another the electrons of one or both are temporarily displaced owing 
to their mutual repulsion. This movement causes small, temporary dipoles to be set up on the surface of the particles, which then attract one another. 
These attractions are called London Dispersion Forces, and exist between all atoms and molecules. Without these forces, we could not liquefy covalent gases of solidify covalent liquids. 

London dispersion forces increase with surface area and with the polarizability of the atom or molecule. Polarizability in turn increases 
with increases number of electrons. In short, larger molecules with larger surface areas have more electrons, which have greater polarizability.
This leads to more London dispersion forces, greater attractions and therefore higher boiling points. 

On descending group 18 the atoms of the elements get bigger, have more electrons and larger surface areas. This increases the London 
dispersion forces between them, making them more difficult to separate and increasing their boiling points. 

When molecules that have permanent dipoles come together, they will arrange themselves so that the negative and the positive ends of the molecule attract one another.

Molecules eventually align in order to find the best compromise between attraction and repulsion. The attractions are called dipole-dipole.
Compounds exhibiting dipole-dipole IMFs have higher melting and boiling points than those exhibiting weaker IMFs.

A similar intermolecular, electrostatic force is created when a polar molecule approaches a nonpolar molecule and induces a dipole in it. 
This is called a dipole induced dipole interaction. The polar molecule induces a temporary dipole in the nonpolar molecule. Larger molecules are more polarizable 
than smaller molecules since they contain more electrons. Larger molecules are more likely to form induced dipoles.

A third force involving dipoles is when a polar molecule interacts with an ion. This is called an ion-dipole force.

Each of these forces that include dipoles all rely on the Coulombic attaction between opposite charges.

Hydrogen is an exceptional element in that when it forms a covalent bond its electron is held to one side of the nuecleus leaving the other side 
completely exposed. Any approaching negatively charged group can get very close to the hydrogen nucleus and this produces an unexpectedly large 
electrostatic attraction. These electrostatic attractions are exaggerated when H is bonded to a more electronegative element that is small enough to 
allow significant intermolecular interaction. Such intermolecular, electrostatic attractions are a special type of dipole-dipole force called hydrogen 
bonds. This is typically the strongest IMF. The unique physical properties of water are due to the fact that it exhibits hydrogen bonding 
between separate water molecules. As a result of these attractions, water has a high boiling point, high specific heat, and many other unusual properties.

The occurrence of hydrogen bonds has two important consequences
\begin{itemize}
    \item It gives substances containing them unusually high boiling points 
    \item Substances containing them tend to be more viscous 
\end{itemize}

Both are explained by the increased attraction betweem molecules caused by hydrogen bonding, making it more difficult to separate them.

\section{Solubility}
AP Topic: 3.10

When the intermolecular forces of two substances are similar, the substances tend to be miscible. Ionic substances tend to be attracted to
and dissolve in polar substances such as water. Nonpolar substances tend to dissolve in nonpolar solvents. 

The important thing to remember here is this. In order for any solute to dissolve in any solvent, solute-solute and solvent-solvent attractions 
must be broken, and solute-solvent attractions must be made. It is the combination of these factors that determines if a solute will dissolve in 
any particular solvent, and only when the solute-solvent interactions are relatively strong compared to other attractions, will the solute dissolve.

Consider a liquid in a sealed container. Even if the liquid is beliw its boiling point a few of the molecules will possess enough energy 
to overcome the intermolecular forces holding them together and escape into the vapor phase above the liquid. The weaker the intermolecular forces 
between the molecules, the easier this process will be and the more molecules will enter the vapor phase. This causes a relatively high vapor pressure.
So, in summary, weak intermolecular forces cause liquids to have low boiling points, they are said to be volatile and will have high vapor pressures, and vice-versa.

Vapor pressure increases significantly with temperature. Increasing the temperature increases the kinetic energy which facilitates escape 
and the speed of the escapees. They bang into the sides of the container with more frequency and more energy. More molecules can attain the energy needed to 
overcome the IMFs in a liquid at a higher temperature since the kinetic energy increases.

In general, as molar mass increases, vapor pressure decreases, because as molecules increase in molar mass, they also increase in number of electrons. As the number 
of electrons increase, the polarizability of the molecule increases so more induced dipole-induced dipole or dispersion forces exist, causing 
stronger attractions to form between molecules. This decreases the number of molecules that escape and thus lowers the vapor pressure.

In the body of a liquid, all of the particles experience forces in three dimensions around them. This results in no net forces on these particles. 
However, these particles that are at the surface of the liquid, have no particles above them, and as such they arep ulled with a net force into the body 
of the liquid. This cohesion between particles of liquid, causes the liquid to contract to the smallest possible size and creates an internal pressure 
at the surface that can resist and external pressure. This is why it is possible to float very small objects with higher density than water, on the 
surface of water, and why water tends to bead into droplets. High surface tension indicates stronger IMFs.

If a liquid is placed into a very thin tube, the combination of cohesive forces within the liquid itself and adhesive forces between the liquid 
and the walls of the tube, can add up to overcome the force of gravity, and the liquid can be drawn up the tube without an external force 
being applied. The narrower the tube, the more the surface area of the glass, the higher the column of water climbs. The weight of the column sets the 
limit for the height achieved.

Both surface tension and capillary action rely upon the strength of interparticular attraction between the liquid particles.

In order for a substrate to interact with an enzyme, an intermolecular attraction must exist. In proteins, the primary, secondary, 
tertiary, and quaternary structures will lead to different three dimensional shapes that, depending on their orientation, may be attracted to polar 
water molecules or repelled.
\section{Properties of Solids}
AP Topic: 3.2

There are two broad categories for solids: crystalline and amorphous.

Crystalling solids have molecules packed together in a regular, repeating way. They are arranged in an orderly, geometric, three dimensional structure. 
The smallest repeating part of a crystalline structure is called a unit cell.

Molecular solids are discrete covalently bonded molecules at each of its lattice points. Units are molecules, held together by intermolecular 
forces using hydrogen bonds, dipole-dipole, or dispersion forces. Molecular solids are made from non-metals. Characterized by strong covalent bonding within 
the molecule yet weak forces between the molecules; therefore they tend to have low melting points and do not conduct electricity. Most are not solid at room temperature.

Atomic solids are where atoms of the substrance are located at the lattice points. Unit particles are atoms. They vary from being soft with very low melting points 
to being very hard with very high melting points and from being poor to excellent conductors.

Covalent network solids are composed of strong directional covalent bonds that are best viewed as one "giant molecule". The elements of group 14 can make 
four covalent bonds, and as such allow them to bond together in large, continuous networks.

Pure semiconductors like silicon are generally poor conductors of electricity, but when 'doped' the conductivity increases and can be controlled. 
WHen the doping is carried out with an element that has an extra valence electron compared to silicon, an n-type conductor is produced.

Conversely, when the doping is carried out with an element that has one less valence electron compared to silicon, a p-type conductor is produced. 
Disrupting the valence shells of the silicon atoms in this manner effectively allows electrons to flow, and the previous insulator becomes a good conductor.
The boundaries between p and n are called p-n junctions, and the control of electrons through these junctions forms the basis of electronics.

Metallic solids have atoms that are surrounded by delocalized valence electrons. Metals are characterized by high thermal and electrical conductivity, malleability, and ductility.
These properties are explained by the non directional covalent bonding found in metallic crystals. It is difficult to separate atoms, but easy to move them provided 
they stay in contact with each other. Melting points and hardness varies widely. 

A metal's structure can be considered to be a close packed lattice of positive ions surrounded by a "sea" of moving, delocalized electrons. These 
electrons and their movement cause metals to be good conductors of electricity. The close packed ions make them good conductors of heat.

Noble gases can form atomic solids with very low melting and boiling points.

Ionic solids contain ions at the points of the lattice that describe the structure of the solid. Each ion in an ionic solid is surrounded by ions of the opposite charge in a 3D crystal lattice. 
This network gives ionic solids hardness. This structure also means that ionic compounds are not malleable or ductile and tend to be brittle, since 
when the ordered structure is disrupted, like charges repel, and the solid splits paart. Ion-ion Coulombic forces are the strongest of all attractive forces.
"IMFs" usually implies covalently bonded substances, but can include ionically bonded substances as well. Ionic bonds can be considered both an intra bond and an inter bond.

In the ionic structure the lattice of ions is held together by strong electrostatic interactions between them. The strong bonds give ionic solids high melting and boiling points 
and low volatility and subsequently low vapor pressures.

Ionic substances can only conduct electricity when molten or in solution, since in the solid the ions are rigidly held and cannot move.

When an ionic substance dissolves, the polar water molecules are attracted to the oppositely charged ions, and penetrate the lattice attaching themselves to the ions.
The process is called hydration, and the ions are said to be hydrated. The ions become free to move when they are hydrated, and the solution will be a good conductor of 
electricity. Since a nonpolar solvent will not be attracted to the ions in an ionic solid, the ionic bonds holding the solid together are not broken and the solid will not dissolve.

Amorphous solids have particles that are not arranged in a regular repeating pattern. Amorphous means "without shape".

Intermolecular forces are especially important in large, biochemical and organic molecules. These intermolecular forces of attraction can occur within a single molecule, 
or between two separate molecules. When occurring within a single molecule, the physical shape and properties of these molecules can be significantly affected.
\section{Kinetic Molecular Theory}
AP Topic: 3.5

All substances have three phases: solid, liquid, or gas. Substances that are solids or liquids under room conditions may also exist as gases, often referred to as vapors. Many 
of the properties of gases differ from those of solids and liquids.
\begin{itemize}
    \item Gases are made up of particles that have large amounts of energy .
    \item A gas has no definite shape or volume, and will expand to fill as much space as possible. 
    \item As a result of the large amount of empty space in a volume of gas, each molecule of a gas behaves largely as though other molecules were absent. 
    \item Gases are easily compressed.
    \item Gases will mix completely with any other gas.
    \item Gases exert pressure on their surroundings.
\end{itemize}

Pressure is the force acting on an object per unit area.

The SI unit of pressure is Pascal.
\begin{itemize}
    \item 1 Pa = 1 N/m$^2$
    \item 1 N = 1 kg$\cdot$m/s$^2$
\end{itemize}

Gravity exerts a force on Earth's atmosphere. A column of air 1 m$^2$ in the cross section extending to the top of the atmosphere exerts a force of 
$10^5$N. Thus the pressure of that column of air is 100 kPa.

Atmospheric pressure is measured with a barometer. If a tube is completely filled with mercury and inverted into a container of mercury open to the atmosphere, the mercury 
will rise until the pressure due to the mass of the mercury column is the same as atmospheric pressure. Standard atmospheric pressure is the pressure required 
to support 760 mm of Hg in a column.
\[1.00 \text{atm} = 760. \text{mmHg} = 760. \text{torr} = 1.0325\times 10^5 \text{Pa} = 101.325 \text{kPa}\] 

In the laboratory, the pressures of gases not open to the atmosphere are measured using a manometer. A manometer consists of a bulb of gas attached to a 
U-tube containing Hg. If the U-tube is closed, then the pressure of the gas is the difference in the height of a liquid. If the U-tube is open to the atmosphere, a correction needs to be added
\begin{itemize}
    \item If P$_{\text{gas}} <$ P$_{\text{atm}}$ then P$_{\text{gas}} +$P$_{\text{h}}=$P$_{\text{atm}}$
    \item If P$_{\text{gas}} >$ P$_{\text{atm}}$ then P$_{\text{gas}} =$P$_{\text{atm}}=$P$_{\text{h}}$
\end{itemize}

Pressure is exerted when gas particles collide with the walls of any container it is held in.

Kinetic molecular theory was developed to explain gas behavior. The five postulates are:

1. Gases are composed of tiny particles whose size is negligible compared to the average distance between them. This means that the volume of the actual, 
individual molecules in a gas can be assumed to be negligible compared to the volume of the container, and therefore the total volume that the gas fills is almost 
all empty space. The observation that gases are compressible is consistent with the assumption that the actual gas particles have a small volume compared to the total volume.

2. Gas particles move randomly, in straight lines, in all directions and at various speeds.

3. The forces of attraction or repulsion between two gas particles are negligible, except when they collide.

4. When gas particles collide with one another, the collisions are elastic. The collisions with the walls of the container create the gas pressure.
Elastic collisions is consistent with the observation that gases, when left alone in a container, do not appear to lose energy and do not spontaneously 
convert to a liquid.

5. The average kinetic energy of a gas particle is proportional to the Kelvin temperature, and as a result, all calculations involving gases should be carried out with temperatures converted to K.

These assumptions do have some limitations. Real gas particles do attract one another to some extent and can stick to one another and are able to sometimes condense to form a liquid.

\section{Ideal Gas Law}
AP Topic: 3.4

Boyle's law states that, at constant temperature, pressure is inversely proportional to volume. This means that as pressure increases, volume decreases, and vice versa.
\[ \text{PV} = \text{a constant}\]  

A plot of V versus P is a hyperbola. Similarly, a plot of V versus 1/P is a straight line through the origin.

In terms of KMT, if the volume is increased, the gas particles collide with the walls of the container less often and the pressure is reduced. The process of breathing also illustrates Boyle's Law.

If we know the volume and pressure of a gas at a given temperature, and then volume or pressure is changed, Boyle's law allows us to calculate the new volume or pressure by applying the simple relationship:
\[\text{P}_1\text{V}_1=\text{P}_2\text{V}_2\] 

P$_1$ and V$_1$ are the original condictions and P$_2$ and V$_2$ are the new conditions.

Boyle's Law has been tested over three centuries. It holds true only at low pressures. An ideal gas is expected to have a constant value of PV.

Charles' law states that, at constant pressure, volume is directly proportional to absolute temperature. This means the volume of a gas increases with increasing temperature, and vice versa.
A plot of V versus T is a straight line.

$\frac{V}{T} = \text{constant}$

In terms of KMT, if the temperature is increased, then the gas particles gain kinetic energy, move around more, and occupy more space.

Charles's law allows us to calculate the new volume or temperature by applying the simple relationship 
\[\frac{V_1}{T_1}=\frac{V_2}{T_2}\]

Avogadro's law states that, at constant temperature and pressure, volume is directly proportional to the number of moles of gas present. This means that the volume of a gas increases 
with increasing number of moles, and vice versa. 
\[\frac{V}{n}=\text{a constant}\]

In terms of KMT, as more moles of a gas are placed into a container, if conditions of temperature and pressure are to remain the same, the gas must occupy a larger volume.

The relationship between the two is below 
\[\frac{V_1}{n_1}=\frac{V_2}{n_2}\]

Gay-Lussac's law states that, at constant volume, pressure is directly proportional to temperature. This means that tempearture increases with increasing pressure, and vice versa.
\[\frac{P}{T}=\text{a constant}\]

In terms of KMT, if the temperature of a gas is raised, then the particles will have more energy, the collisions with the walls of the container will occur with a greater force, and the pressure will increase.
\[\frac{P_1}{T_1}=\frac{P_2}{T_2}\]

Combining the equations of the gas laws above, gives 
\[\frac{P_1V_1}{n_1T_1}=\frac{P_2V_2}{n_2T_2}\]

If the number of moles of gas in an experiment is constant, then the expression becomes 
\[\frac{P_1V_1}{T_1}=\frac{P_2V_2}{T_2}\]

The combination of laws above leads to the formulation of the ideal gas law. Most gases obey this law at temperatures above 273 K and at pressures of 1.00 atm or lower. 
An ideal gas has particles that are assumed to have negligible volume when compared to the total volume, and particles that do not attract one another.
\[\text{PV}=\text{nRT}\]

R is the universal gas constant and is equal to 0.0821 L atm K$^{-1}$ mol$^{-1}$ when P is measured in atm and V is measured in L. 

This equation is useful because it can be manipulated to include other variables. For example we can find a relationship with molar mass and density 
\[MM=\frac{DRT}{P}\]

Since gas molecules are so far apart, we assume they behave independently. Dalton observed that the total pressure of a mixture of gases equals the sum of the pressures that each would exert if present alone.
Partial pressure is the pressure exerted by a particular component of a gas mixture.
\[\text{P}_{\text{total}}=\text{P}_1+\text{P}_2+\text{P}_3\dots \]

Assuming ideal behavior, the equation can be simplified to 
\[P_{\text{total}}=n_{\text{total}}\left(\frac{RT}{V}\right) \]

Additionally, if n$_1$ is the number of moles of gas 1 exerting a partial pressure P$_1$, then 
\[\text{P}_1=\text{X}_1\text{P}_{\text{total}}\]

Where X$_1$ is the mole fraction of gas 1 = n$_1$/n$_{\text{total}}$

It is common to synthesize gases and collect them by displacing a volume of water. To calculate the amount of gas produced, we must correct for the partial pressure of the water vapor 
\[\text{P}_{\text{total}}=\text{P}_{\text{gas}}+\text{P}_{\text{water}}\]

The vapor pressure of water varies with temperature.

In statistics, the Maxwell-Boltzmann distribution is a particular probability distribution. It was first defined and used in physics for describing particle speeds in idealized gases.
A particle speed probabilty distribution indicates which speeds are moe likely: any single particle will have a speed selected randomly from the distribution, and is more likely to be within one range 
of speeds than another. The distribution depends on the temperature of the system and the mass of the particle.

Keep in mind the Maxwell-Boltzmann distribution applies to an ideal gas. In real gases, there are various effects that can make their speed distribution different from the Maxwell-Boltzmann form.

Note that although absolute temperature of a gas is a measure of the average kinetic energy, some molecules will have less or more KE than the average. There is a spread of individual energies 
of gas molecules in any sample of gas. As the temperature increases, the average KE of the gas molecules increases. As KE increases, the velocity of the gas molecules increases.

The root mean square speed ($u_{rms}$) of the velocities of gas particles is the square root of the averages of the squares of the speeds of all the particles at a particular temperature.
\[u_{rms}=\sqrt{\frac{3RT}{MM}}\]

Note $u_{rms}$ is the speed of a gas molecule having average KE. Average KE is related to $u_{rms}$
\[\text{KE}=\frac{1}{2}\text{m}u_{rms}^2\]

where m is the mass of the molecule.

The effect of an increased volume at constant temperature is that as V increases at constant T, the average KE of the gas remains constant, meaning that $u_{rms}$ is a constant.
However, V increases, so the gas molecules have to travel further to hit the walls of the container so P decreases.

The effect of an increased temperature at constant volume is that if T increases, the average KE Of the gas molecules will increase so there are more collisions 
with the container walls. The change in momentum in each collision increases so P increases.

Effusion is the process in which a gas escapes from one vessel to another by passing through a very small opening

Diffusion is the process in which a homogeneous mixture is formed by the random mixing of two different gases.

Graham's law of effusion and diffusion states that the rate of effusion or diffusion of two gases is inversely proportional to the square roots of their respective densities and molecular masses
\[\frac{\text{Rate of effusion of A}}{\text{Rate of effusion of B}}=\sqrt{\frac{\text{density of B}}{\text{density of A}}}=\sqrt{\frac{\text{MM of B}}{\text{MM of A}}}\]

Ligher gas particles effuse and diffuse at higher rates than heavier gas particles.

\begin{itemize}
    \item Diffusion is faster for lighter gas molecules 
    \item Diffusion is significantly slower than the $u_{rms}$ speed. 
    \item Diffusion is slowed by collision 
    \item The average distance traveled by a gas molecule between collisions is called the mean free path. It would be very erratic if we could monitor the path of a single molecule.
\end{itemize}

Imagine a solution, in a closed container. with a gas filling the space above it.

At higher pressures, more gas molecules strike the surface of the solution and enter the solvent, meaning the concentration of the gas dissolved in the solvent is greater.

Gas solubility usually decreases with increase in temperature of the solution, since the gas particles have more energy, and can escape from the solvent, meaning less gas is dissolved in solution.
\section{Deviation from Ideal Gas Law}
AP Topic: 3.6

At high pressures and low temperatures gas particles come close enough to one another to make two postulates of the kinetic molecular theory invalid.
\begin{itemize}
    \item The assumption that gases are composed of tiny particles whose size is negligible compared to the average distance between them begins to fail. When the gas is pressurized into a small space the gas particles size becomes more significant compared to the total volume.
    \item The assumption that the forces of attraction or repulsion between two particles in a gas are very weak of negligible begins to fail. Low temperature means less energy, so the particles are attracted to one another more.
\end{itemize} 

Under these conditions, gases are said to behave non-ideally, or like real gases. There are two consequences.

1. When gases are compressed to high pressures, the size of the gas particles is no longer negligible compared to the total space occupied by the gas. Therefore, the observed total volume occupied by the gas 
under these real conditions is artificially large since the gas particles are now occupying a significant amount of that total volume.

2. The actual pressure of a gas is lower than one would expect when assuming there were no attractive forces between the particles. Because, in a real gas, the 
particles are attracted to one another, they collide with the walls with less force, and the observed pressure is less than in an ideal gas.

These corrections lead to the Van der Waals equation. You do not need to know this equation, but you should know how to relate the relative sizes of $a$ and $b$ to the deviations above.
\[\left(P+\frac{an^2}{V^2}\right)(V-nb)=nRT \] 

Small, nonpolar molecules also behave more ideally than large, polar molecules.
\section{Solutions and Mixtures}
AP Topic: 3.7

Solutions are homogeneous mixtures that are composed of a solute and a solvent.
\begin{itemize}
    \item Solute: component in lesser concentration 
    \item Solvent: component in greater concentration 
\end{itemize}

Solubility is the maximum amount of material that will dissolve in a given amount of solvent at a given temperature to produce a stable solution. In other words, the solution is saturated.

There are some factors affecting solubility. 

First, molecular structure.

Second, pressure. The solubility of a gas increases with increasing temperature. Henry's Law states that the amount of gas dissolved in a solution 
is directly proportional to the pressure of the gas above the solution. Henry's Law is obeyed best for dilute solutions of gases that don't dissociate 
or react with the solvent.
\[C=kP\]
Where $P$ is the partial pressure of the gaseous solute above the solution, $k$ is a constant, and $C$ is the concentration of dissolved gas.

Increasing pressure has very little effect on the solubility of liquids and solids.

Third, temperature. The solubility of a gas in water always decreases with increasing temperature. 

The dissolving of a solid occurs more rapidly with an increase in temperature, but the amount of solid may increase or decrease with an increase in tempearture.
It is very difficult to predict what this solubility may be - experimental evidence is the only sure way.
\begin{itemize}
    \item The amount of solute what will dissolve usually increases with increasing temperature since most solution formation is endothermic.
    \item Solubility generally increases with temperature if the solution process is endothermic.
    \item Solubility generally decreases with temperature if the solution process is exothermic.
\end{itemize}

Molarity is the number of moles of solute per liter of solution and is tempearture dependent. The liquid solvent can expand and contract with changes in temperature.
Most molar solutions are made at 25$^{\circ}$C, so this point is subtle and picky, but important nonetheless. In relatively low molarity solutions, 
there are small numbers of solute particles compared to the number of solvent particles, and in relatively high molarity solutions there are 
relatively large numbers of solute particles compared to the number of solvent particles.
\[M=\frac{\text{moles of solute}}{\text{liters of solution}}\]

Mass percent is the percent by mass of the solute in the solution.
\[\text{Mass Percent} = \frac{\text{grams of solute}}{\text{grams of solution}}\times 100\]

Mole fraction is the ratio of the number of moles of a given component to the total number of moles present.
\[\text{Mole Fraction}= X_a=\frac{n_a}{n_a+n_b+\dots}\]

Molality is the number of moles of solute per kilogram of solvent and is not temperature dependent. 
\[m=\frac{\text{moles of solute}}{\text{kilograms of solvent}}\]

Often solutions are prepared by adding water to more concentrated tools. Calculations involving dilution problems involve three steps.
\begin{enumerate}
    \item Calculate the number of moles present in the final, diluted solution, by applying moles = (concentration)(volume).
    \item Calculate the volume the starting, more concentrated solution that supplies this number of moles by applying moles = (concentration)(volume).
    \item The volume of water that must be added to the concentrated solution is simply the difference between the volume of the final, diluted solution and the volume of the concentrated solution.
\end{enumerate}

In practical terms, the use of highly accurately graduated glassware is required.
\begin{itemize}
    \item A graduated measuring pipet is used to measure volumes for which a volumetric pipet is not avaliable.
    \item A volumetric or transfer pipet gives one and only one measurement but is mighty accurate.
\end{itemize}

This can be summarized with the dilution formula 
\[M_1V_1=M_2V_2\]

When diluting a concentrated acid or base, it is often found that combining water and the acid or base is a very exothermic process. In some 
cases this energy can be very significant and may even cause the water present to turn into the gaseous state. As the steam leaves the system, 
it can cause a spray of concentrated acid or base and represents a significant safety hazard. In order to minimize this hazard, it is vital to always 
add acid to water, and not water to acid.

The dilution sequence of adding a relatively small amount of acid to a relatively large volume of water ensures that the heat generated is kept to a minimum,
and that acid remains in the presence of as much water as possible until it reaches the desired dilution. This will minimize the risks of accidents.

You may be asked to prepare a standard solution using a solid solute. This is a solution whose concentration is accurately known. To do so, mass an appropriately calculated 
amount of solid and place it in a volumetric flask. Add only enough distilled or deionized water to dissolve the solid and swirl to completely 
dissolve the solid. Then add more water, filling to the mark on the flask. Secure the stopper and invert the flask a few times to ensure 
even distribution of the solute throughout the solution. If you dump solid into 1.00 L of water you are neglecting the space the solid will occupy and your molar concentration will not be correct.

When a solute is dissolved in a solvent, the attractive forces between solute and solvent particles are great enough to overcome the attractive forces within the 
pure solvent and within the pure solute. The solute becomes solvated. When the solvent is water, the solute is hydrated.

\begin{itemize}
    \item Substances with similar types of intermolecular forces dissolve in each other.
    \item Water dissolves many salts because the stronger ion-dipole attractions water forms with the ions of hte salt are very similar to the strong attractions between the ions themselves.
    \item Oil does not dissolve in water. Oil is immiscible in water because that any weak dipole-induced dipole attractions that form between oil and water cannot overcome the stronger dipole-dipole hydrogen bonding that water molecules have for each other.
\end{itemize}

The enthalpy change associated with the formation of a solution can be negative or positive.

Enthalpy of hydration is more negative for small ions and highly charged ions.

Some heats of solution are positive. The reasion that the solute dissolves is that the solution process greatly increases the entropy which 
overrides the cost of the small positive heat of solution. This makes the process spontaneous. The solution process involves two factors; the change in heat and the 
change in entropy, and the relative magnitude of these two factors determine whether a solute dissolves in a solvent.
\section{Representations of Solutions}
AP Topic: 3.8

Solutions may be represented by particulate diagrams where properties of the solution such as concentration and interparticle interactions 
can be illustrated.
\section{Separation of Solutions and Mixtures Chromatography}
AP Topic: 3.9

Solutions can be gases, liquids, or solids. For a liquid solution the solute can be a gas, liquid, or a solid. In solutions the solute size 
is very tiny. Like a solution, a colloid is a mixture of two separate substances, but where the solute particles are larger in size than molecules.
If the particle size exceeds 1000 nm, the mixture becomes a suspension. These particle size distinctions mean that a mixture can appear homogeneous 
or heterogeneous depending on the scale on which it is observed.

Liquid solutions, by definition, cannot be separated into their components using a filter paper and have no particles large enough to scatter
visible light. Two methods that can be used to separate such solutions are chromatography and distillation.

All chromatography techniques involving a moving phase and a stationary phase. In the most common applications of paper chromatography, 
the paper is the stationary phase and the solvent is the moving phase. When the solution mixture is exposed to the two phases, each component
of the mixture will have a greater or lesser affinity for the stationary or moving phase. If a component of the mixture has a high 
affinity for the mobile phase, it will move with the moving phase and travel a relatively long distance on the chromatogram. A component with 
less affinity for the mobile phase, or only a larger affinity for the stationary phase, will not move as far. The different distances of travel mean 
that the components of the mixture are physically separated.

It is possible to apply a quantitative analysis to the chromatogram by calculating a R$_f$ value for each component of the mixture.
\[R_f=\frac{\text{Distance traveled by component of mixture}}{\text{Distance traveled by solvent}}\]

The distance travelled by the solvent is determined by measuring the distance from the baseline, and the maximum distance than the solvent has travelled, known as the solvent front.

Distillation is a simple separation technique based upon differences in boiling point of two components of a liquid mixture. The boiling point 
of the individual components of a mixture depends upon the intermolecular attractions between the particles of that component. 

\section{Spectroscopy and the Electromagnetic Spectrum}
AP Topic: 3.11

Spectroscopy is the study of the interaction of electromagnetic radiation and matter. Absorption spectroscopy involves a sample being exposed 
to radiation of varying types, and then observing what happens as the sample interacts with the radiation. Spectroscopy is classified based 
on the energy of the radiation source used - different energies of light probe for different kinds of information about a sample.

Ultraviolet and visible light are used when we want to know more about molecules or metal ions, when the sample is in an aqueous solution.
You can probe the purity or concentration of samples, or whether the substance contains pi-bonds.

Infrared spectroscopy is used when you want to know about molecules, mostly those with covalent bonds, when the sample is in a solution or a solid.
When covalent bonds are exposed to infrared radiation they absorb that energy and tend to bend, stretch and vibrate. The interaction with the IR 
is unique for each type of bond, so IR spectroscopy can be used to distinguish between compounds that have different types of covalent bonds.

Molecular rotations can be caused by the use of microwaves, and can be used to determine the chemical composition and structure of molecules.

\section{Photoelectric Effect}
AP Topic: 3.12

Albert Einstein was a fan of Max Planck's work and proposed that all EM radiation, not just the UV region that Planck focused on, could be 
viewed as a stream of ``particles'' called photons. This explained the photoelectric effect, whereby light bombarding the surface of a metal ejects electrons.
\begin{itemize}
    \item Energy is quantized.
    \item Energy can only occur in discrete units called quanta.
    \item EM radiation exhibits both wave and particle properties.
    \item Keep in mind, the more massive the object, the smaller its associated wavelength and vice versa.
\end{itemize}
Whether the type of radiation, the neergy absorbed or emitted by the matter is governed by two equations.
\[c=\lambda\nu\]
Where $c$ is the speed of light, $\lambda$ is wavelength, and $\nu$ is frequency.
\[\text{E}=h\nu\]
Where E is energy, $h$ is Planck's constant, and $\nu$ is frequency.

Note that there is a minimum amount of energy that can be gained or lost by an atim, and all energy gained or lost must be some integer multiple, n, of 
that minimum. The lost minimum energy change, h$\nu$ is called a quanta of energy. Think of it as a ``packet'' of energy equal to $h\nu$ 
\[\Delta{\text{Energy}}=n(h\nu)\]

Where $h$ is the proportionality constant. This $\nu$ is the lowest frequency that can be absorbed or emitted by the atom. There is no such thing as a 
transfer of E in a fraction of a quantum, only in whole numbers of quanta.

\section{Beer-Lambert Law}
AP Topic: 3.13

The Beer-Lambert law is used to relate the concentrations of colored solutions to the amount of visible light they absorb. The amount 
of absorbance is calculated using the formula
\[A=\epsilon b c\]
Where $A$ is absorbance, $\epsilon$ is molar absorptivity, $b$ is path length, and $c$ is concentration.

When absorbance measurements are made at a fixed wavelength, in a cell of constant path length, both a and b are constant, and the absorbance, A, will 
be directly proportional to c. If a solution of a compound obeys the Beer-Lambert law, a plot of absorbance versus concentration gives a 
straight line with slope ab. The y-intercept is zero. One can use the graph to read corresponding concentrations and absorption values.

A plot of absorption against wavelength can be used to determine the exact color of a solution. The point at which the greatest absorption is observed can be used to determine,
via a color wheel, which wavelength is being reflected and therefore the color of the solution. The color that is observed is due to the wavelengths of light 
that the sample did not absorb. A color wheel can be used to relate absorbed and transmitted colors, the transmitted color being the 
complementary color of the absorbed light.

Althought theoretically the spectrophotometer can be used at a number of different wavelengths, because of the limitation of the electronics, 
the optimum wavelength is where the absorbance is highest. Performing the experiment at the point of highest absorbance offers at least two advantages.
Firstly, Beer's law linear relationship between concentration and absorbance is likely to hold around this point, and secondly, when diluting 
the solution in order to investigate other lower concentrations it is likely that if one starts at a point of maximum absorbance that the 
absorbance will still remain significant and therefore detectable at the lower concentrations.

\pagebreak
\section*{Problems}
\begin{enumerate}
    \item The boiling point of HCl is -85$^{\circ}$C. The boiling point of chlorine, Cl$_2$ is -34$^{\circ}$C. Discuss the differences in the boiling points of these substances in terms of the intermolecular forces present.
    \item A vessel connected to an open-end manometer is filled with gas to a pressure of 0.835 atm. The atmospheric pressure is 755 torr. (a) In which arm of the manometer will the level of mercury be higher? (b) What is the height difference between the two arms?
    \item The pressure on a 411 mL sample of gas is decreased from 812 mmHg to 790 mmHg. What will the new volume of the gas be?
    \item A gas has a volume of 0.572 L at 35.0$^{\circ}$C and 1.00 atm pressure. What is the temperature inside a container where this gas has a volume of 0.535 L at 1.00 atm?
    \item If 2.11 g of neon gas occupies a volume of 12.0 L at 28.0$^{\circ}$C. What volume will 6.58 g of neon occupy under the same conditions?
    \item A gas exerts a pressure of 900 mmHg at 20$^{\circ}$C. What temperature would be required to lower the pressure to 1.00 atm?
    \item A sample of aluminum chloride weighing 0.100 g was vaporized at 350.$^{\circ}$C and 1.00 atm pressure to produce 19.2 mL of vapor. Calculate a value for the molar mass of aluminum chloride.
    \item Ammonium nitrite decomposes upon heating to form N$_2$ gas and water: NH$_4$NO$_2$(s)$\rightarrow$N$_2$(g)+2H$_2$O(g). When a sample of ammonium nitrite is decomposed in a test tube, 511 mL of N$_2$ gas is collected over water at 26$^{\circ}$ and 745 torr total pressure. How many grams of NH$_4$NO$_2$ were decomposed? The vapor pressure of water at 26$^{\circ}$C is 25.2 torr.
    \item What can be said about the $u_{rms}$ of a gas in relation to its molar mass and in relation to its temperature?
    \item The electrolyte in automobile lead storage batteries is a 3.75 M sulfuric acid solution that has a density of 1.230 g/mL. Calculate the mass percent and molality of the sulfuric acid.
    \item The blue color in fireworks is often achieved by heating copper(I) chloride (CuCl) to about 1200$^{\circ}$C. Then the compound emits blue light having a wavelength of 450 nm. What is the increment of energy that is emitted at $4.50\times 10^{2}$ nm by CuCl?
\end{enumerate}
\end{document}