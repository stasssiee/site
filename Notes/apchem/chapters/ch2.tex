\documentclass[../chem.tex]{subfiles}
\graphicspath{{\subfix{../figures/}}}
\begin{document}
\chapter{Compound Structure and Properties}
\section{Types of Chemical Bonds, Intramolecular Force and Potential Energy, and Structure of Ionic Solids}
AP Topic: 2.1, 2.2, \& 2.3

An electron cloud diagram can be thought of as a probability map of where an electron may be found at any one time. 

When two atoms join together with a covalent bond, a pair of electrons is shared between the atoms. Each atom within a covalent bond has a property
known as electronegativity. 

Elements that have electronegativities thata re different but very similar are usually considered to be equally shared. 

When one atom has a much higher electronegativity than the other, then the electrons are attracted toward the more electronegative atom, 
leading to an electron cloud distortion annd a re-distribution of electron cloud density.

The creation of opposite charges at either end of the molecule is called a dipole, and the bond is said to be polar covalent in this case.
With a diagram with no creation of opposite charges, the bond is said to be nonpolar covalent.

The presence of dipoles, and the intermolecular Coulombic attraction between them, determines the type of intermolecular forces 
present in covalently bonded compounds. In turn, the type of intermolecular forces present can greatly influence the properties of the compound.

Most compounds are covalently bonded, especially carbon compounds. In covalent bonding, electrons are shared between atoms to once again achieve 
full s and p subshells, this time by joining together and sharing valence electrons. When they do so they form discrete molecules. Molecular formulas 
are given in the true ratios of atoms. One shared pair of electrons represents a single covalent bond; two shared pairs represent a double bond.
These bonds usually occur between atoms that are non-metals. Unlike ionic substances, which tend to be solids at room temperature, covalent substances 
may exist in any state of matter at room temperature, but melt at low temperatures. These substances are nonconductors of electricity.

Atoms are ttracted to one another when the outer electrons of one atom are electrostatically attracted to the nuclei of another atom. The 
attraction between two atoms makes them increasingly stable, giving lower and lower potential energies.

However, as the atoms continue to approach one another and get increasingly close, there comes a point at which the two nuclei will start to repel one another.
As they start to repel one another the potential energy is raised, and the two atoms become less stable.

A happy medium is reaches at a distance where the forces of attraction and repulsion result in the lowest potential energy. The distance is called the 
bond length, and the potential energy at that point is called the bond strength.

Since for forces of attraction that stablize atoms when they bond are attractions between electrons and nuclei, the greater the number of 
electrons involved, the stronger the attraction and as such, triple covalent bonds tend to be stronger than double bonds, and double bonds 
tend to be stronger than single bonds. Shorter bonds also tend to be stronger than longer ones.

Ionic bonding involves the transfer of electrons between atoms to form ions. 

Atoms have equal numbers of protons and electrons and consequently have no overall charge. When atoms lose or gain electrons, the proton/electron numbers 
are unbalanced causing the particles to become charged. These charged particles are called ions. Since metals have a tendency to lose electrons to 
form positive ions, and nonmetals the opposite, the ionic bond is usually formed between metal and nonmetal. These strong electrostatic forces 
are between the charged particles are called ionic bonds. The stronger ionic bonds are formed between ions that are small and highly charged.

The ions present in an ionic solid are held rigidly in fixed positions in a giant, 3-D lattice. Ionic formulas the given in the simplest 
ratio of elements. These rigid structures mean that ionic compounds are not malleable or ductile and tend to be brittle.

The strong bonds in the lattie gives ionic solids high melting and boiling points and low volatility and subsequently low vapor pressures. 

Ionic substances can only conduct electricity when molten or in solution, since in the solid the ions are rigidly held and cannot move.

When an ionic solid dissolves, the polar water molecules are attracted to the oppositely charged ions, and penetrate the lattice attaching themselves to the ions.
The process is called hydration, and the ions are said to be hydrated. The ions become free to move when they are hydrated, and the solution will 
be a good conductor of electricity. Since a non-polar solvent will not be attracted to the ions in an ionic solid, the ionic bonds holding the solid 
together are not broken and the solid will not dissolve.

The hydration process increases the entropy (see Unit 9).

There are some following cutoffs that are somewhat arbitrary, make sure to keep in mind that bond polarity is relative.
\begin{itemize}
    \item $\Delta\text{EN}<0.5$: nonpolar covalent 
    \item $0.5\leq \Delta\text{EN}\leq 1.7$: polar covalent 
    \item $\Delta\text{EN}>1.7$: ionic
\end{itemize}

To recap, bonds are attractive forces that hold groups of atoms together within a molecule or crystal lattice and make them function as a unit.
Bonding relates to physical properties such as melting point, hardness, and electrical and thermal conductivity as well as solubility characteristics.
The system is achieving the lowest possible energy state by bonding. Energy is released when a bond is formed, therefore, it requires energy to break a bond.

Like inter bonding, intra bonding is based upon Coulombic attraction, but the attractions here are relatively very strong, making intra bonds much stronger than intermolecular forces. 
Recall Coulomb's Law describes forces interacting between static electrically charged particles:
\[F=k_e\left(\frac{q_1q_2}{r^2}\right)\]

The force of attraction between the charges is attractive if the charges have opposite signs and repulsive if like-signed. Coulomb's Law 
can also be used to calculate the energy of an ionic bond 
\[E=2.31\times10^{-19}\text{J}\cdot\text{nm}\left(\frac{q_1q_2}{r^2}\right)\]

There will be a negative sign on the energy once calculated. This indicates an attractive force so that the ion pair has lower energy 
than the separated ions. You can also use Coulomb's Law to calculate the repulsive forces between like charges.
\section{Lewis Diagrams}
AP Topic: 2.5

Lewis structures use dots to represent valence electrons in atoms when they form molecules. Lewis structures are very useful when predicting 
molecular shape of molecular geometry. As discussed, when atoms form molecules they share electrons to achieve full s and p subshells. Since 2 electrons 
are required to fill the s subshell and 6 electrons fill the p subshell, an octet of electrons is the goal. There are some exceptions to the octet rule:
\begin{itemize}
    \item Hydrogen can handle at most 2 electrons. Be only has 4 valence electrons. Boron has 6 valence electrons. 
    \item Expanded octets can only happen if the central element has d-orbitals which means it is from the third period or greater and can thus be surrounded by more than four valence pairs in certain compounds. 
    \item A few stable compounds contain an odd number of valence electrons and cannot obey the octet rule.
\end{itemize}

To draw Lewis Structures:

1. Calculate the total number of valence shell electrons.

2. In a species with more than two atoms, decide which atom is the central atom. Use one pair of electrons to form a covalent bond between the 
terminal atoms that are bonded to the central atom.

3. Arrange the remaining electrons to complete the octets of the terminal atoms and then place any remaining electrons on the central atom.

4. If the central atoms lacks an octet, form multiple bonds by converting non-bonding electrons from terminal atoms into bonding pairs. 

5. One bonding pair of electrons represents one covalent bond that in turn can be represented by a single line. Double bonds share two pairs of electrons, 
represented by two lines and triple bonds share three pairs of electrons, represented by three lines.

6. Any electron pairs that occur in the valence shell of an atom but do not form a bond with another atom are called nonbonding electrons or lone pairs or an unshared pair.

Occasionally when drawing a Lewis structure you may encounter a molecule with a central atom that does not have a complete octet of electrons surrounding it.
The central atom in this case would be considered electron deficient. It can make up the octet by forming bonds with other compounds that have non-bonding pairs of electrons.
This type of bond is called a dative or coordinate bond. These bonds are in all coordination compounds and Lewis acids/bases.

Bonds can be polar while the entire molecule isn't and vice versa. The dipole moment is a separation of charge within the molecule that is a 
product of the size of the charge and the distance of separation. One way for a molecule to be polar is for the bonds within the molecule to be polar and the 
dipoles that are present do not cancel out due to symmetry. The dipole moment can be indicated by an arrow that points toward the negative 
charge center with the tail of the arrow indicating the positive charge center. Molecules that have a lone pair of electrons on the central atom tend to be polar. 
Polar molecules will align themselves with an electric field or in the absence of an electron field, with each other.
\section{VSEPR and Bond Hybridization}
AP Topic: 2.7

The shapes of covalently bonded molecules and ions can be determined by considering the number of electron pairs aroudn the central atom. 
The electron pairs repel one another and try to get as far apart as possible. This theory is called Valence Shell Electron Pair Repulsion theory of VSEPR.
There are some standard shapes for specific numbers of electron pairs and some simple deviations from this theory when non-bonding pairs 
are present around the central atom.

A non-bonding pair of electrons will repel more strongly than a bonding pair. When comparing molecular geometries, it can be seen that this has the effect of 
altering the bond angles. Also, molecules will have different shapes if they have the same total number of electron pairs around the central atom, 
but where the total is made up of different combinations of bonding and lone pairs of electrons on the central atom. Note that multiple lone pairs 
will be arranged with maximum separation within the molecule, which also plays a role in molecular geometry.

When considering a polyatomic molecule the question of shape must be accounted for. Outer shell, atomic orbitals of the central atoms in 
Lewis structures are said to hybridize or undergo hybridization. Hybrid orbitals are a blending of atomic orbitals to create an orbital of intermediate energy. 

The type of hybridization present in a species is quite simple to predict. By considering the total number of electron pairs around the central atom, 
one can determine the total number of orbitals that need to be present, since each electron pair needs on orbital. So, by taking, one s, and as many p 
orbitals as required, one can determine the correct number of orbitals needed, and hence the hybridization.

Whenever a double or triple bond is formed, the first bond is always a sigma bond. All bonds after that are considered to be pi bonds. 
Pi bonds lead to delocalized electron clouds via the overlap of unhybridized p orbitals, giving rise to the potential for some electron movement, 
and for the occasional occurrence, althought rare, of a molecular substance that can conduct electricity. Pi bonds may form only if unhybridized
p orbitals remaing on the bonded atom and when sp or sp$^2$ hybridization is present on the central atom, not sp$^3$ hybridization.

There is another approach used to explain bonding in molecules. Where simple Lewis and VSEPR models fail to account for the 
observed behavior of molecules, another more complex theory must be used.

One such example is the unexpected paramagnetic behavior of oxygen. Its bonding can be explained using molecular orbital theory which describes covalent bonds 
in terms of the combination of individual atomic orbitals to form molecular orbitals rather than the independent overlap of the individual atomic orbitals.

Molecular orbital theory takes into account the idea that electrons and the positive nucleus of one atom strongly perturb or change the spatial 
distribution of the other atom's valence electrons. A new orbital is needed to describe the distribution of the bonding electrons.

\section{Resonance and Formal Charge}
AP Topic: 2.6

When drawing a Lewis structure that involves multiple bonds, it may be possible to draw several different Lewis structures. 
For these structures, the best or correct structure lies in the formal charge.

The formal charges of each atom within a structure can be calculated by:

Formal charge on an atom within a lewis structure is equal to the number of valence electrons around that atom in the free atom minus the number 
of nonbonding electrons around that atom in Lewis structure minus half of the number of bonding electrons around that atom in Lewis structure 

Formal charges show the most likely distribution of charge. 

To determine which structure in a set of possible structures is most likely, choose the structure with atoms that have formal charges of zero,
and/or formal charges with absolute values as low as possible, and/or keep any negative formal charges on the most electronegative atoms.

If a multiple bond is created between two atoms, the bond length observed will be shorter than the corresponding single bond. This is because a double 
bond is stronger than a single bond and hence pulls the atoms closed together. A triple bond is correspondingly shorter and stronger than a double bond. 
Multiple bonds increase the electron density between two nuclei and therefore decrease the nuclear repulsions while enhancing the nucleus to electron 
density attractions - either way, the nuclei move closer together and the bond length decreases. Bond order is simply the number of bonding electron pairs 
shared by two atoms; fractional bond orders will exist when resonance structures exist for a compound.
\section{Structure of Metals and Alloys}
AP Topic: 2.4

A metal's structure can be considered to be a close packed lattice of positive atoms/ions surrounded by a "sea" of moving, delocalized electrons. 
These electrons and their movement cause metals to be good conductors of electricity. The close packed atoms/ions make them good conductors of heat.

The metallic bond is the electrostatic attraction between the positive and negative charges. The flexibility of these bonds makes metals malleable \& ductile.

An alloy is a mixture of metals. There are two types of alloys:
\begin{itemize}
    \item Substitutional alloy - where one metals' atoms are replaced by another metals' atoms. In these cases the metal atoms are usually of similar 
    radius. Alloys of this type are less malleable and ductile than the pure metals, and have densities that typically lay between the densities of the component metals.
    Substitutional alloys are harder than the pure metal because the substituted atoms distort the lattice.
    \item Interstitial alloy - additional, smaller atoms of a different element fill the spaces in the metallic lattice. Interstitial allows 
    have similar, reduced malleability and ductility to substitutional allows since the presence of the smaller atoms make the structure more rigid and less flexible.
\end{itemize}

In both cases the sea of electrons is maintained and the alloys remain good conductors. In some cases, the surface of the allow or metal 
may take on a different property than the remainder of the solid, due to an oxide layer forming, following reaction with oxygen in the air.

\section*{Problems}
\begin{enumerate}
    \item Order the following bonds according to polarity: H-H, O-H, Cl-H, S-H, and F-H.
    \item Draw a lewis diagram for PCl$_6^-$
    \item Predict the molecular structure of the sulfur dioxide molecule. Is this molecule expected to have a dipole moment?
    \item Draw a lewis structure for SO$_2$. Identify the number of bonding \& lone pairs around the central atom and predict the bond angles.
    \item How is the xenon atom in XeF$_4$ hybridized? What is its electron geometry? What is its molecular geometry? What are the bond angles? Is the molecule polar or nonpolar?
\end{enumerate}
\end{document}