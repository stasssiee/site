\documentclass[../chem.tex]{subfiles}
\graphicspath{{\subfix{../figures/}}}
\begin{document}
\chapter{Acids and Bases}
\section{Introduction to Acids and Bases \& pH and pOH of Strong Acids and Bases}
AP Topic: 8.1, 8.2

Acids can be defined in terms of proton (H$^+$) transfer, their strength in terms of ionization, and how they can be sorted into acid-base conjugate pairs.

An acid is a substance that donates hydrogen ions in aqueous solution.

A base is a substance that accepts hydrogen ions in aqueous solution.

The ability of water to act as both an acid and a base is an important concept to grasp, and is called amphoteric behavior.

An acid is a substance that dissolves in water to produce hydronium ions.

A base is a substance that dissolves in water to produce hydroxide ions.

Conjugate acid and base pairs are related by a difference of a hydrogen ion on either side of the equation. 

Recall that there are seven strong acids: HCl, HBr, HI, HNO$_3$, H$_2$SO$_4$, HClO$_3$, and HClO$_4$, and there are eight strong bases: 
LiOH, NaOH, KOH, RbOH, CsOH, Ca(OH)$_2$, Sr(OH)$_2$, and Ba(OH)$_2$. A strong acid or base undergoes complete ionization.

Weak acids and weak bases, on the other hand, have very little ionization and equilibria are set up with the equilibria laying heavily on the left hand side.

The acid ionization constant is written as K$_a$.
\begin{center}
    Percent ionization = $\frac{[H_3O^+]\text{equilibrium}}{[HA]\text{initial}}\times 100\%$
\end{center}

Percent ionization is a measurement of the extent of ionization of an acid. However, percent ionization is dependent on both the strength 
of the acid or base and the initial concentration. If acids are of equal concentration, the weaker acid will have a lower percent ionization. If two samples of the same acid 
are of different concentrations, the solution of lower concentration will have a higher percent ionization.

It should be noted that the concentration of H$_3$O$^+$ and OH$^-$ are dependent upon two separate factors. First, the strength of the acid or base, and 
secondly, the amount of water present. As such, it is possible to have a dilute, strong acid, and to have a concentrated, weak acid that have the same hydronium concentration, and therefore the same pH value. 
Do not confuse concentration for strength.

The pH scale is used to indicate the relative acidity or basicity of an aqueous solution. There is a common misconception that that acid-base 
pH scale ranges from 0 to 14; however you can have values that are less than 0 and greater than 14. Acids have pHs less than 7, bases have pHs greater than 7, 
and 7 on the scale is considered to be neutral, although we will see later how this can vary. Both pH and pOh are defined in a similar manner:
\begin{center}
    pH = -log[H$_3$O$^+$]
\end{center}
\begin{center}
    pOH -log[OH$^-$]
\end{center}

In the case of strong acids and bases, the dissociation is complete, and therefore the concentration of the H$_3$O$^+$ ions or OH$^-$ ions can be 
determined directly from the stoichiometric ratio in the balanced equation and the concentration of acid or base.

\begin{center}
    14 = pH + pOH
\end{center}

An equilibrium constant can also be written for water.
\begin{center}
    K$_w$ = [H$_3$O$^+$][OH$^-$]
\end{center}

Like all equilibrium constants, K$_w$ is temperature dependent. At 298 K, K$_w$ = $1\times 10^{-14}$. Only about 2 in a billion water molecules 
are ionized at any instant. Since pure H$_2$O will have equal concentrations of H$_3$O$^+$ (aq) and $OH^-$ (aq), then under these conditions, we can find that the pH of pure water at 298 K is 7.

From the K$_w$ equation we can find that 
\begin{center}
    pK$_w$ = 14 = pH + pOH 
\end{center}

It should be carefully noted that at other temperatures, K$_w$ will have values other than $1\times 10^{-14}$. As such, at temperatures other than 298 K, pure water 
may not have a pH = 7, but it will still be considered neutral, since in pure water [H$^+$] will always equal [OH$^-$]; neutrality is dependent 
on the equal contractions of these ions as opposed to a pH of 7. In short, in any solution at any temperature 
\begin{itemize}
    \item If [H$^+$] = [OH$^-$], the solution is neutral.
    \item If [H$^+$] $>$ [OH$^-$], the solution is acidic.
    \item If [H$^+$] $<$ [OH$^-$], the solution is basic.
\end{itemize}

At 298 K, the K$_a$ and K$_b$ of an acid-base conjugate pair are related thus:
\begin{center}
    K$_w$ = $1\times 10^{-14}$ = (Ka)(Kb)
\end{center}

Logging both sides we get 
\begin{center}
    pKw = 14 = pKa + pKb
\end{center}
\section{Weak Acid and Base Equilibria}
AP Topic: 8.3

Calculating pH of weak acids involves setting up an equilibrium. It helps to start by writing the balanced equation, setting up the acid equilibrium expression, 
defining initial concentrations, changes, and final concentration in terms of x, substituting values and variables into the K$_\text{a}$ expression, and solving for x. In other words,
use the RICE table method.

Often the -x term in a K$_\text{a}$ expression can be neglected. That simplifies the math tremendously since you are not spared the tedium of having to use the 
quadratic formula. How do you know when to neglect x? Look at the original concentration and compare it to 100K$_\text{a}$ (or 100K$_{\text{b}}$ if you are dealing with a weak base). 
If the initial concentration is large by comparison, you can neglect subtracting the x term.

Because weak acids only partially dissociate, it is not possible to go directly to pH = -log[H$_3$O$^+$(aq)] from the concentration of the acid.

We can derive the following expressions.
\begin{center}
    Ka = $\frac{[\text{H}_3\text{O}^+][\text{A}^-]}{[\text{HA}]}$
\end{center}
Often written as
\begin{center}
    Ka = $\frac{[\text{H}_3\text{O}^+]^2}{[\text{HA}]}$
\end{center}

Note that the stronger the acid, the higher the value of K$_{\text{a}}$. Also,
\begin{center}
    pKa = -log Ka 
\end{center}
Note that the stronger the acid, the lower the value of pK$_{\text{a}}$. K$_{\text{a}}$ and pK$_{\text{a}}$ are inversely related.

For a weak base, dissociation is incomplete.

It is similar to a weak acid expression.
\begin{center}
    Kb = $\frac{[\text{OH}^-][\text{NH}_4^+]}{[\text{NH}_3]}$
\end{center}
Written often as 
\begin{center}
    Kb = $\frac{[OH^-]^2}{[\text{NH}_3]}$
\end{center}
and 
\begin{center}
    pKb = -log Kb
\end{center}

Note that when using RICE tables with weak bases, x represents [OH$^-$], not [H$^+$] as with weak acids. Taking the negative log of x 
will give you the pOH not the pH. Also note that the major species in any solution of a weak acid or base will be the undisassociated form.
\section{Acid-Base Reactions and Buffers}
AP Topic: 8.4

When strong acids and bases react, they do so to create water.

When the moles of base are equal to the moles of acid, neutralization has occurred. When adding unequal amounts of strong acids and strong bases 
together, all of the limiting reagent will be neutralized and the pH of the resulting solution is determined by the excess reagent, using the pH and pOH formulas.

A crucial part of the application of buffer solutions has been the appreciation that buffer solutions are inadvertently produced when mixing 
weak acids and bases. A buffer is a solution containing relatively high concentrations of a conjugate acid-base pair and is discussed further, but if a small 
amount of strong acid or base is added to a buffer solution, the buffer solution resists changes to the pH. When a weak acid and a strong base are mixed there are three possible combinations.
\begin{enumerate}
    \item If the weak acid is in excess, then the resulting solution is a buffer solution. The Henderson-Hasselbalch equation can be used to determine the pH of the resulting solution.
    \begin{center}
        pH = pKa + log$\frac{[\text{A}^-]}{[\text{HA}]}$
    \end{center}
    \item If the strong base is in excess, then the pH of the resulting solution will be calculated from the molarity of the solution formed from the excess moles of the base divided by the total volume of the solution. This is similar to the strong acid with a strong base as shown previously.
    \item If the weak acid and strong base are equimolar, then the solution is at the equivalence point, where the weak acid has been neutralized by the strong base, but an equivalent number of moles of the salt has been formed. The pH of the resulting solution will be determined by the conjugate base of the weak acid.
    \begin{itemize}
        \item Use a RICE table to find the equilibrium concentration of OH$^-$.
        \item Determine the equilibrium constant for the conjugate base, K$_b$ = K$_w$/K$_a$
        \item Use the moles of the conjugate base and the total volume to find the initial molarity
    \end{itemize}
\end{enumerate}

When a strong acid and a weak base react there are also three possibilities.
\begin{enumerate}
    \item If the weak base is in excess, then the resulting solution is a buffer solution and you should use the Henderson-Hasselbalch equation to determine the pH of the resulting solution.
    \item If the strong acid is in excess, then the pH of the resulting solution will be calculated from the molarity of the solution formed from the excess moles of the hydronium ion divided by the total volume of the solution.
    \item If the weak base and strong acid are equimolar, then the reaction is at its equivalence point; the mole of the acid and base are equal. The pH at the equivalence point when a strong acid and a weak base are reacting is determined by the conjugate acid of the weak base.
    \begin{itemize}
        \item Use a RICE table to find the equilibrium concentration of the H$_3$O$^+$.
        \item Start by calculating the initial molarity of the weak conjugate acid by using the mole of the conjugate acid and the total volume.
        \item Then complete the RICE table establishing equilibrium
    \end{itemize}
\end{enumerate}

\section{Acid-Base Titrations}
AP Topic: 8.5

Titration is an experimental technique used to perform a neutralization reaction. Accurately graduated glassware is used in a quantitative
manner to analyze acid-base reactions. Typically a known concentration of a base is added to an unknown concentration of an acid, or vice versa.
The volume of both solutions is measured and used to calculate the unknown concentration.

In a titration, as a base or an acid is added to an acid or base respectively, there is very little change in pH, and a pH change of less than 
approximately 1.5 is expected up to the point that 90\% of the acid or base has been neutralized. This is the buffering region that will be discussed further.
When the moles of titrant are in the exact stoichiometric proportion with the titrate/analyte, then the equivalence point has been reached. At this point there is a rapid change in pH.

These changes can be summarized using a titraiton curve which plots pH against the volume of titrant added. The shape of the titration curve 
differs based on the identity of the titrant and the analyte; factors to consider are the strength of the acid and the base, the number of hydrogen ions/hydroxide ions that will dissociate, and the concentrations of the acid and the base.

Note that when looking at a titration curve, the inflection point corresponds to the equivalence point. WHen a strong acid and a strong base 
react the equivalence point will be at pH = 7. When a strong base is added to a weak acid, the pH at the equivalence point will be above pH = 7,
because as the weak acid is neutralized, its conjugate base is formed, causing the solution to be basic. We can also use this titration curve to determine the pKa of the 
weak acid. The pKa is equal to the pH at half the equivalence point. This is because at this half equivalence point there are equal concentrations of 
conjugate acid and base, because half of the initial concentration of the acid will have been neutralized, and will have formed the conjugate base.

Most acid-base titrations involve the addition of one colorless solution to another colorless solution with no obvious, observable reaction taking place.
Since in an acid-base titration we need to find the equivalence point in order to apply the stoichiometric ratio and then calculate the unknown concentration, we need a method 
of determining when the equivalence point has been reached. We do this with the aid of indicators, a chemical that changes colors at various pHs. Indicators are 
often weak, organic acids, where the ionized and the unionized form have different colors. Usually 1/10 of the initial form of the indicator must be changed to the other form before a new color 
is apparent. Since you typically only add a drop or two of the indicator to the analyte, indicators have very little effect on overall pH of interest. In practice,
an indicator will change color over a small, given range of pH.

Again, the equivalence point is located in the middle of the vertical portion of the titration curves. We need to choose an indicator that changes color 
at a pH value as close to the equivalence point as possible. This observable color change of the indicator is called the end point, and it should correspond 
to the equivalence point as closely as possible. So using the titration plots, suitable indicators can be chosen.
\begin{center}
    Strong acid - Strong base: most indicators

    Weak acid - Strong base - Phenolphthalein

    Strong acid - Weak base - Methyl orange
\end{center}

The useful range of an indicator is usually its pKa $\pm$ 1. When choosing an indicator, determine the pH at the equivalence point of the 
titration and then choose an indicator with a pKa close to that.

For a weak acid-weak base titration there is no sharp change in pH at the equivalence point, therefore no indicator will change color sharply 
at the end-point, therefore no indicator is suitable. A pH meter can be used to determine the end-point in weak acid-weak base titrations.

Acids with more than one ionizable hydrogen will ionize in steps. These polyprotic acids will have multiple equivalence points, and produce a titration curve with multiple, 
vertical portions. On the titration curve below, a diprotic acid is being titrated with a strong base. We recognize it as diprotic because there are two inflection points.
These two points correspond with the equivalence point for each of the protons being donated. We can determine the pKa for each of these protons being donated by using the half equivalence 
point pH method. Because there are multiple protons reacting, the two protons will be donated sequentially, instead of both at the same time. This is demonstrated through a 
diprotic acid having two different pKa values. Each dissociation has its own Ka value, but the first dissociation is the greatest.Subsequent dissociations 
will have much smaller equilibrium constants. As each H$^+$ is removed, the remaining acid gets weaker and therefore has a smaller Ka. As the negative charge 
on the acid increases it becomes more difficult to remove the positively charged proton. Looking at the Ka values of polyprotic acids, it is obvious 
that only the first dissociation will be important in determining the pH of the solution for all acids except sulfuric acid. Because sulfuric acid is a strong acid 
in its first dissociation and a weak acid in its second, we need to consider both if the concentration is more dilute than 1.0 M.

As noted earlier, careful analysis of titration curves shows that the equivalence point is not necessarily at pH = 7. This is due to the fact that 
salts are not always neutral. Some salts do form neutral solutions, but others react with water to form acidic or basic solutions. The reaction is called 
salt hydrolysis. 

\begin{itemize}
    \item Strong acid and Strong base - pH at equivalence point is 7 
    \item Strong base and Weak acid - pH is greater than 7 at equivalence point 
    \item Strong acid and Weak base - pH is less than 7 at equivalence point 
    \item Weak acid and weak base - 
    \begin{itemize}
        \item If Kb for the anion > Ka for the cation then basic at equivalence point 
        \item If Ka for the cation > Kb for the anion then acidic at equivalence point 
        \item If Ka is approximately equal to Kb then neutral at equivalence point 
    \end{itemize}
\end{itemize}

\section{Molecular Structure of Acids and Bases}
AP Topic: 8.6

The structure and bonding that exist in acids and bases can influence their strength.

The chemical structures of acids controls their relative strength. Bond length, bond strength, and electronegativities of atoms in the compound all 
play a role in the degree to which a Bronsted-Lowry acid will dissociate in solution.

THere is a trend that an acid strength increases for binary acids down a group of the periodic table. In this case, bond length and strength appears to play the major role.
The trend suggests that longer and weaker H-X bonds result in a relatively stronger acid than an acid with a shorter and stronger bond.

An oxyacid is a ternary acid with one ore more oxygens in its structure. Generally, the more oxygen present in the polyatomic ion of an oxyacid 
the stronger its acid within that group. This is because the H of the acid is bound to an oxygen and not any other nonmetal present. Oxygen is very 
electronegative and attracts the electrons of the O-H bond toward itself. If you add more oxygens, then the effect is magnified and there is an increasing electron 
density in the region of the molecule that is opposite the H. The added electron density weakens the bond, thus less energy is required to break the bond and the acid dissociates more readily, which we describe as ``strong''.

The relative strength of a carboxylic acid depends upon the stability of the anion that is forms, when it loses its labile. If the anion is relatively stable,
then it will form in solution, and the process the acid will tend to donate H$^+$ ions, making the acid relatively strong. Remember the smaller the pKa, the stronger the acid.
It is also found that when you compare strengths of halogen-substituted caboxylic acids, the numebr of chlorine atoms present increases the strength of the acid, because the 
highly electronegative Cl atoms attract electron density away from the COO$^-$ part of the anion, therefore increasing its stability.

Fluorine is the most electronegative halogen, so carboxylic acids containing fluorine tend to be the strongest.

Nitrogenous bases are substances that contain nitrogen and have the properties of bases.

Note that the vast majority of acids/bases are weak.

If you are given a mixture of weak acids, only the acid with the largest Ka value will contribute an appreciable [H$^+$]. Determine the pH based on this acid and ignore any others.
\section{Properties of Buffers}
AP Topic: 8.8

A buffer solution is one that resists change in pH when either a small amount of acid or base is added to it. Usually a buffer solution is 
made from a weak acid and one of its salts or from a weak base and one of its salts. 

The process of ``mopping up'' added base and acid allow the pH to remain relatively unchanged. Note that pure water has no buffering capacity; 
acids and bases added to water directly affect the pH of the solution.

All buffer solutions mayb e solved with the following equation to lessen confusion:
\begin{center}
    [H$^+$] = K$_a \frac{[\text{Acid}]}{[\text{Base}]}$
\end{center}

Recall from previous discussions that the addition of an ion present in a system causes the equilibrium to shift away from the common ion. 
It is best to explain the common ion effect in terms of the reaction quotient versus the equilibrium constant. 

The addition of a salt containing a common anion to a solution of a weak acid makes the solution less acidic.

\section{pH and pKa \& Henderson-Hasselbalch Equation}
AP Topic: 8.7, 8.9

While RICE tables are always an option, a shortcut way to calculate the pH of a buffer solution is with the Henderson-Hasselbalch equation.
This involves the pKa or pKb of the weak acid or base and the ratio of the concentrations of each component. To calculate the pH of an acidic buffer use this version of the 
Henderson-Hasselbalch equation:
\begin{center}
    pH = pKa + log$\left(\frac{[\text{salt}]}{[\text{acid}]}\right)$
\end{center}

Use this version for a basic buffer:
\begin{center}
    pOH = pKb + log$\left(\frac{[\text{salt}]}{[\text{base}]}\right)$
\end{center}
and then remember that 14 = pH + pOH applies for any solution. Note that you can use mole instead of molarity in the log ratio of each equation.

Since the pKa or pKb is fixed for a weak acid or base, if the concentration of the components are changing but remain in the same ratio, the pH of the buffer will not change.
However, changing the concentrations of the components does affect the capacity of the buffer. Optimum buffering occurs when [HA] = [A$^-$] and the pKa 
of the weak acid used should be as close as possible to the desired pH of the buffer system.

A little knolwedge of logarithms can go a long way here. Considering the acid version of the Henderson-Hasselbalch equation, it can be seen that if the pH of a buffer 
solution is greater than the pKa of the acid, then the salt concentration must be greater than the acid concentration. When the pH is less than the pKa, the opposite is true.

At the point the concentrations of salt and acid or salt and base are equal, the pH = pKa, or the pOH = pKb.

Using the relationship between the pH and pKa of a solution that is a buffer allows us to make qualitative and quantitative assessments of which component of the buffer is present in the 
greatest concentration. We can compare the pH to the pKa to determine which species is predominant at a certain pH, or use the ratio of components to determine the pH.

The Henderson-Hasselbalch equation needs to be used cautiously. A Ka or Kb problem requires a greater understanding of the factors involved and can always 
be used instead of the Henderson-Hasselbalch equation. This equation is only valid for solutions that contain weak, monoprotic acids and their salts or weak bases and their salts.
The buffered solution cannot be too dilute and the Ka/Kb cannot be too large.

As the base is gradually added to the weak acid, some of the weak acid is neutralized, and the result is the salt of the weak acid and water. Because 
not all of the weak acid has been neutralized, it is still present in solution alongside the salt. This combination in a solution is a buffer solution, and its 
pH can be calculated by using a RICE table to determine the remaining concentration of the weak acid and the new concentration of the salt. Then applu the Henderson-Hasselbalch equation.
Alternatively you can use the following:
\[[H^+] = K_a \frac{(mol_{WA})-mol_{base}}{(mol_{WB})+mol_{base}}\]

As the acid is gradually added to the weak base, some of the weak base is neutralized, and the result is the salt of the weak base and water. Because not 
all of the weak base has been neutralized, it is still present in solution alongisde the salt. This combination in solution is a buffer solution,
and its pOH can be calculated by using a RICE table to determine the remaining concentration of the weak base and the new concentration of the salt. Then apply the Henderson-Hasselbalch equation.
Alternatively you can use the following:
\[[H^+] = K_a \frac{mol_{WA}+mol_{acid}}{(mol_{WB})-mol_{acid}}\]

\section{Buffer Capacity}
AP Topic: 8.10

The capacity of a buffer is defined as its ability to continue to react with any extra acid or base that is added to it without a significant 
change in pH. The higher the concentrations of the components of a buffer, the more acid or base it can absorb in the reactions above, and the higher its capacity.
Note that if we change the concentrations of each component of a buffer the capacity will change, but if the ratio of the components remain the same,
according to the Henderson-Hasselbalch equation, the pH will not change.
\section*{Problems}
\begin{enumerate}
    \item In the following reaction label the acid, base, conjugate acid, and conjugate base.
    \begin{center}
        HBr + NH$_3 \rightarrow$ NH$_4^+$ + Br$^-$
    \end{center}
    \item Calculate the pH of a solution made by dissolving 2.00 g of KOH in H$_2$O to a total volume of 250. mL.
    \item Which solution has the lowest pH, 0.0010 M potassium hydroxide, KOH, or 0.0010 M calcium hydroxide, Ca(OH)$_2$?
    \item Calculate the K$_a$ of a solution of 0.250 M of a weak acid with a pH of 5.11.
    \item Calculate the K$_b$ of a solution of 0.250 M of a weak base with a pH of 9.12.
    \item What is the pH of a solution composed of 500.0 mL of 0.250 M sodium hydroxide, NaOH, and 400.0 mL of 0.200 M hydrochloric acid, HCl?
    \item The pK$_a$ of hydrozoic acid, HN$_3$, is 4.72. It reacts with water according to the reaction below:
    \begin{center}
        HN$_3$ (aq) + H$_2$O (l) $\rightleftharpoons$ N$_3^-$ (aq) + H$_3$O$^+$(aq)
    \end{center}
    Calculate the pH of a solution formed from 100.0 mL of 0.900 M HN$_3$ and 50.0 mL of 0.300 M NaOH.

    \item Calculate the pH of a 5.0 M H$_3$PO$_4$ solution and the equilibrium concentrations of the species H$_3$PO$_4$, H$_2$PO$_4^-$, HPO$4^{2-}$, and PO$_4^{3-}$.
    \item Predict whether an aqueous solution of Al$_2$(SO$_4$)$_3$ will be acidic, basic, or neutral with appropriate equations.
    \item How do you account for the difference in the pK$_a$ of acetic acid (CH$_3$COOH, 4.76) and fluoroacetic acid (CH$_2$FCOOH, 2.59)?
    \item Determine the [H$_3$O$^+$] and [C$_2$H$_3$O$_2^-$] in 0.100 M HC$_2$H$_3$O$_2$. The K$_a$ for acetic acid is $1.8\times 10^{-5}$.
    \item Ethanoic acid has a pK$_a$ of 4.75. Find the pH of the solution that results from the addition of 40.0 mL of 0.040 M NaOH to 50.0 mL of 0.075 M ethanoic acid.
    \item Calculate the pH of the solution that results when 0.10 mol gaseous HCl is added to 1.0 L of the buffered solution that contains 0.25 M NH$_3$ (k$_b = 1.8\times10^{-5}$) and 0.40 M NH$_4$Cl.
    \item Three buffers were created from using a weak acid, HA (pK$_a$ = 3.20) and LiA.
    \begin{itemize}
        \item Buffer X = 1.00 moles of HA + 0.50 moles of LiA dissolved to form 1.00 L of solution.
        \item Buffer Y = 0.50 moles of HA + 1.00 moles of LiA dissolved to form 1.00 L of solution.
        \item Buffer Z = 2.00 moles of HA + 4.00 moles of LiA dissolved to form 1.00 L of solution.
    \end{itemize}  
    Which buffer will be best at resisting a change to the pH when an acid is added?
\end{enumerate}
\end{document}