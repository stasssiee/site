\documentclass[../chem.tex]{subfiles}
\graphicspath{{\subfix{../figures/}}}
\begin{document}
\chapter{Kinetics}
\section{Reaction Rates}
AP Topic: 5.1

Many factors can influence the rate of a chemical reaction. Each factor must have a discernible ffect on the microscopic collisions that lead to a chemical reaction.

Some molecules react in a hurry; others react very slowly. Each factor must have a discernible effect on the microscopic collisions that lead to a chemical reaction.

\begin{itemize}
    \item Physical state matters a great deal.
    \item Chemical identity. What exactly is reacting? Usually ions of opposite charge react very rapidly. Also, the more bonds between reacting atoms in a molecule, 
    the slower the reaction rate. Why? More energy is required to separate the molecule into its ``bits''. Substances with strong bonds will react much more slowly. 
\end{itemize}

Increasing the reactants concentration by putting more reactants into the same space increases the collision frequency of the particles, leading to
a faster rate of reaction. A similar effect is observed when increasing the pressure in a gaseous reaction.

Adding an inert gas has no effect on the rate of the reaction since it is not in the reaction mechanism. 

A rise in temperature will result in an increased rate of reaction. The faster molecules move, the more likely they are to collide and the 
more energetic the collisions become. Consider the Maxwell-Boltzmann distribution plot of energies. The area underneath the curve represents the total number of molecules.

When a solid reacts, only the particles on the surface of the solid are avaliable for reaction. If the solid is broken up into smaller pieces, its surface area 
gets larger and more particles are avaliable for collision, therefore the reaction rate increases. 

Catalysts also affect the rate of reaction and are discussed in more detail later. Unlike other participants in a chemical reaction, a catalyst is not 
consumed, therefore it can be used again and again. A catalyst may participate in multiple chemical transformations. The effect of a catalyst may 
vary due to the presence of other substances known as inhibitors or promoters.
\begin{itemize}
    \item A catalyst is a substance that changes the rate of reaction by altering the reaction pathway. Most catalysts work by decreasing the 
    activation energy needed for the reaction to proceed, therefore, more collisions are successful and the reaction rate is increased.
    \item Remember, the catalyst is not part of the chemical reaction and is not used up during the reaction. Usually, the catalyst participates in the rate-determining or slowest step.
    \item Catalysts may be homogeneous or heterogeneous catalysts. A homogeneous catalyst is in the same phase as the reactants. 
    A heterogeneous catalyst is in a different phase than the reactants. Heterogeneous catalysts offers the advantage and products are readily separated from 
    the catalyst, and heterogeneous catalysts are often stable and degrade much slower than homogeneous catalysts.
\end{itemize}

Homogeneous catalysts actually appear in the rate law because their concentration affects the reaction.

\section{Elementary Reactions, Collision Model, Reaction Energy Profile, and Multistep Reaction Energy Profile}
AP Topic: 5.4, 5.5, 5.6, 5.10

Chemical reactions can occur at significantly different speeds or rates. The rate of a reaction can be determined by either monitoring the 
change in concentration of reactants over time, or alternatively, monitoring the change in concentration of products over time.

The basis for the study of the speed or rate of chemical reactions is collision theory. Collision theory tells that a reaction will only take place if three conditions are met.
\begin{enumerate}
    \item The reactants come into contact.
    \item The collision occurs with a certain minimum energy, known as the activation energy.
    \item The collision has the correct molecular orientation. This means that the reactants must collide in a certain physical, three-dimensional orientation for the reaction to take place.
\end{enumerate}

If reactants do not collide, or collide with energies lower than the activation energy, or collide without the correct molecular orientation, 
then no reaction will occur. These collisions are described as unsuccessful, that is, they do not lead to a chemical reaction and the reactants remain unchanged.

All chemical reactions take place via a series of elementary steps. An elementary step is a reaction that forms products in a single step, with only 
one transition state and no intermediates. An energy profile can be used to show the progress of a reaction from reactants, through a transition state and then 
on to products. Reactants with energies closer to the transition state at the beginning will have lower activation energies, and therefore faster rates of reaction.

Elementary steps can fall into one of three categories, but all successful collisions are still subject to the criteria of sufficient energy and correct oriention.
\begin{enumerate}
    \item They can be unimolecular - single species reacts to form products when a rearrangement occurs, activating a reactant molecule. These are called first order reactions. Rearrangements are cuased by collisions between reactant species and solvent or `background' molecules.
    \item They can be bimolecular - two species collide to form products. These are called second order reactions.
    \item They can be trimolecular - three species collide and react to form products. These are called third order.
\end{enumerate}

The fewer the molecules involved in the elementary reaction, the more likely it is that one of the collisions will be in the correct orientation, 
meaning that with increasing molecularity, the changes of correctly orientated collisions goes down. In fact, trimolecular reactions are 
relatively rare, since they involve the need for all three species to be simultaneously in the same area of space, and colliding with the correct energy 
and orientation - in terms of probability, this is relatively unlikely.

In all situations, if the concentration of the reactants is increased, there will be a greater frequency of collisions, and the greater the chances of successful collisions.

If we string together a series of elementary steps we get a more complicated reaction, but the more complex reaction is actually only a series 
of simple, elementary ones. 
\section{Introduction to Rate Law, Introduction to Reaction Mechanisms, Reaction Mechanism and Rate Law, and Steady-State Approximation}
AP Topic: 5.2, 5.7, 5.8, 5.9

The speed of a reaction is expressed in terms of its ``rate'' which is equal to some measurable quantity that is changing with time.

The rate of a chemical reaction is measured by the decrease in concentration of a reactant or an increase in the concentration of a product in a unit of time.
Generally speaking:
\[\text{Rate}=\frac{\text{change in concentration of a species}}{\text{time interval}}=\frac{\Delta[reactants]}{t}\]

When writing rate expressions, they can be written in terms of reactant disappearance or product appearance. Rate if not a constant, it changes with time.
Graphing the data of an experiment will show an average rate of reaction.

Instantaneous Reaction Rate is simply the ``rate at a given instant of time''. 

To determine the value of the rate at a particular instant of time, known as the instantaneous rate, simply compute the slope of the line 
tangent to the curve at that point in time.

Relative Reaction Rate is expressed as the change in concentration of a reactant per unit time or $\frac{\Delta [A]}{\Delta \text{time}}$.

You should focus either on the disappearance of reactants or the appearance of products.
\begin{itemize}
    \item Rate of $\Delta$ of a reactant is always negative.
    \item Rate of $\Delta$ of a product is always positive.
\end{itemize}

Also, the word relative refers to terms that relate to each other in the context of a given chemical system.

Reactions are reversible. So far, we've only considered the forward reaction. The reverse is equally important. When the rate of the forward is equal to the 
rate of the reverse, we have equilibrium. To avoid this complication we will discuss reactions soon after mixing, before things get too messy. 
We will deal with initial reaction rates, so we will not worry too much about the buildup of products and how that starts up the reverse reaction.

The sequence of elementary steps that make up a complex, chemical reaction is known as the mechanism. Each step will either be a relatively fast one, 
or a relatively slow one, but the overall rate of the complex, chemical reaction is only dependent upon the slowest elementary step. For this reason, the 
slowest step is known as the rate determining step.

In order to study reaction rates we need to convert qualitative elementary steps into quantitative entities. This is achieved by the use of a 
rate equation or rate law. All rate equations take the general form,
\[\text{Rate}=\text{k}[\text{A}]^x[\text{B}]^y[\text{C}]^z\]
where k is the rate constant and x, y, and z are the orders with respect to the concentrations of reactants A, B, and C. The order with respect 
to a given reactant is the power to which the concentration of that reactant is raised to in the rate equation. The overall order of the complex, chemical 
reaction is the sum of the individual orders.

Since only the reactants that appear in the rate-determining step are ones that affect the rate, it is only these reactants that ever appear in the rate equation, and vice-versa.

\begin{itemize}
    \item Zero order: The change in concentration of reactant has no effect on the rate. General form: Rate = k.
    \item First order: Rate is directly proportional to the reactants concentration. General form: Rate = k[A].
    \item Second order: Rate is quadrupled when [rxt] is doubled, increases by a factor of 9 when [rxt] is tripled, etc. General form: Rate = k[A]$^2$ or k[A][B].
    \item Fractional orders are rare but exist too.
\end{itemize}

\begin{itemize}
    \item k is most definitely temperature dependent and must be evaluated by experimenet.
\end{itemize}

Units of the rate constant depend on the order of the rate law
\begin{itemize}
    \item Zero order: k has units of M/time 
    \item First order: k has units of 1/time 
    \item Second order: k has units of 1/(M$\cdot$time)
\end{itemize}

Note:
\begin{enumerate}
    \item It is not possible to deduce anything about the order of a reaction from the stoichiometry of the balanced equation that describes the complete, complex, chemical reaction. As such, orders must be determined experimentally or from experimental data.
    
    However, it is possible to deduce orders from the stoichiometry of the balanced equation that describes the slowest elementary step in the mechanism and the stoichiometric number of a substance that appears in the slow step is the power that the concentration of that substance is raised to in the rate equation.
    \item Units and magnitude of the rate constant are important.
    \item A reactant that has no effect on the rate is said to have an equal order of zero. It has no effect on the rate and since any number raised to the power of zero is equal to 1, it can be omitted from the rate equation.
    \item Orders can be fractional.
    \item In all valid mechanisms the sum of the individual elementary steps must add up to the overall, complex, chemical reaction.
    \item An intermediate is formed in one elementary step during the overall reaction, but is then used up in a subsequent elementary step. When an intermediate is found in a rate determining step and therefore in a rate equation, it is usually replaced. The experimental detection of an intermediate can be one way to choose one proposed mechanism over another.
    \item If a substance is present at the beginning of a reaction and present in the same form at the end of the reaction, it can be identified as a catalyst.
    
    Catalysts can appear in rate equations since their concentrations are often more easily determined than intermediates.
    \item A reaction that is second order, and has a rate equation of Rate = k[A][B], can sometimes be carried out with a very large concentration of one of the reactants compared to the other. When this occurs, as the reaction proceeds, 
    the reactant with very large concentration will effectively have a ``constant'' concentration. As a result, and if [B]$>>>>$[A], then the rate law can be written as Rate = k[A], and is known 
    as a pseudo first order reaction. Situations like this can be treated simply as first order reactions.
\end{enumerate}

What we've explored thus far is termed differential rate law of simply rate law. It's the method we use when the data presented is concentration and rate data. 
If the data presented in concentration and time data, we need a new method. So there are two types of rate laws which implies two different approaches are needed.
\begin{itemize}
    \item Differential rate law - data table contains concentration and rate data. Use table logic or ugly algebra to determine the orders of reactants and the value of the rate constant, k.
    \item Integrated rate law - data table contains concentration and time data. Use graphical methods to determine the order of a given reactant. The value of the rate constant k is equal to the absolute value of the slope of the best fit line which is decided by performing 3 linear regressions and analyzing the regression correlation coefficient r. 
\end{itemize}
\section{Concentration Changes Over Time}
AP Topic: 5.3

Let's discuss plotting the concentration of a reactant [reactant] against time.

Set up your axes so that time is always on the x-axis. Plot the concentration of the reactant on the y-axis of the first graph. Plot the netural log of the concentration on the y-axis 
of the second graph and the reciprocal of the concentration on the y-axis of the third graph. You are in search of linear data! If you set the graphs in this order, the y-axis variable leads to 0, 1, 2 orders for that constant.

In zero order, k is negative slope, in first order, k is negative slope, and in second order, k is the slope.

You can now easily solve for either time or concentration once you know the order of the reactant. Just remember y = mx + b. Choose the set of variables 
that gave you the best straight line and insert them into place of x and y in the generalized equation for a straight line. ``A'' is reactant A and A$_0$ is the initial 
concentration of reactant A at time zero.

In zero order: [A] = -kt + [A$_0$]\\
In first order: ln[A] = -kt + ln[A$_0$]\\
In second order: 1/[A] = -kt + 1/[A$_0$]

Also recognize that $\mid$slope$\mid$ = k, since the rate constant is never negative. If you are asked to write the rate expressoin it is simply Rate = k[A]$^{\text{order you determined from analyzing the graphs}}$

Graphing Calculator Tutorial:
\begin{itemize}
    \item Set up your calculator so that time is always in L1.
    \item Use L2, L3, and L4 to display the y-variables. Remember the list for what is placed on the y-axis is alphabetical. L1 is time, L2 is concentration, L3 is natural log of concentration, and L4 is reciprocal concentration.
\end{itemize}

Half-life is defined as the time required for one half of one of the reactants to disappear. You probably remember dealing with half-life in the context of 
$^{14}$C and using it to approximate the age of fossils and such. We will focus on half-life for first order reactions only. 

Another way of looking at graphs to determine reaction orders is to use the definition of rate expressed below. Using that expression, and plotting 
reactant concentration vs time, the slope of such a graph will equal the rate.
\[\text{Rate}=\frac{\text{decrease in concentration of reactant}}{\text{time}}\]

Integrated rate laws for reactions with more than one reactant must still be determined by experiment. We use a technique called swamping.

We must flood the reaction vessel with high concentrations of all but one reactant and perform the experiment. The reactants at high concentrations stay the same. By doing this,
the rate is now dependent on the concentration of the little guy since the big guys aren't changing. We can now rewrite the rate as a pseudo-rate-law and k' is a pseudo-rate-constant.

The Arrhenius equation relates rate constants, activation energy, and temperature together, and can take multiple formats. Three common ones are shown below:
\[\text{k=Ae}^{\text{-Ea/RT}},\qquad \ln\text{k}=\left(\frac{\text{Ea}}{\text{R}}\right)\left(\frac{1}{\text{T}}\right)+\ln \text{A}, \qquad \ln\left(\frac{\text{k}_1}{\text{k}_2}\right)=\left(\frac{\text{Ea}}{\text{R}}\right)\left(\frac{1}{\text{T}_2}-\frac{1}{\text{T}_1}\right)\]

There is one more form of the Arrhenius equation that is used to quantitatively describe how orientation factor might affect a rate constant.
\[\text{k}=\text{pAe}^{\text{-Ea/RT}}\]

p is the orientation factor and is equal to 1 for the simplest required orientations, through to numbers that are many hundreds of thousands of times smaller for more 
complicated reactions that require more intricate and precise orientations in order for collisions to be successful. In short, the bigger the p value, the easier the required 
orientation, the larger the rate constant and the faster the reaction.
\section{Catalysts}
AP Topic: 5.11

Catalysts function by lowering the activation energy of an elementary step in a reaction mechanism, and by providing a new and faster reaction mechanism.

Types of catalysts:
\begin{enumerate}
    \item Acid Base - A reactant will either lose or gain a H$^+$, forming a new intermediate, and as a result the reaction rate is changed. The catalytic reaction may be acid-specific, as in the case of decomposition of the sugar sucrose into glucose and fructose in sulfuric acid.
    \item Surface - A surface catalyst is often a metal, working in a gaesous reactant environment. The gas molecules can be adsorbed onto the surface of the metal where a number of things can occur.
    
    Another effect of adsorbing the gases onto a metal surface is that effectively their concentration is increased. Putting more gas molecules into a smaller area concentrates them, and a higher concentration means more collisions and a faster reaction.
    
    There are usually four steps with surface catalysis.
    \begin{itemize}
        \item Adsorption and activation of the reactants.
        \item Migration of the adsorbed reactants on the surface.
        \item Reaction of the adsorbed substances.
        \item Escape, or desorption, of the products.
    \end{itemize}

    Catalytic converters are also hetergeneous catalysts. 
    
    \item Enzyme - Enzymes are complex protein molecules that act as biological catalysts. They have many active sites on them, which interact with substrate molecules, and lower the activation energy for the biological process.
    Originally thought of as a `lock and key' model, where only specific substrates and enzymes would fit together and work, the induced fit mechanism suggests that both enzymes and substrates are more 
    flexible in their structures to allow a single enzyme to interact with more than one substrate. Their shape is mostly determined by IMFs, which are very temperature sensitive.
    Altering the pH or heating the enzyme easily disrupts the IMFs and causes the enzyme to denature.

    Other enzymes will react with the substrate to form a completely new intermediate with a lower energy for the transition state.
\end{enumerate}

\section*{Problems}
\begin{enumerate}
    \item Write the relative rates of change in concentration of the reactants and products for the following reaction.
    \[4\text{PH}_3\text{(g)}\rightarrow \text{P}_4\text{(g)}+6\text{H}_2\text{g}\]
    \item Identify the catalyst and the intermediate in the following mechanism:
    \begin{align*}
        \text{Step 1:}\qquad \text{SO}_2+\text{V}_2\text{O}_5\rightarrow \text{SO}_3+\text{V}_2\text{O}_4\\
        \text{Step 2:}\qquad \text{V}_2\text{O}_4+\frac{1}{2}\text{O}_2\rightarrow \text{V}_2\text{O}_5
    \end{align*}
    \item Calculate the rate constant for a first-order reaction with a half life of 20.0 minutes. How much time is required for this reaction to be 75\% complete?
    \item The rate constant for the first order transformation of cyclopropane to propene is $5.40\times 10^{-2}\text{hr}^{-1}$. What fraction of the cyclopropane remains after 51.2 hours? What fraction remains after 18.0 hours?
    \item Given a balanced equation for the reaction of the gases nitrogen dioxide and fluorine
    \[2\text{NO}_2\text{(g)}+\text{F}_2\text{(g)}\rightarrow 2\text{NO}_2\text{F(g)}\]
    The experimentally determined rate law is $\qquad$ Rate = k[NO$_2$][F$_2$].
    
    A suggested mechanism for the reaction is 
    \begin{align*}
    \text{NO}_2+\text{F}_2\rightarrow \text{NO}_2\text{F}+\text{F}\qquad \text{Slow}\\
    \text{F}+\text{NO}_2\rightarrow \text{NO}_2\text{F}\qquad \text{Fast}
    \end{align*}

    Is this an acceptable mechanism? That is, does it satisfy the two requirements? Justify your answer.
\end{enumerate}
\end{document}