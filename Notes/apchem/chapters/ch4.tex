\documentclass[../chem.tex]{subfiles}
\graphicspath{{\subfix{../figures/}}}
\begin{document}
\chapter{Chemical Reactions}
\section{Introduction to Reactions}
AP Topic: 4.1

If some aspect of the physical state of matter is altered, but the chemical composition remains the same, then the change is considered to be 
a physical change. The most common physical changes are changes of state.

In a chemical change, which is often called a chemical reaction, the atoms of a substance are rearranged to form new susbtances. A chemical change 
requires that the new substance or substances formed have a different chemical composition to the original substance or substances. Chemical 
changes are often accompanied by observable changes.

Evidence for chemical change can manifest itself in a number of ways. One might see a precipitate form, experience a change of energy in the form of 
light or heat, observe a color change, see the formation of a gas, or observe an electrical current, all of which suggest that a chemical reaction 
has taken place. Such changes are sometimes called driving forces.
\section{Net Ionic Equations}
AP Topic: 4.2

Chemical equations are a shorthand method used to illustrate what happens during a chemical reaction. Reactants react to produce the products. 
There can be a number of steps to writing an equation.
\begin{enumerate}
    \item Write down the equation in words.
    \item Fill in the correct formulas for all the substances.
    \item Balance the equation using coefficients.
\end{enumerate}

Balancing the equation can be tricky and requires practice. It involves the following steps.

\begin{enumerate}
    \item Ensure the correct formulas are being used for all the reactants and products.
    \item Balance each element in turn, remembering to multiply parentheses out carefully. 
    \item When balancing, only place numbers in front of whole formulas. Do not change the formulas of any of the reactants or products or add any extra formulas. The numbers that appear in front of each formula are called the stoichiometric coefficients. They have an extremely important role to play in calculations since they give the reacting ratio.
\end{enumerate}

Lastly, add state symbols; (s) for solid, (l) for liquid, (g) for gas and (aq) for aqueous. State symbols are not always necessary or relevant, and should be applied on a case by case basis.

Many qualitative and quantitative chemical reactions are carried out in an aqueous solution. Water is a convenient medium since it is a cheap 
and readily avaliable solvent, that many solids dissolve in. When dealing with aqueous solution, we often take the molecular equations seen in the previous section, and 
create further nuances. To consider these other ways of representing equations, we have to think about a few aspects of chemicals in solution.

Ionic solutions can be identified by their ability to conduct electricity. If a large number of ions are present in a solution, then the solution will 
be an excellent conductor of electricity. Such a substance is completely ionized and is a strong electrolyte. All soluble ionic compounds, but very few molecular compounds, are strong electrolytes.

If the ionic solution conducts electricity only weakly, there are likely to be a few ions present. Such a substance is partially ionized, and is a weak electrolyte.

Non-electrolytes have no ions present in solution and therefore cannot conduct electricity. Such a substance is not ionized and is a 
nonelectrolyte. Most molecular compounds are either nonelectrolytes or weak electrolytes.

In order to understand what is happening as substances interact with water, we can use ionic equations to show the degree of ionization taking place.

Strong electrolytes are completely dissociated into ions.

Both cations and anions, when surrounded by water, are said to be hydrated.

Weak electrolytes are only partially dissociated into a few ions.

Non-electrolytes are not ionized at all, and therefore there are no ions present. 

When a chemical reaction is shown in its ionic form, it may be possible to simplify it to what is known as a net ionic equation. 

In general the following points are useful
\begin{itemize}
    \item Learn common ions.
    \item Where appropriate, all compounds that produce ions in solution should be written in their ionic form and spectator ions should be ignored.
    \item Balance equations with lowest possible whole numbers.
    \item There is no requirement to use state symbols.
    \item Ions have charges.
\end{itemize}
\section{Representations of Reactions}
AP Topic: 4.3

The law of conservation of mass is a crucial and fundamental part of chemistry. You are expected to relate this idea in terms of symbolic 
representations as well as particulate representations. No matter how the chemical or physical process is represented in an equation, 
it can be translated into a particulate diagram where symbols can be used to represent the equation. It is very important to consider the 
conservation of mass. All particles that were present at the start of the reaction are present at the end.

\section{Physical and Chemical Changes}
AP Topic: 4.4

Evidence for chemical change cna manifest itself in a number of ways. One might see a precipitate form, experience a change of energy in 
the form of heat or light, observe a color change, see the formation of a gas, or observe an electrical current, all of which suggest that 
a chemical reaction has taken place. Such changes are sometimes call driving forces.

In all cases it is vital to distinguish between physical change and chemical change. If there is only an interruption in the intermolecular
forces then the change should be seen as physical, but if there is a rearrangement of the intra bonds, then the change is chemical.

Some changes, such as the dissolution of salt in water, are more difficult to characterize as being physical of chemical, since both intra bonds 
are broken, and intermolecular attractions are broken between the solvent molecules, before ion-dipole forces are created when the ions are surrounded by water molecules.

\section{Stoichiometry}
AP Topic: 4.5 

We have already seen how moles of an element or compound can be calculated and how the moles of a solution can be calculated.

The calculation and use of moles are of enormous importance in chemistry, since if we know the number of moles of a substance that is present 
in a reaction, and we know a balanced chemical equation, it is possible to calculate the moles of another substance present in the equation. Use the following:
\begin{enumerate}
    \item Write a correct and balanced equation.
    \item Find the number of moles present by using a moles relationship for one substance.
    \item Use the stoichiometric coefficients in the balanced equation to find the reacting ratio of the moles. Use this relationship to find the number of moles of the unknown substance.
    \item Re-apply a moles relationship for the unknown substance.
\end{enumerate}

The stoichiometric coefficients are the numbers that are used to balance the chemical reaction and provide the reacting ratio of the moles of the substances.
Stoichiometry is the study of quantities of materials consumed and produced in chemical reactions. A balanced chemical equation that shows the molar ratios 
can be used to make predictions about other substances within the equation.

We have seen previously how Avogadro's law states that equal volumes of all gases at constant temperature and pressure will contain equal numbers of moles. 
The volume of one mole of any gas is called its molar volume, and can be calculated using the ideal gas equation.

We can state that 1 mole of any ideal gas at standard temperature and pressure (STP) occupies a volume of 22.4 L.

When all the reactants in a chemical reaction are completely consumed, then the reactions are said to be in stoichiometric proportions.
On other occasions, it may be that only one particular reactant is completely used up. This happens when one reactant is in excess. The 
reactant that is completely consumed is called the limiting reactant, and it is what determines the quantities of products that form.

In all chemical reactions, the yield of the product will be less than 100\%. The yield is usually less than 100\% since the reactants are often 
not pure, some of the product is lost during purification, the reaction may be reversible and/or side reactions may give by-products. The \% yield 
can be calculated, and is thought of as a guide to the efficiency of the reaction. The higher the \% yield, the greater the efficiency.
\[\% \text{Yield} = \left(\frac{\text{actual yield of product}}{\text{theoretical yield of product}}\right)\times 100\]

Compounds that contain carbon and hydrogen only, when burned completely in oxygen, will yield only carbon dioxide and water. Analysis of the mass of 
CO$_2$ and H$_2$O produced can be used to determine the empirical formula of the substance in question. This method assumes that all the carbon in 
CO$_2$ originated from the carbon in the original compound, and all the hydrogen in the water originated from the hydrogen in the original compound. The method is 
a variation of the steps outlined earlier to determine empirical formulas, as follows.
\begin{enumerate}
    \item Calculate the moles of CO$_2$ produced. Since there is one carbon atom in one molecule of CO$_2$, this is also the number of moles 
    of C atoms present in the original compound.
    \item Calculate the moles of H$_2$O produced. Since there are two hydrogen atoms in one molecule of H$_2$O, multiply this number by two to calculate the number of moles of H atoms present in the original compound.
    \item Calculate the masses of C and H atoms present in the combusted sample by multiplying the moles of each by their molar masses.
    \item If there is another element present in the combusted substance, then calculate its mass by subtracting the mass of C and H from the total mass of the combusted sample. Turn the mass of element into moles by dividing by the appropriate molar mas.
    \item Find the smallest number of moles calculated and divide all the numbers of moles by that number. This gives the molar ratio.
    \item The results from \#5 should be in a convenient ratio, and give the empirical formula.
    \item If necessary, use the molar mass to turn the empirical formula into a molecular formula.
\end{enumerate}

Hydrates are formula units with water associated with them. The water molecules are incorporated into the solid structure. Strong heating 
will drive off the water as a vapor. When the water is completely removed, the salts are said to be anhydrous.
\section{Introduction to Titration}
AP Topic: 4.6

Chemical reactions are often carried out between substances that are in solution. Solutions consist of a solute dissolved in a solvent, usually water.
The concentration of a solution can be measured in terms of the number of grams of the solute that has been dissolved in a particular volume of the 
solution, or more usually, in terms of the number of moles of the solute in a particular volume of the solution. Typical units are mol/L and is referred to as molarity.
\[\text{Moles}=(\text{concentration})(\text{volume})\]

Titration is the name given to the experimental method of analysis that utilizes concentrations of solutions. As above, if we know a balanced 
chemical equation and can calculate the moles of one substance, then by ratio we will know the moles of other substances, and we can use that data 
to calculate an unknown concentration. We need to use a substance with the known concentration that specifically reacts with the solution of unknown
concentration.

If no solid is formed in a reaction, we can use a indicator that changes color at the equivalence point. The observable event that occurs 
at the equivalence point is called the end point. Volumetric analysis is a technique for determining the amount of a certain substance by doing a titration.
\section{Types of Chemical Reaction}
AP Topic: 4.7

There are millions of possible chemical reactions. Classification schemes for chemical reactions are just a way to try to simplify 
thinking about them.

A double replacement/displacement is a reaction where two ions are switched in two compounds to form two new compounds.
\[\text{AB}+\text{CD}\rightarrow \text{AD}+\text{CB}\]
These often involve acids, bases, and salts, but not necessarily. No changes in oxidation numbers occur. All double replacement reactions 
have a driving force that removes a pair of ions from solution.
\begin{itemize}
    \item Precipitation. A precipitate is an insoluble substance formed by the reaction of two aqueous substances. Two ions bond together so strongly that water cannot pull them apart.
    \item Formation of a gas. Gas may form directly or from the decomposition of a product.
    \item Formation of a molecular/covalent substance. When a molecular substance is formed, ions are removed from solution.
\end{itemize}

Acids have formulas that begin with H and have a hydrogen ion that they donate in reactions.

Bases are formulas that end in OH except for ammonia.

A salt is a compound formed when the hydrogen ion(s) in an acid have been replaced by metal ions or the ammonium ion.

A simple redox reaction is a reaction involving the transfer of electrons. In these reactions an oxidizing agent will cause the oxidation of another 
compound and in the process, itself will be reduced and vice-versa. Oxidation numbers change from the reactant side to the product side of the equation 
in a redox reaction.
\begin{itemize}
    \item Single replacement are reactions where an element reacts with a compound. 
    \item Combustion is a reaction with oxygen.
    \item Synthesis and composition are when a reaction with elements and/or compounds combine together to form one product or a single reactant is heated and splits up. Synthesis and decomposition are opposite of one another.
\end{itemize}

A non-simple redox reaction is a reaction involving the transfer of electrons. In these reactions an oxidizing agent will cause the oxidation 
of another compound and in the process, itself will be reduced and vice-versa. Oxidation numbers change from the reactant side to the product side of the equation in a redox reaction.

An oxidizing agent is one that promotes oxidation. Oxidizing agents achieve this by accepting electrons themselves, and in the process, they become reduced.

A reducing agent is one that promotes reduction. Reducing agents achieve this by donating electrons themselves, and in the process, they become oxidized.

In general, nonmetals tend to be good oxidizing agents since they tend to gain electrons and metals tend to be good reducing agents since they tend to give up electrons.
\begin{itemize}
    \item If a reaction takes place in acid solution it means that H$^+$ ions are reactants and water will be one of the products.
    \item If a reaction takes place in basic solution it means OH$^-$ are present and water is one of the products.
    \item Elements in their highest oxidation states can only be reduced and elements in their lowest oxidation states can only be oxidized.
    \item Disproportionation is simultaneous oxidation and reduction of one species.
\end{itemize}

Hydrolysis is a reaction with water.
\begin{itemize}
    \item Hydrides of groups 1 and 2 release hydrogen gas and produce the correspondong hydroxide.
    \item Strong acids will donate H$^+$ in aqueous solution.
\end{itemize}

Transition metal chemistry is a reaction involving transition metal ions in aqueous solution either undergoing ligand exchange or the destruction 
of complexes with acids. A ligand is an ion or molecule attached to a metal atom by coordinate bonding. A coordinate bond is a covalent bond 
in which both electrons come from the same atom.
\begin{itemize}
    \item Aluminum and zinc can act like transition metal ions and form complexes.
    \item Most first row transition metal complexes exist in solution as hexaaqua ions although the water molecule ligands are often omitted.
\end{itemize}

Regardless of the type of reaction, these are steps you should follow when asked to write a net ionic equation:
\begin{enumerate}
    \item Write a complete, balanced equation with states of matter.
    \item Split aqueous ionic or strong acids into ions. Include the charges and make sure the equation remains balanced in mass and charge.
    \item Cancel out any ions that appear split and identical on both sides of the equation. Double check that the net ionic equation is balanced in both mass and charge. Rewrite the equation with the spectator ions removed.
\end{enumerate}
\section{Introduction to Acid-Base Reactions}
AP Topic: 4.8

Acids and bases can be identified in many ways but the most useful way at this stage is the Bronsted-Lowry definition.

It states that an acid is a substance that donates hydrogen ions in aqueous solution and a base is a substance that accepts hydrogen ions in aqueous solution.

Water can act as both an acid and a base and is called amphoteric behavior.

Conjugate acid and base pairs are related by a difference of a hydrogen ion on either side of the equation. 

A strong acid or base undergoes complete ionization. 

Conjugate pairs with a very strong component in a pair will always be accompanied by a equally weak partner. 

Weak acids and weak bases have very little ionization and equilibria are set up with the equilibria laying heavily on the left hand side or the undisassociated form.

One of the most important reactions of acids and bases is their ability to neutralize with one another. A neutralization reaction takes place 
when the hydrogen ions in an acidic solution react with the hydroxide ions and form a basic solution to form water. This makes neutralization 
reactions a special type of double replacement reaction. The other product of a simple neutralization in a salt. Basically 
\[\text{ACID}+\text{BASE}\rightarrow \text{SALT}+\text{WATER}\]

\section{Oxidation-Reduction (Redox) Reactions}
AP Topic: 4.9

Oxidation \& reduction can be defined in a number of ways, one of which is in terms of electrons.

Oxidation is a loss of electrons and reduction is a gain of electrons.

If, during a reaction the oxidation number of an element becomes more positive, then the element has been oxidized. Reduction causes a reduction of oxidation number.

Oxidation number is the number of electrons that an atom loses, or tends to lose, when it is involved in a redox reaction. If the atom gains, or tends to gain electrons,
then the oxidation number is negative, and vice-versa. If there is no change in the number of electrons, then it has not taken part in a redox reaction.
In the case of a simple ion the oxidation number of the species is equal to the ionic charge. In the case of covalent compounds the oxidation number may be regarded 
as the charge that the species would develop if the compound were fully ionic. There are reults that simplify the process of assigning oxidation numbrs.
\begin{enumerate}
    \item The oxidation number of an element when uncombined is always zero.
    \item The sum of the oxidation numbers on a neutral substance is always zero.
    \item In an ion, the sum of the oxidation numbers of any elements present equals the ionic charge.
    \item Some elements very common oxidation numbers in their compounds; Group 1 is always +1; group 2 is always +2; F is always -1; O almost always -2; H almost always +1
    \item In binary compounds with metals, the group 17 elements are -1; group 16 are -2; and group 15 are -3
\end{enumerate}

Disproportionation is a reaction where there is a simultaneous oxidation and reduction of one species.

Synthesis is a reaction where a single product is formed by the reaction of simpler materials, often its elements, but sometimes by compounds.
These reactions are in the general form: $\text{A}+\text{B}\rightarrow \text{AB}$.

Decomposition is a reaction where a single reactant, a compound, is broken down into simpler substances. These reactions are in the general form
$\text{AB}\rightarrow \text{A}+\text{B}$. Decomposition is the reverse of a synthesis reaction.

Single replacement is a reaction where an atom or ion in a compound is displaced by an atom or ion of another element. The general form is 
$\text{A}+\text{BC}\rightarrow\text{AC}+\text{B}$

Some metals will react with water, some with acids, some with both, and some with neither. Predictions can be made about the reactions 
that will and will not happen using the activity series.

Metals are arranged according to their ability to displace hydrogen, the most reactive at the top of the activity series.
\begin{itemize}
    \item All metals above hydrogen in the series will displace it from an acid.
    \item All metals below hydrogen will not displace it from an acid or water.
    \item A metal relatively high in the series will displace one below it from a solution of its ions but the reverse process is not possible.
\end{itemize}

Combustion is a reaction where a compound or element burns in oxygen. The products of such a reaction are simply oxides of that which is combusted.

Very often this ia hydrocarbon that reacts with oxygen to form carbon dioxide and water and a large amount of energy.

Some ionic compounds are very soluble in water while are less so. A precipitation reaction occurs when certain cations and certain anions 
combine to form insoluble compounds. By definition, these compounds do not dissolve in water, and form insoluble solids, or precipitates. 
In order to study these reactions, it is necessary to be aware of the solubility rules.

Using the solubility rules, it becomes possible to make predictions about precipitation reactions. 

A reaction wheere there is rearrangement of both cation anion pairs can be classified as a double replacement reaction. Note that not all 
double replacement reactions result in a solid precipitate being formed.

The formation of colored precipitates and other qualitative tests are an important tool in aqueous chemistry.

Gravimetric analysis involves the addition of a substance to an aqueous solution that causes the formation of a solid. The substance that 
is added is specifically chosen to react with the analyte.

When no more precipitate forms, we can be confident that the analyte has been totally consumed, and this is the point at which the stoichiometric 
molar ratio has been achieved. After their formation, such solids are usually removed from solutions by filtering, washing, and drying. Gravimetric 
analysis is separating the precipitate by filtration.

And finally, a note about how to balance redox reactions by the half reaction method.
\begin{enumerate}
    \item Divide the equation into oxidation and reduction half reactions.
    \item Balance all elements besides hydrogen and oxygen.
    \item Balance O's by adding H$_2$O's to the appropriate side of each equation.
    \item Balance H's by adding H$^+$.
    \item Balance the charge by adding electrons.
    \item Multiply the half reactions to make the number of moles of electrons equal for both half-reactions.
    \item Cancel out any common terms and recombine the two half reactions.
    \item If basic, neutralize any excess H$^+$ by adding the same number of OH$^-$ to each side of the balanced equation.
\end{enumerate}

\section*{Problems}
\begin{enumerate}
    \item Write a balanced equation for Carbon + oxygen $\rightarrow$ carbon monoxide
    \item Describe any observations that you might reasonably expect to make during the reaction between hydrogen gas and oxygen gas to form water.
    \item Discuss the change in forces and bonds when water decomposes into its elements.
    \item A sample of copper(II) sulfate pentahydrate with a mass of 8.512 g is dissolved in enough water to make 500.0 mL of solution. A 25.00 mL portion completely reacts with 20.00 mL of 0.1702 mol L$^{-1}$ solution of iodine ions. In what molar ratio do Cu$^{2+}$ and iodide ions react?
    \item Hydroxides can be used to neutralize acids. What volume of 1.00 M NaOH, would be required to completely neutralize 25.0 mL of 2.00 M HCl?
    \item Calculate the molas mass of a gas that is formed when 0.120 g of its liquid is vaporized, at a temperature of 50.0$^{\circ}$C, and occupies a volume of 38.0 mL at atmospheric pressure.
    \item Sulfur and chlorine react with the equation below. If 202 g of sulfur are allowed to react with 303 of Cl$_2$ in the reaction, what is the limiting reactant? What is the mass of product that will be produced? What mass of the excess reactant will be left over?
    \[\text{S(s)}+3\text{Cl}_2\text{(g)}\rightarrow \text{SCl}_6\text{(l)}\] 
    \item Aluminum will react with oxygen gas according to the equation below. In one such reaction, 23.4 g of Al are allowed to burn in excess oxygen. 39.3 g of alumnium oxide are formed. What is the percentage yield?
    \[4\text{Al}+3\text{O}_2\rightarrow 2\text{Al}_2\text{O}_3\]
    \item When 4-ketopentenoic acid is analyzed by combustion, it is found that a 0.3000 g sample produces 0.579 g of CO$_2$ and 0.142 g of H$_2$O. The acid contains only carbon, hydrogen, and oxygen. What is the empirical formula of the acid?
    \item A sample of the hydrated salt CoCl$_2\cdot$xH$_2$O, with a mass of 11.73 g, is heated to drive off the water of crystallization, cooled, and reweighed until constant mass (6.410 g) is achieved. Calculate the value of x.
    \item Hydroxides can be used to neutralize acids. It is found an indicator changes color at the precise moment that 44.0 mL of NaOH has been added to 25.0 mL of 2.00 mol L$^{-1}$ HCl in a titration. Use these data to calcualte the concentration of NaOH.
    \item Write the net ionic equation for when hydrogen sulfide gas is bubbled through a solution of silver nitrate and what would be the expected pH of the solution at the end of the reaction?
    \item Write the net ionic equation when magnesium metal is added to a solution of iron(III) chloride.
    \item Write the net ionic equation for when concentrated hydrochloric acid is added to solid manganese(IV) oxide.
    \item Write the net ionic equation for when gaseous hydrogen chloride is bubbled into water.
    \item Write type of bond is formed between the ligand and the central transition metal in complexes?
    \item Write the net ionic equation when excess dilute nitric acid is added to a solution containing tetraaminecadium(II) ions.
    \item Acetic acid is refluxed with ethanol and a catalyst for several hours. Write the net ionic equation.
    \item Identify the acid, base, conjugate acid, and conjugate base:
    \[\text{HBr}+\text{NH}_3\rightarrow \text{NH}_4^+ +\text{Br}^-\]
    \item Write a whole formula, ionic, and net ionic equation for the neutralization reaction between aqueous solutions of potassium hydroxide and sulfuric acid.
    \item Write an overall reaction for the formation of aluminum oxide, then write two half equations to identify the redox process.
    \item After a reaction where metal magnesium is placed in a crucible and heated strongly in air until it has been completely oxidized by oxygen in the air, what do you predict will happen to the mass of the crucible and its contents?
    \item Iodate ions react with iodide ions according to the equation below. A sample of sodium iodate with a mass of 0.311 g is dissolved in water and made up to 250. mL. 25.0 mL portions are added to KI that has been dissolved in sulfuric acid. The resultant iodine is titrated against sodium thiosulfate, the average volume being 12.5 mL. Calculate the molarity of the thiosulfate solution.
    \[\text{IO}_3^-\text{(aq)}+5\text{I}^-\text{(aq)}+6\text{H}^+\text{(aq)}\rightarrow 3\text{I}_2\text{(aq)}+3\text{H}_2\text{O(l)}\]
    \item Hydrated ammonium iron(II) sulfate crystals have the formula, (NH$_4$)$_2$SO$_4\cdot$FeSO$_4\cdot$xH$_2$O. 8.325 g of the salt was dissolved in 250.0 mL of acidified water. A 25.00 mL portion of thie solution was titrated with potassium manganate(VII) solution of concentration at 0.4180 M. A volume of 2.250 mL was required. Calculate the value of x.
    \item Predict if a reaction takes place between AlCl$_3$(aq) + LiOH(aq). 
    \item A solid of unknown composition contains some chloride ions. A 0.182 g sample of the solid is dissolved in water and the cloride ions dissolve to produce an aqueous solution. The solution has a large amount of aqueous silver ions added to it until no more precipitate can be formed. After filtering, washing, and drying, it is found that 0.287 g of the precipitate is produced in the reaction. Calculate the mass percentage of chloride ions in the original sample.
    \item Potassium dichromate (K$_2$Cr$_2$O$_7$) is a bright orange compound that can be reduced to a blue-violet solution of Cr$^{3+}$ ions. Under certain conditions, K$_2$Cr$_2$O$_7$ reacts with ethyl alcohol, (C$_2$H$_5$OH) as shown below. Balance this equation using the half-reaction method.
    \[\text{H}^+\text{(aq)}+\text{Cr}_2\text{O}_7^{2-}\text{(aq)}+\text{C}_2\text{H}_5\text{OH(l)}\rightarrow \text{Cr}^{3+}\text{(aq)}+\text{CO}_2\text{(g)}+\text{H}_2\text{O(l)}\]
\end{enumerate}
\end{document}