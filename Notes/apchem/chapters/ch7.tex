\documentclass[../chem.tex]{subfiles}
\graphicspath{{\subfix{../figures/}}}
\begin{document}
\chapter{Equilibrium}
\section{Introduction to Equilibrium \& Direction of Reversible Reactions}
AP Topic: 7.1, 7.2

Many chemical reactions and processes are reversible. 

Many of these processes and reactions will be accompanied by some kind of observable event that will allow you to see that the process
is reversible. Reversible reactions are usually indicated by double arrows, $\rightleftharpoons$.

A dynamic equilibrium exists in a reversible chemical reaction when the rate of the forward reaction is equal to the rate of the reverse reaction.
For a reaction to reach equilibrium, a finite amount of time must elapse.

At the beginning of a reversible reaction the reactant concentrations are high, and as a result the rate of the forward reaction is also high.
At this point this point the product concentrations are low, and the rate of the backward reaction is low.

As the reaction proceeds, the reactant concentrations will fall, and the rate of the forward reaction will begin to decrease. As the product 
concentrations increase, the rate of the backward reaction will increase. The point at which the rates of forward and backward reaction are the same is when 
equilibrium will have been established. At this point, since reactants and products are being converted to one another at the same rate, 
their concentrations do not change.

Once the equilibrium has been established, a sudden introduction of reactants would result in a sudden dramatic increase of the forward reaction rate. 
After the spike in forward rate, a new, constant forward reaction rate will be established. The same effect can be observed in the reverse reaction rate 
when a product is introduced. 

Whether the reaction lies far to the right or to the left depends on three factors.
\begin{itemize}
    \item Initial concentrations 
    \item Relative energies of reactants and products 
    \item Degree of organization of reactants and products
\end{itemize}

When equilibrium has been achieved, on a macroscopic scale it appears that the reaction has `stopped'. Closer inspection on the microscope 
reveals that it is in fact occurring. When equilibrium has been established there are both products and reactants present in the reaction mixture.
If those products and reactants have observable differences, the reaction mixture will often appear as a combination of the two.

\section{Calculating the Equilibrium Constant, Magnitude of the Equilibrium Constant, and Calculating Equilibrium Concentrations}
AP Topic: 7.4, 7.5, 7.7

Consider the equilibrium below, where a, b, c and d are stoichiometric coefficients of substances A, B, C and D respectively.
\begin{center}
    aA (aq) + bB (aq) $\rightleftharpoons$ cC (aq) + dD (aq)
\end{center}

The equilibrium constant (K$_{\text{c}}$) is constant at a given temperature. For the previous reaction at equilibrium, at a given temperature:
\begin{center}
    Kc = $\frac{[C]^c[D]^d}{[A]^a[B]^b}$
\end{center}

Where [] represents the equilibrium concentrations in molarity. K$_{\text{c}}$ has no units.

Note that K$_{\text{c}}$ values do not include values for pure solids or pure liquids, since their concentrations are considered constant and as such are incorporated into the constant.
\begin{itemize}
    \item K$_{\text{c}}$ is used for concentration 
    \item K$_{\text{p}}$ is used for partial pressure
\end{itemize}

When K $>$ 1, the reaction favors the products at equilibrium.

When K $<$ 1, the reaction favors the reactants at equilibrium.

RICE is an acronym for the general steps for solving equilibrium problems.
\begin{itemize}
    \item Set up a ``RICE'' table
    R = write a balanced reaction 

    I = fill in the initial concentrations 

    C = determine the change that is taking place in terms of x

    E = express the equilibrium concentrations in terms of x 
    \item Set up the equilibrium expression and set it equal to its value, if given.
    \item Celebrate if you are given equilibrium concentrations - skip down to the ``E'' line and fill them in. You may be asked to work backwards to determine the ``change'' in equilibrium 
    \item If you are given a K value, then use it to solve for x and use x to calculate the equilibrium concentrations.
    
    Hints:
    \begin{itemize}
        \item Look for very small K values, ``x'' may be negligible.
        \item If ``x'' is large enough to impact the equilibrium values, then you must subtract it from the initial concentration. 
        \item If none of the initial concentrations are zero, then Q must be calculated first to determine the direction of the shift before following the above general steps.
    \end{itemize}
\end{itemize}

The magnitude of K can give a quantitative guide to the amount of products and reactants in any given equilibrium mixture. 
Very large K values indicate a large amount of products, so the reaction essentially ``go to completion''. Very small K values indicate 
large numbers of reactants, so the reactions hardly proceed at all.
\section{Representations of Equilibrium}
AP Topic: 7.8

Equilibrium constants for gaseous reactions are usually found in terms of the partial pressures of the components of the mixture. 
Partial pressures may be given, or if not, can be calculated using 
\begin{center}
    Partial Pressure of A = (mole fraction of A)(Total Pressure)
\end{center}
\begin{center}
    Mole fraction of A = $\frac{\text{moles of A}}{\text{total moles}}$
\end{center}

When a homogeneous gaseous equilibrium is established it is possible to express the equilibrium constant for the reaction in one of two ways, either in terms 
of concentrations as a Kc value, or in terms of partial pressure as a Kp value. It is often helpful to use the two terms interchangeably and this can 
be achieved using the expression
\begin{center}
    Kp = Kc(RT)$^{\Delta{\text{n}}}$
\end{center}
$\Delta$n is the stoichiometric number of moles of gaseous products minus the stoichiometric number of moles of gaseous reactants

Note when $\Delta$n = 0, Then K$_{\text{c}}$ = K$_{\text{p}}$. 
\section{Reaction Quotient and Equilibrium Constant \& Reaction Quotient and Le Chatelier's Principle}
AP Topic 7.3, 7.10

When equilibrium is in the process of being established, we can use the reaction quotient, Q, to determine to what extent the reaction has proceeded.
Considering the equilibrium from earlier,
\begin{center}
    aA (aq) + bB (aq) $\leftrightharpoons$ cC (aq) + dD (aq)
\end{center}
\begin{center}
    Q = $\frac{[\text{C}]^{\text{c}}[\text{D}]^{\text{d}}}{[\text{A}]^{\text{a}}[\text{B}]^{\text{b}}}$
\end{center}

It may seem that Q is essentially the same as K, and it is, but the difference being that K is the ratio of products to reactants when equilibrium 
has been established, and Q is the same ratio at any other point. This allows us to make predictions about what a reaction will do, with any given 
set of conditions, in order to establish the equilibrium position, and convert Q to K. 
\begin{itemize}
    \item When K $<$ Q then there are too many products in the reactant mixture, and the equilibrium must shift backwards, to the reactant side, in order to reduce the product to reactant ratio and lower Q to a point where Q = K.
    \item When K $>$ Q then there are too many reactants in the reaction mixture, and the equilibrium must shift forwards, to the product side, in order to increase the product to reactant ratio and raise Q to a point where Q = K.
    \item When K $=$ Q, equilibrium has been established and there will be no more further observable changes.
\end{itemize}

\section{Properties of the Equilibrium Constant}
AP Topic: 7.6

It is helpful to be aware of the possible different formats that K and Q could take under circumstances that appear very similar. When a reversible 
reaction is written in the opposite direction, the new equilibrium constant is equal to the reciprocal of the original equilibrium constant; if a reaction is
``halved'' or ``doubled'', then the original equilibrium constant must be square rooted or squared respectively in order to find the new equilibrium constant.
\section{Introduction to Le Chatelier's Principles}
AP Topic: 7.9

Le Chatelier's principle states that in any equilibrium system, when a stress is placed upon the system, such as a change in temperature, pressure or concentration,
then there is a shift in the position of the equilibrium to oppose that stress. This shift occurs because the stress will cause Q (or if it is a temperature stress, K) to change,, and the 
equilibrium will have to shift in order to bring Q back into numerical agreement with K.

It should be noted that since a catalyst causes no change in Q or K, it therefore causes no shift in the equilibrium, but that a catalyst does 
increase the rates of both the forward and backward reactions, and therefore the rate that equilibrium is established.

Increasing the total pressure in the reaction vessel by adding an inert gas has no effect on the partial pressures of the gases in the equilibrium system, so no change in Q results, so no shift occurs.

We can rationalize a change in K with increased temperature, by considering the energy profiles in each type of reaction. For an exothermic forward reaction,
the activation energy of the forward reaction is lower than the activation energy of the endothermic, backward reaction; for an endothermic forward reaction, 
the activation energy of the forward reaction is larger than the activation energy of the exothermic backward reaction. In short, the endothermic direction always has the larger activation energy in an equilibrium system.

Coupling this knowledge with the fact that an increase in temperature causes a larger increase in the rate of reaction of those reactions with larger activation energies, 
we can conclude that an increase in temperature causes the rate of the endothermic reaction in an equilibrium system to be increased more than the rate of the exothermic reaction. 
This means that increasing temperature will always shift an equilibrium system in the endothermic direction. For exothermic forward reactions, that means backward, 
to give more reactants and smaller K's, and for endothermic forward reactions that means forward, to give more products and larger K's.

In electrochemistry, predictions about how the voltage of an electrochemical cell will change if the conditions are not standard are discussed. That explanation uses a ``pseudo'' Le Chatelier argument. We can also explain the changes in voltage 
under non-standard conditions by using ``Q versus K'', and that is the preferred argument.
\section{Introduction to Solubility Equilibria}
AP Topic: 7.11

You should remember that all sodium, potassium, ammonium and nitrate salts are completely soluble in water. Soluble is defined as ``greater than 3 grams dissolving in 100 mL of water''. We call everything else 
``insoluble'', even if a good bit of it does dissolve. So ``insoluble'' is not an absolute term. Even ``insoluble'' salts dissolve to some degree, and these slightly 
soluble salts establish a dynamic equilibrium with the hydrated cations and anions in solution.

K$_{\text{sp}}$ is defined as the product of the equilibrium concentrations of the constituent ions raised to their stoichiometric coefficients. 
The smaller the value of K$_{\text{sp}}$ then the smaller the number of ions that have gone into solution, and therefore the less soluble the compound is in water. The concentration 
of ions that have gone into solution is usually expressed in terms of molar solubility, the number of moles of a solute in a 1.0 L of a saturated solution. 
However, solubility may be expressed as the number of grams of a solute in 1.0 L of a saturated solution. Obviously, these two terms have different definitions, and as such 
it is occasionally necessary to convert solubility values to molar solubility values, before using them in K$_{\text{sp}}$ expressions.

With some knowledge of the reaction quotient, Q, you can determine whether a precipitate will form and what concentrations of ions are required to begin the precipitation of an insoluble salt.
\begin{itemize}
    \item If K$_{\text{sp}}$ $>$ Q, the system is not at equilibrium (unsaturated).
    \item If K$_{\text{sp}}$ $=$ Q, the system is at equilibrium (saturated).
    \item If K$_{\text{sp}}$ $<$ Q, the system is not at equilibrium (supersaturated). 
\end{itemize}

Precipitates form when the solution is supersaturated.

Metal-bearing ores often contain the metal in the form of an insoluble salt, and, to complicate matters, the ores often contain several such metal salts.
To selectively precipitate only one type of metal ion, the metal salts can be dissolved to obtain the metal ions and then concentrated in some manner.

Relative solubilities of sparingly soluble salts can be deduced by comparing values of K$_{\text{sp}}$, but be careful! These comparisons
can only be made for salts having the same ratio of cations to anions. For example AgCl and BaSO$_4$ are a good pair to compare in this manner since their ratios 
of ions is 1:1, and their Q/K expressions are both K = x$^2$. The larger K will result in the larger square root of K, hence the larger x value, and hence the greater solubility.

However, such a comparison is not necessarily so easy with salts that have different ratios of ions.

Finally, don't forget that solubility changes with temperature. Some substances become less soluble in cold water while others increase in solubility.

The formation of complex ions can often dissolve otherwise insoluble salts. Often as the complex ions forms, the solubility 
equilibrium shifts to the right and causes the insoluble salt to become more soluble. 

\section{Free Energy of Dissolution}
AP Topic: 7.14

The relative thermodynamic favorability of the dissolution of a salt is determined by the sign of $\Delta$G. As we will see later, $\Delta$G and $\Delta$G$^{\circ}$ depend on both enthalpy and entropy. 
The enthalpy change for dissolution is dependent upon three, independent factors;
\begin{enumerate}
    \item The separation of the solute particles from one another.
    \item The separation of the solvent particles from one another.
    \item The interaction between the solute particles and the solvent particles.
\end{enumerate}

In addition to enthalpy changes, we also need to consider entropy changes. When a solute dissolves, entropy increases.

It is the cumulative effects of enthalpy and entropy factors that determine the value of G$^{\circ}$, and ultimately the thermodynamic favorability of any dissolution process.
\section{Common-Ion Effect}
AP Topic: 7.12

If a solution contains two dissolved substances that share a common ion, then the solubility of a salt becomes a more complex matter to determine. 
Experiments show that the solubility of any salt is always less in the presence of a common ion.

\section{pH and Solubility}
AP Topic: 7.13

When one of the ions in a salt can act as an acid or base, pH can influence solubility. 

In general, insoluble inorganic salts containing anions derived from weak acids tend to be soluble in solutions of strong acids.

Salts are not soluble in strong acid if the anion is the conjugate base of a strong acid!
\section*{Problems}
\begin{enumerate}
    \item Write the expressions for $K$ and $K_p$ for the process of deep blue solid copper(II) sulfate pentahydrate being heated to drive off water vapor to form white solid copper(II) sulfate.
    \item K for the reaction below has a value of 55 at a certain, given temperature. Calculate the number of moles of HBr present in an equilibrium mixture that contains 3.0 moles of hydrogen gas and 0.45 moles of bromine vapor at this temperature.
    \begin{center}
        H$_2$(g) + Br$_2$(g) $\rightleftharpoons$ 2HBr(g)
    \end{center}
    \item Sulfur dioxide and oxygen were mixed in the molar ratio 2:1 and allowed to reach equilibrium at a total pressure of 4.45 atm. At this point, 28.0\% of the sulfur dioxide was converted into sulfur trioxide. Calculate K$_p$ for the reaction.
    \item For the synthesis of ammonia at 500$^{\circ}$C, the equilibrium constant is $6.0\times 10^{-2}$. Predict which direction the system will go when [NH$_3$]$_0$ = $1.0\times 10^{-4}$M, [N$_2$]$_0$ = 5.0M and [H$_2$]$_0$ = $1.0\times 10^{-2}$M.
    \item The following equilibrium concentrations were observed for the Haber process at 127$^{\circ}$C:
    \begin{center}
        [NH$_3$] = $3.1\times 10^{-2}$ mol/L \smallbreak  
        [N$_2$] = $8.5\times 10^{-1}$ mol/L \smallbreak 
        [H$_2$] = $3.1\times10^{-3}$ mol/L
    \end{center}
    Calculate the value of the equilibrium constant at 127$^{\circ}$ for the reaction given by the equation:
    \begin{center}
        $\frac{1}{2}$N$_2$(g)+$\frac{3}{2}$H$_2$(g)$\rightleftharpoons$ NH$_3$(g)
    \end{center}
    \item Assume that gaseous hydrogen iodide is synthesized from hydrogen gas and iodine vapor at a temperature where the equilibrium constant is $1.00\times 10^{-2}$. Suppose 
    HI at $5.000\times 10^{-1}$ atm, H$_2$ at $1.000\times 10^{-2}$ atm, and I$_2$ at $5.000\times 10^{-3}$ atm are mixed in a 5.000-L flask. Calculate the equilibrium pressures of all species.
    \item Predict the shift in equilibrium position that will occur when the volume is reduced
    The preparation of liquid phosphorus trichloride by the reaction:
    \begin{center}
        P$_4$(s)+6Cl$_2$(g)$\rightleftharpoons$4PCl$_3$(l)
    \end{center}
    \item A solution is prepared by adding 750.0 mL of $4.00\times 10^{-3}$M Ce(NO$_3$)$_3$ to 300.0 mL of $2.00\times 10^{-2}$M KIO$_3$.
    Will Ce(IO$_3$)$_3$ (K$_{\text{sp}}=1.9\times 10^{-10}$) precipitate from this solution?
    \item A solution is prepared by mixing 150.0 mL of $1.00\times 10^{-2}$M Mg(NO$_3$)$_2$ and 250.0 mL of $1.00\times 10^{-1}$M NaF. Calculate the concentrations of Mg$^{2+}$ and F$^-$ at equilibrium with solid MgF$_2$ (K$_{\text{sp}}=6.4\times 10^{-9}$).
    \item Purely in terms of Q and K, explain why barium sulfate is less soluble in a solution of sodium sulfate than it is in water.
    \item The solubility product of Pb(OH)$_2$ is $1.2\times 10^{-15}$. Find the molar solubility of a saturated solution of Pb(OH)$_2$ and find the molar solubility of Pb(OH)$_2$ in a solution of pH = 11.00. Explain the changes in the molar solubility of Pb(OH)$_2$.
\end{enumerate}

\end{document}