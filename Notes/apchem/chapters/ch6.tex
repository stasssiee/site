\documentclass[../chem.tex]{subfiles}
\graphicspath{{\subfix{../figures/}}}
\begin{document}
\chapter{Thermochemistry}
\section{Endothermic and Exothermic Processes}
AP Topic: 6.1

Energy is the ability to do work or produce heat; the sum of all potential and kinetic energy in a system is known as the internal energy of the system.
\begin{itemize}
    \item Potential energy - energy by virtue of position. In chemistry this is usually energy stored in bonds. When bonded atoms are separated, the PE is raised because energy must be added to overcome the coulombic attraction between each nucleus and the shared electrons. 
    \item When atoms bond, the above mentioned coulombic attraction results in energy being released thus the system has a subsequently lower PE.
    \item Kinetic energy - energy of motion. KE is proportional to Kelvin temperature; kinetic energy depends on the mass and the velocity of the object: KE = 1/2 mv$^2$.
    \item Units of Energy:
    \begin{itemize}
        \item calorie: amount of heat needed to raise the temperature of 1.00 g of water 1.00$^{\circ}$C.
        \item kilocalorie: 1000 calories and the food label calorie with a capital C.
        \item joule: SI unit of energy; 1 cal = 4.184 J 
    \end{itemize}
\end{itemize}

Law of Conservation of Energy - You may know it as ``energy is never created nor destroyed'' which implies that any change in energy of a system 
must be balanced by the transfer of energy either into or out of the system. This means that the energy of the universe is constant and is known as the First Law of Thermodynamics.

Mnay reactions and changes involve either the absorption or release of energy in the form of heat. We describe these as either endothermic or exothermic.

During these endothermic and exothermic reactions, energy is transferred either from or to the surroundings.

System - the part of the world that you are studying. This might be a beaker containing salt, water and a thermometer, or it could be just the salt that is dissolving. It is important to always define the system when looking at a problem.

Surroundings - everything else around the system. This might be the air, the tabletop, and your hands holding the beaker, or it might include the water, beaker, thermometer, etc.

The universe is the system and surroundings together. Heat or work is always transferred between the system and the surroundings. Therefore, energy (system) = -energy (surroundings), or if not work is done the heat lost is equal to the heat gained, q$_{\text{sys}}$ = -q$_{\text{surr}}$.

Heat, q - Two systems with different temperatures that are in thermal conduct will exchange thermal energy, the quantity of which is called heat. This transfer of energy in a process 
transfers heat because of the temperature difference but, remember, temperature is not a measure of energy, it just reflects motion of particles. 

With increased Kelvin temperature of particles, comes increased kinetic energy - the higher the Kelvin temperature of particles, the more they move.
\begin{center}
    KE per molecule = 1/2 mv$^2$
\end{center}

As the Kelvin temperature approaches 0 K, the average kinetic energy of the particles also approaches zero.

The average kinetic energy of the particles in a high temperature system is greater than the average kinetic energy of particles in a low temperature system.

When two systems that are at different temperatures come into thermal contact with one another, energy is transferred from the hotter one to the cooler one, 
until the temperatures of each are equal. At this point, the avearge kinetic energy of the particles of each system becomes the same, and a single, intermediate temperature is achieved.

The transfer of energy is called heat or `heat transfer' - heat is not a substance, rather it is the transfer of energy. When the two systems are at the same temperature, there is no further transfer of heat and a thermal equilibrium is reached.

In addition to the thermal transfer of energy that takes place when a hot body comes into contact with a cold body, energy can also be transferred via work.
\[\Delta E = q+w\]

Most commonly in chemistry, we think of work in terms of gases and their expansion and contraction. Consider a gas inside a piston.

As the gas expands, the particles collide with the piston and energy is transferred from the gas to the piston. The work done by the gas on the piston can be 
quantitatively assessed. By considering the volume change of the gas, and the external pressure P that the gas is working against, it can be calculated via 
\[ w=-P\Delta V\]

As the gas expands it does work on the piston, energy is transferred to the piston and the piston moves. Because of conservation of energy, 
any energy lost by one system, must be gained in equal magnitude by the other system. In this work scenario, energy flows from one system to the other. Keep in mind, 
\begin{itemize}
    \item If the system is pushing on the surroundings and expanding a piston, then the work is negative.
    \item If the surroundings are pushing on the system and compressing a piston, then the work is positive.
\end{itemize}

In endothermic processes:
\begin{itemize}
    \item Heat is absorbed by the system from the surroundings.
    \item This includes the phase changes - melting, vaporization, and sublimation.
    \item These usually involve changes involving breaking/separation.
\end{itemize}

In exothermic processes:
\begin{itemize}
    \item Heat is released from the system to the surroundings.
    \item It includes the phase changes - freezing, condensation, and deposition.
    \item Changes that involve forming bonds or IMFs.
\end{itemize}

When a solution forms, there are three processes that take place. By combining all three processes, you can determine if the dissolution overall is an endothermic or exothermic process.
\begin{enumerate}
    \item The solvent must expand by overcoming its intermolecular forces. This is an endothermic process.
    \item The solute must expand by overcoming its intermolecular forces. This is an endothermic process.
    \item The solute and solvent must recombine. This process is exothermic.
\end{enumerate}

\section{Energy Diagrams}
AP Topic: 6.2

An endothermic reaction occurs when 
\begin{itemize}
    \item Energy is absorbed by the system from the surroundings
    \item The products are higher in potential energy than the reactants
    \item The temperature of the surroundings usually decreases if heat is transferred
\end{itemize}

An exothermic reaction occurs when 
\begin{itemize}
    \item Energy is released from the system into the surroundings
    \item The products are lower in potential energy than the reactants
    \item The temperature of the surroundings usually increases if heat is transferred
\end{itemize}

The difference in the energy of the products and energy of the reactants is called the enthalpy of reaction, or heat of reaction, and is 
represented as $\Delta_{\text{rxn}}$, with units of kJ/mol$_{\text{rxn}}$.
\begin{center}
    $\Delta^{\circ}_{\text{reaction}}$ = $\sum\Delta H_{\text{f products}}^{\circ}$-$\sum\Delta H_{\text{f reactants}}^{\circ}$
\end{center} 

The sign for $\Delta$H is:
\begin{itemize}
    \item Positive for endothermic reactions.
    \item Negative for exothermic reactions.
\end{itemize}

On the reaction energy diagram,
\begin{itemize}
    \item $\Delta$H is labeled as the difference between the energies of the products and reactants.
    \item Activation energy, E$_{\text{a}}$ is the energy needed to start the reaction, used to break bonds or intermolecular forces in the 
    reactants. The E$_{\text{a}}$ of the forward reaction is labeled as the difference between the reactants and the peak of the graph. 
\end{itemize}

Catalysts are added to reactions to increase the reaction rate. They are not consumed in the reaction. Some catalysts provide an alternative pathway with lower activation energies.

Note: $\Delta$H$_{\text{rxn}}$ is not changed by a catalyst.
\section{Heat Transfer and Thermal Equilibrium}
AP Topic: 6.3

Recall that when two substances at different temperatures are in contact with each other, heat is always transferred from the hotter substance to the 
colder one. If eventually the two substances reach the same temperature, thermal equilibrium has been reached. When two substances are at the same temperature, 
the particles in the substances have the same average kinetic energy.
\section{Heat Capacity and Calorimetry}
AP Topic: 6.4

Calorimetry is the experimental technique used to measure energy changes in a chemical system. The usual process for calorimetry involves a chemical reaction being put into thermal contact with a heat bath.
There are three factors that contribute to the amount of heat, q, transferred; they are the mass of the object, the specific heat capacity, and the change in temperature.

Specific heat capacity is defined as the amount of energy required to raise 1 g of a substance by 1$^{\circ}$C and the experiment is carried out at constant pressure.
Constant pressure is achieved by using open containers. Different substances have different values of c$_{\text{p}}$. The fact that is it a unique value for any 
particular substance means that the transfer of an identical amount of energy to the same mass of two different substances will not result in the same temperature change of the two substances.
\begin{itemize}
    \item Specific heat capacity, c - the energy require to raise the temperature by 1 degree, [J/$^{\circ}$C]
    \item Molar heat capacity - same as above but specific to one mole of substance, [J/mol K] or [J/mol$^{\circ}$C]
    \item Specific heat of water (liquid state) = 4.184 J/g$^{\circ}$C (or 1.00 cal/g$^{\circ})$. Water has one of the highest specific heats known! This property makes life on earth possible and regulates earth's temperature year round.
\end{itemize}

The amount of energy transferred (q) can be related to the temperature change of a substance ($\Delta$T) by the equation below, where m = mass and c = specific heat capacity.
\begin{center}
    q = m c $\Delta$T
\end{center}

If the temperature of the heat bath goes up, then the chemical reaction must have released energy. If the temperature of the heat bath goes down, then the 
chemical reaction must have absorbed energy. In each case, the `system' is the chemical reaction, and the heat bath represents the `surroundings'.

We can apply conservation of energy, and assuming that there are no energy losses, the magnitude of the energy lost or gained by the chemical reaction, must be equal to the magnitude of the energy gained 
or lost by the heat bath. Since these processes will be either exothermic or endothermic, then we can write 
\begin{center}
    q$_{\text{system}}$ = -q$_{\text{surroundings}}$ $\qquad$ or $\qquad$ -q$_{\text{system}}$ = q$_{\text{surroundings}}$
\end{center}
where the negative sign acts in order to show that the processes are of the same magnitude, but are different in direction of heat flow.

\section{Energy of Phase Changes}
AP Topic: 6.5

Chemical systems change their energy through three main processes: heating/cooling, phase transitions, and chemical reactions. We have seen that we can use q = m c $\Delta$T to calculate the 
heat transferred when a temperature change occurs, or in other words, when there is a change in the kinetic energy of the particles. But during a phase change,
there is no change in temperature. Rather, the energy transferred is used to change the position of the particles relative to one another. In order words, 
the potential energy is changing during a phase change. This is shown as the plateaus on the heating or cooling curve. Notice that the segment for boiling is longer than 
the segment for melting. This is because separating liquid particles to form a gas requires more energy than separating molecules from their solid state to liquid state.

Since the temperature is constant during a phase change, we need to use a different formula to calculate the heat transferred during a phase change as follows:
\begin{center}
    q = m (or n) $\Delta$H$_{\text{fusion or vaporization}}$
\end{center}
where $\Delta$H$_{\text{fusion}}$ is ``heat of fusion'' or ``molar heat of fusion'' and is a constant, depending on the substance being considered.

In summary, energy is either being used to change the temperature, but not the phase, or it is being used to change the phase but not the temperature. A plateau represents a stage when two phases are in equilibrium with one another, and a phase change is occurring.

Note: During a phase change use: q = mol ($\Delta$H) and in a single phase use: q = m c $\Delta$ T

If a curve goes from right to left, you can call the heating curve a cooling curve.

Do not neglect to consider the sign of q, as it denotes the direction of the heat being transferred.
\section{Introduction to Enthalpy of Reaction}
Enthalpy, H - flow of energy at constant pressure when two systems are in contact. Every substance is said to have a heat content or enthalpy. Most reactions involve an enthalpy change, $\Delta$H$^{\circ}$, where 
\begin{center}
    $\Delta$H$^{\circ}$ = $\sum\Delta$H$^{\circ}_{\text{f PRODUCTS}}$-$\sum\Delta$H$^{\circ}_{\text{f REACTANTS}}$
\end{center}

The formula above holds only when we are dealing with enthalpies of formation, $\Delta$H$_f$, so be careful with its use. This equation is sometimes referred to as the ``big mama'' equation.
\begin{itemize}
    \item Enthalpy of reaction - amount of heat released or absorbed by a chemical reaction at constant pressure in kJ/mol$_{\text{rxn}}$
    \item Enthalpy of fusion - heat absorbed to melt 1 mole of solid to liquid at the melting point in kJ/mol$_{\text{rxn}}$
    \item Enthalpy of vaporization - heat absorbed to vaporize or boil 1 mole liquid to vapor at the boiling point in kJ/mol$_{\text{rxn}}$
    \item Standard conditions - you already know about STP, but recall that the T in STP is 0$^{\circ}$C and humans are not happy lab workers when it is that cold. So think of standard conditions as standard lab conditions which are 1 atm of pressure, 25$^{\circ}$C and if solutions are involved, their concentrations are 1.0 M.
\end{itemize}

Enthalpy of a reaction can be calculated from several sources including:
\begin{itemize}
    \item Stoichiometry
    \item Calorimetry
    \item Tables of standard values 
    \item Hess's LawBond energies
\end{itemize}

Note at constant pressure, $\Delta$H = q.
\section{Bond Enthalpies}
AP Topic: 6.7

Recall from Unit 2, if a multiple bond is created between two atoms the bond length observed will be shorter than the corresponding single bond. 
This is because a double bond is stronger than a single bond and hence pulls the atoms closer together. A triple bond is correspondingly shorter and stronger than a 
double bond. Multiple bonds increase the electron density attractions - either way, the nuclei move closer together and the bond length decreases. Evidence for the 
``one and one third'' bond order in the carbonate ion is that all the bonds are found to be the same length, not the different lengths that one would expect to find if a 
combination of double and single bonds were present. Bond order is simply the number of bonding electron pairs shaired by two atoms; fractional bond orders will exist when resonance structures exist for a compound.

Bond length is determined by nature striving for a lower energy state; it will be the distance between the two nuclei where energy is a minimum between the two nuclei. 
When two atoms approach each other, two ``bad'' things happen: electron/electron repulsion and proton/proton repulsion. One ``good'' thing happens: 
proton/electron attraction. When the attractive forces offset the repulsive forces, the energy of the two atoms decreases and a bond is formed.

As states previously, one of the ways that the energy change in a reaction can be calculated is by using bond energies, aka bond dissociation energies (BDEs). Here are some things to keep in mind:
\begin{itemize}
    \item Breaking bonds is always endothermic. The converse is also true.
    \item Bond energies are determined from atoms in the gas phase; they are averages within $\pm$10\%.
    \item When calculating reaction energies from bond energies, bonds in reactants are broken while bonds in products are formed; energy released is greater than energy absorbed in exothermic reactions and the converse is also true.
    \item Note this is ``backwards'' from the thermodynamics ``big mama'' equation, because of that misconception again: it takes energy to break bonds, not to make bonds.
\end{itemize}

To recap, atoms are attracted to one another when the outer electrons of one atom are electrostatically attracted to the nuclei of an atom. The attraction between two atoms makes them increasingly stable, giving lower and lower potential energies.

However, as the atoms continue to approach one another and get increasingly close, there comes a point at which the two nuclei will start to repel one another. As they 
start to repel one another the potential energy is raised, and the two atoms become less stable.

A happy medium is reached at a distance where the forces of attraction and repulsion result in the lowest potential energy. The distance is called the bond length, and the potential energy at that point 
the bond strength.

Since the forces of attraction that stabilize atoms when they bond are attractions between electrons and nuclei, the greater the number of electrons 
involved, the stronger the attraction and as such, triple covalent bonds tend to be stronger than double bonds, and double bonds stronger than single bonds. 
Shorter bonds also tend to be stronger than longer ones.

In order to break a bond, energy must be put in.

When making a bond, energy is released.

The energy change in a reaction can be calculated by first summing the energy required to break each of the bonds in the reactants, then summing the energy released 
when making each of the bonds in the products, and then subtracting the two values. Always draw the molecules so you can determine the number and type of each of the bonds present.
\begin{center}
    $\Delta$H = $\sum$Bond Energies$_{\text{broken}}$ - $\sum$Bond Energies$_{\text{formed}}$
\end{center}

\section{Enthalpy of Formation}
AP Topic: 6.8

Standard Enthalpy of Formation, $\Delta$H$_{\text{f}}^{\circ}$ is defined as the enthalpy change when one mole of a substance is formed from its elements, in their standard states.
Note that the enthalpy of formation for any one element in its standard state is zero.

Use the equation shown to calculate the enthalpy of any reaction given enthalpies of formation.
\begin{center}
    $\Delta$H$^{\circ}$ = $\sum\Delta$H$^{\circ}_{\text{f PRODUCTS}}$ - $\sum\Delta$H$^{\circ}_{\text{f REACTANTS}}$
\end{center}

\section{Hess's Law}
AP Topic: 6.9

Hess's Law states that the enthalpy change during a reaction depends only on the nature of the reactants and products and is independent of the route taken.
In order words, enthalpy is independent of the reaction pathway. If you can find a combination of chemical equations that add up to give you the desired overall 
equation, you can also sum up the $\Delta$H's for the individual reactions to get the overall $\Delta$H$_{\text{rxn}}$. To use it, we need to consider enthalpy changes beyond those of formation.

Standard Enthalpy of Combustion, $\Delta$H$_{\text{c}}^{\circ}$ is the enthalpy change when one mole of a substance is completely burned in oxygen. Note that combustion reactions yield
oxides of that which is combusted.

It is useful to remember that compounds containing some combination of carbon, hydrogen, and oxygen, when completely burned in air, produce carbon dioxide and water only. The combustion of other 
reactants may require other knowledge or intelligent guesswork to determine the products of that combustion.

Occasionally, not all enthalpy of formation values are found in the table of thermodynamic data. For most substances, it is not impossible to go into a lab and directly synthesize a compound from its free elements. The heat 
of formation for the substance must be calculated by working backwards from its heat of combustion.

The process of ionic bond formation can be broken down into a number of states. For example, in the formation of sodium chloride there are twoo possible routs;
\begin{enumerate}
    \item A single step process (Enthalpy change = standard enthalpy of formation of NaCl)
    \item A multi-step process involving five separate changes:
        \begin{itemize}
            \item Atomization of sodium
            \item Ionization of sodium 
            \item Dissociation of chlorine molecules 
            \item Formation of gaseous chloride ions from gaseous chlorine atoms 
            \item Bring together the gaseous ions
        \end{itemize}
\end{enumerate}

By assuming a compound is essentially 100\% ionic, it is possible to calculate a theoretical value for the lattice enthalpy. In some cases 
the theoretical values agree very closely with the experimental values, but not always.

Where the match is poor, the idea of a completely ionic bond is incorrect. The differences are caused by polarization, and the ionic bond taking on a degree of covalent character.
This is another example where the idea of a slidincg scale of bond type if a useful one.

A Born-Haber cycle diagram can be constructed from these data. Often, positive values are denoted as going upwards, negative values as going downwards, but you may see the cycle represented in almost any orientation.

If you see a chemical reaction with a $\Delta$H in units of kJ/mol associated with it, it is reasonable to ask the question, ``moles of what''?
The question is an important one, since in the reaction, there are likely to be a number of different reactants and products. 

\section*{Problems}
\begin{enumerate}
    \item A balloon is being inflated to its full extent by heating the air inside it. In the final stages of this process, the volume of the balloon changes from $4.00\times 10^6$L to $4.50\times 10^6$L by the addition of $1.3\times 10^8$J of energy as heat. 
    Assuming the balloon expands against a constant pressure of 1.0 atm, calculate $\Delta E$ for the process.
    \item A solution of ammonium nitrate was created by dissolving 5.02 grams of ammonium nitrate in 100.0 mL of water at 22.3$^{\circ}$C. After forming the solution, the temperature was 17.3$^{\circ}$C. Was the dissolution process endothermic or exothermic?
    \item When 39.0 grams of copper metal at 92.5$^{\circ}$C is dropped into 200. mL of water at 25.0$^{\circ}$C, the two substances reach thermal equilibrium. Which substance has particles with the greatest average speed?
    \item Propane is the gas that is commonly used in gas drills. A sample of propane with a mass of 44.0 g is completely burned in oxygen and in the process 
    it releases 2002 kJ of energy. This chemical reaction is brought into contact with a water bath, and the transfer of energy from the reaction to the water takes place.

    Assuming the specific heat capacity of water to be 4.200 J g$^{-1}$ K$^{-1}$, and that in such an experiment 100\% of the energy generated 
    by burning the propane is transferred to the water causing the water temperature to increase, calculate the change in temperature of 20.00 kg of water.

    \item 50.0 mL of 0.500 M HCl was added to 50.0 mL of 0.500 M NaOH. The initial temperature of the solutions was 19.8$^{\circ}$C. The reaction below occurred:
    \[\text{HCl(aq)}+\text{NaOH(aq)}\rightarrow \text{NaCl(aq)}+\text{H}_2\text{O(l)}\]
    The final temperature of the mixture was 26.3$^{\circ}$C.

    What is the heat of reaction per mole of NaOH?

    \item Upon adding solid potassium hydroxide pellets to water the following reaction takes place:
    \[\text{KOH (s)} \rightarrow \text{KOH(aq)}+43\text{ kJ/mol}\]
    What is the enthalpy change for the dissolution of 14.0 g of KOH?

    \item Without doing any calculations, determine the enthalpy change for the reaction
    \begin{center}
        CH$_3$CH$_2$OH + CH$_3$COOH $\rightarrow$ CH$_3$CO$_2$CH$_2$CH$_3$ + H$_2$O
    \end{center}

    \item All elements are assigned a value of $\Delta H_f^{\circ}$ equal to zero. Why?
    \item Write an equation to represent the standard enthalpy of combustion of Al(s)
    \item Calculate the standard enthalpy of combustion of 2-propanol, given the following data: 
    
    Enthalpies of combustion for C(graphite) = -393 kJ mol$^{-1}$ and H$_2$(g) = -286 kJ mol$^{-1}$. Enthalpy of formation of 2-propanol = -318 kJ mol$^{-1}$.
    \item The enthalpy of formation of hexane is -199 kJ and the enthalpy of formation of 1-hexene is -73 kJ. Would calculating the $\Delta$H using Hess's Law or in average bond energy terms be more accurate?

\end{enumerate}
\end{document}