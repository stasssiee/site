\documentclass[../stats.tex]{subfiles}
\graphicspath{{\subfix{../figures/}}}
\begin{document}
\chapter{Inference for Quantitative Data: Means}
\section{Constructing a One Sample t-Interval}
Sampling Distribution for Means:
\begin{itemize}
    \item Shape - normal population indicates a normal sampling distribution. $n<30$ then data needs to be stated as approximately normal. $n\geq 30$ satisfies Central Limit Theorem (CLT) which guarantees approximately normal 
    \item Center - as long as you are taking a random sample or using random assignment, $\mu_{\overline{x}}=\mu$
    \item When sampling, as long as 10\% condition is met, $\sigma_{\overline{x}}=\frac{\sigma}{\sqrt{x}}$
\end{itemize}

As long as the above conditions are met, the sampling distribution of $\overline{x}: \overline{X} \sim N\left(\mu,\frac{\sigma}{\sqrt{n}}\right)$

When we do not know the population standard deivation (which we usually don't), we must estimate it from our sample using the sample standard deviation, $s$. However, when we do so, 
the test statistic ($z$) that we previously used changes. The new test statistic is now called the t-statistic and has a new distribution associated with it.

The new t-distribution is not exactly like the standard normal curve, but it is very close:
\begin{itemize}
    \item It it still centered at 0 
    \item It is bell shaped 
    \item Its spread is slightly greater than the standard normal distribution 
    \item It has more area in the tails 
\end{itemize}

The special thing about t-distributions is the fact that it is a family of distributions. There is a unique density curve for each sample size (and the dependence on sample size is taken care of by degrees of freedom: $n-1$.)

The tails on a t-distribution are ``fatter'' than a standard normal distribution. This is true because the small our sample size, the more variation we have, hence the fatter tails. The larger our sample size gets, the closer the t-distribution will have towards the standard normal distribution.

Constructing a confidence interval
\begin{itemize}
    \item Define the parameter. $\mu$ = true mean of $\{$population parameter in context$\}$
    \item Check the assumptions and conditions: 
    \begin{itemize}
        \item Random: The sample should be a random sample of the pouplation or random assignment in an experiment.
        \item Independence (10\% Condition): The sample size, $n$, must be no longer than 10\% of the population.
        \item Normality: THere are multiple ways to verify this condition.
        \begin{itemize}
            \item Stated in problem: It may be stated in the problem that the sampling distribution is approx. normal 
            \item Central Limit Theorem: When $n$ is large, the sampling distribution of the sample means is approx. normal.
            \item Visual representation: You are given a graphical representation (histogram or boxplot) that depicts a shape that is approximately normal. You may also be given data that you have to graph yourself (include a sketch). You are looking for no strong skew or outliers. 
        \end{itemize}
        \end{itemize}
    \item Name the inference method: One Sample t-Interval 
    \item Calculate the interval: point estimate $\pm$ margin of error: $\overline{x}\pm t^*_{n-1}\left(\frac{s}{\sqrt{n}}\right)$. Use 2nd-Vars-4:invT$()$ in the calculator to calculate the critical value.
    \item Write your conclusion in context
\end{itemize}
You can determine the sample size from the formula $ME=t^*_{n-1}\left(\frac{s}{\sqrt{n}}\right)$.

\section{Constructing a One Sample t-Test}
Constructing a significance test:
\begin{enumerate}
    \item Define the parameter: $\mu$ = true mean of $\{$population parameter in context$\}$.
    \item State the hypothesis: the null and alternative.
    \item Check the assumptions and conditions (same as last topic)
    \item Method: One Sample t-Test
    \item Test statistic is statistic-parameter divided by standard deviation of statistic and $t=\frac{\overline{x}-\mu_0}{\left(\frac{s}{\sqrt{n}}\right)}$.
    \item Obtain the p-value from tcdf.
    \item Make a decision.
    \item State your conclusion on context.
\end{enumerate}

\section{Inference for Paired Data}
Comparative studies are more convincing than single-sample investigations. For that reason, one-sample inference is less common than comparative inference.
Study designs that involve making two observations on the same individual or one observation on each of two similar individuals, result in paired data.

When paired data result from measuring the same quantitative variable twice, we can make comparisons by analyzing the differences in each pair. If the conditions for inference are met, we can use one-sample t-procedures to perform inference about the mean difference: $\mu_D$. These methods are called matched pairs procedures.

\end{document}