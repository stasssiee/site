\documentclass[../stats.tex]{subfiles}
\graphicspath{{\subfix{../figures/}}}
\begin{document}
\chapter{Inference for Categorical Data: Proportions}
\section{Constructing a One Proportion z-Interval}
Definition: A confidence interval for a population parameter is an interval of plausible values for that unknown parameter.

It is constructed in such a way so that, with a chosen degree of confidence, the value of the parameter will be captured inside the interval.

The chosen degree of confidence is called the confidence level. The confidence level gives information about how much ``confidence'' we will have in the method used to construct the interval.

To create an interval of plausible values for a parameter, we need two components:
\begin{itemize}
    \item A point estimate is a single value used to estimate the population parameter such as a sample population.
    \item A margin of error represents the maximum expected difference between the true population parameter and the sample estimate.
\end{itemize}

To calculate the interval you use the formula 
\[ \hat{p}\pm z^*\cdot\sqrt{\frac{\hat{p}(1-\hat{p})}{n}}\]

where $\hat{p}$ is the point estimate, $z^*$ is the critical value and $z^*\cdot\sqrt{\frac{\hat{p}(1-\hat{p})}{n}}$ is the margin of error.

To calculate the critical value, you can use invNorm. Generally for a 90\% confidence interval, $z^*=1.64$, for a 95\% confidence interval, $z^*=1.96$ and for a 99\% confidence interval, $z^*=2.58$.
\end{document}