\documentclass[../stats.tex]{subfiles}
\graphicspath{{\subfix{../figures/}}}
\begin{document}
\chapter{Probability, Random Variables, and Probability Distributions}
\section{Basic Probability and Simulations}
\begin{itemize}
    \item Outcomes are governed by change, but in many repetitions, a pattern emerges.
    \item Sample space: all of the possible outcomes.
    \item Event: a specific desired outcome or set of outcomes.
    \item Notation: probability of Event A $\rightarrow$ P(A).
    \item Range of probabilities: probability of event is between 0 and 1.
    \item Sum of probabilities: probability of whole sample is 1.
\end{itemize}

Theoretical probability is what should happen given the sample space: \# of desired outcomes/total \# outcomes.

Empirical probability is when you are performing a simulation or experiment, it is what does happen given the trials: \# successes/\# trials.

The law of large numbers states that simulated probabilities tend to get closer to the true probability as the number of trials increases. 

The notation for the probability that an event does not occur is $P(A^C)$. The probabiliy an event does not occur is $P(A^C)=1-P(A)$.
\begin{itemize}
    \item The idea of probability is that randomness is predictable in the long run.
    \item Probability does not allow us to make short run predictions.
    \item Probability tells us random behavior events out in the long run.
    \item Future outcomes are not affected by past behavior.
\end{itemize}

The imitation of chance behavior, based on a model that accurately reflects the situation, is called a simulation. Simulations are usually 
done with a table of random digits, random number generator, dice, deck of cards, spinner, etc.

Four principles of simulation:
\begin{itemize}
    \item State: identify the probability calculation.
    \item Plan: Describe how to use your chance process.
    \item Do: Perform the simulation 
    \item Conclude: use the results of your simulation to answer the question.
\end{itemize}
\section{The Addition Rule}
\begin{itemize}
    \item When two events have no outcomes in common, we refer to them as mutually exclusive events.
    \item P(A and B) = 0
    \item P(A or B) = P(A)+P(B)
\end{itemize}

If A and B are any two events resulting from some chance process, then
\begin{center}
    P(A or B) = P(A) + P(B) - P(A and B)
\end{center}
Notice how if A and B are mutually exclusive, P(A and B) = 0 and P(A or B) = P(A) + P(B).


\section{Venn Diagrams and the Multiplication Rule}
\begin{itemize}
    \item We can represent events with a Venn diagram - a display of potential probabilities.
    \item The box around the Venn diagram represents the total sample space where P(S) = 1.
    \item The circles themselves represent the probability of each event.
\end{itemize}

Union: $\cup$ is the probability that either occurs.

Intersection: $\cap$ is the probability both will occur.

\begin{itemize}
    \item Two events are independent if they do not influence one another.
    \item The occurrence of one has no effect on the occurrence of the other.
    \item If A and B are independent events, then the probability that both A and B occur is found using the multiplication rule:
    \begin{center}
        P(A and B) = P(A $\cup$ B) = P(A)$\cdot$P(B)
    \end{center}
\end{itemize}

\begin{itemize}
    \item Two events are dependent if they influence one another.
    \item The occurrence of one affects the occurrence of the other.
    \item If A and B are dependent events, then the probability that both A and B occur is found using:
    \begin{center}
        P(A and B) = P(A $\cup$ B) = P(A)$\cdot$P(B$|$A)
    \end{center}
    \item Where P(B$|$A) is the probability B ``given'' that A has occurred.
    \item The conditional probability of an event is the probability that one event will happen, if it is known that another event has happened.
\end{itemize}

\section{Conditional Probability and Tree Diagrams}
\begin{itemize}
    \item If A and B are dependent events, then the probability that both A and B occur is found using 
    P(A$\cup$B)=P(A)$\cdot$P(B$|$A) where P(B$|$A) is the probability of B ``given'' that A has already occurred.
    \item The conditional probability of an event is the probability that one event will happen, if it is known that another event has happened.
    \item We can rearrange the multiplication rule a little and get a formula for conditional probability.
    \begin{center}
        P(B|A)=$\frac{\text{P(A$\cup$ B)}}{\text{P(A)}}$
    \end{center}
\end{itemize}
\begin{itemize}
    \item If we draw two cards with replacement, we know that those events are independent.
    \item If we draw two cards without replacement, we know that those events are dependent.
    Using the following formula we can prove independence:
    \begin{center}
        P(A$\cup$B) = P(A)$\cdot$P(B) or P(A) = P(A$|$B)
    \end{center}
\end{itemize}
\section{Discrete and Continuous Random Variables}

\end{document}
