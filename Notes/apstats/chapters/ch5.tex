\documentclass[../stats.tex]{subfiles}
\graphicspath{{\subfix{../figures/}}}
\begin{document}
\chapter{Sampling Distributions}
\section{Sampling Distributions of Sample Proportions}
As we begin to use sample data to draw conclusions about a larger population, we must be clear about whether a number describes a sample or a population.

\textbf{Parameter: } is a number that describes some characteristic of a population 

\textbf{Statistic: } is a number that describes some characteristic of a sample 

\textbf{Unbiased Estimator: } a sample proportion or sample mean that is equal to the population proportion or population mean 

$\hat{p}$ (sample proportion) estimates $p$ (population proportion)

$\overline{x}$ (sample mean) estimates $\mu$ (population mean)

$s_x$ (sample standard deviation) estimates $\sigma$ (population standard deviation)

Rather than showing real repeated samples, imagine what would happen if we were to actually draw many samples. Now imagine what would happen if 
we looked at the sample proportions for these samples. The histogram we'd get if we could see all the proportions from all possible samples is called the 
sampling distribution of the sample proportions.
\begin{itemize}
    \item We would expect the histogram of the sample proportions to center at the true proportion, $p$, in the population.
    \item The spread is calculated as standard deviation based on the true proportion, $p$, and the sample size, $n$. As the sample size gets larger the standard deviation will get smaller.
    \item The shape of the histogram would be unimodal and symmetric.
    \item More specifically, a normal model is just the right one for the histogram of sample proportions.
\end{itemize}

\textbf{Assumptions and Conditions}
\begin{itemize}
    \item Randomness: The sample should be a simple random sample of the population 
    \item Independence (10\% Condition): The sample size, $n$, must be no larger than 10\$ of the population 
    \item Normality (Large Counts Condition): The sample size has to be big enough so that both number of successes and number of failures are at least 10. We also refer to this as the Success/Fail Condition.
\end{itemize}

Provided that the sampled values are independent and the sample size is large enough, the sampling distribution of $p$ is modeled by a normal model with:

Sample proportions: $\hat{p}=\frac{\#\text{successes}}{\text{sample size}}$

Mean of sample proportions: $\mu_{\hat{p}}=p$

Standard deviation of sample proportions: $\sigma_{\hat{p}}=\sqrt{\frac{p(1-p)}{n}}$

\end{document}