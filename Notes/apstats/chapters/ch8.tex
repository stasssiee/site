\documentclass[../stats.tex]{subfiles}
\graphicspath{{\subfix{../figures/}}}
\begin{document}
\chapter{Inference for Categorical Data: Chi-Square}
\section{Chi Square Test for Goodness of Fit}
A goodness-of-fit test is used to test the hypothesis that an observed frequency distribution fits to some claimed distribution.

To measure the difference betwen the observed and expected counts, and to determine if the difference is significant, we will introduce a new test statistic, called the chi-square statistic:
\[ \chi^2 = \sum \frac{(O-E)^2}{E} \]
Where $O$ represents each observed count in the distribution and $E$ represents each corresponding expected cout.
\begin{itemize}
    \item The sampling distribution of the chi-square statistic is not a normal distribution.
    \item It is a right-skewed distribution that allows only for positive values because the statistic cannot be negative.
\end{itemize}

When all the counts are at least 5, the sampling distribution of the $\chi^2$ statistic is close to a chi-square distribution with degrees of freedom (df) equal to the number of categories minus 1.
\begin{itemize}
    \item The chi-square distributions are a family of distributions that take only positive values and are skewed to the right.
    \item A particular chi-square distribution is specified by giving its degrees of freedom.
    \item The chi-square goodness-of-fit test uses the chi-square distribution with df = number of categories - 1
\end{itemize}

The null hypothesis in a chi-square goodness-of-fit test should stake a claim about the distribution of a single categorical variable in the population of interest.

The alternative hypothesis in a chi-square goodness-of-fit test is that the categorical variable does not have the specified distribution, and is easily given in words.

Conditions:
\begin{itemize}
    \item Random - the data came from a well-designed random sample or randomized experiment 
    \item Independent - When sampling without replacement, the 10\% condition is met 
    \item Large counts - All expected counts are at least 5 lets us say that the sampling distribution will follow a chi-squared distributions
\end{itemize}

When all conditions are met, the chi-squared goodness of fit test can be performed with the hypotheses:
\begin{itemize}
    \item $H_0$: the claimed distribution is correct
    \item $H_a$: at least one proportion in the claimed distribution is incorrect
\end{itemize}


\section{Chi Square Test for Homogeneity}

\section{Chi Square Test for Independence}
\end{document}