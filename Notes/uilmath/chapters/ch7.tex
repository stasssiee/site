\documentclass[../uilmath.tex]{subfiles}
\graphicspath{{\subfix{../figures/}}}
\begin{document}
\chapter{Tips and Strategies}
\section*{Conversions}
\begin{itemize}
    \item 1 hour = 60 minutes 
    \item 1 minute = 60 seconds 
    \item 1 foot = 12 inches 
    \item 1 yard = 3 feet = 36 inches
    \item 1 pound = 16 ounces 
    \item 1 gallon = 4 quarts = 128 ounces  
    \item 1 quart = 2 pints = 32 ounces 
    \item 1 pint = 2 cups = 16 ounces 
    \item 1 cup = 8 ounces 
    \item 1 gallon = 231 cubic inches
    \item 1 square mile = 640 acres 
    \item 1 inch = 2.54 centimeters 
    \item 1 foot = 30.48 centimeters
    \item Normal body temperature = 98.6$\degree$F = 37$\degree$C 
    \item Boiling point of water = 212$\degree$F = 100$\degree$C 
    \item Freezing point of water = 32$\degree$F = 0$\degree$C 
    \item 1 cubic foot = 1728 cubic inches 
    \item 1 cubic yard = 27 cubic feet 
    \item 16 tablespoons = 1 cup 
    \item 1 square foot = 144 square inches
    \item 1 square yard = 9 square feet 
    \item 3 teaspoons = 1 tablespoon  
    \item 1 mile = 1760 yards = 5280 feet
    \item 10 millimeters = 1 centimeter  
    \item 100 centimeters = 1000 millimeters = 1 meter
    \item 1 hectometer = 100 meters  
    \item 1000 meters = 1 kilometer 
    \item 1 dekameter = 10 meters 
    \item 10 decimeters = 1 meter 
    \item 1 year = 12 months = 365 days 
    \item Leap year = 366 days     
\end{itemize}

\section*{Days in Months}
\begin{itemize}
    \item January - 31 
    \item February - 28 or 29 
    \item March - 31 
    \item April - 30 
    \item May - 31 
    \item June - 30
    \item July - 31 
    \item August - 31
    \item September - 30
    \item October - 31
    \item November - 30
    \item December - 31
\end{itemize}

\section*{Prime Numbers Under 100}
2, 3, 5, 7, 11, 13, 17, 19, 23, 29, 31, 37, 41, 43, 47, 53, 59, 61, 67, 71, 73, 79, 83, 89. 97

\section*{Geometry}
\begin{enumerate}
    \item Sum of exterior angles of a regular polygon is 360$\degree$.
    \item Sum of interior angles of a regular polygon = 180$\degree(n-2)$
    \item Measure of exterior angle = $\frac{360\degree}{n}$
    \item Measure of interior angle = $\frac{180\degree \cdot (n-2)}{n}$
    \item Area of a regular polygon 
    \begin{itemize}
        \item Given side: $\frac{ns^2}{4\tan(180\degree/n)}$
        \item Given apothem: $na^2\tan(180\degree/n)$
        \item Given radius: $\frac{nr^2\sin(360\degree/n)}{2}$
    \end{itemize}
    \item Square 
    \begin{itemize}
        \item Area = side$^2$ = $\frac{(\text{diagonal})^2}{2}$
        \item Perimeter = $4s$
        \item Length of diagonal = $s\sqrt{2}$
    \end{itemize}
    \item Triangle 
    \begin{itemize}
        \item Area = $\frac{1}{2}bh$
    \end{itemize}
    \item Equilaterial Triangle 
    \begin{itemize}
        \item Area = $\frac{s^2\sqrt{3}}{4}=\frac{h^2\sqrt{3}}{3}$
        \item Perimeter = $3s$
    \end{itemize}
    \item Rectangle 
    \begin{itemize}
        \item Area = lw 
        \item Perimeter = 2(l+w)
    \end{itemize}
    \item Parallelogram 
    \begin{itemize}
        \item Area = bh 
    \end{itemize}
    \item Trapezoid 
    \begin{itemize}
        \item Area = $\frac{\text{height}(\text{base}_1+\text{base}_2)}{2}$
    \end{itemize}
    \item Rhombus 
    \begin{itemize}
        \item Area = $\frac{\text{(diagonal)}^2}{2}$
    \end{itemize}
    \item Circle 
    \begin{itemize}
        \item Area = $\pi r^2$
        \item Circumference = $2\pi r$ = $\pi d$
    \end{itemize}
    \item Rectangular solid 
    \begin{itemize}
        \item Surface Area = $2(lw+lh+wh)$
        \item Inner diagonal = $\sqrt{\text{length}^2+\text{width}^2+\text{height}^2}$
        \item Volume = $lwh$
    \end{itemize}
    \item Cube 
    \begin{itemize}
        \item Total surface area = 6$e^2$
        \item Volume = $e^3$
        \item Inner diagonal = $e\sqrt{3}$
    \end{itemize}
    \item Sphere 
    \begin{itemize}
        \item Surface area = $4\pi r^2$
        \item Volume = $\frac{4}{3}\pi r^3$
    \end{itemize}
    \item Right Circular Cylinder 
    \begin{itemize}
        \item Lateral Area = $2\pi rh$
        \item Total surface area = $2\pi r^2+2\pi rh$
        \item Volume = $\pi r^2 h$
    \end{itemize}
    \item Right Circular Cone 
    \begin{itemize}
        \item Lateral Area = $\pi r l$ ($l$ is the slant height)
        \item Total Surface Area = $\pi r l +\pi r$
        \item Volume = $\frac{1}{3}\pi r^2$
    \end{itemize}
\end{enumerate}

\section*{More Advanced Formulas}
\begin{enumerate}
    \item Compound Interest: $A = P\left(1+\frac{r}{n}\right)^{nt}$
    \item Compound interest continuously: $A = Pe^{rt}$
    \item Laws of Sines: $\frac{\sin (A)}{a} = \frac{\sin(B)}{b} = \frac{\sin(C)}{c}$
    \item Laws of Cosines: $c^2=a^2+b^2-2ab \sin(c)$
    \item Heron's Formula: Area = $\sqrt{s(s-a)(s-b)(s-c)}$, where $A$ is the area of a triangle with sides $a$, $b$, and $c$; $s$ = semi-perimeter = $\frac{a+b+c}{2}$
    \item Radius (r) of circle inscribed in a triangle with sides $a$, $b$, and $c$:
    
    $r=\frac{\sqrt{s(s-a)(s-b)(s-c)}}{s}$

    \item The area of a triangle given the length of two sides ($a$ and $b$) and an included angle, $C$.
    
    $A = \frac{1}{2}ab\sin(C)$

    \item Area of a sector of a circle given the radius, $r$, of the circle and the measure of the intercepted arc in radians, $\theta = \frac{1}{2}r^2\theta$.
    \item Area of a segment of a circle given the radius, $r$, of the circle and the intercepted arc in radius, $\theta = \frac{1}{2}r^2(\theta - \sin\theta)$
\end{enumerate}

\section*{Random Formulas}


\end{document}