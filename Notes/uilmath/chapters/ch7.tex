\documentclass[../uilmath.tex]{subfiles}
\graphicspath{{\subfix{../figures/}}}
\begin{document}
\chapter{Tips and Strategies}
\section*{Conversions}
\begin{itemize}
    \item 1 hour = 60 minutes 
    \item 1 minute = 60 seconds 
    \item 1 foot = 12 inches 
    \item 1 yard = 3 feet = 36 inches
    \item 1 pound = 16 ounces 
    \item 1 gallon = 4 quarts = 128 ounces  
    \item 1 quart = 2 pints = 32 ounces 
    \item 1 pint = 2 cups = 16 ounces 
    \item 1 cup = 8 ounces 
    \item 1 gallon = 231 cubic inches
    \item 1 square mile = 640 acres 
    \item 1 inch = 2.54 centimeters 
    \item 1 foot = 30.48 centimeters
    \item Normal body temperature = 98.6$\degree$F = 37$\degree$C 
    \item Boiling point of water = 212$\degree$F = 100$\degree$C 
    \item Freezing point of water = 32$\degree$F = 0$\degree$C 
    \item 1 cubic foot = 1728 cubic inches 
    \item 1 cubic yard = 27 cubic feet 
    \item 16 tablespoons = 1 cup 
    \item 1 square foot = 144 square inches
    \item 1 square yard = 9 square feet 
    \item 3 teaspoons = 1 tablespoon  
    \item 1 mile = 1760 yards = 5280 feet
    \item 10 millimeters = 1 centimeter  
    \item 100 centimeters = 1000 millimeters = 1 meter
    \item 1 hectometer = 100 meters  
    \item 1000 meters = 1 kilometer 
    \item 1 dekameter = 10 meters 
    \item 10 decimeters = 1 meter 
    \item 1 year = 12 months = 365 days 
    \item Leap year = 366 days     
\end{itemize}

\section*{Days in Months}
\begin{itemize}
    \item January - 31 
    \item February - 28 or 29 
    \item March - 31 
    \item April - 30 
    \item May - 31 
    \item June - 30
    \item July - 31 
    \item August - 31
    \item September - 30
    \item October - 31
    \item November - 30
    \item December - 31
\end{itemize}

\section*{Prime Numbers Under 100}
2, 3, 5, 7, 11, 13, 17, 19, 23, 29, 31, 37, 41, 43, 47, 53, 59, 61, 67, 71, 73, 79, 83, 89. 97

\section*{Geometry}
\begin{enumerate}
    \item Sum of exterior angles of a regular polygon is 360$\degree$.
    \item Sum of interior angles of a regular polygon = 180$\degree(n-2)$
    \item Measure of exterior angle = $\frac{360\degree}{n}$
    \item Measure of interior angle = $\frac{180\degree \cdot (n-2)}{n}$
    \item Area of a regular polygon 
    \begin{itemize}
        \item Given side: $\frac{ns^2}{4\tan(180\degree/n)}$
        \item Given apothem: $na^2\tan(180\degree/n)$
        \item Given radius: $\frac{nr^2\sin(360\degree/n)}{2}$
    \end{itemize}
    \item Square 
    \begin{itemize}
        \item Area = side$^2$ = $\frac{(\text{diagonal})^2}{2}$
        \item Perimeter = $4s$
        \item Length of diagonal = $s\sqrt{2}$
    \end{itemize}
    \item Triangle 
    \begin{itemize}
        \item Area = $\frac{1}{2}bh$
    \end{itemize}
    \item Equilaterial Triangle 
    \begin{itemize}
        \item Area = $\frac{s^2\sqrt{3}}{4}=\frac{h^2\sqrt{3}}{3}$
        \item Perimeter = $3s$
    \end{itemize}
    \item Rectangle 
    \begin{itemize}
        \item Area = lw 
        \item Perimeter = 2(l+w)
    \end{itemize}
    \item Parallelogram 
    \begin{itemize}
        \item Area = bh 
    \end{itemize}
    \item Trapezoid 
    \begin{itemize}
        \item Area = $\frac{\text{height}(\text{base}_1+\text{base}_2)}{2}$
    \end{itemize}
    \item Rhombus 
    \begin{itemize}
        \item Area = $\frac{\text{(diagonal)}^2}{2}$
    \end{itemize}
    \item Circle 
    \begin{itemize}
        \item Area = $\pi r^2$
        \item Circumference = $2\pi r$ = $\pi d$
    \end{itemize}
    \item Rectangular solid 
    \begin{itemize}
        \item Surface Area = $2(lw+lh+wh)$
        \item Inner diagonal = $\sqrt{\text{length}^2+\text{width}^2+\text{height}^2}$
        \item Volume = $lwh$
    \end{itemize}
    \item Cube 
    \begin{itemize}
        \item Total surface area = 6$e^2$
        \item Volume = $e^3$
        \item Inner diagonal = $e\sqrt{3}$
    \end{itemize}
    \item Sphere 
    \begin{itemize}
        \item Surface area = $4\pi r^2$
        \item Volume = $\frac{4}{3}\pi r^3$
    \end{itemize}
    \item Right Circular Cylinder 
    \begin{itemize}
        \item Lateral Area = $2\pi rh$
        \item Total surface area = $2\pi r^2+2\pi rh$
        \item Volume = $\pi r^2 h$
    \end{itemize}
    \item Right Circular Cone 
    \begin{itemize}
        \item Lateral Area = $\pi r l$ ($l$ is the slant height)
        \item Total Surface Area = $\pi r l +\pi r$
        \item Volume = $\frac{1}{3}\pi r^2$
    \end{itemize}
\end{enumerate}

\section*{More Advanced Formulas}
\begin{enumerate}
    \item Compound Interest: $A = P\left(1+\frac{r}{n}\right)^{nt}$
    \item Compound interest continuously: $A = Pe^{rt}$
    \item Laws of Sines: $\frac{\sin (A)}{a} = \frac{\sin(B)}{b} = \frac{\sin(C)}{c}$
    \item Laws of Cosines: $c^2=a^2+b^2-2ab \sin(c)$
    \item Heron's Formula: Area = $\sqrt{s(s-a)(s-b)(s-c)}$, where $A$ is the area of a triangle with sides $a$, $b$, and $c$; $s$ = semi-perimeter = $\frac{a+b+c}{2}$
    \item Radius (r) of circle inscribed in a triangle with sides $a$, $b$, and $c$:
    
    $r=\frac{\sqrt{s(s-a)(s-b)(s-c)}}{s}$

    \item The area of a triangle given the length of two sides ($a$ and $b$) and an included angle, $C$.
    
    $A = \frac{1}{2}ab\sin(C)$

    \item Area of a sector of a circle given the radius, $r$, of the circle and the measure of the intercepted arc in radians, $\theta = \frac{1}{2}r^2\theta$.
    \item Area of a segment of a circle given the radius, $r$, of the circle and the intercepted arc in radius, $\theta = \frac{1}{2}r^2(\theta - \sin\theta)$
\end{enumerate}

\section*{Random Formulas}
\begin{enumerate}
    \item Arithmetic mean of $a$ and $b = \frac{a+b}{2}$
    
    Note: Arithmetic Mean = $\frac{a_1+a_2+a_3+\dots + a_n}{n}$

    \item Geometric mean of $a$ and $b$ = $\sqrt{ab}$
    
    Note: Geometric Mean = $\sqrt[n]{a_1a_2a_3\dots a_n}$

    \item Harmonic mean of $a$ and $b$ = $\frac{\text{Geometric Mean}^2}{\text{Arithmetic Mean}}=\frac{2ab}{a+b}$
    
    Note: Harmonic mean of 3 terms = $\frac{3a_1a_2a_3}{a_1a_2+a_1a_3+a_2a_3}$

    \item Mode: number that appears the most 
    \item Range: difference of smallest and largest number given 
    \item If $ax^2+bx+c=0$, then 
    \begin{itemize}
        \item $x=\frac{-b\pm \sqrt{b^2-4ac}}{2a}$
        \item Discriminant = $b^2-4ac$
        \item Sum of the roots = $-\frac{b}{a}$
        \item Product of the roots = $\frac{c}{a}$
    \end{itemize}

    \item If $ax^3+bx^2+cx+d=0$, then 
    \begin{itemize}
        \item Sum of the roots = $-\frac{b}{a}$
        \item Product of the roots = $-\frac{d}{a}$
        \item Sum of the product of the roots taken two at a time = $\frac{c}{a}$
    \end{itemize}

    \item $(a+b)^2 = a^2+2ab+b^2$; $(a-b)^2=a^2-2ab+b^2$
    \item $(a+b)^3=a^3+3a^2b+3ab^2+b^3$
    \item $(a-b)^3=a^3-3a^2b+3ab^2-b^3$
    \item Given points $(x_1,y_1)$ and $(x_2,y_2)$
    \begin{itemize}
        \item Slope = $m$ = $\frac{y_2-y_1}{x_2-x_1}$
        \item Midpoint: $\left(\frac{x_1+x_2}{2}, \frac{y_1+y_2}{2}\right)$
        \item Distance = $d$ = $\sqrt{(x_2-x_1)^2+(y_2-y_1)^2}$
    \end{itemize}

    \item Probability = $\frac{\text{Favorable}}{\text{Total Outcomes}}$; Odds = $\frac{\text{Favorable}}{\text{Unfavorable}}$
    \item If Tom can do a job in $A$ hours and Jane can do the same job in $B$ hours, how long will it take them to do the job together?
    \[ \frac{\text{Both}}{A(\text{alone})}+\frac{\text{Both}}{B\text{alone}}=1\]

    \item The sum of the coefficients of $(Ax+By)^n = (A+B)^n$
    \item A tangent and a secant intersect in a point in the exterior of a circle.
    \[ \frac{\text{External Segment}}{\text{Tangent}}=\frac{\text{Tangent}}{\text{Secant}}\]

    \item Orthocenter: The point where the altitudes of a triangle intersect 
    \item Centroid: The point where the medians of a triangle intersect 
    \item Circumcenter: The point where the perpendicular bisectors of the sides of a triangle intersect 
    \item Incenter: The point where the angle bisectors of a triangle intersect 
    \item Supplementary angles: Two angles the sum of whose measures is $180\degree$
    \item Complementary angles: two angles the sum of whose measures is $90\degree$
    \item Arithmetic sequence 
    \begin{itemize}
        \item $t_n=a+(n-1)d$
    \end{itemize}
    \item Arithmetic series 
    \begin{itemize}
        \item $S_n = \frac{n}{2}(a+t_n)=\frac{n}{2}[2a+(n-1)d]$
    \end{itemize}
    \item Geometric Sequence 
    \begin{itemize}
        \item $t_n=ar^{n-1}$
    \end{itemize}
    \item Geometric Series 
    \begin{itemize}
        \item $S_n = \frac{a(1-r^{n+1})}{1-r}$
    \end{itemize}
    \item Infinite Geometric Series 
    \begin{itemize}
        \item $S=\frac{a}{1-r}$
    \end{itemize}
    \item Circle: $(x-h)^2+(y-k)^2=r^2$
    \begin{itemize}
        \item Center: $(h,k)$
        \item radius = $r$
    \end{itemize}
    \item Parabola: $(x-h)^2=4p(y-k)$
    \begin{itemize}
        \item Length of latus rectum = $|4p|$
        \item Vertex: $(h,k)$
    \end{itemize}
    \item $\sin 2A = 2\sin A\cos A$
    \item $\sin^2 A + \cos^2 A = 1$
    \item $\tan A = \frac{\sin A}{\cos A}$; $\cot A = \frac{\cos A}{\sin A}$; $\sec A = \frac{1}{\cos A}$; $\csc A = \frac{1}{\sin A}$
    \item Angle of Inclination, $B$, of line with slope $m$.
    \begin{itemize}
        \item $\tan B = m$
    \end{itemize}
    \item The distance between a point $(x_1,y_1)$ and a line, $Ax+By+C=0$
    \begin{itemize}
        \item $d = \frac{|Ax_1+By_1+c|}{\sqrt{A^2+B^2}}$
    \end{itemize}
    \item In a $45\degree$-$45\degree$-$90\degree$ triangle 
    \begin{itemize}
        \item hypotenuse = leg$\sqrt{2}$
    \end{itemize}
    \item In a $30\degree$-$60\degree$-$90\degree$ triangle 
    \begin{itemize}
        \item hypotenuse = 2(short leg)
        \item long leg = (short leg)$\sqrt{3}$
    \end{itemize}
    \item Lucas numbers: $1,3,4,7,11,18,29,47,76,123,199,322,521,\dots$.
    \begin{itemize}
        \item $L_n = \left[\left(\frac{1+\sqrt{5}}{2}\right)^2\right]$; Note: The ``nint'' function is the ``nearest integer function'' and is denoted by 
        $[x]$ which means the nearest integer to the number $x$. The Golden Mean is equal to $\frac{1+\sqrt{5}}{2}$ which is approximately equal to 1.618.
    \end{itemize}

    \item Fibonacci numbers: $1,1,2,3,5,8,13,21,34,55,89,144,233,377,\dots$
    \begin{itemize}
        \item $F_n = \left[\frac{\left(\frac{1+\sqrt{5}}{2}\right)^n}{\sqrt{5}}\right]$ Note: The ``nint function'' is the ``nearest integer function'' and is denoted by $[x]$
        which means the nearest integer to the number $x$.
    \end{itemize}
    \item Deficient number: If the sum of the positive integral divisors of a number is less than twice the number, the number is a deficient number.
    \item Perfect number: If the sum of the positive integral divisors of a number is equal to twice the number, the number is a perfect number 
    \item Abundant number: If the sum of the positive integral divisors of a number is greater than twice the number, the number is an abundant number 
    \item Factorials
    \begin{itemize}
        \item $0!=0$
        \item $1!=1$
        \item $2!=2$
        \item $3!=6$
        \item $4!=24$
        \item $5!=120$
        \item $6!=720$
    \end{itemize}
    \item Permuations 
    \begin{itemize}
        \item Permutation of $n$ things taken $r$ at a time = $\frac{n!}{(n-r)!}$
    \end{itemize}
    \item Combinations 
    \begin{itemize}
        \item Combination of $n$ things taken $r$ at a time = $\frac{n!}{r!(n-r)!}$
    \end{itemize}
    \item Laws of Exponents
    \begin{itemize}
        \item $(a^m)(a^n)=a^{m+n}$
        \item $\frac{a^m}{a^n}=a^{m-n}$
        \item $(a^m)^n=a^{mn}$
    \end{itemize}
    \item Laws of Logarithms 
    \begin{itemize}
        \item $\log_b M + \log_b N = \log_b MN$
        \item $\log_b M - \log_b N = \log_b \frac{M}{N}$
        \item $\log_b M^p = p\log_b M$
        \item Change of base: $\log_b M = \frac{\log M}{\log b}$
    \end{itemize}
    \item $\lim_{x\to \infty} \frac{ax^n+bx^{n-1}+\dots}{cx^n+dx^{n-1}+\dots}=\frac{a}{c}$
    \item Interest = Principal $\times$ Rate $\times$ Time 
    \item Distance = Rate $\times$ Time 
    \item Cevian: A line segment joining the vertex of a triangle to any point on the opposite side 
    \item General equation of a conic section is $Ax^2+Bxy+Cy^2+Dx+Ey+F=0$
    \begin{itemize}
        \item If $\theta$ is the angle of rotation, then $\cot 2\theta = \frac{A-C}{B}$
        \item If $B^2-4AC<0$, the conic is either an ellipse or a circle .
        \item If $B^2-4AC=0$, the conic is a parabola.
        \item If $B^2-4AC>0$, the conic is a hyperola.
    \end{itemize}
    \item $r_1(\cos\theta_1 + i\sin\theta_1)\cdot r_2(\cos\theta_2+i\sin\theta_2)=r_1r_2[\cos(\theta_1+\theta_2)+i\sin(\theta_1+\theta_2)]$
    \item $\frac{r_1(\cos\theta_1+i\sin\theta_1)}{r_2(\cos\theta_2+i\sin\theta_2)}=\frac{r_1}{r_2}[\cos(\theta_1-\theta_2)+i\sin(\theta_1-\theta_2)]$
    \item $(r\cis \theta)^n = r^n\cis n\theta$
\end{enumerate}
\end{document}