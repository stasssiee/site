\documentclass[../abcalc.tex]{subfiles}
\graphicspath{{\subfix{../figures/}}}
\begin{document}
\chapter{Applications of Integration}
\section{Average Value of a Function}
\begin{theorem}
    If $f$ is continuous on $[a,b]$, then there exists some number $c$ in the open interval $(a,b)$ such that 
    \[f(c)=\frac{1}{b-a}\int^b_a f(x)\mathrm{d}x\]
\end{theorem}

Problem 1: Given the function $r(x)=2\sin x-1$ where $r$ is the rate at which Mr. Brust's waistline is changing (inches per month) and $x$ is time (months):

(a) What is the average rate that Mr. Brust's waistline changes from the 10th month to the 12th month?

(b) What is the average change of this rate during the first 5 months?
\section{Net Change}
The integral of the rate of change is the next change.

\begin{itemize}
    \item $\int v(t)\mathrm{d}t$ is the displacement.
    \item $\int |v(t)|\mathrm{d}t$ is the total distance.
\end{itemize}

Problem 1: A particle's velocity is given by $v(t)=t^3-2t^2+1$. If $x(t)$ represents the position of the particle along the $x$-axis, find the following:

(a) The position of the particle after 3 seconds if $x(0)=5$.

(b) The position of the particle after 2 seconds if $x(1)=-2$.

Problem 2: If $H(-1)=12$ and $H'(t)=cos(\pi t)$, what is $H\left(\frac{3}{2}\right)$?
\section{Area between Two Curves}
Steps to find the area between curves:
\begin{itemize}
    \item Sketch the graph.
    \item Determine if the graph is perpendicular to x or y.
    \item Set up the integral.
    \item Solve the integral.
\end{itemize}
If it is perpendicular to the x axis the formula is $A = \int_{x_1}^{x_2}[\text{(Top Curve)}-(\text{Bottom Curve})]\mathrm{d}x$

If it is perpendicular to the y-axis the formula is $A = \int_{x_1}^{x_2}[\text{(Right Curve)}-(\text{Left Curve})]\mathrm{d}y$

Problem 1: Find the area enclosed by the curves $y=x^2+3$ and $y=4x^2$.

Problem 2: Find the area enclosed by the curves $y=|x|$ and $y=x^2-2$.
\section{Volume - Disk \& Washer}
If a region in a plane is revolved about a line, the resulting figure is a solid of revolution, and the line is called the axis of revolution.

Horizontal Axis of Revolution: $V=\pi \int_a^b [R(x)]^2\mathrm{d}x$

Vertical Axis of Revolution: $V= \pi\int_c^d [R(y)]^2 \mathrm{d}y$

For the washer method if there is a gap in the shape:

Horizontally: $V = \pi\int_a^b \left([R(x)]^2-[r(x)]^2\right)\mathrm{d}x$

Vertically: $V = \pi\int_c^d \left([R(y)]^2-[r(y)]^2\right)\mathrm{d}y$

Problem 1: Set up the integral to find the volume of the solid generated by revolving the region bounded by the graphs of the equations about the line $x=6$. $y=x \quad y=0 \quad y=4 \quad x=6$

Problem 2: Set up the integral to find the volume of the solid generated by revolving the region bounded by the graphs of the equations about the $x$-axis. $y=\sin x \quad y=0 \quad x=0 \quad x=\pi$
\section{Volume - Cross Sections}
\begin{itemize}
    \item Volume for a rectangle: $\int_a^b \frac{(T-B)}{b}h \mathrm{d}x$.
    \item Volume for a square: $\int_a^b \frac{(T-B)^2}{s}\mathrm{d}x$.
    \item Volume for a semi-circle: $\frac{1}{8}\pi \int_a^b (T-B)^2\mathrm{d}x$.
    \item Volume for a isosceles triangle: $\frac{1}{2}\int_a^b (T-B)^2\mathrm{d}x$.
    \item Volume for an equilaterial triangle: $\frac{\sqrt{3}}{4}\int_a^b (T-B)^2\mathrm{d}x$.
\end{itemize}

Problem 1: Set up an integral to find the volume of the solid whose base is bounded by the graphs of $y=x^3$ and $y=0$ and $x=1$ with the 
indicated cross sections taken perpendicular to the $y$-axis.

(a) Squares

(b) Rectangles of height 1

(c) Semicircles
\end{document}