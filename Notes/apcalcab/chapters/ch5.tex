\documentclass[../abcalc.tex]{subfiles}
\graphicspath{{\subfix{../figures/}}}
\begin{document}
\chapter{Analytical Applications of Differentiation}
\section{Extrema on an Interval}
\begin{definition}
    Let $f$ be defined on an interval $I$ containing $c$.
    \begin{enumerate}
        \item $f(c)$ is the minimum of $f$ on $I$ if $f(c)\leq f(x)$ for all $x$ in $I$.
        \item $f(c)$ is the maximum of $f$ on $I$ if $f(c)\geq f(x)$ for all $x$ in $I$.
    \end{enumerate}
    The minimum and maximum of a function on an interval are the extreme values or extrema of the function on the interval. The minimum 
    and maximum of a function on an interval are also caleld the absolute minimum and absolute maximum, or the global minimum and global maximum, on the interval.
\end{definition}
\begin{definition}
    \begin{enumerate}
        \item If there is an open interval containing $c$ on which $f(c)$ is a maximum, then $(c,f(c))$ is called a relative maximum of $f$, 
        or you can say that $f$ has a relative maximum at $(c,f(c))$.
        \item If there is an open interval containing $c$ on which $f(c)$ is a minimum, then $(c,f(c))$ is called a relative minimum of $f$, 
        or you can say that $f$ has a relative minimum at $(c,f(c))$.
    \end{enumerate}
    The relative maximum and relative minimum points are sometimes called local maximum and local minimum points, respectively.
\end{definition}

\begin{definition}
    Let $f$ be defined at $c$. If $f'(c)=0$ or if $f$ is not differentiable at $c$, then $c$ is a critical number of $f$ and the point $(c,f(c))$ is a critical point of $f$.
\end{definition}

Relative extrema occur only at critical numbers.

\begin{theorem}
    If $f$ is continuous on a closed interval $[a,b]$, then $f$ attains an absolute maximum value $f(c)$ or an absolute minimum value $f(d)$ 
    for some numbers $c$ and $d$ in $[a,b]$.
\end{theorem}

To find the extrema of a continuous function $f$ on a closed interval $[a,b]$ use the following steps:
\begin{enumerate}
    \item Find $f'(x)$ and the critical numbers of $f$ in $[a,b]$.
    \item Evaluate $f$ at each critical number in $(a,b)$.
    \item Evaluate $f$ at each endpoint in $[a,b]$.
    \item The least of these values is the minimum. The greatest is the maximum.
\end{enumerate}

Problem 1: Find the absolute maximums and minimums of $f$ on the given closed interval, and state where these values occur.

$f(x) = \sin^2 x + \cos x \quad [0,2\pi]$.
\section{1st Derivative Test}
\begin{itemize}
    \item When the function is increasing, the derivative is positive.
    \item When the function is decreasing, the derivative is negative.
    \item When the function changes from increasing to decreasing (or vice versa), the derivative is zero or undefined.
\end{itemize}

The first derivative test:
\begin{enumerate}
    \item Find the critical points.
    \item Draw a number line with those critical points.
    \item Identify the intervals to consider.
    \item Choose a test value in each interval.
    \item Plug the test value into the derivative to find the sign.
    \item Make conclusions about the function and relative extrema.
\end{enumerate}

When a function changes concavity that is called a point of inflection. A point of inflection occurs whenever the second derivative is zero or undefined.

Problem 1: Find the interval(s) where the function $f(x)=\frac{1}{2}x-\sin x$ is increasing and decreasing and the interval(s) where the function is concave up and down and has point(s) of inflection on the interval $[0,2\pi]$.
\section{2nd Derivative Test}
\begin{itemize}
    \item If the function has horizontal tangent ($f'=0$) and is concave up $(f''>0)$, then the function has a relative minimum at that $x$-value.
    \item If the function has horizontal tangent ($f'=0$) and is concave down $(f''<0)$, then the function has a relative maximum at that $x$-value.
\end{itemize}

The second derivative test:
\begin{enumerate}
    \item Find the possible point(s) of inflection.
    \item Draw a number line with those point(s) of inflection.
    \item Identify the intervals to consider.
    \item Choose a test value in each interval.
    \item Plug the test value into the second derivative to find the sign.
    \item Make conclusions about the function and point(s) of inflection.
\end{enumerate}

Problem 1: Find the interval in which $f'(x)=x \ln x$ is concave down given that the domain of this function $f$ is $x>0$.
\section{Mean Value Theorem and Rolle's Theorem}
\begin{theorem}
    Rolle's Theorem: Let $f$ be continuous on the closed interval $[a,b]$ and differentiable in the open interval $(a,b)$.
    If $f(a) = f(b)$, then there is at least one number in $c$ in $(a,b)$ such that $f'(c)=0$.
\end{theorem}
\begin{theorem}
    Mean Value Theorem: If $f$ is continuous on the closed interval $[a,b]$ and differentiable on the open interval $(a,b)$, then there exists 
    a number $c$ in $(a,b)$ such that 
    \[f'(c)=\frac{f(b)-f(a)}{b-a}\]
\end{theorem}

Problem 1: Let $f$ be a twice-differentiable function such that $f(2)=5$ and $f(5)=2$. Let $g$ be the function given by $g(x)=f(f(x))$.
\begin{itemize}
    \item Explain why there must be a value $c$ for $2<c<5$ such that $g'(c)=1$.
    \item Show that $g'(2)=g'(5)$. Use this result to explain why there must be a value $k$ for $2<k<5$ such that $g''(k)=0$.
\end{itemize}
\section{Optimization}
Optimization is the process of finding the ``optimal'' value of a quantity. Optimal values are often either the maximum or minimum values of a certain functino.
\begin{enumerate}
    \item Determine what you are trying to maximize of minimize.
    \item Determine what your answer must look like.
    \item Draw a visual if possible.
    \item Write the equation you must maximize or minimize.
    \item Use information from the problem to get the equation from \#4 in terms of one variable.
    \item Take the derivative of the new equation from \#5.
    \item Set the derivative equal to zero to identify critical values.
    \item Choose the critical value that is also the particular extreme value you need from \#1 and justify your choice.
    \item Answer the question identified in \#2.
\end{enumerate}
Problem 1: What point on the graph $y=\sqrt{x}$ is closest to $(5,0)$?

Problem 2: A manufacturer wants to design an open box having a square base and a surface area of 108 square inches. What dimensions 
will produce a box with maximum volume?

Problem 3: A power station is on onside of a river that is 1/2 mile wide, and a factory is 6 miles downstream on the other side. It costs 
\$60,000 per mile to run power lines over land and \$85,000 per mile to run them underwater. FInd the most economical path for the transmission 
line from the power station to the factory.
\section{Sketching Graphs and Their Derivatives}
Problem 1: Sketch the graph of a function $f$ with the given characteristics
\begin{itemize}
    \item $f(2)=f(4)=0$
    \item $f'(x)>0$ if $x<3$
    \item $f'(3)$ does not exist
    \item $f'(x)<0$ if $x>3$
    \item $f''(x)>0, x\neq 3$
\end{itemize}
\end{document}