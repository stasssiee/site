\documentclass[../abcalc.tex]{subfiles}
\graphicspath{{\subfix{../figures/}}}
\begin{document}
\chapter{Contextual Applications of Differentiation}
\section{Related Rates}
Questions that ask for the calculation of the rate at which one variable changes, based on the rate at which another variable is known to change,
are usually called related rates. Solutions are found by writing an equation that relates the variables of the problem, then differentiating them 
with respect to another variable. Since time is rarely a variable in the equation you write, you will have to differentiate implicitly with respect to time.

Process:
\begin{enumerate}
    \item Draw a picture.
    \item Make a list of known and unknown rates and quantities. Translate the given information in the problem into ``calculus-speak''.
    \item Write a formula or equation relating to the variables from step \#2.
    \item Differentiate implicitly with respect to time.
    \item Now you can plug in numbers and do calculations. 
    \item Translate from ``calculus-speak'' back to English and answer the question that is being asked.
\end{enumerate}

Formulas:
\begin{itemize}
    \item Perimeter of a rectangle: $P=2l+2w$.
    \item Circumference of a circle: $C=2\pi r$.
    \item Area of a rectangle: $A=lw$ or $A=bh$.
    \item Area of a circle: $A=\pi r^2$.
    \item Area of a triangle: $A=\frac{1}{2}bh$.
    \item Pythagorean Theorem: $a^2+b^2=c^2$.
    \item Volume of a cylinder: $V=\pi r^2h$.
    \item Volume of a cone: $V=\frac{1}{3}\pi r^2 h$.
    \item Volume of a sphere: $V=\frac{4}{3}\pi r^3$.
\end{itemize}

Problem 1: A cube has an edge of 40 feet at $t = 0$, and the edge is decreasing at a constant rate of 4 feet per minute. After 2 minutes,
what would the rate of change of the volume in cubic feet per minute be?

Problem 2: A camera is filming the progress as a daredevil attempts to scale the wall of a skyscraper. The climber is moving vertically 
at a constant rate of 16 feet per minute, and the camera if 400 feet from the base of the skyscraper. Through how many radians per minute 
is the camera angle changing when the climber is 300 feet up the building?

Problem 3: Water is poured into a conical tank that is 24 feet tall and has a diameter at the top of 20 feet. The radius of the surface of the water in the tank 
is increasing at 0.75 feet per minute. At what rate is the area of the surface changing when the radius is 4.2 feet?
\section{Particle Motion - Position, Velocity, \& Acceleration}
If $x(t)$ represents the position of a particle along the $x$-axis at any time $t$, then the following statements are true.
\begin{enumerate}
    \item ``Initially'' means when $t=0$.
    \item ``At the origin'' means $x(t)=0$.
    \item ``At rest'' means $v(t)=0$.
    \item If the velocity of the particle is positive, then the particle is moving to the right.
    \item If the velocity of the particle is negative, then the particle is moving to the left.
    \item To find the average velocity over a time interval, divide the change in velocity by the change in time.
    \item Instantaneous velocity is the velocity at a single moment in time.
    \item If the acceleration of the particle is positive, then the velocity is increasing.
    \item If the acceleration of the particle is negative, then the velocity is decreasing.
    \item In order for a particle to change direction, the velocity must change sigs.
\end{enumerate}
Problem 1: Draw a position, velocity and acceleration graph for the following scenario: Start at the 10 yard mark of the measuring tape.
Walk forward quickly for 8 seconds. Then, turn around and walk slowly back toward the beginning of the tape for 8 seconds. Turn around again 
and walk away at a medium rate from the beginning of the tape until time runs out.
\end{document}