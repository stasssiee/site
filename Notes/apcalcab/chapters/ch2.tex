\documentclass[../abcalc.tex]{subfiles}
\graphicspath{{\subfix{../figures/}}}
\begin{document}
\chapter{Differentiation: Definition and Fundamental Properties}
\section{Average Rate of Change and Secant Lines}
The average rate of change is known as the secant line or slope. Represented as a formula this can be written as 
\[\frac{\Delta y}{\Delta x} = \frac{f(x_2)-f(x_1)}{x_2-x_1} = \frac{f(b)-f(a)}{b-a}\]

When a function is linear, we often refer to the average rate of change as simply the rate of change. 

A tangent line is the instantaneous rate of change.

Problem 1: Suppose an object's position at time $t$ is described by $s(t)=t^2-5t+1$. What is the object's average velocity between time 0 and 3 seconds later?

Problem 2: A rock is thrown straight up, with an initial velocity of 20 meters per second and from an initial height of 2 meters. The height 
$h$ of the rock after $t$ seconds by the equation: $h(t)=2+20t-4.9t^2$. What is the rock's average velocity during the first two seconds of its flight?
\section{Definition of Derivative}
When finding the average rate of change we need two points.

The formula to find the instantaneous rate of change is
\[\lim_{h\to 0}\frac{f(x+h)-f(x)}{h}\]

This is called the difference quotient. A derivative is a common word used to mean instantaneous rate of change. 

Symbolically we can write:
\begin{itemize}
    \item The derivative of $f(x)$ is $f'(x)$.
    \item The derivative of $y$ is $y'$.
    \item The derivative of $y$ is $\frac{dy}{dx}$.
\end{itemize}

Given a specific $x$-value where $x=a$, then 
\[\lim_{x\to a}\frac{f(x)-f(a)}{x-a}\]

Problem 1: Find the instantaneous rate of change of the function $f(x)=\frac{1}{2}x^2-1$ on $[x,x+h]$.

Problem 2: What function is the definition of the derivative being applied to in 
\[\lim_{h\to 0} \frac{\sqrt{2(x+h)}-\sqrt{2x}}{h}\] 

Problem 3: Use the limit defintiion to find the derivative of the function $f(x)=-x^2+2x-3$.
\section{Derivative Rules}
The definition of the derivative explains why the derivative represents an instantaneous slope. However, there are some quick and easy rules that make finding derivatives much less time consuming.

Basic Derivative Rules:
\begin{itemize}
    \item Constant Rule: $\frac{d}{dx}(c) = 0$
    \item Constant Multiple Rule: $\frac{d}{dx}[cf(x)] = cf'(x)$
    \item Power Rule: $\frac{d}{dx}(x^n)=nx^{n-1}$
    \item Sum Rule: $\frac{d}{dx}[f(x)+g(x)]=f'(x)+g'(x)$
    \item Difference Rule: $\frac{d}{dx}[f(x)-g(x)]=f'(x)-g'(x)$
\end{itemize}

Extra Derivatives:
\begin{itemize}
    \item $\frac{d}{dx}(\sin x)=\cos x$
    \item $\frac{d}{dx}(\cos x)=-\sin x$
    \item $\frac{d}{dx}(\tan x)=\sec^2 x$
    \item $\frac{d}{dx}(\csc x)=-\csc x\cot x$
    \item $\frac{d}{dx}(\sec x)=\sec x\tan x$
    \item $\frac{d}{dx}(\cot x)=-\csc^2 x$
    \item $\frac{d}{dx}(e^x)=e^x$
    \item $\frac{d}{dx}(\ln x)=\frac{1}{x}$
\end{itemize}

Problem 1: Differentiate $f(x)=\sqrt[3]{x}$.

Problem 2: If $f(x)=5x^3$, then $f'(2)=$?

Problem 3: A projectile starts at time $t=0$ and moves along the $x$-axis so that its velocity at any time $t\geq 0$ is $v(t)=t^3-6\csc t+e^t$.
What is the acceleration of the particle at $t=1$?
\section{Product Rule}
The product rule states 
\[\frac{d}{dx}(f(x)g(x))=f(x)g'(x)+f'(x)g(x)\]

Problem 1: Differentiate $g(z)=\sqrt[3]{z^2}\sin z$.

Problem 2: For what values does $f(t)=-e^t(2t+1)$ have a horizontal tangent? 

Problem 3: Let $s(t)=\frac{1}{\pi}+3\sin t$ represent the position of an object moving on a line. At what time(s) is the object at rest?
\section{Quotient Rule}
The quotient rule states 
\[\frac{d}{dx}\left(\frac{f(x)}{g(x)}\right) = \frac{g(x)f'(x)-f(x)g'(x)}{(g(x))^2}\]

Problem 1: Differentiate $\frac{3(1-\sin z)}{e^z}$.

Problem 2: For what value(s) of $x$ does the function $f(x)=\frac{x^2}{x+1}$ have a horizontal tangent?

Problem 3: Let $s(t)=\frac{1}{\pi}+3\sin t$ represent the position of an object moving on a line. What is the velocity of the object 
when the acceleration is 3 on $[0, 2\pi]$?
\section{Tangent Lines}
Problem 1: Let $f$ be the function defined by $f(x)=4x^3-5x+3$. Write the equation of the line tangent to the graph of $f$ at the point where $x=-1$.

Problem 2: Write the equation of the line tangent to the graph of $y=\frac{3x-2}{2x-3}$ at the point $(-1,1)$.
\section{Linear Approximation}
Problem 1: Use the differential equation $\frac{dy}{dx}=\frac{3x^2+1}{2y}$ to write an equation for the line tangent to the graph of $f$
at $f(1)=-1$ and use it to approximate $f(1.2)$.
\section{Differentiability \& Continuity}
Now that you know about continuity and differentiability, let's expand the idea of differentiability and continuity. A function to be differentiable requires that the function cannot have a sharp turn or a vertical tangent.
A function must be continuous to be differentiable. 

Problem 1: Let $f$ be the function defined, where $c$ and $d$ are constants. If $f$ is differentiable at $x=2$, what is the value of $c+d$? 
\( f(x)=
\begin{cases}
    cx+d& x\leq 2\\
    x^2-cx& x>2
\end{cases}\)
\end{document}