\documentclass[../abcalc.tex]{subfiles}
\graphicspath{{\subfix{../figures/}}}
\begin{document}
\chapter{Integration and Accumulation of Change}
\section{Left and Right Riemann Sums}
Problem 1: Let $f(x)=x^2+x$. Consider the region bounded by the graph of $f$, the $x$-axis and the line $x=2$. Divide the interval $[0,2]$ 
into 4 equal subintervals.
\begin{itemize}
    \item Obtain an estimate for the region using a right Riemann sum. Is this an over or underestimate. Why?
    \item Obtain an estimate for the region using a left Riemann sum. Is this an over or underestimate. Why?
\end{itemize}
\section{Midpoint Sums}
Problem 1: Given the function $f(x)=x^2+1$, estimate the area bounded by the graph of the curve and the $x$-axis on $[0,2]$ by using a 
midpoint Riemann sum with $n=2$ equal subintervals.
\section{Trapezoidal Sums}
You jump out of an airplane. Before your parachute opens, you fall faster and faster, but your acceleration decreases as you fall because 
of air resistance. 
The following data gives you acceleration in m/sec$^2$ after $t$ seconds.
\begin{itemize}
    \item Time: 0, Acceleration: 9.81
    \item Time: 2, Acceleration: 8.03
    \item Time: 4, Acceleration: 6.53
    \item Time: 6, Acceleration: 5.38
    \item Time: 8, Acceleration: 4.41
    \item Time: 10, Acceleration: 3.61
\end{itemize}
Use the trapezoid method with 5 subintervals of equal length to estimate your speed after ten seconds. Is this an underestimate of overestimate? Why?
\section{Definite Integrals}
The notation of a definite integral is $\int_a^b f(x)\mathrm{d}x$.

The defintion of a definite integral is the exact area between a curve $f(x)$ and the $x$-axis on an interval $(a,b)$.

Properties of definite integrals:
\begin{itemize}
    \item Zero Integral: $\int_a^a f(x)\mathrm{d}x = 0$
    \item Negation: $\int_b^a f(x)\mathrm{d}x = -\int_a^b f(x)\mathrm{d}x$
    \item Multiply by a constant: $\int_a^b kf(x)\mathrm{d}x = k\int_a^b f(x)\mathrm{d}x$
    \item Decomposition: $\int_a^c f(x)\mathrm{d}x = \int_a^b f(x)\mathrm{d}x + \int_b^c f(x)\mathrm{d}x$
    \item Addition and Subtraction: $\int_a^b f(x)\pm g(x)=\int_a^b f(x)\pm \int_a^b g(x)$
\end{itemize}

Problem 1: Given $\int_{-2}^1 f(x)\mathrm{d}x = 4 \quad \int_1^5 f(x)\mathrm{d}x=-3 \quad \int_{-2}^1 g(x)\mathrm{d}x=8$, find $\int_{-2}^1 [f(x)+2g(x)]\mathrm{d}x$.
\section{Fundamental Theorem of Calculus}
The basic rule of integration is $\int x^n\mathrm{d}x=\frac{x^{n+1}}{n+1}+C$.

Indefinite integral: $\int f'(x)\mathrm{d}x = f(x)+C$

\begin{theorem}
    If a function $f'$ is continuous on the closed interval $[a,b]$ and $f$ is an antiderivative of $f'$ on the interval $[a,b]$, then 
    \[\int^a_b f'(x)\mathrm{d}x = f(x)|^b_a = f(b)-f(a)\]
\end{theorem}

Problem 1: Evaluate the definite integral $\int_0^3 |x-2|\mathrm{d}x$.
\section{Antiderivatives and Specific Solutions}
Memorize the following integrals:
\begin{itemize}
    \item $\int \frac{1}{x}\mathrm{d}x = |\ln x| + C$
    \item $\int e^x \mathrm{d}x = e^x + C$
    \item $\int \cos x \mathrm{d}x = \sin x + C$
    \item $\int \sin x \mathrm{d}x = -\cos x + C$
\end{itemize}

Problem 1: Find $y$, if $y'' = \cos x$ and $y'\left(\frac{\pi}{2}\right) = 2$ and $y\left(\frac{\pi}{2}\right)=3\pi$.

Problem 2: A particle moves along the $x$-axis with an acceleration of $a(t)=12t-4$. The particle's velocity is 18 centimeters per second at 
$t=2$. The initial position of the particle is 8 cm. What is the position of the particle at $t=3$?

Problem 3: Evaluate $\int \frac{1-\sin^2 x}{\cos x}\mathrm{d}x$.
\section{2nd Fundamental Theorem of Calculus}
\begin{theorem}
    If $f(x)=\int_a^b f'(t)\mathrm{d}t$, $f$ is a continuous function, and $g$, $h$ are differentiable functions, then 
    \[f'(x)=f(b)\cdot b' - f(a)\cdot a'\]
    Break down: 
    \begin{itemize}
        \item Plug in the top bound and multiply by the derivative of the top bound 
        \item Plug in the bottom bound and multiply by the derivative of the bottom bound
        \item Derivative of an Integral = \#1 - \#2 
    \end{itemize}

    Use this theorem if you have an integral with a variable on at least one bound or are taking the derivative of an integral.
\end{theorem}

Problem 1: Find $f'(x): f(x)=\int_1^{4x}h(t)\mathrm{d}t$

Problem 2: Take the derivative with respect to $x$: $\int_{-x}^x 5t\mathrm{d}t$
\section{Trigonometric Integrals}
Trig Integrals:
\begin{itemize}
    \item $\int \cos x\mathrm{d}x = \sin x + C$
    \item $\int -\sin x\mathrm{d}x = \cos x + C$
    \item $\int \sec x\tan x\mathrm{d}x = \sec x + C$
    \item $\int -\csc x\cot x\mathrm{d}x = \csc x + C$
    \item $\int \sec^2 x \mathrm{d}x = \tan x + C$
    \item $\int -\csc^2 x\mathrm{d}x = \cot x + C$
\end{itemize}

Problem 1: $\int^0_{-\pi/16} \sec 4x \tan 4x \mathrm{d}x$

Problem 2: $\int \frac{20x^3}{\sqrt{1-25x^8}}\mathrm{d}x$
\section{Integration with U-Substitution}
\begin{itemize}
    \item Find the outside function and the inside function. The outside function should be the derivative of the inside.
    \item Set $u$ equal to the inside function.
    \item Find $\frac{\mathrm{d}u}{\mathrm{d}x}$. This is the derivative of $u$ with respect to $x$.
    \item Solve for $\mathrm{d}u$ Make sure this matches the outside function $+\mathrm{d}x$.
    \item Substitute the $u$ variables into the integral for the $x$ variables.
    \item Integrate.
    \item Substitute $x$ back into the answer.
\end{itemize}

Problem 1: $\int x(1+2x^2)^2\mathrm{d}x$

Problem 2: $\int \tan^4 x\sec^2 x\mathrm{d}x$

Problem 3: $\int \frac{\csc^2 x}{\cot x}\mathrm{d}x$
\section{Definite Integrals with U Substitution}
Problem 1: $\int_0^1 \frac{x}{x^2+1}^3 \mathrm{d}x$

Problem 2: $\int_1^e \frac{\ln x}{x}\mathrm{d}x$
\section{Complex U Substitution}
Problem 1: $\int (x+1)\sqrt{2-x}\mathrm{d}x$

Problem 2: $\int x^2 \sqrt{1-x}\mathrm{d}x$
\end{document}