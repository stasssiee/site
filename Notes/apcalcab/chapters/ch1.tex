\documentclass[../abcalc.tex]{subfiles}
\graphicspath{{\subfix{../figures/}}}
\begin{document}
\chapter{Limits and Continuity}
\section{Introduction to Limits}
Let's start with an example.

Given the function $f(x)=\frac{x^2-1}{x-1}$, find $f(1)$.

We will end up getting 0/0. This is a difficulty! 0/0 is called indeterminate, so we need another way to answer this.

Let's start by approaching $x=1$ from the left and approach $x=1$ from the right. If we put this in tabular data, we end up approaching 2.

Now we can see that as $x$ gets close to 1, then the function $f(x)=\frac{x^2-1}{x-1}$ gets close to 2.

We are now faced with an interesting situation: When $x=1$, the answer is undefined, but we can see that it is going to be 2.

We want to give the answer ``2'', but we can't, so instead mathematicians say exactly what is going on by using the special world ``limit''.
\begin{center}
    The limit of $\frac{x^2-1}{x-1}$ is 2.
\end{center}

Symbolically, this is written as $lim_{x\to 1}\frac{x^2-1}{x-1}=2$. It is a special way of saying, ``ignore what happens when we get there, but as we get closer and closer, the answer gets closer and closer to 2''.

\begin{definition}
    If when the $x$ values are approaching $x=c$ from either side $f(x)$ becomes arbitrarily close to a single number $y=L$, then the limit $f(x)$ as $x$ approaches $c$ is $L$.
    \[\lim_{x\to c} f(x) = L\]
\end{definition}

Limits can also be used even when we know the value when we get there. Nobody said they are only for difficult functions.

Problem 1: What is the limit of $x^2-3x+4$ as $x$ approaches 2?

Problem 2: What is the limit of $\frac{(x+h)^2-x^2}{h}$ as $h$ approaches 0?
\section{Finding Limits and Limit Properties}
Let $b$ and $c$ be real numbers, let $n$ be a positive integer, and let $f$ and $g$ be functions with the following limits:
\begin{align*}
\lim_{x\to c}f(x)=L \quad \text{and} \quad \lim_{x\to c}g(x)=K
\end{align*}
\begin{enumerate}
    \item scalar multiple: $\lim_{x\to c}[bf(x)]=b[\lim_{x\to c}f(x)]=bL$
    \item sum or difference: $\lim_{x\to c}[f(x)\pm g(x)]=\lim_{x\to c}f(x)\pm \lim_{x\to c}g(x)=L\pm K$
    \item product: $\lim_{x\to c}[f(x)g(x)]=\lim_{x\to c}f(x)\cdot \lim_{x\to c}g(x)=L\cdot K$
    \item quotient: $\lim_{x\to c}\frac{f(x)}{g(x)}=\frac{\lim_{x\to c}f(x)}{\lim_{x\to c}g(x)}=\frac{L}{K}$
    \item power: $\lim_{x\to c}[f(x)]^n = [\lim_{x\to c}f(x)]^n = L^n$
    \item composition: If $f$ is a continuous function, $\lim_{x\to c}f(g(x))=f[\lim_{x\to c}g(x)]=f(K)$
\end{enumerate}

Problem 1: If the limit of $f(x)$ as $x$ approaches $c$ is 9, what is the limit of $4f(x)$ as $x$ approaches $c$?

Problem 2: Find $A$ so that the limit as $x$ approaches 2 for $\frac{x^2+Ax-10}{x-2}$ exists.
\section{One Sided Limits}
Some basic notation:
\begin{align*}
    \lim_{x\to a^-}f(x) \text{means the limit from the left side and} \lim_{x\to a^+} \text{means the limit from the right side.}
\end{align*}

Given horizontal asymptotes, given the power of the highest degree in the numerator and denominator - when the power is higher in the numerator, it will approach infinity, when the power is higher in the 
denominator, it will approach zero, and when the degrees are the same, then the asymptote is the coefficients of the highest degrees divided by each other.

Problem 1: Find the indicated limit.
\[\lim_{x\to -4^+}\left(\frac{3x-1}{x+4}\right)\]
\section{Continuity}
Continuity means that you draw a graph without picking up a pencil.

\begin{itemize}
    \item Removable discontinuity - hole, algebraically, you can find this when a factor in the top cancels out a factor at the bottom.
    \item Jump discontinuity - break, algebraically, it is a piecewise function.
    \item Infinite discontinuity - vertical asymptotes, algebraically, you set the denominator equal to zero.
    \item Mix discontinuity - a mix of any of the above three.
\end{itemize}

\begin{definition}
    A function $f(x)$ is continuous at $x=c$ if and only if $\lim_{x\to c}f(x) = f(c)$.
\end{definition}

In order to prove continuity, you must show three things:
\begin{enumerate}
    \item $\lim_{x\to c}f(x)$ exists.
    \item $f(c)$ is defined.
    \item $\lim_{x\to c}f(x)=f(c)$.
\end{enumerate}

Problem 1: 
Given \( f(x) = 
\begin{cases}
    6+cx & x < 1, \\
    9+2\ln x & x \geq 1.
\end{cases} \)
Find \( c \).
\section{Limits with Infinity}
To find a limit that goes to infinity - in general, a function does not have a limit when the degree of the exponent in the numerator is 
higher than the denominator. It will have a limit of zero when the degree of the exponent in the denominator is higher than the numerator.
It would be the ratio of the coefficients of the highest degree of exponent if the highest exponent degrees are the same.

Problem 1: Find 
\[\lim_{x\to \infty}\frac{3^x-3}{3^x+1}\]
\section{Special Trig Limits}
Know that the special limit
\begin{align*}
\lim_{x\to 0}\frac{\sin x}{x}
\end{align*}
yields a value of 1.

Problem 1: Find
\[\lim_{x\to 0}\frac{\sin 4x}{2x}\]
\section{Intermediate Value Theorem (IVT)}
The Intermediate Value Theorem states that if $f$ is continuous on $[a,b]$, and $k$ is any number between $f(a)$ and $f(b)$, then there is at least one 
number $c$ in $[a,b]$ such that $f(c)=k$.

Problem 1: Use the Intermediate Value Theorem to show that there exists a solution to $\cos x = x$ on the interval $\left[0, \frac{\pi}{2}\right]$.
\end{document}