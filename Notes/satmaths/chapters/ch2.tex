\documentclass[../satmath.tex]{subfiles}
\graphicspath{{\subfix{../figures/}}}
\begin{document}
\chapter{Linear Equations and Algebra}
\begin{enumerate}[label=\bfseries\arabic*.]

\item Line $l$ is defined by the equation $3y-6x=18$. Line $m$ is perpendicular to line $l$ in the $xy$-plane, what is the slope of line $m$?

\item A line with a slope of $\frac{3}{4}$ lies in the $xy$-plane. If $k$ is a constant and the points $(4,5)$ and $(k,-4)$ lie on the line, 
what is the value of $k$?

\item Given the equation $y=8x-96$, what is the $x$ value of the $x$-intercept of the graph in the $xy$-plane?

\item
    
\[f(x)=-1.5x+24\]
Natalie used simple syrup to make a sweet tea. The function above shows the volume, in fluid ounces, of simple syrup that was 
left after Natalie made $x$ cups of sweet tea. In this context, which statement is the best interpretation of the $y$-intercept 
of $y=f(x)$ in the $xy$-plane?

    $\textbf{(A) } \text{Natalie used 1.5 fluid ounces of simple syrup for each cup of sweet tea.} \\ \textbf{(B) } \text{Natalie had 1.5 fluid ounces of simple syrup when she started making tea.} 
\\    \textbf{(C) }\text{Natalie had 24 fluid ounces of simple syrup when she started making tea.} \\ \textbf{(D) }\text{Natalie used 24 fluid ounces of simple syrup for each cup of tea.}$


\item The table below gives some values of a linear equation. Which equation defines $y$?
\begin{table}[h]
    \centering
    \begin{tabular}{|l|l|}
    \hline
    $x$ & $y$ \\ \hline
    0 & 23 \\ \hline
    1 & 27 \\ \hline
    2 & 31 \\ \hline
    \end{tabular}
\end{table}

$\textbf{(A) } y=2x+31 \qquad \textbf{(B) } y=4x+23 \qquad \textbf{(C) } 23x+27 \qquad \textbf{(D) }27x+31$

\item What is the distance between the points $(-2,4)$ and $(6,-17)$?

$\textbf{(A) }\sqrt{285} \qquad \textbf{(B) } \sqrt{233} \qquad \textbf{(C) } \sqrt{377} \qquad \textbf{(D) } \sqrt{505}$

\item In the standard $xy$-coordinate plane, line $p$ passes through $(4,2)$ and $(-2,7)$. If line $r$ is parallel to line $p$, which of the following 
is a possible equation for line $r$?

$\textbf{(A) } y=\frac{6}{5}x-4 \qquad \textbf{(B) } y=\frac{5}{6}x-4 \qquad \textbf{(C) } y=-\frac{5}{6}x+4 \qquad \textbf{(D) } y=-\frac{6}{5}x+4$

\item In the standard $xy$-coordinate plane, line $h$ passes through $(0,2)$ and $(2,3)$. If $h$ is perpendicular to $k$, and line $k$ passes 
through $(-2,5)$, what is the $y$-intercept of line $k$?

$\textbf{(A )} (0,2) \qquad \textbf{(B) } (0,1) \qquad \textbf{(C) } (0,0) \qquad \textbf{(D) } (2,0)$

\item The function $g(k)=-0.12k+108.9$ models approximately how many gallons of rocket fuel are left in a rocket after traveling $k$ kilometers. 
According to the function, about how many gallons of fuel are used to travel each kilometer?

$\textbf{(A) } 0.12 \qquad \textbf{(B) } 9 \qquad \textbf{(C) } 12 \qquad \textbf{(D) } 108.9$

\item 

\begin{table}[h]
    \centering
    \begin{tabular}{|l|l|l|l|}
    \hline
    $x$ & -1 & 2 & 7 \\ \hline
    $y$ & -5 & $k+3$ & 3 \\ \hline
    \end{tabular}
\end{table}

The table shows values of $x$ and their corresponding values of $y$. If there is a linear relationship between $x$ and $y$, what is the value of $k+3$?

\item If $4x+6y=38$ and $y=3x-1$, what is the value of $xy$?

\item If $5a+8b=22$ and $a+4b=14$, which of the following ordered pairs $(a,b)$ is the solution to this system of equations?

$\textbf{(A)} (-6,5) \qquad \textbf{(B) } (-2,4) \qquad \textbf{(C) } (2,3) \qquad \textbf{(D) } (5,-1)$

\item 

\begin{align*}
4y-5x=24\\
kx=2y+4
\end{align*}
In the given system of equations, $k$ is a constant. If the system has no solution, what is the value of $k$?

\item 
\begin{align*}
y\leq 2x+4\\
y\geq -\frac{1}{3}x-2
\end{align*}
Which point $(x,y)$ is a solution to the given system of inequalities shown above?

$\textbf{(A) } (-8,0) \qquad \textbf{(B) } (0,-8) \qquad \textbf{(C) } (0,8) \qquad \textbf{(D) } (8,0)$

\item 
\begin{align*}
3x-4y=7\\
5x+\frac{1}{2}y=26
\end{align*}
What is the value of $x$ in the $(x,y)$ solution to the system of equations above?

\item If Seya gives each of her classmates 2 pieces of candy, she'll have 5 pieces left over. If she wants to give each of her classmates 3 pieces 
of candy, she'll need 10 more pieces of candy. How many pieces of candy does Seya have?

\item 
\begin{align*}
2x-3y=15\\
ax+4y=c
\end{align*}
In the system of linear equations above, $a$ and $c$ are constants such that the system has more than one solution. What is the value of $a$?
 
\item  

\[y\leq 8x-10\]
In which table is every $x$ and $y$ pair a solution to the inequality above?

\[
\textbf{(A) }
\begin{array}{|c|c|c|c|}
\hline
x & 2 & 4 & 6 \\ \hline
y & 6 & 24 & 38 \\ \hline
\end{array}
\]

\[
\textbf{(B) }
\begin{array}{|c|c|c|c|}
\hline
x & 2 & 4 & 6 \\ \hline
y & 5 & 32 & 36 \\ \hline
\end{array}
\]

\[
\textbf{(C) }
\begin{array}{|c|c|c|c|}
\hline
x & 2 & 4 & 6 \\ \hline
y & 5 & 22 & 30 \\ \hline
\end{array}
\]

\[
\textbf{(D) }
\begin{array}{|c|c|c|c|}
\hline
x & 2 & 4 & 6 \\ \hline
y & 10 & 32 & 38 \\ \hline
\end{array}
\]

\item Line $m$ has a slope of 3 and passes through the point $(1,-1)$. Which of the following is the equation for line $m$?

$\textbf{(A) } y=3x-2 \qquad \textbf{(B) } y=3x+4 \qquad \textbf{(C) } y=3x-4 \qquad \textbf{(D) } y=-\frac{1}{3}x-\frac{2}{3}$

\item
\[f(m)=1570+55m\]
The function above models the amount of money in Jordan's bank account after $m$ months. In this context, what is the best interpretation 
of the slope of the function?

$\textbf{(A) } \text{Jordan had \$1,570 in her bank account to begin with.}\\
\textbf{(B) } \text{Jordan adds \$1,570 to her bank account each month.}\\
\textbf{(C) } \text{Jordan had \$55 in her bank account to begin with.}\\
\textbf{(D) } \text{Jordan adds \$55 to her bank account each month.}$

\item 
\begin{table}[h]
    \centering
    \begin{tabular}{|l|l|l|l|l|}
    \hline
    $x$ & 3 & 6 & 9 & 12 \\ \hline
    $f(x)$ & 34 & 25 & 16 & 7 \\ \hline
    \end{tabular}
\end{table}

The table above shows 4 values of $x$ and their corresponding values of $f(x)$. The linear function is defined by the equation $f(x)=hx+43$.
What is the value of $h$?

\item
\begin{align*}
4x-3y=18\\
3x+2y=5
\end{align*}
Given the system of equations above, what is the value of $\frac{x}{y}$?

\item A line in the $xy$-plane passes through the origin and is perpendicular to a line that has a slope of $\frac{2}{7}$. Which of the 
following points must lie on this line?

$\textbf{(A) } (-14,2) \qquad \textbf{(B) } (4,-14)\qquad \textbf{(C) } (4,14)\qquad \textbf{(D) } (2,-14)$ 

\item James earns money by mowing lawns and cleaning pools. He earns \$35 per lawn that he mows and \$50 per pool that he cleans. He needs to earn a 
minimum of \$500 per week to cover his expenses. Which of the following inequalities represents the possible number of lawns he mows, $m$,
and the number of pools he cleans, $p$, each week to meet or exceed his budget requirement?

$\textbf{(A) } 35m+50p \geq 500 \qquad \textbf{(B) }35m+50P > 500 \qquad \textbf{(C) }\frac{35}{m}+\frac{50}{p}\geq 500 \qquad \textbf{(D) }\frac{35}{m}+\frac{50}{p}>500$

\item 
\begin{align*}
2x+3y=15\\
ax-7y=c 
\end{align*}
The system of equations above has no solution. What is the value of $a$?

\item 
\begin{align*}
3y<4\\
x<6y+3
\end{align*}
For the system of inequalities shown above, which of the following points falls within the solution set?

$\textbf{(A) } (-2,11)\qquad \textbf{(B) } (1,12)\qquad \textbf{(C) } (-5,-1)\qquad \textbf{(D) } (-10,-3)$


\item If $4x-2y=8$ and $2y=4$, what is the value of $x$?

\item If line $m$ is parallel to line $n$ and the equation of line $m$ is $y=\frac{2}{3}x+1$, which of the following could be the equation of line $n$?
 
$\textbf{(A) } 2x+3y=4 \qquad \textbf{(B) } 3x+2y=5 \qquad \textbf{(C) } 3x-2y=5 \qquad \textbf{(D) } 2x-3y=5$

\item The average number of fish, $f$ per aquarium at an indoor nature reserve can be estimated by the equation $f = 0.75m+15$ where $m$ represents 
the number of months since January of 2014 and $m\leq 11$. Which of the following statements best explains the function that the number 0.75 serves 
within the context of the problem?

$\textbf{(A) } \text{It is the estimated average number of fish per aquarium in January of 2014.}\\
\textbf{(B) } \text{It is the estimated monthly decrease in the average number of fish per aquarium.}\\
\textbf{(C) } \text{It is the estimated monthly increase in the avearge number of fish per aquarium.}\\
\textbf{(D) } \text{It is the estimated avearge number of fish per aquarium in December of 2014.}$

\item JT's fruit stand sells only peaches and oranges. Peaches sell for \$4 per pound, and oranges sell for \$1.50 per pound. To make a profit, 
JT must sell at least twice as many pounds of peaches as pounds of oranges and must have sales of no less than \$100 per day. If $x$ is the number of 
pounds of peaches he sells and $y$ is the number of pounds of oranges he sells per day, which of the following systems of inequalities represents this situation?

$\textbf{(A) } x\geq 2y \qquad 4x+1.5y\leq 100 \qquad \textbf{(B) } x\geq 2y \qquad 4x+1.5y\geq 100 \qquad \textbf{(C) } y\geq 2x \qquad 4x+1.5y\geq 100 \qquad \textbf{(D) } y\geq 2x \qquad 4x+1.5y\leq 100$

\item
\begin{align*}
3x-4y=14\\
x=-8y
\end{align*}
Based on the system of equations above, what is the value of $xy$?

\item Which of the following systems of linear equations has no solution?

$\textbf{(A) } x=4\qquad y=2x+3 \qquad \textbf{(B) } y=2x-5 \qquad y=2x+3 \qquad \textbf{(C) }x=-5 \qquad y=5 \qquad \textbf{(D) }y=4x-5 \qquad y=2x-5$

\item The minimum value of $b$ is 8 less than 3 times another number $c$. Which inequality gives the possible values of $b$?

$\textbf{(A) } b\geq 8-3c \qquad \textbf{(B) } b\leq 8-3c \qquad \textbf{(C) } b\leq 3c-8 \qquad \textbf{(D) } b\geq 3c-8$

\item Which of the following is a characteristic of the graph of the equation $y+x=k(x-y)$ if $k$ is a constant greater than 1?

\textbf{I.} It has a $y$-intercept of 1.\\
\textbf{II.} It passes through the origin.\\
\textbf{III.} It has a slope between 0 and 1.

$\textbf{(A) } \text{II only} \qquad \textbf{(B) } \text{III only} \qquad \textbf{(C) } \text{I and III} \qquad \textbf{(D) } \text{II and III}$
\end{enumerate}

\section*{Solutions}
\begin{enumerate}[label=\bfseries\arabic*.]
\item $\frac{1}{2}$
\item -8
\item 12
\item C
\item B
\item D
\item C
\item B
\item A
\item -2
\item 10
\item B
\item 10/4
\item D
\item 5
\item 15
\item $-\frac{8}{3}$
\item C
\item C
\item D
\item -9
\item $-\frac{3}{2}$
\item B
\item A
\item $-\frac{14}{3}$
\item C
\item 3
\item D
\item C
\item B
\item -2
\item B
\item D
\item D
\end{enumerate}
\end{document}